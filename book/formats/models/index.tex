\chapter { Modelle }
\paragraph { In Warcraft 3 können zwei unterschiedliche Formate zur Darstellung von Modellen verwendet werden. Zum einen existiert }
\paragraph { Die im Internet zu findenden Spezifikationen basieren auf einem von Jimmy "Nub" Campbell (jcampbelly@gmail.com) erstellten Dokument und stammen, wie auch die anderen Formatspezifikationen, nicht offiziell von Blizzard Entertainment. }
\paragraph { Laut Campbell, basiert sämtlicher Inhalt bzw. sämtliches Wissen über das MDX-Format in der Spezifikation auf statistischen Analysen. Zudem gibt es einige unbekannte Werte, welche jedoch bei sämtlichen Dateien durchgehend gleich sind und somit vermutlich nur eine geringe Bedeutung haben. }
\paragraph { In diesem Kapitel wird sowohl auf die Bedeutung als auch auf die Spezifikation der beiden Formate eingegangen. Es wird dabei versucht, die Daten möglich exakt wiederzugeben, damit es dem Lese theoretisch ermöglicht wird, eine eigene Implementation der Formate zu entwickeln. }
\section { Das Modell }
\paragraph { Zunächst einmal gilt es die Frage zu klären, was im Folgenden eigentlich unter einem Modell zu verstehen ist. Ein Modell ist im Folgenden die Darstellung  bzw. Abbildung eines dreidimensionalen Körpers, welcher für gewöhnlich in Warcraft 3 dateiweise getrennt wird. Das bedeutet, dass jedes Modell in genau einer Datei gespeichert wird. Dies kann entweder eine MDX- oder eine MDL-Datei sein. }

\section { Objekt-Elemente }
Einige Modell-Elemente haben eine Reihe von Eigenschaften gemeinsam, sodass sich diese auch als sogenannte Objekt-Elemente zusammenfassen lassen, da die Reihe von Eigenschaften allgemeine Daten eines Objekts enthalten wie zum Beispiel einen Namen und eine einzigartige Id. Diese Eigenschaften werden im MDX-Format binär direkt hintereinander gespeichert, was die Implementation vereinfacht.
Objekt-Elemente enthalten folgende Eigenschaften:


\chapter { Sequenzen }
\paragraph { Sequenzen beschreiben Animationen mit einem Namen und einer Dauer, damit diese im Spiel abspielbar werden. Oftmals werden Sequenzen fälschlicherweise als Animationen bezeichnet, eventuell um Verwirrung zu vermeiden, da auch Video- bzw. Zwischensequenzen des Spiels gemeint sein könnten. So gibt es in JASS z. B. die Funktionen SetUnitAnimation und QueueUnitAnimation, die trotz ihres Namens Sequenz- und keine Animationsname, da es solche auch gar nicht gibt, als Parameter entgegen nehmen. }
\paragraph { Es gibt einige Sequenznamen, die in Warcraft 3 eine besondere Bedeutung haben und in bestimmten Fällen automatisch abgespielt werden. }
\paragraph { Beim Abspielen einer Sequenz eines Modells durch das Spiel geschieht nichts anderes als das Abspielen der von der Sequenz verwendeten Animation in dem angegebenen Zeitraum der Sequenz. }
\paragraph { Folgende Liste zeigt alle Sequenznamen mit besonderer Bedeutung im Spiel. Groß- und Kleinschreibung muss dabei nicht beachtet werden und (PRÜFEN!!!) die Zahl hinter dem Sequenznamen kann auch als englisches Wort ausgeschrieben werden. Der Wertebereich reicht dabei von X - Y. Wertebereiche ermöglichen ein zufälliges Abspielen einer Sequenz eines Bereiches, damit z. B. nicht immer die gleiche Stehanimation einer EInheit im Spiel abgespielt wird. }
\paragraph { Wird keine Zahl angegeben, so wird die Sequenz standardmäßig als erste interpretiert. Bei mehreren Sequenzen mit dem gleichen Namen wird stets die erste verwendet (PRÜFEN!!!). }
\begin{enumerate}
	\item Stand
	\item Attack
	\item Attack Slam
	\item Spell
	\item Portrait Talk
	\item Morph
	\item Walk
	\item Death
	\item Decay
	\item Victory
\end{enumerate}

\paragraph { Es gibt noch weitere Namen, die den obigen Sequenznamen folgen können und der Sequenz dadurch eine für das Spiel zusätzliche Bedeutung geben: }
\begin{enumerate}
	\item Alternate
	\item Small
	\item Medium
	\item Large
\end{enumerate}
