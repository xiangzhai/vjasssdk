\chapter{Modelle}

\section{Vorwort}
Da Warcraft 3 ein 3d-Echtzeit-Strategiespiel ist, liegt es nahe, dass für die 3d-Grafiken im Spiel ein bestimmtes Format verwendet werden muss.
Blizzard Entertainment hat dieses bestimmte Format selbst entwickelt und die Dokumentation bzw. Spezifikation dessen nicht öffentlich zugänglich gemacht.
Eigentlich handelt es sich um zwei verschiedene Formate, die aber nur exakt das Gleiche beschreiben können.
Diese beiden Formate gibt es nicht etwa, um die Komplexität spaßeshalber zu erhöhen, sondern viel mehr,
um einerseits eine relativ hohe Geschwindigkeit bei der Darstellung einer 3d-Grafik und andererseits
einen einfache Möglichkeit der Bearbeitung dieser zu gewährleisten.
Wie sich der Leser nun sicher schon denken kann handelt es sich beim ersten Format um ein für Menschen
lesbares Textformat, wie etwa XML. Es heißt MDL und ermöglicht Grafikern bzw. Entwicklern eine relativ
einfache Form der Bearbeitung mittels eines ganz normalen Texteditors, insofern man natürlich die Bedeutung
der Elemente kennt.
Das zweite Format ist das binäre Gegenstück dazu und ermöglicht daher eine relativ geringe Ladezeit und zudem noch kleinere
Dateigrößen.
Es heißt MDX und sollte besser nicht mit einem einfachen Texteditor bearbeitet werden.
Beide Formate lassen sich sowohl im Spiel als auch im Welteditor darstellen, wenn auch beide Programme bei
bestimmten Situationen dadurch zum Absturz gebracht werden können.
In diesem Kapitel wird sowohl auf die Bedeutung als auch auf die Spezifikation der beiden Formate eingegangen. Es wird dabei versucht, die Daten möglich exakt wiederzugeben, damit es dem Leser theoretisch ermöglicht wird, eine eigene Implementation der Formate zu entwickeln, so wie es der Autor selbst mit dem
Projekt "wc3lib" versucht.

\section{Spezifikation}
Wie bereits genannt, gibt es keine offizielle Spezifikation. Blizzard veröffentlichte lediglich die sogenannten "Art Tools".
Eine Erweiterung für das recht bekannte 3d-Graikprogramm "3ds Max". Leider ist es maximal mit Version 5 des Grafikprogramms kompatibel,
welche schon lange nicht mehr vertrieben wird.
Zudem binden diese "Art Tool" den Grafiker an ein kommerzielles Programm. Alternativen wie etwa Blender,
können so nicht verwendet werden.
So liegt es sehr nahe, dass Personen, die 3d-Grafiken für Warcraft 3 erstellen wollen irgendwann hergehen
und Konvertierungs- oder sogar Anzeigeprogramme für das Grafikformat schreiben wollen.
Dazu musste das Format jedoch erstmal spezifiziert werden. Das ging vermutlich leichter als z. B. beim MPQ-Format,
da man die Unterschiede einzelner Werte sofort sichtbar machen konnte und man durch die "Art Tools" und das
nicht binäre Format MDL sehr einfache Möglichkeiten hatte, die Auswirkungen der Werte zu analysieren.
Schließlich entstand eine Spezifikation von Jimmy "Nub" Campbell (jcampbelly@gmail.com).
Laut Campbell, basiert sämtlicher Inhalt bzw. sämtliches Wissen über das MDX-Format in der Spezifikation auf statistischen Analysen. Zudem gibt es einige unbekannte Werte, welche jedoch bei sämtlichen Dateien durchgehend gleich sind und somit vermutlich nur eine geringe Bedeutung haben.
Eine neuere und auch bessere Spezifikation wurde später von Magnus Ostberg (aka Magos, MagosX@GMail.com), der den Warcraft-3-Modifizierern
noch heute für sein Programm "War3ModelEditor" bekannt ist, erstellt.
Dadurch, dass er jenes in C++ geschriebene und auf die Windows-Plattform beschränkte Programm entwickelte,
konnte er sich vermutlich ein noch wesentlich präzieseres Bild der beiden Formate machen als Campbell.
Das Programm kann 3d-Grafiken nicht nur relativ korrekt darstellen, es ermöglicht auch eine einfache Form
der Bearbeitung verschiedener Elemente und besitzt einen integrierten MPQ-Browser für Archive.
Magos erstellte auch eine BLP-Spezifikation. Das Texturformat wird selbstverständlich ebenfalls benötigt,
wenn man 3d-Grafiken aus Warcraft 3 anzeigen möchte.
Die Spezifikationen der beiden Formate in diesem Buch basieren hauptsächlich auf Ostbergs Versionen, jedoch auch auf eigenen Erfahrungen und derer anderer.
Sie befinden sich am Ende dieses Kapitels.
Für eigene Implementationen empfehle ich jedem, auch mal einen Blick in den Quell-Code verfügbarer Programme
zu werfen.

\section{Implementationen}
Wie bereits oben beschrieben wurde, existiert das Programm "War3ModelEditor". Zudem gibt es noch das Programm "Warcraft 3 Viewer", welches angeblich auf die vorhandene API der Warcraft-3-Engine
setzt, um Grafiken darzustellen.
Es enthält ebenfalls einen MPQ-Browser, ist jedoch insgesamt weniger komfortabel als Ostbergs'.
Des Weiteren existieren viele kleinere Werkzeuge zur Bearbeitung bestimmter Eigenschaften oder schlichtweg zur Konvertierung.
Bei manchen davon handelt es sich um Skripte, die nur mit zugehörigen Programmen verwendet werden können.
So stieß ich im Jahr 2009 als ich mit den Arbeiten an meinem Programm "wc3lib" begann auf das unfertige
Projekt "warblender", bei welchem es sich um eine Reihe von Python-Skripte handelt, die zusammen eine
Erweiterungen für das freie und kostenlose 3d-Grafikprogramm "Blender" ermöglichen.
Leider funktionierte jene Version der Erweiterung nicht einmal mit der aktuellen von "Blender". Nachdem
ich einige Zeilen umgeschrieben und an die neue Python-API "Blenders" angepasst hatte, konnte ich immerhin
die einfache Geometrie einer Datei importieren.
Der Leser sollte bedenken, dass die Implementation eines Konvertierungsprogramm zwischen zwei 3d-Grafikformaten oftmals
eine beträchtliche Menge Aufwand mit sich bringt, da manche Unterschiede nicht einfach ausgeglichen oder umgangen
werden können.
Hätte ich nicht mit dem Projekt "wc3lib" begonnen, um sämtliche Formate zu implementieren, so hätte für mich die Weiterentwicklung des Projekts "warblender"
vermutlich an erster Stelle gestanden.
Zwar kann ein Grafiker, der Blender benutzt, sein Modell erst in das 3ds-Max-Format und dann über irgendwelche "legalen" Versionen von "3ds Max" in das MDX- oder MDL-Format
konvertieren, jedoch ist das nicht gerade ein empfehlenswerter Arbeitsaufwand.
Das Problem der beiden Darstellungsprogramme "War3ModelEditor" und "Warcraft 3 Viewer" ist, dass sie keine
Konvertierung in ein externes Format erlauben. Dies ist vermutlich deshalb genauso schwer wie ohne die Programme zu bewerkstelligen,
weil sie keinen Nutzen von einer 3d-Rendering-Engine machen, die eigene Formate mit sich bringt.
Das Programm "War3ModelEditor" rendert die Grafiken mit einfachen Direct3d-Befehlen.
Daher setzte ich bei meinem Projekt auf die 3d-Rendering-Engine "OGRE", die eigene Mesh-Formate mit sich bringt und so
Konvertierungen von dargestellten Modellen von sich aus erlaubt.
Selbstverständlich, muss man dazu die Warcraft-3-Grafik korrekt mit dem Format der Rendering-Engine abgleichen.
Hier ist nun eine Liste verschiedener Programme zur Bearbeitung und Anzeige der beiden Formate:
\begin{enumerate}
\item War3ModelEditor - in C++ geschrieben, verwendet eine ältere Version des Programms "StormLib", Direct3d und die Windows-API).
\item MdxLib - Ostbergs Nachfolgeprojekt, in C# geschrieben.
\end{enumerate}

\section{Das Modell}
Zunächst einmal gilt es die Frage zu klären, was im Folgenden eigentlich unter einem Modell zu verstehen ist. Ein Modell ist im Folgenden die Darstellung  bzw. Abbildung eines dreidimensionalen Körpers, welcher für gewöhnlich in Warcraft 3 dateiweise getrennt wird. Das bedeutet, dass jedes Modell in genau einer Datei gespeichert wird. Dies kann entweder eine MDX- oder eine MDL-Datei sein.
Es gilt dabei zu beachten, dass die Texturen, also die 2d-Grafiken bzw. Bilder, die auf das Modell gelegt werden,
um diesem eine Art Haut zu verleihen und somit die Grenzen der möglichen Punkte im virtuellen dreidimensionalen Raum nicht zu sprengen, separat abgespeichert werden.
Eine Textur wird also in der Regel auch in einer eigenen Datei abgespeichert, kann aber von einem
Modell mehrmals verwendet werden.
Ein Modell enthält unterschiedliche Elemente, die es beschreiben. Diese Elemente sind in anderen 3d-Grafikformaten meist ähnlich oder sogar gleich,
da die Technik für die Darstellung und Bearbeitung von 3d-Grafiken auf Softwareebene zum Zeitpunkt der Entwicklung des Formats bereits fortgeschritten genug war, sodass eine gewisser inoffizieller Standard existierte,
der bei den Formaten MDX und MDLX sicherlich auch durch Blizzard's Verwendung des Bearbeitungsprogramms
3ds Max und dessen eigenes Format (.3ds) beeinflusst wurde.
In den einzelnen Kapiteln wird auf die Bedeutung jedes Elements genauer eingegangen.

\section{Knoten-Elemente}
Einige Modell-Elemente haben eine Reihe von Eigenschaften gemeinsam, sodass sich diese auch als sogenannte Knoten-Elemente zusammenfassen lassen, da die Reihe von Eigenschaften allgemeine Daten eines Knotens enthält, wie zum Beispiel einen Namen und eine einzigartige Id. Diese Eigenschaften werden im MDX-Format binär direkt hintereinander gespeichert, was die Implementation vereinfacht.
Dennoch kommen bestimmte Eigenschaften wie Verschiebungen und Rotationen nur in speziellen Fällen vor.
Knoten-Elemente unterstützen spezielle Funktionen wie Kind-Eltern-Beziehungen zu anderen Knoten-Elementen, was eine besondere Bedeutung bei Sequenzen hat.
Zudem beziehen sich einige weitere Elemente des Modells, wie z. B. Achsenpunkte, auf eine bestimmte Knoten-Id und somit auf ein bestimmtes Knoten-Element des Modells.
Das MDLX-Format bringt hierbei keine einzigartige Neuheit mit sich, da dieses Prinzip bei den meisten
anderen 3d-Formaten ebenfalls verwendet wird.
Der Typ jedes Knotens wird ebenfalls gespeichert, was Programmierern eine leichtere Bestimmung bei einer
nötigen Konvertierung erlaubt.
Lediglich bei Partikel-Emittern muss darauf geachtet werden, um welchen der beiden möglichen Implementationen
es sich handelt.

Knoten-Elemente enthalten folgende Eigenschaften:
\enumerate{begin}
\item einen Namen (maximal 80 Zeichen lang)
\item eine einzigartige Id (in Form einer Zahl)
\item einen Elternknoten (optional)
\item einen Typ
\item Verschiebungen, Drehungen und Skalierungen, falls diese nicht vom Elternknoten geerbt werden (optional)

\subsection{Typ}
Zum Einen wird im Typ-Flag der Typ des Knotens angegeben (z. B. Knochen oder Licht) und zum Anderen bestimmte Eigenschaften,
die das Verhalten des Knotens bzw. dessen Darstellung beeinflussen.
Folgende Elemente eines Modells sind Knotenelemente:
\enumerate{begin}
\item Helfer
\item Knochen
\item Lichter
\item Ereignisobjekte
\item Anhänge
\item Partikel-Emitter (zwei unterschiedliche Typen)
\item Kollisionsformen
\item Bändel-Emitter
\enumerate{end}

Folgende Eigenschaften können im Typ bestimmt werden:
\enumerate{begin}
\item Die Z-Koordinaten (der Bewegungen?) befinden sich immer relativ zur Benutzeransicht - BillboardedLockZ
\item Die Y-Koordinaten (der Bewegungen?) befinden sich immer relativ zur Benutzeransicht - BillboardedLockY
\item Die X-Koordinaten (der Bewegungen?) befinden sich immer relativ zur Benutzeransicht - BillboardedLockX
\item Billboarded,
\item CameraAnchored,
\item Bestimmte Bewegungen werden vom Elternknoten nicht mitübernommen - DontInherit { <Rotation|Translation|Scaling> },

\subsection{Billboarded}
Billboarded (engl. etwa "als Anschlagtafel angebracht") wird in der comptergesteuerten 3d-Grafik dazu verwendet, Grafiken immer an der Benutzeransicht auszurichten. Das bedeutet, dass sich Eigenschaften des Knoten-Elements immer automatisch in Richtung der Benutzeransicht ausrichten. Dabei können alle drei Achsen einzeln definiert werden. Dies wird z. B. für relativ einfach aufgebaute Grafiken, die nur aus einer zweidimensionalen Fläche bestehen, verwendet und spart Leistung.


\chapter{Sequenzen}
\paragraph { Sequenzen beschreiben Animationen mit einem Namen und einer Dauer, damit diese im Spiel abspielbar werden. Oftmals werden Sequenzen fälschlicherweise als Animationen bezeichnet, eventuell um Verwirrung zu vermeiden, da auch Video- bzw. Zwischensequenzen des Spiels gemeint sein könnten. So gibt es in JASS z. B. die Funktionen SetUnitAnimation und QueueUnitAnimation, die trotz ihres Namens Sequenz- und keine Animationsname, da es solche auch gar nicht gibt, als Parameter entgegen nehmen. }
\paragraph { Es gibt einige Sequenznamen, die in Warcraft 3 eine besondere Bedeutung haben und in bestimmten Fällen automatisch abgespielt werden. }
\paragraph { Beim Abspielen einer Sequenz eines Modells durch das Spiel geschieht nichts anderes als das Abspielen der von der Sequenz verwendeten Animation in dem angegebenen Zeitraum der Sequenz. }
\paragraph { Folgende Liste zeigt alle Sequenznamen mit besonderer Bedeutung im Spiel. Groß- und Kleinschreibung muss dabei nicht beachtet werden und (PRÜFEN!!!) die Zahl hinter dem Sequenznamen kann auch als englisches Wort ausgeschrieben werden. Der Wertebereich reicht dabei von X - Y. Wertebereiche ermöglichen ein zufälliges Abspielen einer Sequenz eines Bereiches, damit z. B. nicht immer die gleiche Stehanimation einer EInheit im Spiel abgespielt wird. }
\paragraph { Wird keine Zahl angegeben, so wird die Sequenz standardmäßig als erste interpretiert. Bei mehreren Sequenzen mit dem gleichen Namen wird stets die erste verwendet (PRÜFEN!!!). }
\begin{enumerate}
\item Stand
\item Attack
\item Attack Slam
\item Spell
\item Portrait Talk
\item Morph
\item Walk
\item Death
\item Decay
\item Victory
\end{enumerate}

\paragraph { Es gibt noch weitere Namen, die den obigen Sequenznamen folgen können und der Sequenz dadurch eine für das Spiel zusätzliche Bedeutung geben: }
\begin{enumerate}
\item Alternate
\item Small
\item Medium
\item Large
\end{enumerate}
