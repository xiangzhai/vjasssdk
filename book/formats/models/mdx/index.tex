\chapter{MDX}
Im Gegensatz zum bereits vorgestellen MDL-Format, ist das MDX-Format ein Binärformat. Das bedeutet, dass Dateien dieses Formates binär abgespeichert werden. Dies hat einerseits zur Folge, dass die Dateien meist deutlich kleiner und somit auch schneller zu laden und andererseits, dass sie von Menschen so gut wie nicht lesbar sind.
Der Große Vorteil des MDX- gegenüber des MDL-Formates liegt auf der Hand:Es erhöht die Spiel- und Editor-Leistung, das MDX-Format zu verwenden.
Die im Internet zu findenden Spezifikationen basieren auf einem von Jimmy "Nub" Campbell (jcampbelly@gmail.com) erstellen Dokument und stammen, wie auch die anderen Formatspezifikationen, nicht offiziell von Blizzard Entertainment.
Laut des Dokuments, basiert sämtlicher Inhalt bzw. sämtliches Wissen über das MDX-Format auf statistischen Analysen. Zudem gibt es einige unbekannte Werte, welche jedoch bei sämtlichen Dateien durchgehend gleich sind.

Kennzeichen
Jeder Eigenschaftssatz in einer MDX-Datei beginnt mit einem einzigartigen Kennzeichen, bestehende aus vier ASCII-Zeichen.
