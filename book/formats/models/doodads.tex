\chapter{Doodads und Zerstörbare}
Doodads und Zerstörbare sind spezielle Objekttypen, die der Dekoration von Karten in Warcraft 3 dienen. Es ist möglich bis zu 10 verschiedene Variationen des selben Doodad- oder Zerstörbares-Typen zu definieren.
Dabei handelt es sich um unterschiedliche Modelldateien, die mit dem Postfix ihrer Variationsnummer im Dateinamen unterschieden werden.
Alle müssen sich den selben Verzeichnispfad teilen.
Die Verwendung von Variationen erlaubt eine automatische Platzierung von zufälligen Variationen, was die Abwechslung in Landschaften erhöht.
Die Anzahl der unterschiedlichen Variationen muss im Objekteditor beim jeweiligen Objekt eingetragen werden.
\section{Bäume}
Eine spezielle Form eines Zerstörbaren ist ein Baum. Bei Bäumen werden automatisch drei verschiedene Modelldateien geladen (falls vorhanden).
Eine Version stellt den stehenden Baum ohne Animationen dar, eine andere den abgeholzten bzw. toten Baum ohne Animationen und eine weitere den Baum mit all seinen Animationen.
Diese Technik wird verwendet, um Bäume besonders schnell anzeigen zu können, da die Entwickler von Warcraft 3 vermutlich davon ausgehen, dass vor allem viele Bäume in Karten dargestellt werden, was auf die meisten Nahkampf- bzw. Standardkarten des Spiels auch zutrifft.
Die Standversion der Modelldatei (ohne Animationen) trägt das Postfix S im Dateinamen, die Todesversion (ohne Animationen) das Postfix D. Die Standardversion mit allen Animationen trägt kein Postfix.
Das Baumpostifx kann mit dem Variationspostfix kombiniert werden und muss nach diesem stehen.