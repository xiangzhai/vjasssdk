\chapter { Sequenzen }
\paragraph { Sequenzen beschreiben Animationen mit einem Namen und einer Dauer, damit diese im Spiel abspielbar werden. Oftmals werden Sequenzen fälschlicherweise als Animationen bezeichnet, eventuell um Verwirrung zu vermeiden, da auch Video- bzw. Zwischensequenzen des Spiels gemeint sein könnten. So gibt es in JASS z. B. die Funktionen SetUnitAnimation und QueueUnitAnimation, die trotz ihres Namens Sequenz- und keine Animationsname, da es solche auch gar nicht gibt, als Parameter entgegen nehmen. }
\paragraph { Es gibt einige Sequenznamen, die in Warcraft 3 eine besondere Bedeutung haben und in bestimmten Fällen automatisch abgespielt werden. }
\paragraph { Beim Abspielen einer Sequenz eines Modells durch das Spiel geschieht nichts anderes als das Abspielen der von der Sequenz verwendeten Animation in dem angegebenen Zeitraum der Sequenz. }
\paragraph { Folgende Liste zeigt alle Sequenznamen mit besonderer Bedeutung im Spiel. Groß- und Kleinschreibung muss dabei nicht beachtet werden und (PRÜFEN!!!) die Zahl hinter dem Sequenznamen kann auch als englisches Wort ausgeschrieben werden. Der Wertebereich reicht dabei von X - Y. Wertebereiche ermöglichen ein zufälliges Abspielen einer Sequenz eines Bereiches, damit z. B. nicht immer die gleiche Stehanimation einer EInheit im Spiel abgespielt wird. }
\paragraph { Wird keine Zahl angegeben, so wird die Sequenz standardmäßig als erste interpretiert. Bei mehreren Sequenzen mit dem gleichen Namen wird stets die erste verwendet (PRÜFEN!!!). }
\begin{enumerate}
	\item Stand
	\item Attack
	\item Attack Slam
	\item Spell
	\item Portrait Talk
	\item Morph
	\item Walk
	\item Death
	\item Decay
	\item Victory
\end{enumerate}

\paragraph { Es gibt noch weitere Namen, die den obigen Sequenznamen folgen können und der Sequenz dadurch eine für das Spiel zusätzliche Bedeutung geben: }
\begin{enumerate}
	\item Alternate
	\item Small
	\item Medium
	\item Large
\end{enumerate}