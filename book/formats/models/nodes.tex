\section{Knoten-Elemente}
Einige Modell-Elemente haben eine Reihe von Eigenschaften gemeinsam, sodass sich diese auch als sogenannte Knoten-Elemente zusammenfassen lassen, da die Reihe von Eigenschaften allgemeine Daten eines Knotens enthält, wie zum Beispiel einen Namen und eine einzigartige Id. Diese Eigenschaften werden im MDX-Format binär direkt hintereinander gespeichert, was die Implementation vereinfacht.
Dennoch kommen bestimmte Eigenschaften wie Verschiebungen und Rotationen nur in speziellen Fällen vor.
Knoten-Elemente unterstützen spezielle Funktionen wie Kind-Eltern-Beziehungen zu anderen Knoten-Elementen, was eine besondere Bedeutung bei Sequenzen hat.
Zudem beziehen sich einige weitere Elemente des Modells, wie z. B. Achsenpunkte, auf eine bestimmte Knoten-Id und somit auf ein bestimmtes Knoten-Element des Modells.
Das MDLX-Format bringt hierbei keine einzigartige Neuheit mit sich, da dieses Prinzip bei den meisten
anderen 3d-Formaten ebenfalls verwendet wird.
Der Typ jedes Knotens wird ebenfalls gespeichert, was Programmierern eine leichtere Bestimmung bei einer
nötigen Konvertierung erlaubt.
Lediglich bei Partikel-Emittern muss aufgepasst werden, um welchen der beiden möglichen Implementationen
es sich handelt.

Knoten-Elemente enthalten folgende Eigenschaften:
* einen Namen (maximal 80 Zeichen lang)
* eine einzigartige Id (in Form einer Zahl)
* [einen Elternknoten]
* [Verschiebungen, Drehungen und Skalierungen, falls diese nicht vom Elternknoten geerbt werden]

Folgende Elemente eines Modells sind Objektelemente:
* Helfer
* Knochen
* Lichter
* Ereignisobjekte
* Anhänge
* Partikel-Emitter (zwei unterschiedliche Typen)
* Kollisionsformen
* Bändel-Emitter

