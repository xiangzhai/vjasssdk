\section{Knoten-Elemente}
Einige Modell-Elemente haben eine Reihe von Eigenschaften gemeinsam, sodass sich diese auch als sogenannte Knoten-Elemente zusammenfassen lassen, da die Reihe von Eigenschaften allgemeine Daten eines Knotens enthält, wie zum Beispiel einen Namen und eine einzigartige Id. Diese Eigenschaften werden im MDX-Format binär direkt hintereinander gespeichert, was die Implementation vereinfacht.
Dennoch kommen bestimmte Eigenschaften wie Verschiebungen und Rotationen nur in speziellen Fällen vor.
Knoten-Elemente unterstützen spezielle Funktionen wie Kind-Eltern-Beziehungen zu anderen Knoten-Elementen, was eine besondere Bedeutung bei Sequenzen hat.
Zudem beziehen sich einige weitere Elemente des Modells, wie z. B. Achsenpunkte, auf eine bestimmte Knoten-Id und somit auf ein bestimmtes Knoten-Element des Modells.
Das MDLX-Format bringt hierbei keine einzigartige Neuheit mit sich, da dieses Prinzip bei den meisten
anderen 3d-Formaten ebenfalls verwendet wird.
Der Typ jedes Knotens wird ebenfalls gespeichert, was Programmierern eine leichtere Bestimmung bei einer
nötigen Konvertierung erlaubt.
Lediglich bei Partikel-Emittern muss darauf geachtet werden, um welchen der beiden möglichen Implementationen
es sich handelt.

Knoten-Elemente enthalten folgende Eigenschaften:
\enumerate{begin}
\item einen Namen (maximal 80 Zeichen lang)
\item eine einzigartige Id (in Form einer Zahl)
\item einen Elternknoten (optional)
\item einen Typ
\item Verschiebungen, Drehungen und Skalierungen, falls diese nicht vom Elternknoten geerbt werden (optional)

\subsection{Typ}
Zum Einen wird im Typ-Flag der Typ des Knotens angegeben (z. B. Knochen oder Licht) und zum Anderen bestimmte Eigenschaften,
die das Verhalten des Knotens bzw. dessen Darstellung beeinflussen.
Folgende Elemente eines Modells sind Knotenelemente:
\enumerate{begin}
\item Helfer
\item Knochen
\item Lichter
\item Ereignisobjekte
\item Anhänge
\item Partikel-Emitter (zwei unterschiedliche Typen)
\item Kollisionsformen
\item Bändel-Emitter
\enumerate{end}

Folgende Eigenschaften können im Typ bestimmt werden:
\enumerate{begin}
\item Die Z-Koordinaten (der Bewegungen?) befinden sich immer relativ zur Benutzeransicht - BillboardedLockZ
\item Die Y-Koordinaten (der Bewegungen?) befinden sich immer relativ zur Benutzeransicht - BillboardedLockY
\item Die X-Koordinaten (der Bewegungen?) befinden sich immer relativ zur Benutzeransicht - BillboardedLockX
\item Billboarded,
\item CameraAnchored,
\item Bestimmte Bewegungen werden vom Elternknoten nicht mitübernommen - DontInherit { <Rotation|Translation|Scaling> },

\subsection{Billboarded}
Billboarded (engl. etwa "als Anschlagtafel angebracht") wird in der comptergesteuerten 3d-Grafik dazu verwendet, Grafiken immer an der Benutzeransicht auszurichten. Das bedeutet, dass sich Eigenschaften des Knoten-Elements immer automatisch in Richtung der Benutzeransicht ausrichten. Dabei können alle drei Achsen einzeln definiert werden. Dies wird z. B. für relativ einfach aufgebaute Grafiken, die nur aus einer zweidimensionalen Fläche bestehen, verwendet und spart Leistung.

