\chapter{Aufzählungen}
Aufzählungselemente dürfen nur konstante Ausdrücke als Werte enthalten.
Alle Aufzählungselemente müssen den geerbten Typ haben.
Wird der geerbte Typ weggelassen, so ist die Aufzählung standardmäßig vom nativen Typ "integer".
Aufzählungen sind konstante Behältertypen (siehe "Behältertypen") und können daher auch in "foreach"-Schleifen verwendet werden.
Wird ein Aufzählungselement bei der Deklaration gesetzt, so werden alle darauffolgenden nicht explizit gesetzten Elemente,
auf die noch nicht verwendeten nachfolgenden Werte gesetzt, falls solche existieren.
Bei Ganz- und Fließkommazahlen wird dabei einfach in 1er-Schritten hochgezählt.
Beim Typ "boolean" gilt die Reihenfolge "true", "false". Bei vordefinierten Typen (siehe "Typen") gilt die aufsteigende
Reihenfolge Ihrer Ganzzahlwerte.
Ist das letzte Element einer Reihenfolge erreicht, so erhalten alle weiteren Elemente ebenfalls das letzte
und der Compiler gibt eine Warnung aus.
Der "size"-Operator kann verwendet werden, um die Anzahl der Aufzählungselemente zu erhalten. 

\section{Notation}
enum <Bezeichner> (<geerbter Typ>) [Deklarationsvoraussetzungen]
{
    <Elementname> [= <Elementwert>] [,] // Nach dem letzten Element muss kein Komma gesetzt werden.
} [globale Instanzen durch Kommate getrennt];