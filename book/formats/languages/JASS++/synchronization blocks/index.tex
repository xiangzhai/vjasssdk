\chapter{Synchronisationsblöcke}
Sogenannte Synchronisationsblöcke ermöglichen das Ausführen von Code für einen bestimmten Spieler bzw. eine bestimmte Spielergruppe.
Dabei können dem Schlüsselwort Spieler- (player) und Streitmacht (force)-Ausdrücke folgen, für die der auszuführende Code lokal ausgeführt werden soll.
Umgesetzt wird das Ganze vom Compiler über die Funktion "GetLocalPlayer". Lokalblöcke werden in JASS++ benötigt, um den Compiler den enthaltenen Code zu validieren, da nicht sämtlicher Code innerhalb eines solchen Blocks stehen darf.
Es gibt eine Reihe nativer Funktionen, die ausdrücklich zugelassen ist.
Zudem dürfen nur Variablen bestimmter Typen eine Zuweisung erhalten und der Compiler sollte eine Warnung ausgeben, falls diese außerhalb des Blocks noch einmal verwendet werden.
Außerdem wird zur Kompilier- und Laufzeit eine Warnung ausgegeben, falls ein Synchronisationsblock für einen nicht menschlichen Spieler ausgeführt wird.
Dabei wird zur Laufzeit auch überprüft, ob der Spieler das Spiel bereits verlassen hat.

\section{Notation}
local (<Spieler-/Streitmachtausdruck 1>[, <Spieler-/Streitmachtausdruck 2>, … <Spieler-/Streitmachtausdruck n>])
{
	<Anweisungen>
}

\section{Liste erlaubter nativer Typen, deren Werte verändert werden dürfen}
* string
* integer
* real
* boolean
TODO

\section{Liste erlaubter nativer Funktionen}
* MultiboardDisplay
* LeaderboardDisplay
* StartSound
TODO 
