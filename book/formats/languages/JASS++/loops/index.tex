\chapter{Schleifen}
Schleifen erlauben dem Programmierer, Anweisungen automatisiert mehrmals ausführen zu lassen.
Für gewöhnlich wird in Programmiersprachen zwischen kopf- und fußgesteuerten Schleifen unterschieden.
Eine Ausnahme bilden dabei in JASS++ Endlosschleifen, da sie keinen Wahrheitsausdruck benötigen.
Die Verwendung der Schlüsselwörter "case", "default", und "else" erlauben dem Programmierer zusätzliche Möglichkeiten,
auf den Verlauf der Schleifenausführungen zu reagieren, ohne selbst entsprechende Variable zu deklarieren und zu setzen.
Mit dem Schlüsselwort "break" können Schleifenabläufe abgebrochen werden. Mit dem Schlüsselwort "continue" können Schleifen-
abläufe erneut am Anfang des Schleifenkörpers begonnen werden.

\section{Bedingte Schleifen}

\subsection{while-Schleifen}
while-Schleifen sind kopfgesteuerte Schleifen, bei welchen zunächst ein Wahrheitsausdruck angegeben werden muss.
Solange dieser wahr ist, wird der Schleifenkörper ausgeführt.

\subsubsection{Notation}
while (<Wahrheitsausdruck>)
{
	[break; | continue;]
}
case (<Ganzzahlausdruck>) // Schleife wurde <Ganzzahlausdruck> mal durchlaufen.
{
}
default // Schleife wurde nicht durch eine break-Anweisung abgebrochen
{
}
else // Schleife wurde durch eine break-Anweisung abgebrochen
{
}

\subsection{do-while-Schleifen}
do-while-Schleifen sind fußgesteuerte Schleifen, bei welchen zunächst der Schleifenkörper ausgeführt wird und danach
solange der angegebene Wahrheitsausdruck wahr ist.

\subsubsection{Notation}
do
{
	[break; | continue;]
}
while (<Wahrheitsausdruck>);
case (<Ganzzahlausdruck>) // Schleife wurde <Ganzzahlausdruck> mal durchlaufen.
{
}
default // Schleife wurde nicht durch eine break-Anweisung abgebrochen
{
}
else // Schleife wurde durch eine break-Anweisung abgebrochen
{
}

\subsection{until-Schleifen}
until-Schleifen sind kopfgesteuerte Schleifen, bei welchen zunächst ein Wahrheitsausdruck angegeben werden muss.
Solange dieser falsch ist, wird der Schleifenkörper ausgeführt.

\subsubsection{Notation}
until (<Wahrheitsausdruck>)
{
	[break; | continue;]
}
case (<Ganzzahlausdruck>) // Schleife wurde <Ganzzahlausdruck> mal durchlaufen.
{
}
default // Schleife wurde nicht durch eine break-Anweisung abgebrochen
{
}
else // Schleife wurde durch eine break-Anweisung abgebrochen
{
}

\subsection{do-until-Schleifen}
do-until-Schleifen sind fußgesteuerte Schleifen, bei welchen zunächst der Schleifenkörper ausgeführt wird und danach
solange der angegebene Wahrheitsausdruck falsch ist.

\subsubsection{Notation}
do
{
	[break; | continue;]
}
until (<Wahrheitsausdruck>);
case (<Ganzzahlausdruck>) // Schleife wurde <Ganzzahlausdruck> mal durchlaufen.
{
}
default // Schleife wurde nicht durch eine break-Anweisung abgebrochen
{
}
else // Schleife wurde durch eine break-Anweisung abgebrochen
{
}

\section{Zählerschleifen}

\subsection{for-Schleifen}
for (<Anweisung>; <Wahrheitsausdruck>; <Anweisung>)
{
	[break; | continue;]
}
case (<Ganzzahlausdruck>) // Schleife wurde <Ganzzahlausdruck> mal durchlaufen.
{
}
default // Schleife wurde nicht durch eine "break"-Anweisung abgebrochen
{
}
else // Schleife wurde durch eine "break"-Anweisung abgebrochen
{
}

\section{Behälter-Schleifen}

\subsection{foreach-Schleifen}
foreach (<Variablenname>; <Behälterausdruck>) // siehe "Typen - Behältertypen"
{
	[break; | continue;]
}
case (<Ganzzahlausdruck>) // Schleife wurde <Ganzzahlausdruck> mal durchlaufen.
{
}
default // Schleife wurde nicht durch eine "break"-Anweisung abgebrochen
{
}
else // Schleife wurde durch eine "break"-Anweisung abgebrochen
{
}

\section{Endlosschleifen}
Diese Schleifen können nur mit "break" abgebrochen werden. Besitzt eine Schleife dieser Art keine oder nur eine indirekte
"break"-Anweisung, so muss der Compiler eine Warnung ausgeben.

\subsection{forever-Schleifen}
Die Verwendung des "default"-Blocks ist selbstverständlich nicht möglich.

\subsubsection{Notation}
forever
{
	[break; | continue;]
}
case (<Ganzzahlausdruck>) // Schleife wurde <Ganzzahlausdruck> mal durchlaufen.
{
}
else // Schleife wurde durch eine "break"-Anweisung abgebrochen
{
}