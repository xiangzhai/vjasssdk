\chapter{Variable}
Variable existieren, falls nicht anders angegeben, vom Programmstart bis zum Programmende. Nicht statische Variable in Funktionen und
Klassen leben vom Funktionsaufruf bis zum Ende des Funktionsablaufes bzw. von der Erzeugung der Klasseninstanz
bis zur Löschung dieser.
Variable, die als "static" deklariert wurden, existieren ebenfalls vom Programmstart bis zum Programmende.
Anders als in C++ können in JASS++ keine Variable in beliebig verschachtelten Gültigkeitsbereichen deklariert werden, die am Ende dieser
automatisch gelöscht werden.
Sie können ausschließlich global oder in Gültigkeitsbereichen deklariert werden und tragen implizit den Qualifizerer "dynamic",
können aber auch als "static" deklariert werden. Es gilt zu beachten, dass diese statischen Variable wie andere globale
bei ihrer Deklaration bzw. dem Programmstart ein einziges Mal initialisiert werden und nicht wie bei lokalen z. B. bei jedem Funktionsaufruf.
Man sollte des Weiteren beachten, dass auch Variable innerhalb von Funktionen und Methoden Zugriffsmodifikatoren haben können (siehe "Funktionen").
Variable innerhalb von Funktionen und Methoden, die nicht statisch sind, dürfen nur am Anfang des Funktionskörpers deklariert werden.
Statische dagegen an beliebiger Stelle innerhalb der Funktion bzw. Methode.
Da nicht statische Variable in Funktionen und Methoden nur vom Aufruf bis zum Ablaufsende leben, kann auf sie auch nicht zugegriffen werden, wenn sie
öffentlich sind (siehe "Zugriffsmodifikatoren"). Auch nicht wenn es sich um eine Funktionsvariable handelt!

\section{Initialisierungsnotation}
{ <Ausdruck 1>[, <Ausdruck 2>] } | <Ausdruck>

\section{Bezeichnernotation}
<Variablenbezeichner>[[[<Variablengröße der 1. Dimension>]][<Variablengröße der n. Dimension>]] [= <Initialisierungsnotation>]

\section{Notation}
[Zugriffsmodifikatoren] [static] [const | var] <Variablentyp> | <Typendeklaration> <Bezeichnernotation 1>[, <Bezeichnernotation 2>, <Bezeichnernotation n>];

\section{var}
Das Schlüsselwort "var" ist optional und dient besserer Lesbarkeit des Codes.
Konstanten werden dagegen zwingend mit "const" deklariert.

\section{Konstanten}
Statische und nicht statische Variablendeklarationen können mit dem Schlüsselwort "const" als Konstantendeklarationen definiert
werden.
In diesem Fall entfällt das optionale Schlüsselwort "var" und die Konstante muss bei ihrer Deklaration initialisiert werden.
Der Inhalt von Konstanten darf nicht nachträglich geändert werden.

\section{Gültigkeitsbereiche}
Variable können global oder in Gültigkeitsbereichen deklariert werden.
Klassen, Funktionen und Methoden können statische Variable enthalten:
static var integer test;
static const integer test = 10;

\section{Typen}
Bei kopierbasierten Typen wird das Literal "null" für eine Leerung des Variableninhalts benutzt. Bei referenzbasierten Typen
dagegen 0 (siehe Abschnitt "Literale").
Es gilt zu beachten, dass die Objekte, auf welche bei Variablen mit referenzierbasierten Typen verwiesen wird, mit speziellen Funktionen
und nicht mit einer einfachen 0-Setzung gelöscht werden.
Die Sprache garantiert jedoch, dass Variable nach einer Löschung des verwiesenen Objekts den Inhalt 0 haben, insofern die Löschung
erfolgreich war (keine Ausnahme wurde geworfen) (siehe "Ausnahmen" und "Methoden - Destruktor").

\section{Arrays}
Arrays dienen einer generischen Schnittstelle für beliebige Behältertypen. Daher können Variable eines jeden Typs (ausgenommen "code")
als Array deklariert werden.
Array-Größenangaben müssen Ganzzahlwerte sein. Sie können weggelassen werden, falls sich die Größe aus der Initialisierung heraus
ermitteln lässt.
Der Variableninhalt wird bei Arrays wie in C/C++ mittels geschweifter Klammern eingeschlossen. Es können mehrdimensionale
Arrays deklariert werden. Die Größe von Arrays ist durchgehend dynamisch. Außerdem können Arrays im Gegensatz zu Arrays in JASS
als Parameter und Rückgabewert übergeben werden. Zudem kann einem Array der Inhalt eines anderen Arrays zugewiesen werden.
Des Weiteren gelten für Arrays die selben Regeln wie für alle Behältertypen (siehe "Behältertypen"), jedoch sind Arrays kopier-
und nicht referenzbasiert!

\subsection{Anmerkung zur Compiler-Implementation}
Arrays in JASS++ müssen mit Hash-Tabellen oder Spiel-Caches umgesetzt werden. Es gilt dabei zu beachten, dass sie kopierbasiert sind
und am Ende ihres Gültigkeitsbereiches wieder freigegeben werden müssen.

\subsection{Initialisierungsnotation}
{ { <Wert 1>, <Wert 2>, <Wert 3>, <Wert n> }, { <Wert 1>, <Wert 2>, <Wert 3>, <Wert n> } };

\subsection{Notation}
<Typ> <Bezeichner>[<Größe der 1. Dimension>][ [<Größe der 2. Dimension>] ][ [<Größe der 3. Dimension>] ] [= Initialisierungsnotation]