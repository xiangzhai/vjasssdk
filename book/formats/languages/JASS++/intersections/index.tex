\chapter{Verzweigungen}

\section{Wenn-Dann-Ansonsten-Verzweigungen}
Wenn-Dann-Ansonsten-Verzweigunen werden zur Überprüfung unterschiedlicher Ausdrücke benutzt.
Sie werden mit dem Schlüsselwort "if" eingeleitet, welchem in jedem Fall ein in Klammern eingeschlossener Wahrheitsausdruck (Typ "boolean")
folgen muss. Ist dieser Ausdruck zur Laufzeit wahr, so werden die Anweisungen im durch eckige Klammern eingeschlossenen Anweisungsblock
ausgeführt.
Optional können mit "else if" weitere Ausdrücke überprüft werden, falls der erste nicht zutrifft. Des Weiteren können mit einem ebenfalls optionalen
"else" Anweisungen ausgeführt werden, falls keiner der überprüften Ausdrücke wahr ist.
Falls ein konstanter Ausdruck überprüft wird, sollte der Compiler einen Hinweis ausgeben, falls der konstante Ausdruck nicht wahr ist, eine Warnung.

Notationen:
if (<Wahrheitsausdruck>)
{
}
[else if (<Wahrheitsausdruck>)
{
}
else (<Wahrheitsausdruck>)
{
}]

\section{Fall-Verzweigungen}
Fall-Verzweigungen können anders als in herkömmlichen Sprachen wie C/C++ jede Art von Datentyp behandeln. Außerdem müssen die Ausdrücke in den "case"-
Verzweigungen nicht konstant sein.
Entfällt die "break"-Anweisung, so beginnt die Fallverzweigung erneut, was einen deutlichen Unterschied zu anderen Sprachen darstellt.
Dies dient dem Fall, dass die Variable der Fallverzweigung verändert wurde.
Dies gilt natürlich nicht, wenn sich eine "return"-Anweisung im Block befindet.
Der Compiler sollte eine Warnung ausgeben, falls sich weder eine direkte "break"- noch eine direkte "return"-Anweisung in einem der Blöcke befindet, da
es eventuell zu einer ewigen Rekursion führen könnte.
Mit direkt ist gemeint, dass sie sich nicht in einer Wenn-Dann-Ansonsten-Verzweigung befindet.

Notation:
switch (<Ausdruck>)
{
	case (<Ausdruck>)
	{
		break;
	}

	case (<Ausdruck>) // Ein Ausdruck kann auch aus mehreren durch Kommata getrennten Werten bestehen: 1,10,2
	{
		break;
	}

	default
	{
		break;
	}
} 
