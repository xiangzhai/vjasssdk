\chapter{Typen}

In JASS++ können sämtliche Typen aus JASS ohne besondere Einschränkungen (bis auf "const"-Deklarationen) verwendet werden.
Es können jedoch auch eigene Typen können definiert werden. Insgesamt existieren folgende Arten von Typen:
\begin{enumerate}
\item native Typen
\item Funktionen (auch native)
\item Funktionstypen
\item Aufzählungen
\item Klassen
\end{enumerate}
Eine Variable muss stets einen bestimmten Typ besitzen. Dieser kann als Suffix das Schlüsselwort "const" erhalten, was praktisch als
anderer Typ gilt und nur beschränkte Operationen mit der Variable erlaubt.
In JASS++ wird, wie in JASS, zwischen referenz- und kopierbasierten Typen unterschieden. Ist der Typ einer Variable
referenzbasiert und konstant, so darf das Objekt, auf welches referenziert wird nicht verändert werden, die Variable selbst jedoch schon.
So kann ihr ein neuer Inhalt zugewiesen werden, für diesen jedoch keine Methode aufgerufen werden, die nicht als "const" deklariert wurde (siehe "Methoden").
Kopierbasierte Typen sind somit immer implizit als "const" deklariert, da von ihnen sowieso eine Kopie erzeugt wird und nicht eine Referenz.
Daher bewirkt das Suffix "const" bei kopierbasierten Typen nichts, ist aber eine gültige Syntax.
Dies wurde so entschieden, da es z. B. bei der Definition generischer Containerklassen zu Problemen kommen kann, wenn es darum geht, einen
unveränderbaren Wert zurückzugeben.
Der Typ gilt ebenfalls als anderer bei Funktionsdeklarationen (siehe "Funktionen"), jedoch wird die Version der Funktion niemals aufgerufen, falls eine andere
mit selbigem Bezeichner aber einem nichtkonstanten Typ als Parameter existiert. Der Compiler sollte zumindest einen Hinweis ausgeben, wenn
kopierbasierte, konstante Typen verwendet werden.
Der Inhalt kopierbasierter Variablen wird wie der referenzbasierter beim Übergeben an eine andere (z. B. an Funktionsparameter) kopiert,
jedoch wird bei referenzbasierten lediglich eine Referenz auf ein global existierendes Objekt kopiert, welches, falls möglich,
explizit vom Entwickler wieder freigegeben werden muss.
Da auch Funktionszeiger referenzbasierter Typen sind, können sie ebenfalls als "const" deklariert werden (siehe "Funktionstypen").
Wird eine Variable eines referenzbasierten Typs freigegeben bzw. gelöscht, so wird bei einer erfolgreichen Freigabe der Variableninhalt auf 0
gesetzt.
Dies gilt auch für native referenzbasierte Typen.
Ein Typ muss nicht zwingend einen Bezeichner besitzen, insofern mindestens eine Instanz davon am Ende der Typendeklaration erzeugt wird.
Der Typ kann dann von nirgendwo sonst aus im Code verwendet werden. Man spricht dabei auch von anonymen Typen. Instanzen können bei
Typendeklarationen in der Regel am Ende erzeugt werden (siehe "Variablen").

\section{Typendeklaration-Notation}
[<benötigte Schlüsselwörter>] [<Typbezeichner>] [<benötigte Angaben>] [<Variablendeklaration 1>, ...];

\section{Typenverwendung-Notation}
<Typbezeichner> [const]

\section{Typkonvertierungen}
Sämtliche Typkonvertierungen müssen explizit vorgenommen werden, insofern es sich nicht
um eine Kind-zu-Eltern-Konvertierung handelt. Sämtliche Funktions- und
Klasseninstanzenreferenzen können zum Typ integer konvertiert werden (ebenfalls explizit).
Kindinstanzen können zu Elterninstanzen implizit konvertiert werden. Dies gilt auch für
native Kindtypen wie z. B. unit zu widget.
Andersherum muss explizit umgewandelt werden und gilt nur noch für Klasseninstanzen, nicht für
native Typen mit Ausnahme derer für die native Konvertierungsfunktionen zur Verfügung stehen.
Zu beachten gilt, dass es bei Eltern-zu-Kind-Konvertierungen zu fehlenden Daten für z. B.
virtuelle Methoden kommen kann. Schlimmer noch wäre ein Zugriff auf ein Instanzelement
einer ursprünglichen Nichtklasseninstanz, da so auf ein falsches Element einer anderen
Instanz zugegriffen werden würde. Dies kann selbst zur Laufzeit nicht im Debug-Modus heraus-
gefunden werden.
Daher sollte der Compiler bei solchen Konvertierungen eine Warnung ausgeben.
Außerdem können Variablen mit nicht-konstantem Inhalt implizit zu solchen konvertiert werden.
Andersherum ist eine Konvertierung gar nicht möglich.
Schlägt eine Konvertierung fehl (z. B. eine Ganzzahl in einen Schadenstypen), so wird ein Nullwert
zurückgeliefert. 0 bei referenzbasierten und "null" bei kopierbasierten Typen.
Eine Liste der möglichen Konvertierungen nativer Typen, welche durch die bestehenden nativen Funktionen
von JASS möglich sind und nach der Konvertierungssyntax von JASS++ durchgeführt werden können finden
Sie im Abschnitt "Operatoren" und im Unterabschnitt des jeweiligen Typs.

\subsection{Notation}
<Konvertierungstyp>(Ausdruck)

\subsection{Beispiele}
MyClass myClassValue = new MyClass;
integer myValue = integer(myClassValue);
MyParentClass myParentClassValue = myClassValue;
myClassValue = myParentClassValue; // illegal!
myClassValue = MyClass(myParentClassValue);
unit myUnit = null;
widget myWidget = myUnit;
myUnit = myWidget; // illegal!
myUnit = widget(myWidget); // illegal!
unit const myUnit = myOtherNonConstantUnit;
unit myUnit= myOtherConstantUnit; // illegal!

\section{Behältertypen}
Ein Behältertyp ist ein Typ, welcher mindestens die Verwendung des "size"- und des "[]"-Operators erlaubt.
Die Verwendung des "+=" bzw. "+" und des "-=" bzw. "-"-Operators ist optional möglich.
Dabei muss der "size"-Operator eine Ganzzahl zurückgeben und der "[]"-Operator eine einzige als
Parameter entgegennehmen und den Typ der enthaltenen Elemente des Behälters zurückgeben.
Ob die beiden Operatoren als "const" implementiert sind ist frei wählbar, jedoch können, falls
einer der beiden nicht als "const" implementiert ist, konstante Instanzen des Typs nicht in
"foreach"-Schleifen verwendet werden.
Anderenfalls geht dies ohne Probleme durch die vorgegebene Implementation (siehe "foreach-Schleife").
Mit dem "+" bzw. dem "+="-Operator kann man neue Instanzen eines Behälters erzeugen bzw. vorhandene um neue Einträge erweitern.
Mit dem "-" bzw. dem "-="-Operator kann man neue Instanzen eines Behälters erzeugen bzw. vorhandene um bestehende Einträge verringern.
Es gibt drei native Behältertypen, die jeweils Werte eines bestimmten nativen Typs enthalten:
\begin{enumerate}
\item string - string ([] liefert ein einzelnes Zeichen)
\item force - player
\item group - unit
\end{enumerate}

Alle drei können somit in "foreach-Schleifen" verwendet werden.
Außerdem zählen auch Aufzählungen zu den Behältertypen (siehe "Aufzählungen"). Jedoch sind diese nicht erweiterbar.

\subsection{Operatoren}
Die Additions- und Subtraktionsoperatoren entfernen immer genau die Anzahl der Elemente des jeweiligen Ausdrucks.
Daher wird ein Element, das mehrfach in einem Behälter vorhanden ist genau einmal aus diesem entfernt, wenn der "-"-Operator
verwendet wird.
Außerdem kann der zweite Ausdruck bei ihnen auch ein einzelnes Element sein!
\begin{enumerate}
\item + - Addiert zwei Behälter und erzeugt daraus einen neuen Behälter mit dem Inhalt beider Behälter.
\item - - Subtrahiert zwei Behälter und erzeugt daraus einen neuen Behälter mit dem Inhalt des ersten Behälters ohne die Elemente des zweiten Behälters.
\item += - Addiert einen anderen Behälter zu einem bestehenden.
\item -= - Subtrahiert einen anderen Behälter von einem bestehenden.
\item size - Liefert die Anzahl der Elemente eines Behälters.
\item [] - Liefert ein Element oder einen Bereich eines Behälters.
\item []= - Setzt ein Element oder einen Bereich eines Behälters.
\item >= Vergleicht die Größe zweier Behälter.
\item <= Vergleicht die Größe zweier Behälter.
\item > Vergleicht die Größe zweier Behälter.
\item < Vergleicht die Größe zweier Behälter.
\end{enumerate}

\subsection{Intervalle}
Mit dem "[]"-Operator können nicht nur einzelne Elemente gesetzt bzw. ausgelesen werden, sondern auch Bereiche bzw. Intervalle.
Dabei müssen statt eines Indizes ein Start- und ein Endwert angegeben werden. Diese Art der Anweisung liefert einen neu erzeugten
Behälter mit den entsprechenden Elementen bzw. nimmt einen solchen als Ausdruck entgegen.
Es sollte eine Laufzeitwarnung ausgegeben werden, falls dabei Werte außerhalb des Behälterbereichs übergeben werden.
Ein solcher Fehlzugriff liefert stets einen Nullwert.
Eine Fehlsetzung setzt stets nichts.

\subsection{Zugriffsnotation}
<Ausdruck>[<Index>]
<Ausdruck>[<Index>] = <Ausdruck>;
<Behälterausdruck>[<Startposition>;<Endposition>]
<Behälterausdruck>[<Startposition>;<Endposition>] = <Behälterausdruck>

\section{Vordefinierte Typen}
Bei folgenden Typen wird die verwendete "common.j"-Datei analysiert und anhand der dort definierten JASS-Konstanten
ermittelt, welche Werte zur Verfügung stehen. Sollte eine Variable eines der Typen einen anderen Wert erhalten
haben als in der Datei zur Verfügung steht, so gibt der Compiler eine Fehlermeldung aus.
Dabei werden auch Konvertierungen von Ganzzahlen zu einem der Typen überprüft. Sollte die Konvertierung erst zur
Laufzeit überprüfbar sein, gibt der Compiler zumindest eine Warnung aus.
Hier ist eine Liste der nativen Typen:
playercolor, race, playergameresult, alliancetype, version, attacktype, damagetype, weapontype, pathingtype, racepreference,
mapcontrol, gametype, mapflag, placement, startlocprio, mapdensity, gamedifficulty, gamespeed, playerslotstate, volumegroup,
igamestate, fgamestate, playerstate, unitstate, aidifficulty, playerscore, gameevent, playerevent, playerunitevent, unitevent
widgetevent, dialogevent, gameevent, playerevent, playerunitevent, limitop, unittype, itemtype, camerafield, blendmode, raritycontrol,
texmapflags, fogstate, effecttype, soundtype 