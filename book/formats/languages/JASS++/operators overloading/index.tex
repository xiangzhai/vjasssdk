\chapter{Operatorenüberladung}
Eine Funktions- oder Methodendeklaration, welche das Schlüsselwort operator und ein zugehöriges Operatorzeichen oder einen zugehörigen Typnamen enthält, überlädt die Bedeutung eines
Operators für die angegebenen Parametertypen.
Dies kann global oder innerhalb eines Gültigkeitsbereiches geschehen (mit den jeweiligen Zugriffsmodifikatoren).
Wird ein Typ und kein Operatorzeichen als Bezeichner gewählt, so ermöglicht dies Klassenobjekten eine explizite Konvertierung zum Typ der den Bezeichner trägt.
Es gilt zu beachten, dass innerhalb der Operatordefinition nicht der überladene Operator gilt (also keine Rekursion möglich). Dies soll ermöglichen,
dass man den originalen zumindest nocht innerhalb der eigenen Definition aufrufen kann.
Folgende Operatoren können nach folgenden Schemata als nicht statische Methoden überladen werden:
Bezeichner			Rückgabetyp			Parametertypen		Parameteranzahl
=				self				self			1
+ (linksbündig)			self				-			0
- (linksbündig)			self				-			0
+ (rechtsbündig)		self				self			1
- (rechtsbündig)
*				self				self			1
/				self				self			1
%				self				self			1
++				self				self			1
--				self				self			1
+=				self				self			1
-=				self				self			1
*=				self				self			1
/=				self				self			1
%=				self				self			1
==				boolean				self			1
!=				boolean				self			1
>=				boolean				self			1
<=				boolean				self			1
>				boolean				self			1
<				boolean				self			1
!				boolean				self			1
true (steht für den boolschen Ausdruck ohne Operatorzeichen)
[]				beliebig			beliebig		1 bzw. 2 für Intervalle
[]=				-				beliebig		2 bzw. 3 für Intervalle, der letzte Parameter wird stets als der Zuweisungswert verwendet
size				integer				-			0
()				beliebig			beliebig		beliebig

save				-				beliebig		beliebig
load				-				beliebig		beliebig
exists				boolean				-			-
flush				-				belibig			beliebig

Zu beachten gilt es hierbei, dass diverse Logikoperatoren bereits standardmäßig implementiert sind.
Die eigenen Definitionen überladen selbstverständlich die standardmäßigen.

\section{Anmerkung zum []-Operator}
Der []-Operator dient normalerweise dem Zugriff auf Indizes und ist standardmäßig für keine Klasseninstanz definiert.
Es ist möglich dem []-Operator zwei Parameter zu geben, um Intervallzugriffe zu ermöglichen (siehe "Typen - Behältertypen").

\section{Anmerkung zum size-Operator}
Der size-Operator macht für gewöhnlich eine vom Entwickler definierte Aussage über eine bestimmte Größe der Instanz bzw. eines ihrer enthaltenen
Elemente.
Auch hierbei gelten bestimmte Regeln, damit die Klasse ein Behältertyp ist (siehe "Typen - Behälterypen").

Als Typ können sowohl native als auch eigene Typen genommen werden. Zu beachten gilt es hierbei, dass sich Klasseninstanzen standardmäßig
zum Typ "integer" konvertieren lassen.
Eine eigene Definition des Operators "integer" hat eine höhere Priorität als die Standardimplementierung.

\section{Konvertierungsoperatorennotation} 
<Zugriffsmodifikatoren> operator <Typbezeichner>() [const]
{
	<Anweisungen>
}
