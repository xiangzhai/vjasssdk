\chapter { Deklarationen }

In JASS++ gibt es zwei Arten von Deklarationen: dynamische und statische. Statische existieren vom Programmstart bis zum
Programmende (bzw. Spielstart bis zum Spielende). Dynamische haben eine bestimmte, oftmals zur Lauftzeit definierte Verfügbar-
keitsdauer.
In Funktionen geht diese vom Funktionsaufruf bis zum Funktionsablaufsende. In Klassen von der Erzeugung des Objekts bis zu
dessen Freigabe.
Statische Deklarationen erhalten den Qualifzierer "static", dynamische den Qualifzierer "dynamic". Es gelten einige Implikations-
regeln, um unnötige Tipparbeit zu vermeiden. Dennoch ist auch eine explizite Voranstellung des jeweiligen Qualifizerers erlaubt.
Es gibt nur eine begrenzte Reihe von Deklarationsarten, die beide Arten der Qualifikation zulassen.

Folgende Sprachkonstrukte werden implizit als dynamisch deklariert:
* Variablen
* Methoden

Folgende Sprachkonstrukte werden implizit als statisch deklariert:
* eigene Gültigkeitsbereiche
* Typen
* Funktionen
* Pseudonyme
* Aufzählungen
* Klassen

Folgende Sprachkonstrukte erlauben auch die explizite Qualifikation durch den nicht impliziten Qualifizierer.
* Variablenart
* Methoden

\section { Notation }
[static | dynamic] <Deklaration> 
