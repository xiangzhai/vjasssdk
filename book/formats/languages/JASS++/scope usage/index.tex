\chapter{Gültigkeitsbereichverwendung}
Gültigkeitsbereiche können innerhalb eines anderen Gültigkeitsbereiches oder im globalen "verwendet" werden.
Dies bedeutet, dass Elementen des Gültigkeitsbereiches kein Präfix des Gültigkeitsbereiches mehr voran-
gestellt werden muss.
Dabei ist auch die Verwendung von verschachtelten Gültigkeitsbereichen möglich.
Die Verwendung wirkt sich sowohl auf den aktuellen als auch auf alle Untergültigkeitsbereiche dessen aus.
Sie kann durch den --Operator in einem der Subgültigkeitsbereiche aufgehoben werden, wobei sämtliche Ebenen unterhalb der angegebenen
aufgehoben werden.
Der *-Operator dagegen bewirkt, dass sämtliche Untergültigkeitsbereiche implizit ebenfalls verwendet werden.
Eine Kombination der beiden Operatoren ist theoretisch möglich, da der --Operator implizit den *-Operator verwendet, aber nicht notwendig.
Der Compiler sollte in diesem Fall eine Warnung ausgeben.
Zudem ist es unter keinen Umständen möglich, bei der Gültigkeitsbereichverwendung den aktuellen Gültigkeitsbereich oder einen Subgültigkeitsbereich
dessen mit dem *-Operator oder dem --Operator anzugeben.
In diesem Fall muss der Compiler eine Fehlermeldung ausgeben.
Auch Funktionen können als Gültigkeitsbereich verwendet werden (siehe "Gültigkeitsbereiche").
VERALTET: Eine besondere Ausnahme bei der Gültigkeitsbereichverwendung bilden Funktionen.
VERALTET: Ihr innerer Gültigkeitsbereich kann nicht von außerhalb verwendet werden.
Elemente die in einem Gültigkeitsbereich enthalten sind, müssen niemals zwingend dessen Bezeichner angeben.

\section{Notation}
<Gültigkeitsbereichbezeichner>[* | -]

\section{Beispiele}
scope gna.gnu.gni*
class gna.gnu.gni.Test
enum gna.gnu.gni.Test.Type

Test var = new Test()
scope gna.gnu.gni-
Test var2 = new Test() // Fehler, undefinierter Bezeichner!