\chapter{Variablensperrung}

Da das Schlüsselwort "threaded" die Deklaration einer Funktion ermöglicht, welche fast gleichzeit wie andere ausgeführt
werden kann und es in Warcraft 3 The Frozen Throne selbst den Datentyp "trigger" gibt, dessen Aktionen ebenfalls
fast gleichzeitig ausgeführt werden können (eigentlich handelt es sich aufgrund der einzig möglichen Compilerimplementation
bei beiden Möglichkeiten um den Datentyp "trigger"), unterstützt die Sprache auch sogenannte Variablensperrungen,
die verhindern, dass eine Variable über einen bestimmten Zeitraum aus mehreren Funktionen gesetzt wird.
Ausgelesen werden können Variablen zu jedem Zeitpunkt, jedoch sollte der Testmodus nach Einstellung eine Warnung
ausgeben können, wenn eine gesperrte Variable außerhalb ihres Sperrblocks ausgelesen wird.
Die Sperrung einer Variable ist bis zum Ende des sogenannten Sperrblocks gültig, welcher in einer als "threaded"
deklarierten Funktion auch "sleep"-Anweisungen enthalten kann.
Zudem kann bei der Verwendung eines "threaded"-Blocks eine Sekundendauer in Form eines Fließkommazahlenliterals
angegeben werden. Wird dies getan, so wartet die Funktion solange bis die Variable entsperrt wurde und überprüft
diese in dem angegebenen Intervall. Sollte versucht werden eine Variable zu sperren, die bereits gesperrt ist und
kein Intervall angegeben werden, so würde der Sperrblock übersprungen werden.

\section{Notation}
threaded (<Variablenname>[; <Fließkommaintervall>])
{
	<Anweisungen>
}