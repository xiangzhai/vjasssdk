\chapter{Qualifizierer-Gültigkeitsbereiche}
Unter Qualifizierer-Gültigkeitsbereichen versteht man unbezeichnete Gültigkeitsbereiche,
welche nicht als eine weitere Überdeckungsschicht gelten, jedoch sämtliche enthaltenen Deklarationen der ersten Ebene
als etwas Bestimmtes qualifzieren.
Dabei muss die Qualifizierung eines eingeschlossenen definierten Objekts von der Syntax gestattet sein.
Die Qualifizierung wird den Deklarationen direkt vorangestellt.
Mit folgenden Schlüsselwörtern können solche Qualifizierer-Gültigkeitsbereiche definiert werden:
Siehe dazu "Zusammenfassung der Qualifizierer" im Abschnitt "Schlüsselwörter".
Die Schlüsselwörter können auch in Kombination hintereinander verwendet werden, falls dies die Syntax der Sprache
erlaubt.
Außerdem können sämtliche (auch selbst definierte) Typen als Qualifizierer verwendet werden.
Des Weiteren gilt es zu beachten, dass manche Qualifizierer eventuell auch eine andere Art der Zusammenfassung erlauben
(z. B. Trennung durch Kommata am Ende des Gültigkeitsbereiches).
In einem solchen Fall steht es dem Entwickler frei, sich zwischen den beiden Varianten zu entscheiden.

\section{Notation}
<Qualifizierer 1> <Qualifizierer 2> ... <Qualifizierer n>
{
	<Code-Element-Definition>
}

\section{Beispiele}
static const integer
{
	ich = 10;
	du = 10;
	er = 12;
}

debug
{
	void bla() {};
	void bla2() {};
	void bla3() {};
}

function
{
	void test() { }
	void test2() { }
}