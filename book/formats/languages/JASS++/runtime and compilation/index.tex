\chapter{Laufzeit und Kompilierung}
Im Gegensatz zu JASS ist JASS++ keine reine Interpreter-Sprache mehr, da sie zunächst zu JASS-Code
umgewandelt werden muss. Daher wird in den nachfolgenden Inhalten oft zwischen der Lauf- und der Kompilier-
zeit unterschied. Mit der Kompilierzeit ist jene Zeit gemeint, in welcher der JASS++-Code in JASS-Code
umgewandelt wird. Mit der Laufzeit dagegen ist jene Zeit gemeint, in der der bereits zu JASS-Code kompilierte
JASS++-Code ausgeführt bzw. interpretiert wird. Dies wird in der Regel von Warcraft 3 The Frozen Throne bzw. dessen Engine
bewerkstelligt, könnte jedoch theoretisch auch von einer anderen, selbst gebauten getan werden.