\chapter{Pseudonyme}

Das Schlüsselwort "alias" ermöglicht es einem einen anderen Bezeichner für einen bestehenden zu definieren. Dieser gilt innerhalb
des Gültigkeitsbereiches, in welchem sich die Pseudonymdefinition befindet, er kann aber natürlich auch wie der originale von außerhalb
verwendet werden, insofern er öffentlich ist.
Der neue Bezeichner darf bereits existieren und überdeckt innerhalb des Gültigkeitsbereiches den alten. Der Compiler sollte in
diesem Fall jedoch eine Warnung ausgeben.
Der Bezeichner dagegen muss existieren.

\section{Notation}
alias <Neuer Bezeichner> <Bezeichner>;