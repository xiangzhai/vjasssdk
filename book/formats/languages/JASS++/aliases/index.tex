\chapter{Pseudonyme}

Das Schlüsselwort "alias" ermöglicht es einem einen anderen Bezeichner für einen bestehenden zu definieren. Dieser gilt innerhalb
des Gültigkeitsbereiches, in welchem sich die Pseudonymdefinition befindet, er kann aber natürlich auch wie der originale von außerhalb
verwendet werden, insofern er öffentlich ist.
Der neue Bezeichner darf bereits existieren und überdeckt innerhalb des Gültigkeitsbereiches den alten. Der Compiler sollte in
diesem Fall jedoch eine Warnung ausgeben.
Der Bezeichner dagegen muss existieren.
Es gilt zu beachten, dass der Pseudonym-Bezeichner ebenfalls zweimal im selben Gültigkeitsbereich existieren kann,
insofern er ein Pseudonym für eine andere, unterscheidbare Typendefinition ist.
In diesem Fall gilt der Bezeichner also für beide Deklarationen und der Compiler entscheidet nach Anwendungsfall.
Dies kann bei Funktions- und Vorlageninstanzdeklarationen der Fall sein, da diese sich nicht nur durch ihren Bezeichner
unterscheiden müssen (siehe "Funktionen" und "Vorlagen").
Ein Pseudonym für eine Funktionsdeklaration kann daher hergestellt werden, auch wenn der Bezeichner des Pseudonyms bereits
im selben Gültigkeitsbereich existiert, insofern sich die Funktionsdeklaration unterscheidet.
Damit vereinen Pseudonyme in JASS++ die Möglichkeiten von "typedef"-Anwendungen in C/C++ und richtigen Pseudonymen, da auch
Gültigkeitsbereichsbezeichner überdeckt werden können.
Pseudonym-Deklarationen müssen nicht angefordert werden. Sie existieren gültigkeitsbereichsweit.

\section{Notation}
[Zugriffsmodifikatoren] alias <Neuer Bezeichner> (<Bezeichner>[, <Bezeichner n>]);