\chapter{KIs}
Da das Spiel Warcraft 3 The Frozen Throne ein recht umfangreiches KI-Modul besitzt, gibt es in JASS++
selbstverständlich eine kleinere Unterstützung für KI-Skripts.
KI-Blöcke werden mit dem Schlüsselwort "ai" eingeleitet und haben zwingend einen Bezeichner.
Innerhalb eines KI-Blocks dürfen explizit Variablen und Funktionen aus der verwendeten "common.ai"-Datei
verwendet werden bzw. natürlich auch eigene, die innerhalb des KI-Blocks deklariert wurden.
Jedoch keinerlei selbstdefinierte Funktionen von außerhalb, da der Block wie ein externes Skript behandelt wird
und man ansonsten sämtliches Verwendetes mit in das Skript hinein kopieren müsste bzw. dabei alle Abhängigkeiten
genaustens prüfen müsste.
Für Funktionen aus der "common.j" und "Blizzard.j" gelten gewisse Einschränkungen (siehe unten).
Der Inhalt des KI-Blocks kann über seinen Bezeichner mit den Schlüsselwörtern "execute" und "for" für einen Spieler gestartet werden.
Der Aufruf entspricht einem Aufruf der nativen JASS-Funktion "StartCampaignAI" mit dem Spieler und dem Bezeichner "Scripts/Custom/<Bezeichner>.ai" als Parameter.
Daher muss der Compiler dafür sorgen, dass aus dem KI-Block sozusagen eine externe Datei nach dem obigen Schema im Verzeichnis "Scripts/Custom" generiert wird.
Der KI-Block ist also auch manuell mit der Nativen aufrufbar und sollte sich im MPQ-Archiv der Karte befinden. Der Compiler sollte einen Fehler ausgeben,
falls die Datei bereits existiert und nicht vom Compiler generiert wurde.
Die Compilergenerierung wird am Anfang der Datei mit dem Kommentar "// JASS++ AI" markiert.
Der Inhalt eines KI-Blocks darf nur innerhalb des Blocks selbst aufgerufen werden.
Bestimmte Funktionen aus der "common.j" und "Blizzard.j" dürfen in KI-Blöcken weder direkt noch indirekt verwendet werden:
\begin{enumerate}
\item Native Funktionen, die eine Zeichenkette zurückgeben
\item Native Funktionen, die sogenannte Callbacks annehmen (code, trigger, boolexpr, ForGroup, etc.)
\item ExecuteFunc
\end{enumerate}

Siehe auch http://jass.sourceforge.net/doc/library.shtml.
Es gilt noch zu beachten, dass globale Variablen pro KI-Block separat existieren und die KI-Blöcke so nicht in der Lage sind,
globale Variablen gegenseitig oder für das Kartenskript zu überschreiben.
KI-Blöcke könnnen nicht als Deklarationsvoraussetzung verwendet werden, da sie in jedem Fall ausgeführt werden können und
ihre Deklaration nicht im Skript stattfindet.
Sie selbst können ebenfalls keine Deklarationsvoraussetzung haben, da alle verfügbaren Funktionen bereits vor ihnen
deklariert werden.

\section{Deklarationsnotation}
ai <Bezeichner>
	<Anweisungen>

\section{Aufrufsnotation}
execute <Bezeichner> for <Spieler>
