\chapter{Operatoren}

\section{Für Gültigkeitsbereiche}
[<Bezeichner des Gültigkeitsbereiches>].<Bezeichner eines im Gültigkeitsbereich deklarierten Elements>

\section{Für die meisten nativen und alle selbstdefinierten Typen}
save <Ausdruck> <Hash-Tabelle | Spiel-Cache> <Ganzzahlausdruck | Zeichenkettenausdruck>
load <Ausdruck> <Hash-Tabelle | Spiel-Cache> <Ganzzahlausdruck | Zeichenkettenausdruck>

\section{Für die Typen integer und real}
Ausdruck = Ausdruck
+Ausdruck
-Ausdruck
Ausdruck + Ausdruck
Ausdruck - Ausdruck
Ausdruck * Ausdruck
Ausdruck / Ausdruck
Ausdruck % Ausdruck
Ausdruck++
Ausdruck--
Ausdruck += Ausdruck
Ausdruck -= Ausdruck
Ausdruck *= Ausdruck
Ausdruck /= Ausdruck
Ausdruck %= Ausdruck
Ausdruck == Ausdruck
Ausdruck != Ausdruck
Ausdruck >= Ausdruck
Ausdruck <= Ausdruck
Ausdruck > Ausdruck
Ausdruck < Ausdruck

\section{Für den Typ integer}
real(Ausdruck)						Entspricht I2R(Ausdruck)
string(Ausdruck)					Entspricht I2S(Ausdruck)
race(Ausdruck)						Entspricht ConvertRace(Ausdruck)
alliancetype(Ausdruck)					Entspricht ConvertAllianceType(Ausdruck)
racepreference(Ausdruck)				Entspricht ConvertRacePref(Ausdruck)
igamestate(Ausdruck)					Entspricht ConvertIGameState(Ausdruck)
fgamestate(Ausdruck)					Entspricht ConvertFGameState(Ausdruck)
playerstate(Ausdruck)					Entspricht ConvertPlayerState(Ausdruck)
playerscore(Ausdruck)					Entspricht ConvertPlayerScore(Ausdruck)
playergameresult(Ausdruck)				Entspricht ConvertPlayerGameResult(Ausdruck)
unitstate(Ausdruck)					Entspricht ConvertUnitState(Ausdruck)
aidifficulty(Ausdruck)					Entspricht ConvertAIDifficulty(Ausdruck)
gameevent(Ausdruck)					Entspricht ConvertGameEvent(Ausdruck)
playerevent(Ausdruck)					Entspricht ConvertPlayerEvent(Ausdruck)
playerunitevent(Ausdruck)				Entspricht ConvertPlayerUnitEvent(Ausdruck)
widgetevent(Ausdruck)					Entspricht ConvertWidgetEvent(Ausdruck)
dialogevent(Ausdruck)					Entspricht ConvertDialogEvent(Ausdruck)
unitevent(Ausdruck)					Entspricht ConvertUnitEvent(Ausdruck)
limitop(Ausdruck)					Entspricht ConvertLimitOp(Ausdruck)
unittype(Ausdruck)					Entspricht ConvertUnitType(Ausdruck)
gamespeed(Ausdruck)					Entspricht ConvertGameSpeed(Ausdruck)
placement(Ausdruck)					Entspricht ConvertPlacement(Ausdruck)
startlocprio(Ausdruck)					Entspricht ConvertStartLocPrio(Ausdruck)
gamedifficulty(Ausdruck)				Entspricht ConvertGameDifficulty(Ausdruck)
gametype(Ausdruck)					Entspricht ConvertGameType(Ausdruck)
mapflag(Ausdruck)					Entspricht ConvertMapFlag(Ausdruck)
mapvisibility(Ausdruck)					Entspricht ConvertMapVisibility(Ausdruck)
mapsetting(Ausdruck)					Entspricht ConvertMapSetting(Ausdruck)
mapdensity(Ausdruck)					Entspricht ConvertMapDensity(Ausdruck)
mapcontrol(Ausdruck)					Entspricht ConvertMapControl(Ausdruck)
playercolor(Ausdruck)					Entspricht ConvertPlayerColor(Ausdruck)
playerslotstate(Ausdruck)				Entspricht ConvertPlayerSlotState(Ausdruck)
volumegroup(Ausdruck)					Entspricht ConvertVolumeGroup(Ausdruck)
camerafield(Ausdruck)					Entspricht ConvertCameraField(Ausdruck)
blendmode(Ausdruck)					Entspricht ConvertBlendMode(Ausdruck)
raritycontrol(Ausdruck)					Entspricht ConvertRarityControl(Ausdruck)
texmapflags(Ausdruck)					Entspricht ConvertTexMapFlags(Ausdruck)
fogstate(Ausdruck)					Entspricht ConvertFogState(Ausdruck)
effecttype(Ausdruck)					Entspricht ConvertEffectType(Ausdruck)
version(Ausdruck)					Entspricht ConvertVersion(Ausdruck)
itemtype(Ausdruck)					Entspricht ConvertItemType(Ausdruck)
attacktype(Ausdruck)					Entspricht ConvertAttackType(Ausdruck)
damagetype(Ausdruck)					Entspricht ConvertDamageType(Ausdruck)
weapontype(Ausdruck)					Entspricht ConvertWeaponType(Ausdruck)
soundtype(Ausdruck)					Entspricht ConvertSoundType(Ausdruck)
pathingtype(Ausdruck)					Entspricht ConvertPathingType(Ausdruck)

\section{Für den Typ real}
integer(Ausdruck)					Entspricht R2I(Ausdruck)
string(Ausdruck)					Entspricht R2S(Ausdruck)
string(Ausdruck, <Ganzzahlausdruck>, <Ganzzahlausdruck>)Entspricht R2SW(Ausdruck, <Ganzzahlausdruck>, <Ganzzahlausdruck>)

\section{Für den Typ boolean}
Ausdruck = Ausdruck
Ausdruck
!Ausdruck
Ausdruck == Ausdruck
Ausdruck != Ausdruck

\section{Für den Typ string}
Ausdruck = Ausdruck
Ausdruck + Ausdruck
Ausdruck - Ausdruck				Entfernt eine Teilzeichenkette
Ausdruck += Ausdruck
Ausdruck -= Ausdruck				Entfernt eine Teilzeichenkette
Ausdruck == Ausdruck
Ausdruck != Ausdruck
Ausdruck >= Ausdruck				Die Vergleichsoperatoren beziehen sich in diesem Fall auf die Wertigkeit des Unicode-Zeichens (Warcraft 3 TFT verwendet kein ASCII)
Ausdruck <= Ausdruck
Ausdruck > Ausdruck
Ausdruck < Ausdruck
Ausdruck * Ausdruck				Der Ausdruck muss eine Ganzzahl sein. Liefert eine neue Zeichenkette, in welcher die alte Ausdruck mal kopiert und aneinander gehängt wurde.
Ausdruck *= Ausdruck				Die Ausdruck erhält den vervielfachten Inhalt.
hash Ausdruck					Entspricht Ausdruck.hash und StringHash(Ausdruck)
Ausdruck[Index]					Entspricht SubString(Ausdruck, index, index + 1). Liefert immer "null", wenn sich der Index außerhalb des String-Bereichs befindet.
size Ausdruck					Entspricht Ausdruck.size und StringLength(Ausdruck)
integer(Ausdruck)				Entspricht S2I(Ausdruck)
real(Ausdruck)					Entspricht S2R(Ausdruck)

\section{Für den Typ code}
Ausdruck = Ausdruck				Der Ausdruck muss eine zuvor deklarierte Funktion sein, die keine Parameter entgegen nimmt und keinen Rückgabetyp besitzt. In JASS wird für gewöhnlich nicht überprüft, ob die Funktion zuvor deklariert wurde, was zu Laufzeitfehlern führen kann.

\section{Für den Typ handle}
Ausdruck = Ausdruck
Ausdruck == Ausdruck
Ausdruck != Ausdruck
integer(Ausdruck)				Entspricht GetHandleId(Ausdruck)
id Ausdruck					Entspricht Ausdruck.id und GetHandleId(Ausdruck)
Ausdruck.id

\section{Für den Typ hashtable}
exists Ausdruck <Ganzzahlausdruck>
Ausdruck.exists <Ganzzahlausdruck>
flush Ausdruck <Ganzzahlausdruck>

\section{Für den Typ gamecache}
exists Ausdruck <Zeichenkettenausdruck>
Ausdruck.exists <Zeichenkettenausdruck>
flush Ausdruck <Zeichenkettenausdruck>

\section{Für Behältertypen}
Ausdruck[Index]					Liefert das Element des Indizes Index. Befindet sich der Index außerhalb des Bereichs des Behälters, so liefert der Ausdruck "null". Es sollte eine Laufzeitwarnung ausgegeben werden.
Ausdruck[Index]=				Setzt das Element des Indizes Index. Befindet sich der Index außerhalb des Bereichs des Behälters, so wird nichts gesetzt. Es sollte eine Laufzeitwarnung ausgegeben werden.
Ausdruck[Startposition;Endposition]		Liefert einen Bereich. Befindet sich ein Wert außerhalb des Behälterbereichs, liefert der Ausdruck "null". Es sollte eine Laufzweitwarnung ausgegeben werden.
Ausdruck[Startposition;Endposition]=		Setzt einen Bereich des Behälters. Befindet sich ein Wert außerhalb des Behälterbereichs, so wird nichts gesetzt. Es sollte eine Laufzeitwarnung ausgegeben werden.
size Ausdruck					Liefert die Größe des Behälters als Ganzzahl.
Ausdruck.size					Liefert die Größe des Behälters als Ganzzahl.
Ausdruck + Ausdruck
Ausdruck - Ausdruck
Ausdruck += Ausdruck
Ausdruck -= Ausdruck
Ausdruck >= Ausdruck
Ausdruck <= Ausdruck
Ausdruck > Ausdruck
Ausdruck < Ausdruck

\section{Für Arrays}
Ausdruck[Indexdimension 1][Indexdimension 2]...[Indexdimension n]

\section{Für Lade-Blöcke}
<Lade-Block-Bezeichner> : <Deklarationsvoraussetzungen>

\section{Für KI-Blöcke}
execute <KI-Block-Bezeichner> for <Spielerausdruck>

\section{Für Funktionsdeklarationen}
<Rückgabetyp> <Funktionsname><Parameterliste> : <Deklarationsvoraussetzungen>

\section{Für Aufzählungsdeklarationen}
<Aufzählungsbezeichner> (<Erbung) : <Deklarationsvoraussetzungen>

\section{Für Klassendeklarationen}
<Klassenbezeichner> (<Erbung>) : <Deklarationsvoraussetzungen>

\section{Für Aufzählungen}
size <Aufzählungsbezeichner>			Liefert die Anzahl der Elemente der Aufzählung
<Aufzählungsbezeichner>.size			Liefert die Anzahl der Elemente der Aufzählung

\section{Für Aufzählungselemente}
<Geerbter Typ>(Ausdruck)
Ausdruck = Ausdruck
Ausdruck == Ausdruck
Ausdruck != Ausdruck

\section{Für "threaded" Funktionen}
execute Funktion(<Parameter>)
executewait [<real-Ausdruck>] Funktion(<Parameter>)
evaluate Funktion(<Parameter>)
executions Funktion
Funktion.executions
evaluations Funktion
Funktion.evaluations
sleeps Funktion
Funktion.sleeps
sleepon Funktion
sleepoff Funktion
reset Funktion

\section{Für Funktionstypen}
integer(Ausdruck)				Liefert die interne Ganzzahl des Funktionsprototypen.
id Ausdruck					Liefert die interne Ganzzahl des Funktionsprototypen.
Ausdruck.id

\section{Innerhalb von "threaded" Funktionen}
sleep <real-Ausdruck>
id						Liefert die interne Ganzzahl des Funktionsprototypen.

\section{Innerhalb von Funktionen}
wait <real-Ausdruck>

\section{Für Klassen(instanzen)}
<Klassenname>.<Aufzählung | statische Methode | statisches Element>
<Ausdruck>.<Elementname | Methodenname>
new <Klassenname>				Entspricht Klassenname.new()
delete <Ausdruck>				Entspricht Ausdruck.delete()
copy Ausdruck					Entspricht Ausdruck.copy()
Ausdruck = Ausdruck
Ausdruck					Ist wahr, falls die Ausdruck auf ein Objekt zeigt.
!Ausdruck					Ist wahr, falls die Ausdruck auf kein Objekt zeigt.
Ausdruck == Ausdruck				Ist wahr, falls die beide Ausdruckn auf dasselbe Objekt zeigen.
Ausdruck != Ausdruck				Ist wahr, falls die beide Ausdruckn nicht auf dasselbe Objekt zeigen.
integer(Ausdruck)				Liefert die interne Ganzzahl der Klasseninstanz.
id Ausdruck					Entspricht Ausdruck.id. Liefert die interne Ganzzahl der Klasseninstanz.