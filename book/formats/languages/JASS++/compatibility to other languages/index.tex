\chapter { Kompatibilität zu anderen Sprachen }

\section { Kompatibilität zu JASS }
JASS-Funktionen und -Variablen, die innerhalb der #jass und #endjass-Präprozessoranweisungen
deklariert wurden, können von JASS++-Code normal verwendet werden. Dabei gilt zu beachten,
dass die Typumwandlungsregeln von JASS++ bei der Parameterübergabe gelten.
Native Funktionen und Typen können ebenfalls durch eine reine JASS-Deklaration verwendet werden
und müssen nicht erst mit den JASS++-Sprachelementen deklariert werden. Diese eignen sich eher
zur Deklaration neuer Typen und nativer Funktionen, insofern es dem Entwickler möglich ist, dies
z. B. mit Hilfe des Werkzeugs Grimoire zu bewerkstelligen.
Aus JASS heraus kann ebenfalls mittels der #jasspp- und #endjasspp-Präprozessoranweisungen JASS++-Code
verwendet werden, allerdings sollten die zu JASS kompilierten JASS++-Funktionen und -Variablen
aus JASS-Code heraus nicht verwendbar sein.
Es gilt hierbei noch anzumerken, dass die Verwendung von JASS++-Code in globalen JASS-Funktionen
aufgrund der Gültigkeitsbereichssortierung zu größeren Problemen führen kann, da JASS-Funktionen vor sämtlichen
Gültigkeitsbereich-Funktionen deklariert werden.

\section { Kompatibilität zu vJass }
Momentan ist noch nicht geplant JASS++- und vJass-Code kombinierbar zu machen.
Code zwischen den Präprozessoranweisungen #vjass und #endjass wird daher bei einer JASS++-
Kompilierung ignoriert bzw. zu JASS-Code umgewandelt und genau wie JASS-Code vor sämtlichen JASS++-
Elementen deklariert, definiert und aufgerufen.

\section { Kompatibilität zu Zinc }
Momentan ist noch nicht geplant JASS++- und Zinc-Code kombinierbar zu machen.
Code zwischen den Präprozessoranweisungen #zinc und #endzinc wird daher bei einer JASS++-
Kompilierung ignoriert bzw. zu JASS-Code umgewandelt und genau wie JASS-Code vor sämtlichen JASS++-
Elementen deklariert, definiert und aufgerufen.

\section { Kompatibilität zu CJass }
Momentan ist noch nicht geplant JASS++- und CJass-Code kombinierbar zu machen.
Code zwischen den Präprozessoranweisungen #cjass und #endcjass wird daher bei einer JASS++-
Kompilierung ignoriert bzw. zu JASS-Code umgewandelt und genau wie JASS-Code vor sämtlichen JASS++-
Elementen deklariert, definiert und aufgerufen. 
