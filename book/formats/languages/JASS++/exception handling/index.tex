\chapter{Ausnahmebehandlung}
Ausnahmebehandlung ist ein eleganter Weg, alternative Verhaltensweisen des Programms zu behandeln.
Ausnahmen können von einer beliebigen Funktion geworfen werden, die als "throw"-Funktion deklariert wurde (siehe "Funktionen").
Die Ausnahme kann mittels eines try-catch-default-Blocks aufgefangen werden.
Der Compiler sollte eine Warnung ausgeben, falls keine Ausnahme eines angegebenen Typs innerhalb des try-Blocks geworfen werden
könnte.
Außerdem sollte er eine Warnung ausgeben, wenn ein Ausnahmetyp nicht oder gar keiner behandelt wird. Dabei gilt es zu beachten, dass
Ausnahmetypen, die indirekt von der Funktion geworfen werden können, für den Funktionsinhalt selbst überprüft werden, damit keine
riesige Liste von Warnungen bei großen Verschachtelungen entsteht.

\section{Notation}
try
{
	<Anweisungen>
}
catch (<Typ> <Variablenname>)
{
}
default // Alle anderen Ausnahmen
{
}