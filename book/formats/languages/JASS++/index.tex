\documentstyle[12pt,german,a4]{article}

\begin{document}
	\title{JASS++}
	\author{Tamino Dauth \and \verb+tamino@cdauth.eu+\thanks}
	\date{2010-07-07}
	\maketitle
	\begin{abstract}
		Dieses Dokument enthält die Definition der Skriptsprache JASS++.
	\end{abstract}
	\tableofcontents
	\section{\dots}

	Ein minimales \LaTeX{}--Dokument.

	\chapter{Literale}
Ein Literal ist ein sich selbst erklärender Ausdruck, der nicht speziell benannt werden muss also keinen Bezeichner benötigt.
Für folgende native Typen gelten folgende Literalnotationsregeln:

Kopierbasierte Typen:
TODO: Exponentialschreibweise für alle Ganzzahlausdrücke erlauben und auch für Fließkommazahlen?
\begin{enumerate}
\item string : "<Zeichenkettenliteral>" | null
\item integer : 0<Oktalliteral> | 0%<Binärliteral> | 0x<Hexadezimalliteral> | <Dezimalliteral>[e <Exponent>] | '<Id-Literal>' | null
\item real : <Fließkommaliteral> [r | R] | null
\item boolean : true | false | null
\item handle (und Nicht-agent-Kindtypen): null
\end{enumerate}

Referenzbasierte Typen:
\begin{enumerate}
\item code : 0
\item agent (und Kindtypen) : 0
\item Klassenobjekte : 0
\item Funktionszeiger : 0
\end{enumerate}

Dabei können Zeichenkettenausdrücke mehrfach hintereinander geschrieben werden. Diese werden dann automatisch zu einem einzelnen
zusammengefasst. Dies vereinfacht die Aufteilung auf mehrere Zeilen, ohne unnötige Zeilenumbrüche und Einrückungen dazwischen und
dient einer besseren Übersicht:
"<Ausdruckteil 1>"
"<Ausdruckteil n>"

\section{Nullwerte}
Wie oben dargestellt erhalten kopierbasierte Typen einen Nullwert mit dem Literal "null" und referenzbasierte Typen mit dem Literal "0".
Der Zugriff auf Nullwerte kann in manchen Fällen zu einem undefinierten Verhalten führen.
In der Regel ist er im Testmodus jedoch abgesichert und führt zu Fehlermeldungen (siehe "Klassen").

\section{Id-Literale}
Für Id-Literale gelten bei der Validierung selbige Regeln wie in der Sektion "Objekttypen-Literale" definiert. Es wird also
eine Warnung vom Compiler ausgegeben, falls sich die Id statisch überprüfen lässt und in keiner der entsprechenden Dateien
zu finden ist.

\section{Objekttypen-Literale}
Des Weiteren können Eigenschaften sogenannter Objekttypen, die einen bestimmten nativen, kopierbasierten Typ haben abgefragt werden.
Objekttypen erhalten in Warcraft 3 The Frozen Throne eine vierstellige, hexadezimale Id. Daher können Objekttypen im Wertebereich von 16^4 definiert werden.
Jede Id identifiziert genau einen Objekttyp.
Ein Objekttyp kann in JASS++ folgendes sein:
\begin{enumerate}
\item ein Einheitentyp
\item ein Gegenstandstyp
\item ein Doodad-Typ
\item ein Zerstörbares-Typ
\item ein Fähigkeitentyp
\item ein Zauberverstärkertyp
\item ein Forschungstyp
\item ein Wettereffekttyp
\item ein Blitzeffekttyp
\item ein Geländeset
\end{enumerate}

Die Objekttypen stammen aus eine der folgenden Dateien (aufsteigend nach Priorität geordnet):
\begin{enumerate}
\item war3map.u
\item war3map.d
\item UI/UnitData.slk
\item *
\item *
\end{enumerate}
TODO

Dabei werden die angegebenen Dateien in den folgenden Archiven in absteigender Reihenfolge durchsucht:
\begin{enumerate}
\item Karte
\item Kampagne
\item War3Patch.mpq
\item War3Locale.mpq (richtigen Namen finden)
\item War3X.mpq
\item War3.mpq
\end{enumerate}
TODO

Die Eigenschaften eines Objekttyps können einen der folgenden nativen, kopierbasierten Typen haben:
\begin{enumerate}
\item string
\item integer
\item real
\item boolean
\end{enumerate}
TODO

Die Eigenschaften eines Objekttyps haben selbst nochmals eine eigene Id.
Ein Objekttypen-Eigenschafts-Literal wird immer als ein Ausdruck eines der angegebenen Typen interpretiert.
Kann der Objekttypeintrag bzw. der Objekttyp-Eigenschaftseintrag nicht gefunden werden, so gibt der Compiler eine Warnung aus.
Das Literal ist in diesem Fall ein Nullwert ("null").
Wird keine Eigenschafts-Id angegeben, so wird der Ausdruck als Array interpretiert. Dies geht jedoch nur falls alle Objekttypen-Eigenschaften
den gleichen nativen, kopierbasierten Typ haben.
Ansonsten kann das Literal nur als eigener Typ interpertiert werden (siehe "Klassen - Zuweisungen").
Die Notation sieht folgendermaßen aus:
<<Objekttyp-Id>[, %]>
<<Objekttyp-Id>,<Eigenschafts-Id>[, %])>

Das Prozentzeichen ist optional und gibt an, dass Fließkommazahlen als Prozentanteile und daher als Ganzzahlen zwischen 0 und 100 interpretiert werden
sollen. Aus 0.30 würde in diesem Fall also die Ganzzahl 30 werden.
Es gilt bei einer reinen Angabe der Objekttyp-Id für alle Eigenschaften.
Es gilt zu beachten, dass Objekttypen-Ids auch als Escape-Sequenzen verwendet werden können.

\section{Farbliterale}
Farbliterale haben stets den nativen Typ "integer". Sie sind daher ein 32-Bit-Wert, welcher eine RGBA-Farbe enthält.
Der RGBA-Wert muss hexadezimal angegeben werden, wobei die ersten beiden Ziffern für den Alphawert, die zweiten beiden für den Grünwert,
die dritten beiden für den Blauwert und die letzten beiden für den Blauwert stehen.
Notation:
|c<Farbe>|r

\section{Escape-Sequenzen}
Eine Escape-Sequenz ist ein Zeichen, welchem das \-Zeichen vorangestellt wird und welches ausschließlich ein einem
Zeichenkettenliteral vorkommen darf.
Es gibt folgende Escape-Sequenzen:
\begin(enumerate}
\item n				Zeilenumbruch
\item t				Horizontaler Tabulator
\item \				Backslash
\item "				Doppeltes Anführungszeichen
\end{enumerate}

Zusätzlich existieren spezielle Escape-Sequenzen für Objekttypen-Ids (siehe "Literale - Objekttypen-Ids") und Farben (siehe "Literale - Farbliterale").
Farbkodierungen und Objekttypen-Ids in Zeichenketten werden zum Kompilierzeitpunkt validiert bzw. kompiliert.
Die speziellen Escapesequenzen werden in den Zeichenketten zum Kompilierzeitpunkt ersetzt, insofern sie gültig sind.
Falls nicht, bleibt der Ausdruck erhalten und der Compiler gibt eine Warnung aus.
Wie bei normalen Objektypen-Literalen kann ein Prozentzeichen angehängt werden.
Bei Farben kann jedoch, anders als bei gewöhnlichen Farbliteralen, Text eingeschlossen werden. Dieser sollte dann in der entsprechenden Farbe
im Spiel bei einer Ausgabe angezeigt werden.

\section{Notation}
|c<Farbe>[<Text>]|r
<<Objekt-Id>,<Feld-Id>[, %]>

	\chapter{Literale}
Ein Literal ist ein sich selbst erklärender Ausdruck, der nicht speziell benannt werden muss also keinen Bezeichner benötigt.
Für folgende native Typen gelten folgende Literalnotationsregeln:

Kopierbasierte Typen:
TODO: Exponentialschreibweise für alle Ganzzahlausdrücke erlauben und auch für Fließkommazahlen?
\begin{enumerate}
\item string : "<Zeichenkettenliteral>" | null
\item integer : 0<Oktalliteral> | 0%<Binärliteral> | 0x<Hexadezimalliteral> | <Dezimalliteral>[e <Exponent>] | '<Id-Literal>' | null
\item real : <Fließkommaliteral> [r | R] | null
\item boolean : true | false | null
\item handle (und Nicht-agent-Kindtypen): null
\end{enumerate}

Referenzbasierte Typen:
\begin{enumerate}
\item code : 0
\item agent (und Kindtypen) : 0
\item Klassenobjekte : 0
\item Funktionszeiger : 0
\end{enumerate}

Dabei können Zeichenkettenausdrücke mehrfach hintereinander geschrieben werden. Diese werden dann automatisch zu einem einzelnen
zusammengefasst. Dies vereinfacht die Aufteilung auf mehrere Zeilen, ohne unnötige Zeilenumbrüche und Einrückungen dazwischen und
dient einer besseren Übersicht:
"<Ausdruckteil 1>"
"<Ausdruckteil n>"

\section{Nullwerte}
Wie oben dargestellt erhalten kopierbasierte Typen einen Nullwert mit dem Literal "null" und referenzbasierte Typen mit dem Literal "0".
Der Zugriff auf Nullwerte kann in manchen Fällen zu einem undefinierten Verhalten führen.
In der Regel ist er im Testmodus jedoch abgesichert und führt zu Fehlermeldungen (siehe "Klassen").

\section{Id-Literale}
Für Id-Literale gelten bei der Validierung selbige Regeln wie in der Sektion "Objekttypen-Literale" definiert. Es wird also
eine Warnung vom Compiler ausgegeben, falls sich die Id statisch überprüfen lässt und in keiner der entsprechenden Dateien
zu finden ist.

\section{Objekttypen-Literale}
Des Weiteren können Eigenschaften sogenannter Objekttypen, die einen bestimmten nativen, kopierbasierten Typ haben abgefragt werden.
Objekttypen erhalten in Warcraft 3 The Frozen Throne eine vierstellige, hexadezimale Id. Daher können Objekttypen im Wertebereich von 16^4 definiert werden.
Jede Id identifiziert genau einen Objekttyp.
Ein Objekttyp kann in JASS++ folgendes sein:
\begin{enumerate}
\item ein Einheitentyp
\item ein Gegenstandstyp
\item ein Doodad-Typ
\item ein Zerstörbares-Typ
\item ein Fähigkeitentyp
\item ein Zauberverstärkertyp
\item ein Forschungstyp
\item ein Wettereffekttyp
\item ein Blitzeffekttyp
\item ein Geländeset
\end{enumerate}

Die Objekttypen stammen aus eine der folgenden Dateien (aufsteigend nach Priorität geordnet):
\begin{enumerate}
\item war3map.u
\item war3map.d
\item UI/UnitData.slk
\item *
\item *
\end{enumerate}
TODO

Dabei werden die angegebenen Dateien in den folgenden Archiven in absteigender Reihenfolge durchsucht:
\begin{enumerate}
\item Karte
\item Kampagne
\item War3Patch.mpq
\item War3Locale.mpq (richtigen Namen finden)
\item War3X.mpq
\item War3.mpq
\end{enumerate}
TODO

Die Eigenschaften eines Objekttyps können einen der folgenden nativen, kopierbasierten Typen haben:
\begin{enumerate}
\item string
\item integer
\item real
\item boolean
\end{enumerate}
TODO

Die Eigenschaften eines Objekttyps haben selbst nochmals eine eigene Id.
Ein Objekttypen-Eigenschafts-Literal wird immer als ein Ausdruck eines der angegebenen Typen interpretiert.
Kann der Objekttypeintrag bzw. der Objekttyp-Eigenschaftseintrag nicht gefunden werden, so gibt der Compiler eine Warnung aus.
Das Literal ist in diesem Fall ein Nullwert ("null").
Wird keine Eigenschafts-Id angegeben, so wird der Ausdruck als Array interpretiert. Dies geht jedoch nur falls alle Objekttypen-Eigenschaften
den gleichen nativen, kopierbasierten Typ haben.
Ansonsten kann das Literal nur als eigener Typ interpertiert werden (siehe "Klassen - Zuweisungen").
Die Notation sieht folgendermaßen aus:
<<Objekttyp-Id>[, %]>
<<Objekttyp-Id>,<Eigenschafts-Id>[, %])>

Das Prozentzeichen ist optional und gibt an, dass Fließkommazahlen als Prozentanteile und daher als Ganzzahlen zwischen 0 und 100 interpretiert werden
sollen. Aus 0.30 würde in diesem Fall also die Ganzzahl 30 werden.
Es gilt bei einer reinen Angabe der Objekttyp-Id für alle Eigenschaften.
Es gilt zu beachten, dass Objekttypen-Ids auch als Escape-Sequenzen verwendet werden können.

\section{Farbliterale}
Farbliterale haben stets den nativen Typ "integer". Sie sind daher ein 32-Bit-Wert, welcher eine RGBA-Farbe enthält.
Der RGBA-Wert muss hexadezimal angegeben werden, wobei die ersten beiden Ziffern für den Alphawert, die zweiten beiden für den Grünwert,
die dritten beiden für den Blauwert und die letzten beiden für den Blauwert stehen.
Notation:
|c<Farbe>|r

\section{Escape-Sequenzen}
Eine Escape-Sequenz ist ein Zeichen, welchem das \-Zeichen vorangestellt wird und welches ausschließlich ein einem
Zeichenkettenliteral vorkommen darf.
Es gibt folgende Escape-Sequenzen:
\begin(enumerate}
\item n				Zeilenumbruch
\item t				Horizontaler Tabulator
\item \				Backslash
\item "				Doppeltes Anführungszeichen
\end{enumerate}

Zusätzlich existieren spezielle Escape-Sequenzen für Objekttypen-Ids (siehe "Literale - Objekttypen-Ids") und Farben (siehe "Literale - Farbliterale").
Farbkodierungen und Objekttypen-Ids in Zeichenketten werden zum Kompilierzeitpunkt validiert bzw. kompiliert.
Die speziellen Escapesequenzen werden in den Zeichenketten zum Kompilierzeitpunkt ersetzt, insofern sie gültig sind.
Falls nicht, bleibt der Ausdruck erhalten und der Compiler gibt eine Warnung aus.
Wie bei normalen Objektypen-Literalen kann ein Prozentzeichen angehängt werden.
Bei Farben kann jedoch, anders als bei gewöhnlichen Farbliteralen, Text eingeschlossen werden. Dieser sollte dann in der entsprechenden Farbe
im Spiel bei einer Ausgabe angezeigt werden.

\section{Notation}
|c<Farbe>[<Text>]|r
<<Objekt-Id>,<Feld-Id>[, %]>

	\chapter{Literale}
Ein Literal ist ein sich selbst erklärender Ausdruck, der nicht speziell benannt werden muss also keinen Bezeichner benötigt.
Für folgende native Typen gelten folgende Literalnotationsregeln:

Kopierbasierte Typen:
TODO: Exponentialschreibweise für alle Ganzzahlausdrücke erlauben und auch für Fließkommazahlen?
\begin{enumerate}
\item string : "<Zeichenkettenliteral>" | null
\item integer : 0<Oktalliteral> | 0%<Binärliteral> | 0x<Hexadezimalliteral> | <Dezimalliteral>[e <Exponent>] | '<Id-Literal>' | null
\item real : <Fließkommaliteral> [r | R] | null
\item boolean : true | false | null
\item handle (und Nicht-agent-Kindtypen): null
\end{enumerate}

Referenzbasierte Typen:
\begin{enumerate}
\item code : 0
\item agent (und Kindtypen) : 0
\item Klassenobjekte : 0
\item Funktionszeiger : 0
\end{enumerate}

Dabei können Zeichenkettenausdrücke mehrfach hintereinander geschrieben werden. Diese werden dann automatisch zu einem einzelnen
zusammengefasst. Dies vereinfacht die Aufteilung auf mehrere Zeilen, ohne unnötige Zeilenumbrüche und Einrückungen dazwischen und
dient einer besseren Übersicht:
"<Ausdruckteil 1>"
"<Ausdruckteil n>"

\section{Nullwerte}
Wie oben dargestellt erhalten kopierbasierte Typen einen Nullwert mit dem Literal "null" und referenzbasierte Typen mit dem Literal "0".
Der Zugriff auf Nullwerte kann in manchen Fällen zu einem undefinierten Verhalten führen.
In der Regel ist er im Testmodus jedoch abgesichert und führt zu Fehlermeldungen (siehe "Klassen").

\section{Id-Literale}
Für Id-Literale gelten bei der Validierung selbige Regeln wie in der Sektion "Objekttypen-Literale" definiert. Es wird also
eine Warnung vom Compiler ausgegeben, falls sich die Id statisch überprüfen lässt und in keiner der entsprechenden Dateien
zu finden ist.

\section{Objekttypen-Literale}
Des Weiteren können Eigenschaften sogenannter Objekttypen, die einen bestimmten nativen, kopierbasierten Typ haben abgefragt werden.
Objekttypen erhalten in Warcraft 3 The Frozen Throne eine vierstellige, hexadezimale Id. Daher können Objekttypen im Wertebereich von 16^4 definiert werden.
Jede Id identifiziert genau einen Objekttyp.
Ein Objekttyp kann in JASS++ folgendes sein:
\begin{enumerate}
\item ein Einheitentyp
\item ein Gegenstandstyp
\item ein Doodad-Typ
\item ein Zerstörbares-Typ
\item ein Fähigkeitentyp
\item ein Zauberverstärkertyp
\item ein Forschungstyp
\item ein Wettereffekttyp
\item ein Blitzeffekttyp
\item ein Geländeset
\end{enumerate}

Die Objekttypen stammen aus eine der folgenden Dateien (aufsteigend nach Priorität geordnet):
\begin{enumerate}
\item war3map.u
\item war3map.d
\item UI/UnitData.slk
\item *
\item *
\end{enumerate}
TODO

Dabei werden die angegebenen Dateien in den folgenden Archiven in absteigender Reihenfolge durchsucht:
\begin{enumerate}
\item Karte
\item Kampagne
\item War3Patch.mpq
\item War3Locale.mpq (richtigen Namen finden)
\item War3X.mpq
\item War3.mpq
\end{enumerate}
TODO

Die Eigenschaften eines Objekttyps können einen der folgenden nativen, kopierbasierten Typen haben:
\begin{enumerate}
\item string
\item integer
\item real
\item boolean
\end{enumerate}
TODO

Die Eigenschaften eines Objekttyps haben selbst nochmals eine eigene Id.
Ein Objekttypen-Eigenschafts-Literal wird immer als ein Ausdruck eines der angegebenen Typen interpretiert.
Kann der Objekttypeintrag bzw. der Objekttyp-Eigenschaftseintrag nicht gefunden werden, so gibt der Compiler eine Warnung aus.
Das Literal ist in diesem Fall ein Nullwert ("null").
Wird keine Eigenschafts-Id angegeben, so wird der Ausdruck als Array interpretiert. Dies geht jedoch nur falls alle Objekttypen-Eigenschaften
den gleichen nativen, kopierbasierten Typ haben.
Ansonsten kann das Literal nur als eigener Typ interpertiert werden (siehe "Klassen - Zuweisungen").
Die Notation sieht folgendermaßen aus:
<<Objekttyp-Id>[, %]>
<<Objekttyp-Id>,<Eigenschafts-Id>[, %])>

Das Prozentzeichen ist optional und gibt an, dass Fließkommazahlen als Prozentanteile und daher als Ganzzahlen zwischen 0 und 100 interpretiert werden
sollen. Aus 0.30 würde in diesem Fall also die Ganzzahl 30 werden.
Es gilt bei einer reinen Angabe der Objekttyp-Id für alle Eigenschaften.
Es gilt zu beachten, dass Objekttypen-Ids auch als Escape-Sequenzen verwendet werden können.

\section{Farbliterale}
Farbliterale haben stets den nativen Typ "integer". Sie sind daher ein 32-Bit-Wert, welcher eine RGBA-Farbe enthält.
Der RGBA-Wert muss hexadezimal angegeben werden, wobei die ersten beiden Ziffern für den Alphawert, die zweiten beiden für den Grünwert,
die dritten beiden für den Blauwert und die letzten beiden für den Blauwert stehen.
Notation:
|c<Farbe>|r

\section{Escape-Sequenzen}
Eine Escape-Sequenz ist ein Zeichen, welchem das \-Zeichen vorangestellt wird und welches ausschließlich ein einem
Zeichenkettenliteral vorkommen darf.
Es gibt folgende Escape-Sequenzen:
\begin(enumerate}
\item n				Zeilenumbruch
\item t				Horizontaler Tabulator
\item \				Backslash
\item "				Doppeltes Anführungszeichen
\end{enumerate}

Zusätzlich existieren spezielle Escape-Sequenzen für Objekttypen-Ids (siehe "Literale - Objekttypen-Ids") und Farben (siehe "Literale - Farbliterale").
Farbkodierungen und Objekttypen-Ids in Zeichenketten werden zum Kompilierzeitpunkt validiert bzw. kompiliert.
Die speziellen Escapesequenzen werden in den Zeichenketten zum Kompilierzeitpunkt ersetzt, insofern sie gültig sind.
Falls nicht, bleibt der Ausdruck erhalten und der Compiler gibt eine Warnung aus.
Wie bei normalen Objektypen-Literalen kann ein Prozentzeichen angehängt werden.
Bei Farben kann jedoch, anders als bei gewöhnlichen Farbliteralen, Text eingeschlossen werden. Dieser sollte dann in der entsprechenden Farbe
im Spiel bei einer Ausgabe angezeigt werden.

\section{Notation}
|c<Farbe>[<Text>]|r
<<Objekt-Id>,<Feld-Id>[, %]>

	\chapter{Literale}
Ein Literal ist ein sich selbst erklärender Ausdruck, der nicht speziell benannt werden muss also keinen Bezeichner benötigt.
Für folgende native Typen gelten folgende Literalnotationsregeln:

Kopierbasierte Typen:
TODO: Exponentialschreibweise für alle Ganzzahlausdrücke erlauben und auch für Fließkommazahlen?
\begin{enumerate}
\item string : "<Zeichenkettenliteral>" | null
\item integer : 0<Oktalliteral> | 0%<Binärliteral> | 0x<Hexadezimalliteral> | <Dezimalliteral>[e <Exponent>] | '<Id-Literal>' | null
\item real : <Fließkommaliteral> [r | R] | null
\item boolean : true | false | null
\item handle (und Nicht-agent-Kindtypen): null
\end{enumerate}

Referenzbasierte Typen:
\begin{enumerate}
\item code : 0
\item agent (und Kindtypen) : 0
\item Klassenobjekte : 0
\item Funktionszeiger : 0
\end{enumerate}

Dabei können Zeichenkettenausdrücke mehrfach hintereinander geschrieben werden. Diese werden dann automatisch zu einem einzelnen
zusammengefasst. Dies vereinfacht die Aufteilung auf mehrere Zeilen, ohne unnötige Zeilenumbrüche und Einrückungen dazwischen und
dient einer besseren Übersicht:
"<Ausdruckteil 1>"
"<Ausdruckteil n>"

\section{Nullwerte}
Wie oben dargestellt erhalten kopierbasierte Typen einen Nullwert mit dem Literal "null" und referenzbasierte Typen mit dem Literal "0".
Der Zugriff auf Nullwerte kann in manchen Fällen zu einem undefinierten Verhalten führen.
In der Regel ist er im Testmodus jedoch abgesichert und führt zu Fehlermeldungen (siehe "Klassen").

\section{Id-Literale}
Für Id-Literale gelten bei der Validierung selbige Regeln wie in der Sektion "Objekttypen-Literale" definiert. Es wird also
eine Warnung vom Compiler ausgegeben, falls sich die Id statisch überprüfen lässt und in keiner der entsprechenden Dateien
zu finden ist.

\section{Objekttypen-Literale}
Des Weiteren können Eigenschaften sogenannter Objekttypen, die einen bestimmten nativen, kopierbasierten Typ haben abgefragt werden.
Objekttypen erhalten in Warcraft 3 The Frozen Throne eine vierstellige, hexadezimale Id. Daher können Objekttypen im Wertebereich von 16^4 definiert werden.
Jede Id identifiziert genau einen Objekttyp.
Ein Objekttyp kann in JASS++ folgendes sein:
\begin{enumerate}
\item ein Einheitentyp
\item ein Gegenstandstyp
\item ein Doodad-Typ
\item ein Zerstörbares-Typ
\item ein Fähigkeitentyp
\item ein Zauberverstärkertyp
\item ein Forschungstyp
\item ein Wettereffekttyp
\item ein Blitzeffekttyp
\item ein Geländeset
\end{enumerate}

Die Objekttypen stammen aus eine der folgenden Dateien (aufsteigend nach Priorität geordnet):
\begin{enumerate}
\item war3map.u
\item war3map.d
\item UI/UnitData.slk
\item *
\item *
\end{enumerate}
TODO

Dabei werden die angegebenen Dateien in den folgenden Archiven in absteigender Reihenfolge durchsucht:
\begin{enumerate}
\item Karte
\item Kampagne
\item War3Patch.mpq
\item War3Locale.mpq (richtigen Namen finden)
\item War3X.mpq
\item War3.mpq
\end{enumerate}
TODO

Die Eigenschaften eines Objekttyps können einen der folgenden nativen, kopierbasierten Typen haben:
\begin{enumerate}
\item string
\item integer
\item real
\item boolean
\end{enumerate}
TODO

Die Eigenschaften eines Objekttyps haben selbst nochmals eine eigene Id.
Ein Objekttypen-Eigenschafts-Literal wird immer als ein Ausdruck eines der angegebenen Typen interpretiert.
Kann der Objekttypeintrag bzw. der Objekttyp-Eigenschaftseintrag nicht gefunden werden, so gibt der Compiler eine Warnung aus.
Das Literal ist in diesem Fall ein Nullwert ("null").
Wird keine Eigenschafts-Id angegeben, so wird der Ausdruck als Array interpretiert. Dies geht jedoch nur falls alle Objekttypen-Eigenschaften
den gleichen nativen, kopierbasierten Typ haben.
Ansonsten kann das Literal nur als eigener Typ interpertiert werden (siehe "Klassen - Zuweisungen").
Die Notation sieht folgendermaßen aus:
<<Objekttyp-Id>[, %]>
<<Objekttyp-Id>,<Eigenschafts-Id>[, %])>

Das Prozentzeichen ist optional und gibt an, dass Fließkommazahlen als Prozentanteile und daher als Ganzzahlen zwischen 0 und 100 interpretiert werden
sollen. Aus 0.30 würde in diesem Fall also die Ganzzahl 30 werden.
Es gilt bei einer reinen Angabe der Objekttyp-Id für alle Eigenschaften.
Es gilt zu beachten, dass Objekttypen-Ids auch als Escape-Sequenzen verwendet werden können.

\section{Farbliterale}
Farbliterale haben stets den nativen Typ "integer". Sie sind daher ein 32-Bit-Wert, welcher eine RGBA-Farbe enthält.
Der RGBA-Wert muss hexadezimal angegeben werden, wobei die ersten beiden Ziffern für den Alphawert, die zweiten beiden für den Grünwert,
die dritten beiden für den Blauwert und die letzten beiden für den Blauwert stehen.
Notation:
|c<Farbe>|r

\section{Escape-Sequenzen}
Eine Escape-Sequenz ist ein Zeichen, welchem das \-Zeichen vorangestellt wird und welches ausschließlich ein einem
Zeichenkettenliteral vorkommen darf.
Es gibt folgende Escape-Sequenzen:
\begin(enumerate}
\item n				Zeilenumbruch
\item t				Horizontaler Tabulator
\item \				Backslash
\item "				Doppeltes Anführungszeichen
\end{enumerate}

Zusätzlich existieren spezielle Escape-Sequenzen für Objekttypen-Ids (siehe "Literale - Objekttypen-Ids") und Farben (siehe "Literale - Farbliterale").
Farbkodierungen und Objekttypen-Ids in Zeichenketten werden zum Kompilierzeitpunkt validiert bzw. kompiliert.
Die speziellen Escapesequenzen werden in den Zeichenketten zum Kompilierzeitpunkt ersetzt, insofern sie gültig sind.
Falls nicht, bleibt der Ausdruck erhalten und der Compiler gibt eine Warnung aus.
Wie bei normalen Objektypen-Literalen kann ein Prozentzeichen angehängt werden.
Bei Farben kann jedoch, anders als bei gewöhnlichen Farbliteralen, Text eingeschlossen werden. Dieser sollte dann in der entsprechenden Farbe
im Spiel bei einer Ausgabe angezeigt werden.

\section{Notation}
|c<Farbe>[<Text>]|r
<<Objekt-Id>,<Feld-Id>[, %]>

	\chapter{Literale}
Ein Literal ist ein sich selbst erklärender Ausdruck, der nicht speziell benannt werden muss also keinen Bezeichner benötigt.
Für folgende native Typen gelten folgende Literalnotationsregeln:

Kopierbasierte Typen:
TODO: Exponentialschreibweise für alle Ganzzahlausdrücke erlauben und auch für Fließkommazahlen?
\begin{enumerate}
\item string : "<Zeichenkettenliteral>" | null
\item integer : 0<Oktalliteral> | 0%<Binärliteral> | 0x<Hexadezimalliteral> | <Dezimalliteral>[e <Exponent>] | '<Id-Literal>' | null
\item real : <Fließkommaliteral> [r | R] | null
\item boolean : true | false | null
\item handle (und Nicht-agent-Kindtypen): null
\end{enumerate}

Referenzbasierte Typen:
\begin{enumerate}
\item code : 0
\item agent (und Kindtypen) : 0
\item Klassenobjekte : 0
\item Funktionszeiger : 0
\end{enumerate}

Dabei können Zeichenkettenausdrücke mehrfach hintereinander geschrieben werden. Diese werden dann automatisch zu einem einzelnen
zusammengefasst. Dies vereinfacht die Aufteilung auf mehrere Zeilen, ohne unnötige Zeilenumbrüche und Einrückungen dazwischen und
dient einer besseren Übersicht:
"<Ausdruckteil 1>"
"<Ausdruckteil n>"

\section{Nullwerte}
Wie oben dargestellt erhalten kopierbasierte Typen einen Nullwert mit dem Literal "null" und referenzbasierte Typen mit dem Literal "0".
Der Zugriff auf Nullwerte kann in manchen Fällen zu einem undefinierten Verhalten führen.
In der Regel ist er im Testmodus jedoch abgesichert und führt zu Fehlermeldungen (siehe "Klassen").

\section{Id-Literale}
Für Id-Literale gelten bei der Validierung selbige Regeln wie in der Sektion "Objekttypen-Literale" definiert. Es wird also
eine Warnung vom Compiler ausgegeben, falls sich die Id statisch überprüfen lässt und in keiner der entsprechenden Dateien
zu finden ist.

\section{Objekttypen-Literale}
Des Weiteren können Eigenschaften sogenannter Objekttypen, die einen bestimmten nativen, kopierbasierten Typ haben abgefragt werden.
Objekttypen erhalten in Warcraft 3 The Frozen Throne eine vierstellige, hexadezimale Id. Daher können Objekttypen im Wertebereich von 16^4 definiert werden.
Jede Id identifiziert genau einen Objekttyp.
Ein Objekttyp kann in JASS++ folgendes sein:
\begin{enumerate}
\item ein Einheitentyp
\item ein Gegenstandstyp
\item ein Doodad-Typ
\item ein Zerstörbares-Typ
\item ein Fähigkeitentyp
\item ein Zauberverstärkertyp
\item ein Forschungstyp
\item ein Wettereffekttyp
\item ein Blitzeffekttyp
\item ein Geländeset
\end{enumerate}

Die Objekttypen stammen aus eine der folgenden Dateien (aufsteigend nach Priorität geordnet):
\begin{enumerate}
\item war3map.u
\item war3map.d
\item UI/UnitData.slk
\item *
\item *
\end{enumerate}
TODO

Dabei werden die angegebenen Dateien in den folgenden Archiven in absteigender Reihenfolge durchsucht:
\begin{enumerate}
\item Karte
\item Kampagne
\item War3Patch.mpq
\item War3Locale.mpq (richtigen Namen finden)
\item War3X.mpq
\item War3.mpq
\end{enumerate}
TODO

Die Eigenschaften eines Objekttyps können einen der folgenden nativen, kopierbasierten Typen haben:
\begin{enumerate}
\item string
\item integer
\item real
\item boolean
\end{enumerate}
TODO

Die Eigenschaften eines Objekttyps haben selbst nochmals eine eigene Id.
Ein Objekttypen-Eigenschafts-Literal wird immer als ein Ausdruck eines der angegebenen Typen interpretiert.
Kann der Objekttypeintrag bzw. der Objekttyp-Eigenschaftseintrag nicht gefunden werden, so gibt der Compiler eine Warnung aus.
Das Literal ist in diesem Fall ein Nullwert ("null").
Wird keine Eigenschafts-Id angegeben, so wird der Ausdruck als Array interpretiert. Dies geht jedoch nur falls alle Objekttypen-Eigenschaften
den gleichen nativen, kopierbasierten Typ haben.
Ansonsten kann das Literal nur als eigener Typ interpertiert werden (siehe "Klassen - Zuweisungen").
Die Notation sieht folgendermaßen aus:
<<Objekttyp-Id>[, %]>
<<Objekttyp-Id>,<Eigenschafts-Id>[, %])>

Das Prozentzeichen ist optional und gibt an, dass Fließkommazahlen als Prozentanteile und daher als Ganzzahlen zwischen 0 und 100 interpretiert werden
sollen. Aus 0.30 würde in diesem Fall also die Ganzzahl 30 werden.
Es gilt bei einer reinen Angabe der Objekttyp-Id für alle Eigenschaften.
Es gilt zu beachten, dass Objekttypen-Ids auch als Escape-Sequenzen verwendet werden können.

\section{Farbliterale}
Farbliterale haben stets den nativen Typ "integer". Sie sind daher ein 32-Bit-Wert, welcher eine RGBA-Farbe enthält.
Der RGBA-Wert muss hexadezimal angegeben werden, wobei die ersten beiden Ziffern für den Alphawert, die zweiten beiden für den Grünwert,
die dritten beiden für den Blauwert und die letzten beiden für den Blauwert stehen.
Notation:
|c<Farbe>|r

\section{Escape-Sequenzen}
Eine Escape-Sequenz ist ein Zeichen, welchem das \-Zeichen vorangestellt wird und welches ausschließlich ein einem
Zeichenkettenliteral vorkommen darf.
Es gibt folgende Escape-Sequenzen:
\begin(enumerate}
\item n				Zeilenumbruch
\item t				Horizontaler Tabulator
\item \				Backslash
\item "				Doppeltes Anführungszeichen
\end{enumerate}

Zusätzlich existieren spezielle Escape-Sequenzen für Objekttypen-Ids (siehe "Literale - Objekttypen-Ids") und Farben (siehe "Literale - Farbliterale").
Farbkodierungen und Objekttypen-Ids in Zeichenketten werden zum Kompilierzeitpunkt validiert bzw. kompiliert.
Die speziellen Escapesequenzen werden in den Zeichenketten zum Kompilierzeitpunkt ersetzt, insofern sie gültig sind.
Falls nicht, bleibt der Ausdruck erhalten und der Compiler gibt eine Warnung aus.
Wie bei normalen Objektypen-Literalen kann ein Prozentzeichen angehängt werden.
Bei Farben kann jedoch, anders als bei gewöhnlichen Farbliteralen, Text eingeschlossen werden. Dieser sollte dann in der entsprechenden Farbe
im Spiel bei einer Ausgabe angezeigt werden.

\section{Notation}
|c<Farbe>[<Text>]|r
<<Objekt-Id>,<Feld-Id>[, %]>

	\chapter{Literale}
Ein Literal ist ein sich selbst erklärender Ausdruck, der nicht speziell benannt werden muss also keinen Bezeichner benötigt.
Für folgende native Typen gelten folgende Literalnotationsregeln:

Kopierbasierte Typen:
TODO: Exponentialschreibweise für alle Ganzzahlausdrücke erlauben und auch für Fließkommazahlen?
\begin{enumerate}
\item string : "<Zeichenkettenliteral>" | null
\item integer : 0<Oktalliteral> | 0%<Binärliteral> | 0x<Hexadezimalliteral> | <Dezimalliteral>[e <Exponent>] | '<Id-Literal>' | null
\item real : <Fließkommaliteral> [r | R] | null
\item boolean : true | false | null
\item handle (und Nicht-agent-Kindtypen): null
\end{enumerate}

Referenzbasierte Typen:
\begin{enumerate}
\item code : 0
\item agent (und Kindtypen) : 0
\item Klassenobjekte : 0
\item Funktionszeiger : 0
\end{enumerate}

Dabei können Zeichenkettenausdrücke mehrfach hintereinander geschrieben werden. Diese werden dann automatisch zu einem einzelnen
zusammengefasst. Dies vereinfacht die Aufteilung auf mehrere Zeilen, ohne unnötige Zeilenumbrüche und Einrückungen dazwischen und
dient einer besseren Übersicht:
"<Ausdruckteil 1>"
"<Ausdruckteil n>"

\section{Nullwerte}
Wie oben dargestellt erhalten kopierbasierte Typen einen Nullwert mit dem Literal "null" und referenzbasierte Typen mit dem Literal "0".
Der Zugriff auf Nullwerte kann in manchen Fällen zu einem undefinierten Verhalten führen.
In der Regel ist er im Testmodus jedoch abgesichert und führt zu Fehlermeldungen (siehe "Klassen").

\section{Id-Literale}
Für Id-Literale gelten bei der Validierung selbige Regeln wie in der Sektion "Objekttypen-Literale" definiert. Es wird also
eine Warnung vom Compiler ausgegeben, falls sich die Id statisch überprüfen lässt und in keiner der entsprechenden Dateien
zu finden ist.

\section{Objekttypen-Literale}
Des Weiteren können Eigenschaften sogenannter Objekttypen, die einen bestimmten nativen, kopierbasierten Typ haben abgefragt werden.
Objekttypen erhalten in Warcraft 3 The Frozen Throne eine vierstellige, hexadezimale Id. Daher können Objekttypen im Wertebereich von 16^4 definiert werden.
Jede Id identifiziert genau einen Objekttyp.
Ein Objekttyp kann in JASS++ folgendes sein:
\begin{enumerate}
\item ein Einheitentyp
\item ein Gegenstandstyp
\item ein Doodad-Typ
\item ein Zerstörbares-Typ
\item ein Fähigkeitentyp
\item ein Zauberverstärkertyp
\item ein Forschungstyp
\item ein Wettereffekttyp
\item ein Blitzeffekttyp
\item ein Geländeset
\end{enumerate}

Die Objekttypen stammen aus eine der folgenden Dateien (aufsteigend nach Priorität geordnet):
\begin{enumerate}
\item war3map.u
\item war3map.d
\item UI/UnitData.slk
\item *
\item *
\end{enumerate}
TODO

Dabei werden die angegebenen Dateien in den folgenden Archiven in absteigender Reihenfolge durchsucht:
\begin{enumerate}
\item Karte
\item Kampagne
\item War3Patch.mpq
\item War3Locale.mpq (richtigen Namen finden)
\item War3X.mpq
\item War3.mpq
\end{enumerate}
TODO

Die Eigenschaften eines Objekttyps können einen der folgenden nativen, kopierbasierten Typen haben:
\begin{enumerate}
\item string
\item integer
\item real
\item boolean
\end{enumerate}
TODO

Die Eigenschaften eines Objekttyps haben selbst nochmals eine eigene Id.
Ein Objekttypen-Eigenschafts-Literal wird immer als ein Ausdruck eines der angegebenen Typen interpretiert.
Kann der Objekttypeintrag bzw. der Objekttyp-Eigenschaftseintrag nicht gefunden werden, so gibt der Compiler eine Warnung aus.
Das Literal ist in diesem Fall ein Nullwert ("null").
Wird keine Eigenschafts-Id angegeben, so wird der Ausdruck als Array interpretiert. Dies geht jedoch nur falls alle Objekttypen-Eigenschaften
den gleichen nativen, kopierbasierten Typ haben.
Ansonsten kann das Literal nur als eigener Typ interpertiert werden (siehe "Klassen - Zuweisungen").
Die Notation sieht folgendermaßen aus:
<<Objekttyp-Id>[, %]>
<<Objekttyp-Id>,<Eigenschafts-Id>[, %])>

Das Prozentzeichen ist optional und gibt an, dass Fließkommazahlen als Prozentanteile und daher als Ganzzahlen zwischen 0 und 100 interpretiert werden
sollen. Aus 0.30 würde in diesem Fall also die Ganzzahl 30 werden.
Es gilt bei einer reinen Angabe der Objekttyp-Id für alle Eigenschaften.
Es gilt zu beachten, dass Objekttypen-Ids auch als Escape-Sequenzen verwendet werden können.

\section{Farbliterale}
Farbliterale haben stets den nativen Typ "integer". Sie sind daher ein 32-Bit-Wert, welcher eine RGBA-Farbe enthält.
Der RGBA-Wert muss hexadezimal angegeben werden, wobei die ersten beiden Ziffern für den Alphawert, die zweiten beiden für den Grünwert,
die dritten beiden für den Blauwert und die letzten beiden für den Blauwert stehen.
Notation:
|c<Farbe>|r

\section{Escape-Sequenzen}
Eine Escape-Sequenz ist ein Zeichen, welchem das \-Zeichen vorangestellt wird und welches ausschließlich ein einem
Zeichenkettenliteral vorkommen darf.
Es gibt folgende Escape-Sequenzen:
\begin(enumerate}
\item n				Zeilenumbruch
\item t				Horizontaler Tabulator
\item \				Backslash
\item "				Doppeltes Anführungszeichen
\end{enumerate}

Zusätzlich existieren spezielle Escape-Sequenzen für Objekttypen-Ids (siehe "Literale - Objekttypen-Ids") und Farben (siehe "Literale - Farbliterale").
Farbkodierungen und Objekttypen-Ids in Zeichenketten werden zum Kompilierzeitpunkt validiert bzw. kompiliert.
Die speziellen Escapesequenzen werden in den Zeichenketten zum Kompilierzeitpunkt ersetzt, insofern sie gültig sind.
Falls nicht, bleibt der Ausdruck erhalten und der Compiler gibt eine Warnung aus.
Wie bei normalen Objektypen-Literalen kann ein Prozentzeichen angehängt werden.
Bei Farben kann jedoch, anders als bei gewöhnlichen Farbliteralen, Text eingeschlossen werden. Dieser sollte dann in der entsprechenden Farbe
im Spiel bei einer Ausgabe angezeigt werden.

\section{Notation}
|c<Farbe>[<Text>]|r
<<Objekt-Id>,<Feld-Id>[, %]>


	\chapter{Literale}
Ein Literal ist ein sich selbst erklärender Ausdruck, der nicht speziell benannt werden muss also keinen Bezeichner benötigt.
Für folgende native Typen gelten folgende Literalnotationsregeln:

Kopierbasierte Typen:
TODO: Exponentialschreibweise für alle Ganzzahlausdrücke erlauben und auch für Fließkommazahlen?
\begin{enumerate}
\item string : "<Zeichenkettenliteral>" | null
\item integer : 0<Oktalliteral> | 0%<Binärliteral> | 0x<Hexadezimalliteral> | <Dezimalliteral>[e <Exponent>] | '<Id-Literal>' | null
\item real : <Fließkommaliteral> [r | R] | null
\item boolean : true | false | null
\item handle (und Nicht-agent-Kindtypen): null
\end{enumerate}

Referenzbasierte Typen:
\begin{enumerate}
\item code : 0
\item agent (und Kindtypen) : 0
\item Klassenobjekte : 0
\item Funktionszeiger : 0
\end{enumerate}

Dabei können Zeichenkettenausdrücke mehrfach hintereinander geschrieben werden. Diese werden dann automatisch zu einem einzelnen
zusammengefasst. Dies vereinfacht die Aufteilung auf mehrere Zeilen, ohne unnötige Zeilenumbrüche und Einrückungen dazwischen und
dient einer besseren Übersicht:
"<Ausdruckteil 1>"
"<Ausdruckteil n>"

\section{Nullwerte}
Wie oben dargestellt erhalten kopierbasierte Typen einen Nullwert mit dem Literal "null" und referenzbasierte Typen mit dem Literal "0".
Der Zugriff auf Nullwerte kann in manchen Fällen zu einem undefinierten Verhalten führen.
In der Regel ist er im Testmodus jedoch abgesichert und führt zu Fehlermeldungen (siehe "Klassen").

\section{Id-Literale}
Für Id-Literale gelten bei der Validierung selbige Regeln wie in der Sektion "Objekttypen-Literale" definiert. Es wird also
eine Warnung vom Compiler ausgegeben, falls sich die Id statisch überprüfen lässt und in keiner der entsprechenden Dateien
zu finden ist.

\section{Objekttypen-Literale}
Des Weiteren können Eigenschaften sogenannter Objekttypen, die einen bestimmten nativen, kopierbasierten Typ haben abgefragt werden.
Objekttypen erhalten in Warcraft 3 The Frozen Throne eine vierstellige, hexadezimale Id. Daher können Objekttypen im Wertebereich von 16^4 definiert werden.
Jede Id identifiziert genau einen Objekttyp.
Ein Objekttyp kann in JASS++ folgendes sein:
\begin{enumerate}
\item ein Einheitentyp
\item ein Gegenstandstyp
\item ein Doodad-Typ
\item ein Zerstörbares-Typ
\item ein Fähigkeitentyp
\item ein Zauberverstärkertyp
\item ein Forschungstyp
\item ein Wettereffekttyp
\item ein Blitzeffekttyp
\item ein Geländeset
\end{enumerate}

Die Objekttypen stammen aus eine der folgenden Dateien (aufsteigend nach Priorität geordnet):
\begin{enumerate}
\item war3map.u
\item war3map.d
\item UI/UnitData.slk
\item *
\item *
\end{enumerate}
TODO

Dabei werden die angegebenen Dateien in den folgenden Archiven in absteigender Reihenfolge durchsucht:
\begin{enumerate}
\item Karte
\item Kampagne
\item War3Patch.mpq
\item War3Locale.mpq (richtigen Namen finden)
\item War3X.mpq
\item War3.mpq
\end{enumerate}
TODO

Die Eigenschaften eines Objekttyps können einen der folgenden nativen, kopierbasierten Typen haben:
\begin{enumerate}
\item string
\item integer
\item real
\item boolean
\end{enumerate}
TODO

Die Eigenschaften eines Objekttyps haben selbst nochmals eine eigene Id.
Ein Objekttypen-Eigenschafts-Literal wird immer als ein Ausdruck eines der angegebenen Typen interpretiert.
Kann der Objekttypeintrag bzw. der Objekttyp-Eigenschaftseintrag nicht gefunden werden, so gibt der Compiler eine Warnung aus.
Das Literal ist in diesem Fall ein Nullwert ("null").
Wird keine Eigenschafts-Id angegeben, so wird der Ausdruck als Array interpretiert. Dies geht jedoch nur falls alle Objekttypen-Eigenschaften
den gleichen nativen, kopierbasierten Typ haben.
Ansonsten kann das Literal nur als eigener Typ interpertiert werden (siehe "Klassen - Zuweisungen").
Die Notation sieht folgendermaßen aus:
<<Objekttyp-Id>[, %]>
<<Objekttyp-Id>,<Eigenschafts-Id>[, %])>

Das Prozentzeichen ist optional und gibt an, dass Fließkommazahlen als Prozentanteile und daher als Ganzzahlen zwischen 0 und 100 interpretiert werden
sollen. Aus 0.30 würde in diesem Fall also die Ganzzahl 30 werden.
Es gilt bei einer reinen Angabe der Objekttyp-Id für alle Eigenschaften.
Es gilt zu beachten, dass Objekttypen-Ids auch als Escape-Sequenzen verwendet werden können.

\section{Farbliterale}
Farbliterale haben stets den nativen Typ "integer". Sie sind daher ein 32-Bit-Wert, welcher eine RGBA-Farbe enthält.
Der RGBA-Wert muss hexadezimal angegeben werden, wobei die ersten beiden Ziffern für den Alphawert, die zweiten beiden für den Grünwert,
die dritten beiden für den Blauwert und die letzten beiden für den Blauwert stehen.
Notation:
|c<Farbe>|r

\section{Escape-Sequenzen}
Eine Escape-Sequenz ist ein Zeichen, welchem das \-Zeichen vorangestellt wird und welches ausschließlich ein einem
Zeichenkettenliteral vorkommen darf.
Es gibt folgende Escape-Sequenzen:
\begin(enumerate}
\item n				Zeilenumbruch
\item t				Horizontaler Tabulator
\item \				Backslash
\item "				Doppeltes Anführungszeichen
\end{enumerate}

Zusätzlich existieren spezielle Escape-Sequenzen für Objekttypen-Ids (siehe "Literale - Objekttypen-Ids") und Farben (siehe "Literale - Farbliterale").
Farbkodierungen und Objekttypen-Ids in Zeichenketten werden zum Kompilierzeitpunkt validiert bzw. kompiliert.
Die speziellen Escapesequenzen werden in den Zeichenketten zum Kompilierzeitpunkt ersetzt, insofern sie gültig sind.
Falls nicht, bleibt der Ausdruck erhalten und der Compiler gibt eine Warnung aus.
Wie bei normalen Objektypen-Literalen kann ein Prozentzeichen angehängt werden.
Bei Farben kann jedoch, anders als bei gewöhnlichen Farbliteralen, Text eingeschlossen werden. Dieser sollte dann in der entsprechenden Farbe
im Spiel bei einer Ausgabe angezeigt werden.

\section{Notation}
|c<Farbe>[<Text>]|r
<<Objekt-Id>,<Feld-Id>[, %]>

	\chapter{Literale}
Ein Literal ist ein sich selbst erklärender Ausdruck, der nicht speziell benannt werden muss also keinen Bezeichner benötigt.
Für folgende native Typen gelten folgende Literalnotationsregeln:

Kopierbasierte Typen:
TODO: Exponentialschreibweise für alle Ganzzahlausdrücke erlauben und auch für Fließkommazahlen?
\begin{enumerate}
\item string : "<Zeichenkettenliteral>" | null
\item integer : 0<Oktalliteral> | 0%<Binärliteral> | 0x<Hexadezimalliteral> | <Dezimalliteral>[e <Exponent>] | '<Id-Literal>' | null
\item real : <Fließkommaliteral> [r | R] | null
\item boolean : true | false | null
\item handle (und Nicht-agent-Kindtypen): null
\end{enumerate}

Referenzbasierte Typen:
\begin{enumerate}
\item code : 0
\item agent (und Kindtypen) : 0
\item Klassenobjekte : 0
\item Funktionszeiger : 0
\end{enumerate}

Dabei können Zeichenkettenausdrücke mehrfach hintereinander geschrieben werden. Diese werden dann automatisch zu einem einzelnen
zusammengefasst. Dies vereinfacht die Aufteilung auf mehrere Zeilen, ohne unnötige Zeilenumbrüche und Einrückungen dazwischen und
dient einer besseren Übersicht:
"<Ausdruckteil 1>"
"<Ausdruckteil n>"

\section{Nullwerte}
Wie oben dargestellt erhalten kopierbasierte Typen einen Nullwert mit dem Literal "null" und referenzbasierte Typen mit dem Literal "0".
Der Zugriff auf Nullwerte kann in manchen Fällen zu einem undefinierten Verhalten führen.
In der Regel ist er im Testmodus jedoch abgesichert und führt zu Fehlermeldungen (siehe "Klassen").

\section{Id-Literale}
Für Id-Literale gelten bei der Validierung selbige Regeln wie in der Sektion "Objekttypen-Literale" definiert. Es wird also
eine Warnung vom Compiler ausgegeben, falls sich die Id statisch überprüfen lässt und in keiner der entsprechenden Dateien
zu finden ist.

\section{Objekttypen-Literale}
Des Weiteren können Eigenschaften sogenannter Objekttypen, die einen bestimmten nativen, kopierbasierten Typ haben abgefragt werden.
Objekttypen erhalten in Warcraft 3 The Frozen Throne eine vierstellige, hexadezimale Id. Daher können Objekttypen im Wertebereich von 16^4 definiert werden.
Jede Id identifiziert genau einen Objekttyp.
Ein Objekttyp kann in JASS++ folgendes sein:
\begin{enumerate}
\item ein Einheitentyp
\item ein Gegenstandstyp
\item ein Doodad-Typ
\item ein Zerstörbares-Typ
\item ein Fähigkeitentyp
\item ein Zauberverstärkertyp
\item ein Forschungstyp
\item ein Wettereffekttyp
\item ein Blitzeffekttyp
\item ein Geländeset
\end{enumerate}

Die Objekttypen stammen aus eine der folgenden Dateien (aufsteigend nach Priorität geordnet):
\begin{enumerate}
\item war3map.u
\item war3map.d
\item UI/UnitData.slk
\item *
\item *
\end{enumerate}
TODO

Dabei werden die angegebenen Dateien in den folgenden Archiven in absteigender Reihenfolge durchsucht:
\begin{enumerate}
\item Karte
\item Kampagne
\item War3Patch.mpq
\item War3Locale.mpq (richtigen Namen finden)
\item War3X.mpq
\item War3.mpq
\end{enumerate}
TODO

Die Eigenschaften eines Objekttyps können einen der folgenden nativen, kopierbasierten Typen haben:
\begin{enumerate}
\item string
\item integer
\item real
\item boolean
\end{enumerate}
TODO

Die Eigenschaften eines Objekttyps haben selbst nochmals eine eigene Id.
Ein Objekttypen-Eigenschafts-Literal wird immer als ein Ausdruck eines der angegebenen Typen interpretiert.
Kann der Objekttypeintrag bzw. der Objekttyp-Eigenschaftseintrag nicht gefunden werden, so gibt der Compiler eine Warnung aus.
Das Literal ist in diesem Fall ein Nullwert ("null").
Wird keine Eigenschafts-Id angegeben, so wird der Ausdruck als Array interpretiert. Dies geht jedoch nur falls alle Objekttypen-Eigenschaften
den gleichen nativen, kopierbasierten Typ haben.
Ansonsten kann das Literal nur als eigener Typ interpertiert werden (siehe "Klassen - Zuweisungen").
Die Notation sieht folgendermaßen aus:
<<Objekttyp-Id>[, %]>
<<Objekttyp-Id>,<Eigenschafts-Id>[, %])>

Das Prozentzeichen ist optional und gibt an, dass Fließkommazahlen als Prozentanteile und daher als Ganzzahlen zwischen 0 und 100 interpretiert werden
sollen. Aus 0.30 würde in diesem Fall also die Ganzzahl 30 werden.
Es gilt bei einer reinen Angabe der Objekttyp-Id für alle Eigenschaften.
Es gilt zu beachten, dass Objekttypen-Ids auch als Escape-Sequenzen verwendet werden können.

\section{Farbliterale}
Farbliterale haben stets den nativen Typ "integer". Sie sind daher ein 32-Bit-Wert, welcher eine RGBA-Farbe enthält.
Der RGBA-Wert muss hexadezimal angegeben werden, wobei die ersten beiden Ziffern für den Alphawert, die zweiten beiden für den Grünwert,
die dritten beiden für den Blauwert und die letzten beiden für den Blauwert stehen.
Notation:
|c<Farbe>|r

\section{Escape-Sequenzen}
Eine Escape-Sequenz ist ein Zeichen, welchem das \-Zeichen vorangestellt wird und welches ausschließlich ein einem
Zeichenkettenliteral vorkommen darf.
Es gibt folgende Escape-Sequenzen:
\begin(enumerate}
\item n				Zeilenumbruch
\item t				Horizontaler Tabulator
\item \				Backslash
\item "				Doppeltes Anführungszeichen
\end{enumerate}

Zusätzlich existieren spezielle Escape-Sequenzen für Objekttypen-Ids (siehe "Literale - Objekttypen-Ids") und Farben (siehe "Literale - Farbliterale").
Farbkodierungen und Objekttypen-Ids in Zeichenketten werden zum Kompilierzeitpunkt validiert bzw. kompiliert.
Die speziellen Escapesequenzen werden in den Zeichenketten zum Kompilierzeitpunkt ersetzt, insofern sie gültig sind.
Falls nicht, bleibt der Ausdruck erhalten und der Compiler gibt eine Warnung aus.
Wie bei normalen Objektypen-Literalen kann ein Prozentzeichen angehängt werden.
Bei Farben kann jedoch, anders als bei gewöhnlichen Farbliteralen, Text eingeschlossen werden. Dieser sollte dann in der entsprechenden Farbe
im Spiel bei einer Ausgabe angezeigt werden.

\section{Notation}
|c<Farbe>[<Text>]|r
<<Objekt-Id>,<Feld-Id>[, %]>

	\chapter{Literale}
Ein Literal ist ein sich selbst erklärender Ausdruck, der nicht speziell benannt werden muss also keinen Bezeichner benötigt.
Für folgende native Typen gelten folgende Literalnotationsregeln:

Kopierbasierte Typen:
TODO: Exponentialschreibweise für alle Ganzzahlausdrücke erlauben und auch für Fließkommazahlen?
\begin{enumerate}
\item string : "<Zeichenkettenliteral>" | null
\item integer : 0<Oktalliteral> | 0%<Binärliteral> | 0x<Hexadezimalliteral> | <Dezimalliteral>[e <Exponent>] | '<Id-Literal>' | null
\item real : <Fließkommaliteral> [r | R] | null
\item boolean : true | false | null
\item handle (und Nicht-agent-Kindtypen): null
\end{enumerate}

Referenzbasierte Typen:
\begin{enumerate}
\item code : 0
\item agent (und Kindtypen) : 0
\item Klassenobjekte : 0
\item Funktionszeiger : 0
\end{enumerate}

Dabei können Zeichenkettenausdrücke mehrfach hintereinander geschrieben werden. Diese werden dann automatisch zu einem einzelnen
zusammengefasst. Dies vereinfacht die Aufteilung auf mehrere Zeilen, ohne unnötige Zeilenumbrüche und Einrückungen dazwischen und
dient einer besseren Übersicht:
"<Ausdruckteil 1>"
"<Ausdruckteil n>"

\section{Nullwerte}
Wie oben dargestellt erhalten kopierbasierte Typen einen Nullwert mit dem Literal "null" und referenzbasierte Typen mit dem Literal "0".
Der Zugriff auf Nullwerte kann in manchen Fällen zu einem undefinierten Verhalten führen.
In der Regel ist er im Testmodus jedoch abgesichert und führt zu Fehlermeldungen (siehe "Klassen").

\section{Id-Literale}
Für Id-Literale gelten bei der Validierung selbige Regeln wie in der Sektion "Objekttypen-Literale" definiert. Es wird also
eine Warnung vom Compiler ausgegeben, falls sich die Id statisch überprüfen lässt und in keiner der entsprechenden Dateien
zu finden ist.

\section{Objekttypen-Literale}
Des Weiteren können Eigenschaften sogenannter Objekttypen, die einen bestimmten nativen, kopierbasierten Typ haben abgefragt werden.
Objekttypen erhalten in Warcraft 3 The Frozen Throne eine vierstellige, hexadezimale Id. Daher können Objekttypen im Wertebereich von 16^4 definiert werden.
Jede Id identifiziert genau einen Objekttyp.
Ein Objekttyp kann in JASS++ folgendes sein:
\begin{enumerate}
\item ein Einheitentyp
\item ein Gegenstandstyp
\item ein Doodad-Typ
\item ein Zerstörbares-Typ
\item ein Fähigkeitentyp
\item ein Zauberverstärkertyp
\item ein Forschungstyp
\item ein Wettereffekttyp
\item ein Blitzeffekttyp
\item ein Geländeset
\end{enumerate}

Die Objekttypen stammen aus eine der folgenden Dateien (aufsteigend nach Priorität geordnet):
\begin{enumerate}
\item war3map.u
\item war3map.d
\item UI/UnitData.slk
\item *
\item *
\end{enumerate}
TODO

Dabei werden die angegebenen Dateien in den folgenden Archiven in absteigender Reihenfolge durchsucht:
\begin{enumerate}
\item Karte
\item Kampagne
\item War3Patch.mpq
\item War3Locale.mpq (richtigen Namen finden)
\item War3X.mpq
\item War3.mpq
\end{enumerate}
TODO

Die Eigenschaften eines Objekttyps können einen der folgenden nativen, kopierbasierten Typen haben:
\begin{enumerate}
\item string
\item integer
\item real
\item boolean
\end{enumerate}
TODO

Die Eigenschaften eines Objekttyps haben selbst nochmals eine eigene Id.
Ein Objekttypen-Eigenschafts-Literal wird immer als ein Ausdruck eines der angegebenen Typen interpretiert.
Kann der Objekttypeintrag bzw. der Objekttyp-Eigenschaftseintrag nicht gefunden werden, so gibt der Compiler eine Warnung aus.
Das Literal ist in diesem Fall ein Nullwert ("null").
Wird keine Eigenschafts-Id angegeben, so wird der Ausdruck als Array interpretiert. Dies geht jedoch nur falls alle Objekttypen-Eigenschaften
den gleichen nativen, kopierbasierten Typ haben.
Ansonsten kann das Literal nur als eigener Typ interpertiert werden (siehe "Klassen - Zuweisungen").
Die Notation sieht folgendermaßen aus:
<<Objekttyp-Id>[, %]>
<<Objekttyp-Id>,<Eigenschafts-Id>[, %])>

Das Prozentzeichen ist optional und gibt an, dass Fließkommazahlen als Prozentanteile und daher als Ganzzahlen zwischen 0 und 100 interpretiert werden
sollen. Aus 0.30 würde in diesem Fall also die Ganzzahl 30 werden.
Es gilt bei einer reinen Angabe der Objekttyp-Id für alle Eigenschaften.
Es gilt zu beachten, dass Objekttypen-Ids auch als Escape-Sequenzen verwendet werden können.

\section{Farbliterale}
Farbliterale haben stets den nativen Typ "integer". Sie sind daher ein 32-Bit-Wert, welcher eine RGBA-Farbe enthält.
Der RGBA-Wert muss hexadezimal angegeben werden, wobei die ersten beiden Ziffern für den Alphawert, die zweiten beiden für den Grünwert,
die dritten beiden für den Blauwert und die letzten beiden für den Blauwert stehen.
Notation:
|c<Farbe>|r

\section{Escape-Sequenzen}
Eine Escape-Sequenz ist ein Zeichen, welchem das \-Zeichen vorangestellt wird und welches ausschließlich ein einem
Zeichenkettenliteral vorkommen darf.
Es gibt folgende Escape-Sequenzen:
\begin(enumerate}
\item n				Zeilenumbruch
\item t				Horizontaler Tabulator
\item \				Backslash
\item "				Doppeltes Anführungszeichen
\end{enumerate}

Zusätzlich existieren spezielle Escape-Sequenzen für Objekttypen-Ids (siehe "Literale - Objekttypen-Ids") und Farben (siehe "Literale - Farbliterale").
Farbkodierungen und Objekttypen-Ids in Zeichenketten werden zum Kompilierzeitpunkt validiert bzw. kompiliert.
Die speziellen Escapesequenzen werden in den Zeichenketten zum Kompilierzeitpunkt ersetzt, insofern sie gültig sind.
Falls nicht, bleibt der Ausdruck erhalten und der Compiler gibt eine Warnung aus.
Wie bei normalen Objektypen-Literalen kann ein Prozentzeichen angehängt werden.
Bei Farben kann jedoch, anders als bei gewöhnlichen Farbliteralen, Text eingeschlossen werden. Dieser sollte dann in der entsprechenden Farbe
im Spiel bei einer Ausgabe angezeigt werden.

\section{Notation}
|c<Farbe>[<Text>]|r
<<Objekt-Id>,<Feld-Id>[, %]>


	\chapter{Literale}
Ein Literal ist ein sich selbst erklärender Ausdruck, der nicht speziell benannt werden muss also keinen Bezeichner benötigt.
Für folgende native Typen gelten folgende Literalnotationsregeln:

Kopierbasierte Typen:
TODO: Exponentialschreibweise für alle Ganzzahlausdrücke erlauben und auch für Fließkommazahlen?
\begin{enumerate}
\item string : "<Zeichenkettenliteral>" | null
\item integer : 0<Oktalliteral> | 0%<Binärliteral> | 0x<Hexadezimalliteral> | <Dezimalliteral>[e <Exponent>] | '<Id-Literal>' | null
\item real : <Fließkommaliteral> [r | R] | null
\item boolean : true | false | null
\item handle (und Nicht-agent-Kindtypen): null
\end{enumerate}

Referenzbasierte Typen:
\begin{enumerate}
\item code : 0
\item agent (und Kindtypen) : 0
\item Klassenobjekte : 0
\item Funktionszeiger : 0
\end{enumerate}

Dabei können Zeichenkettenausdrücke mehrfach hintereinander geschrieben werden. Diese werden dann automatisch zu einem einzelnen
zusammengefasst. Dies vereinfacht die Aufteilung auf mehrere Zeilen, ohne unnötige Zeilenumbrüche und Einrückungen dazwischen und
dient einer besseren Übersicht:
"<Ausdruckteil 1>"
"<Ausdruckteil n>"

\section{Nullwerte}
Wie oben dargestellt erhalten kopierbasierte Typen einen Nullwert mit dem Literal "null" und referenzbasierte Typen mit dem Literal "0".
Der Zugriff auf Nullwerte kann in manchen Fällen zu einem undefinierten Verhalten führen.
In der Regel ist er im Testmodus jedoch abgesichert und führt zu Fehlermeldungen (siehe "Klassen").

\section{Id-Literale}
Für Id-Literale gelten bei der Validierung selbige Regeln wie in der Sektion "Objekttypen-Literale" definiert. Es wird also
eine Warnung vom Compiler ausgegeben, falls sich die Id statisch überprüfen lässt und in keiner der entsprechenden Dateien
zu finden ist.

\section{Objekttypen-Literale}
Des Weiteren können Eigenschaften sogenannter Objekttypen, die einen bestimmten nativen, kopierbasierten Typ haben abgefragt werden.
Objekttypen erhalten in Warcraft 3 The Frozen Throne eine vierstellige, hexadezimale Id. Daher können Objekttypen im Wertebereich von 16^4 definiert werden.
Jede Id identifiziert genau einen Objekttyp.
Ein Objekttyp kann in JASS++ folgendes sein:
\begin{enumerate}
\item ein Einheitentyp
\item ein Gegenstandstyp
\item ein Doodad-Typ
\item ein Zerstörbares-Typ
\item ein Fähigkeitentyp
\item ein Zauberverstärkertyp
\item ein Forschungstyp
\item ein Wettereffekttyp
\item ein Blitzeffekttyp
\item ein Geländeset
\end{enumerate}

Die Objekttypen stammen aus eine der folgenden Dateien (aufsteigend nach Priorität geordnet):
\begin{enumerate}
\item war3map.u
\item war3map.d
\item UI/UnitData.slk
\item *
\item *
\end{enumerate}
TODO

Dabei werden die angegebenen Dateien in den folgenden Archiven in absteigender Reihenfolge durchsucht:
\begin{enumerate}
\item Karte
\item Kampagne
\item War3Patch.mpq
\item War3Locale.mpq (richtigen Namen finden)
\item War3X.mpq
\item War3.mpq
\end{enumerate}
TODO

Die Eigenschaften eines Objekttyps können einen der folgenden nativen, kopierbasierten Typen haben:
\begin{enumerate}
\item string
\item integer
\item real
\item boolean
\end{enumerate}
TODO

Die Eigenschaften eines Objekttyps haben selbst nochmals eine eigene Id.
Ein Objekttypen-Eigenschafts-Literal wird immer als ein Ausdruck eines der angegebenen Typen interpretiert.
Kann der Objekttypeintrag bzw. der Objekttyp-Eigenschaftseintrag nicht gefunden werden, so gibt der Compiler eine Warnung aus.
Das Literal ist in diesem Fall ein Nullwert ("null").
Wird keine Eigenschafts-Id angegeben, so wird der Ausdruck als Array interpretiert. Dies geht jedoch nur falls alle Objekttypen-Eigenschaften
den gleichen nativen, kopierbasierten Typ haben.
Ansonsten kann das Literal nur als eigener Typ interpertiert werden (siehe "Klassen - Zuweisungen").
Die Notation sieht folgendermaßen aus:
<<Objekttyp-Id>[, %]>
<<Objekttyp-Id>,<Eigenschafts-Id>[, %])>

Das Prozentzeichen ist optional und gibt an, dass Fließkommazahlen als Prozentanteile und daher als Ganzzahlen zwischen 0 und 100 interpretiert werden
sollen. Aus 0.30 würde in diesem Fall also die Ganzzahl 30 werden.
Es gilt bei einer reinen Angabe der Objekttyp-Id für alle Eigenschaften.
Es gilt zu beachten, dass Objekttypen-Ids auch als Escape-Sequenzen verwendet werden können.

\section{Farbliterale}
Farbliterale haben stets den nativen Typ "integer". Sie sind daher ein 32-Bit-Wert, welcher eine RGBA-Farbe enthält.
Der RGBA-Wert muss hexadezimal angegeben werden, wobei die ersten beiden Ziffern für den Alphawert, die zweiten beiden für den Grünwert,
die dritten beiden für den Blauwert und die letzten beiden für den Blauwert stehen.
Notation:
|c<Farbe>|r

\section{Escape-Sequenzen}
Eine Escape-Sequenz ist ein Zeichen, welchem das \-Zeichen vorangestellt wird und welches ausschließlich ein einem
Zeichenkettenliteral vorkommen darf.
Es gibt folgende Escape-Sequenzen:
\begin(enumerate}
\item n				Zeilenumbruch
\item t				Horizontaler Tabulator
\item \				Backslash
\item "				Doppeltes Anführungszeichen
\end{enumerate}

Zusätzlich existieren spezielle Escape-Sequenzen für Objekttypen-Ids (siehe "Literale - Objekttypen-Ids") und Farben (siehe "Literale - Farbliterale").
Farbkodierungen und Objekttypen-Ids in Zeichenketten werden zum Kompilierzeitpunkt validiert bzw. kompiliert.
Die speziellen Escapesequenzen werden in den Zeichenketten zum Kompilierzeitpunkt ersetzt, insofern sie gültig sind.
Falls nicht, bleibt der Ausdruck erhalten und der Compiler gibt eine Warnung aus.
Wie bei normalen Objektypen-Literalen kann ein Prozentzeichen angehängt werden.
Bei Farben kann jedoch, anders als bei gewöhnlichen Farbliteralen, Text eingeschlossen werden. Dieser sollte dann in der entsprechenden Farbe
im Spiel bei einer Ausgabe angezeigt werden.

\section{Notation}
|c<Farbe>[<Text>]|r
<<Objekt-Id>,<Feld-Id>[, %]>

	\chapter{Literale}
Ein Literal ist ein sich selbst erklärender Ausdruck, der nicht speziell benannt werden muss also keinen Bezeichner benötigt.
Für folgende native Typen gelten folgende Literalnotationsregeln:

Kopierbasierte Typen:
TODO: Exponentialschreibweise für alle Ganzzahlausdrücke erlauben und auch für Fließkommazahlen?
\begin{enumerate}
\item string : "<Zeichenkettenliteral>" | null
\item integer : 0<Oktalliteral> | 0%<Binärliteral> | 0x<Hexadezimalliteral> | <Dezimalliteral>[e <Exponent>] | '<Id-Literal>' | null
\item real : <Fließkommaliteral> [r | R] | null
\item boolean : true | false | null
\item handle (und Nicht-agent-Kindtypen): null
\end{enumerate}

Referenzbasierte Typen:
\begin{enumerate}
\item code : 0
\item agent (und Kindtypen) : 0
\item Klassenobjekte : 0
\item Funktionszeiger : 0
\end{enumerate}

Dabei können Zeichenkettenausdrücke mehrfach hintereinander geschrieben werden. Diese werden dann automatisch zu einem einzelnen
zusammengefasst. Dies vereinfacht die Aufteilung auf mehrere Zeilen, ohne unnötige Zeilenumbrüche und Einrückungen dazwischen und
dient einer besseren Übersicht:
"<Ausdruckteil 1>"
"<Ausdruckteil n>"

\section{Nullwerte}
Wie oben dargestellt erhalten kopierbasierte Typen einen Nullwert mit dem Literal "null" und referenzbasierte Typen mit dem Literal "0".
Der Zugriff auf Nullwerte kann in manchen Fällen zu einem undefinierten Verhalten führen.
In der Regel ist er im Testmodus jedoch abgesichert und führt zu Fehlermeldungen (siehe "Klassen").

\section{Id-Literale}
Für Id-Literale gelten bei der Validierung selbige Regeln wie in der Sektion "Objekttypen-Literale" definiert. Es wird also
eine Warnung vom Compiler ausgegeben, falls sich die Id statisch überprüfen lässt und in keiner der entsprechenden Dateien
zu finden ist.

\section{Objekttypen-Literale}
Des Weiteren können Eigenschaften sogenannter Objekttypen, die einen bestimmten nativen, kopierbasierten Typ haben abgefragt werden.
Objekttypen erhalten in Warcraft 3 The Frozen Throne eine vierstellige, hexadezimale Id. Daher können Objekttypen im Wertebereich von 16^4 definiert werden.
Jede Id identifiziert genau einen Objekttyp.
Ein Objekttyp kann in JASS++ folgendes sein:
\begin{enumerate}
\item ein Einheitentyp
\item ein Gegenstandstyp
\item ein Doodad-Typ
\item ein Zerstörbares-Typ
\item ein Fähigkeitentyp
\item ein Zauberverstärkertyp
\item ein Forschungstyp
\item ein Wettereffekttyp
\item ein Blitzeffekttyp
\item ein Geländeset
\end{enumerate}

Die Objekttypen stammen aus eine der folgenden Dateien (aufsteigend nach Priorität geordnet):
\begin{enumerate}
\item war3map.u
\item war3map.d
\item UI/UnitData.slk
\item *
\item *
\end{enumerate}
TODO

Dabei werden die angegebenen Dateien in den folgenden Archiven in absteigender Reihenfolge durchsucht:
\begin{enumerate}
\item Karte
\item Kampagne
\item War3Patch.mpq
\item War3Locale.mpq (richtigen Namen finden)
\item War3X.mpq
\item War3.mpq
\end{enumerate}
TODO

Die Eigenschaften eines Objekttyps können einen der folgenden nativen, kopierbasierten Typen haben:
\begin{enumerate}
\item string
\item integer
\item real
\item boolean
\end{enumerate}
TODO

Die Eigenschaften eines Objekttyps haben selbst nochmals eine eigene Id.
Ein Objekttypen-Eigenschafts-Literal wird immer als ein Ausdruck eines der angegebenen Typen interpretiert.
Kann der Objekttypeintrag bzw. der Objekttyp-Eigenschaftseintrag nicht gefunden werden, so gibt der Compiler eine Warnung aus.
Das Literal ist in diesem Fall ein Nullwert ("null").
Wird keine Eigenschafts-Id angegeben, so wird der Ausdruck als Array interpretiert. Dies geht jedoch nur falls alle Objekttypen-Eigenschaften
den gleichen nativen, kopierbasierten Typ haben.
Ansonsten kann das Literal nur als eigener Typ interpertiert werden (siehe "Klassen - Zuweisungen").
Die Notation sieht folgendermaßen aus:
<<Objekttyp-Id>[, %]>
<<Objekttyp-Id>,<Eigenschafts-Id>[, %])>

Das Prozentzeichen ist optional und gibt an, dass Fließkommazahlen als Prozentanteile und daher als Ganzzahlen zwischen 0 und 100 interpretiert werden
sollen. Aus 0.30 würde in diesem Fall also die Ganzzahl 30 werden.
Es gilt bei einer reinen Angabe der Objekttyp-Id für alle Eigenschaften.
Es gilt zu beachten, dass Objekttypen-Ids auch als Escape-Sequenzen verwendet werden können.

\section{Farbliterale}
Farbliterale haben stets den nativen Typ "integer". Sie sind daher ein 32-Bit-Wert, welcher eine RGBA-Farbe enthält.
Der RGBA-Wert muss hexadezimal angegeben werden, wobei die ersten beiden Ziffern für den Alphawert, die zweiten beiden für den Grünwert,
die dritten beiden für den Blauwert und die letzten beiden für den Blauwert stehen.
Notation:
|c<Farbe>|r

\section{Escape-Sequenzen}
Eine Escape-Sequenz ist ein Zeichen, welchem das \-Zeichen vorangestellt wird und welches ausschließlich ein einem
Zeichenkettenliteral vorkommen darf.
Es gibt folgende Escape-Sequenzen:
\begin(enumerate}
\item n				Zeilenumbruch
\item t				Horizontaler Tabulator
\item \				Backslash
\item "				Doppeltes Anführungszeichen
\end{enumerate}

Zusätzlich existieren spezielle Escape-Sequenzen für Objekttypen-Ids (siehe "Literale - Objekttypen-Ids") und Farben (siehe "Literale - Farbliterale").
Farbkodierungen und Objekttypen-Ids in Zeichenketten werden zum Kompilierzeitpunkt validiert bzw. kompiliert.
Die speziellen Escapesequenzen werden in den Zeichenketten zum Kompilierzeitpunkt ersetzt, insofern sie gültig sind.
Falls nicht, bleibt der Ausdruck erhalten und der Compiler gibt eine Warnung aus.
Wie bei normalen Objektypen-Literalen kann ein Prozentzeichen angehängt werden.
Bei Farben kann jedoch, anders als bei gewöhnlichen Farbliteralen, Text eingeschlossen werden. Dieser sollte dann in der entsprechenden Farbe
im Spiel bei einer Ausgabe angezeigt werden.

\section{Notation}
|c<Farbe>[<Text>]|r
<<Objekt-Id>,<Feld-Id>[, %]>

	\chapter{Literale}
Ein Literal ist ein sich selbst erklärender Ausdruck, der nicht speziell benannt werden muss also keinen Bezeichner benötigt.
Für folgende native Typen gelten folgende Literalnotationsregeln:

Kopierbasierte Typen:
TODO: Exponentialschreibweise für alle Ganzzahlausdrücke erlauben und auch für Fließkommazahlen?
\begin{enumerate}
\item string : "<Zeichenkettenliteral>" | null
\item integer : 0<Oktalliteral> | 0%<Binärliteral> | 0x<Hexadezimalliteral> | <Dezimalliteral>[e <Exponent>] | '<Id-Literal>' | null
\item real : <Fließkommaliteral> [r | R] | null
\item boolean : true | false | null
\item handle (und Nicht-agent-Kindtypen): null
\end{enumerate}

Referenzbasierte Typen:
\begin{enumerate}
\item code : 0
\item agent (und Kindtypen) : 0
\item Klassenobjekte : 0
\item Funktionszeiger : 0
\end{enumerate}

Dabei können Zeichenkettenausdrücke mehrfach hintereinander geschrieben werden. Diese werden dann automatisch zu einem einzelnen
zusammengefasst. Dies vereinfacht die Aufteilung auf mehrere Zeilen, ohne unnötige Zeilenumbrüche und Einrückungen dazwischen und
dient einer besseren Übersicht:
"<Ausdruckteil 1>"
"<Ausdruckteil n>"

\section{Nullwerte}
Wie oben dargestellt erhalten kopierbasierte Typen einen Nullwert mit dem Literal "null" und referenzbasierte Typen mit dem Literal "0".
Der Zugriff auf Nullwerte kann in manchen Fällen zu einem undefinierten Verhalten führen.
In der Regel ist er im Testmodus jedoch abgesichert und führt zu Fehlermeldungen (siehe "Klassen").

\section{Id-Literale}
Für Id-Literale gelten bei der Validierung selbige Regeln wie in der Sektion "Objekttypen-Literale" definiert. Es wird also
eine Warnung vom Compiler ausgegeben, falls sich die Id statisch überprüfen lässt und in keiner der entsprechenden Dateien
zu finden ist.

\section{Objekttypen-Literale}
Des Weiteren können Eigenschaften sogenannter Objekttypen, die einen bestimmten nativen, kopierbasierten Typ haben abgefragt werden.
Objekttypen erhalten in Warcraft 3 The Frozen Throne eine vierstellige, hexadezimale Id. Daher können Objekttypen im Wertebereich von 16^4 definiert werden.
Jede Id identifiziert genau einen Objekttyp.
Ein Objekttyp kann in JASS++ folgendes sein:
\begin{enumerate}
\item ein Einheitentyp
\item ein Gegenstandstyp
\item ein Doodad-Typ
\item ein Zerstörbares-Typ
\item ein Fähigkeitentyp
\item ein Zauberverstärkertyp
\item ein Forschungstyp
\item ein Wettereffekttyp
\item ein Blitzeffekttyp
\item ein Geländeset
\end{enumerate}

Die Objekttypen stammen aus eine der folgenden Dateien (aufsteigend nach Priorität geordnet):
\begin{enumerate}
\item war3map.u
\item war3map.d
\item UI/UnitData.slk
\item *
\item *
\end{enumerate}
TODO

Dabei werden die angegebenen Dateien in den folgenden Archiven in absteigender Reihenfolge durchsucht:
\begin{enumerate}
\item Karte
\item Kampagne
\item War3Patch.mpq
\item War3Locale.mpq (richtigen Namen finden)
\item War3X.mpq
\item War3.mpq
\end{enumerate}
TODO

Die Eigenschaften eines Objekttyps können einen der folgenden nativen, kopierbasierten Typen haben:
\begin{enumerate}
\item string
\item integer
\item real
\item boolean
\end{enumerate}
TODO

Die Eigenschaften eines Objekttyps haben selbst nochmals eine eigene Id.
Ein Objekttypen-Eigenschafts-Literal wird immer als ein Ausdruck eines der angegebenen Typen interpretiert.
Kann der Objekttypeintrag bzw. der Objekttyp-Eigenschaftseintrag nicht gefunden werden, so gibt der Compiler eine Warnung aus.
Das Literal ist in diesem Fall ein Nullwert ("null").
Wird keine Eigenschafts-Id angegeben, so wird der Ausdruck als Array interpretiert. Dies geht jedoch nur falls alle Objekttypen-Eigenschaften
den gleichen nativen, kopierbasierten Typ haben.
Ansonsten kann das Literal nur als eigener Typ interpertiert werden (siehe "Klassen - Zuweisungen").
Die Notation sieht folgendermaßen aus:
<<Objekttyp-Id>[, %]>
<<Objekttyp-Id>,<Eigenschafts-Id>[, %])>

Das Prozentzeichen ist optional und gibt an, dass Fließkommazahlen als Prozentanteile und daher als Ganzzahlen zwischen 0 und 100 interpretiert werden
sollen. Aus 0.30 würde in diesem Fall also die Ganzzahl 30 werden.
Es gilt bei einer reinen Angabe der Objekttyp-Id für alle Eigenschaften.
Es gilt zu beachten, dass Objekttypen-Ids auch als Escape-Sequenzen verwendet werden können.

\section{Farbliterale}
Farbliterale haben stets den nativen Typ "integer". Sie sind daher ein 32-Bit-Wert, welcher eine RGBA-Farbe enthält.
Der RGBA-Wert muss hexadezimal angegeben werden, wobei die ersten beiden Ziffern für den Alphawert, die zweiten beiden für den Grünwert,
die dritten beiden für den Blauwert und die letzten beiden für den Blauwert stehen.
Notation:
|c<Farbe>|r

\section{Escape-Sequenzen}
Eine Escape-Sequenz ist ein Zeichen, welchem das \-Zeichen vorangestellt wird und welches ausschließlich ein einem
Zeichenkettenliteral vorkommen darf.
Es gibt folgende Escape-Sequenzen:
\begin(enumerate}
\item n				Zeilenumbruch
\item t				Horizontaler Tabulator
\item \				Backslash
\item "				Doppeltes Anführungszeichen
\end{enumerate}

Zusätzlich existieren spezielle Escape-Sequenzen für Objekttypen-Ids (siehe "Literale - Objekttypen-Ids") und Farben (siehe "Literale - Farbliterale").
Farbkodierungen und Objekttypen-Ids in Zeichenketten werden zum Kompilierzeitpunkt validiert bzw. kompiliert.
Die speziellen Escapesequenzen werden in den Zeichenketten zum Kompilierzeitpunkt ersetzt, insofern sie gültig sind.
Falls nicht, bleibt der Ausdruck erhalten und der Compiler gibt eine Warnung aus.
Wie bei normalen Objektypen-Literalen kann ein Prozentzeichen angehängt werden.
Bei Farben kann jedoch, anders als bei gewöhnlichen Farbliteralen, Text eingeschlossen werden. Dieser sollte dann in der entsprechenden Farbe
im Spiel bei einer Ausgabe angezeigt werden.

\section{Notation}
|c<Farbe>[<Text>]|r
<<Objekt-Id>,<Feld-Id>[, %]>

	\chapter{Literale}
Ein Literal ist ein sich selbst erklärender Ausdruck, der nicht speziell benannt werden muss also keinen Bezeichner benötigt.
Für folgende native Typen gelten folgende Literalnotationsregeln:

Kopierbasierte Typen:
TODO: Exponentialschreibweise für alle Ganzzahlausdrücke erlauben und auch für Fließkommazahlen?
\begin{enumerate}
\item string : "<Zeichenkettenliteral>" | null
\item integer : 0<Oktalliteral> | 0%<Binärliteral> | 0x<Hexadezimalliteral> | <Dezimalliteral>[e <Exponent>] | '<Id-Literal>' | null
\item real : <Fließkommaliteral> [r | R] | null
\item boolean : true | false | null
\item handle (und Nicht-agent-Kindtypen): null
\end{enumerate}

Referenzbasierte Typen:
\begin{enumerate}
\item code : 0
\item agent (und Kindtypen) : 0
\item Klassenobjekte : 0
\item Funktionszeiger : 0
\end{enumerate}

Dabei können Zeichenkettenausdrücke mehrfach hintereinander geschrieben werden. Diese werden dann automatisch zu einem einzelnen
zusammengefasst. Dies vereinfacht die Aufteilung auf mehrere Zeilen, ohne unnötige Zeilenumbrüche und Einrückungen dazwischen und
dient einer besseren Übersicht:
"<Ausdruckteil 1>"
"<Ausdruckteil n>"

\section{Nullwerte}
Wie oben dargestellt erhalten kopierbasierte Typen einen Nullwert mit dem Literal "null" und referenzbasierte Typen mit dem Literal "0".
Der Zugriff auf Nullwerte kann in manchen Fällen zu einem undefinierten Verhalten führen.
In der Regel ist er im Testmodus jedoch abgesichert und führt zu Fehlermeldungen (siehe "Klassen").

\section{Id-Literale}
Für Id-Literale gelten bei der Validierung selbige Regeln wie in der Sektion "Objekttypen-Literale" definiert. Es wird also
eine Warnung vom Compiler ausgegeben, falls sich die Id statisch überprüfen lässt und in keiner der entsprechenden Dateien
zu finden ist.

\section{Objekttypen-Literale}
Des Weiteren können Eigenschaften sogenannter Objekttypen, die einen bestimmten nativen, kopierbasierten Typ haben abgefragt werden.
Objekttypen erhalten in Warcraft 3 The Frozen Throne eine vierstellige, hexadezimale Id. Daher können Objekttypen im Wertebereich von 16^4 definiert werden.
Jede Id identifiziert genau einen Objekttyp.
Ein Objekttyp kann in JASS++ folgendes sein:
\begin{enumerate}
\item ein Einheitentyp
\item ein Gegenstandstyp
\item ein Doodad-Typ
\item ein Zerstörbares-Typ
\item ein Fähigkeitentyp
\item ein Zauberverstärkertyp
\item ein Forschungstyp
\item ein Wettereffekttyp
\item ein Blitzeffekttyp
\item ein Geländeset
\end{enumerate}

Die Objekttypen stammen aus eine der folgenden Dateien (aufsteigend nach Priorität geordnet):
\begin{enumerate}
\item war3map.u
\item war3map.d
\item UI/UnitData.slk
\item *
\item *
\end{enumerate}
TODO

Dabei werden die angegebenen Dateien in den folgenden Archiven in absteigender Reihenfolge durchsucht:
\begin{enumerate}
\item Karte
\item Kampagne
\item War3Patch.mpq
\item War3Locale.mpq (richtigen Namen finden)
\item War3X.mpq
\item War3.mpq
\end{enumerate}
TODO

Die Eigenschaften eines Objekttyps können einen der folgenden nativen, kopierbasierten Typen haben:
\begin{enumerate}
\item string
\item integer
\item real
\item boolean
\end{enumerate}
TODO

Die Eigenschaften eines Objekttyps haben selbst nochmals eine eigene Id.
Ein Objekttypen-Eigenschafts-Literal wird immer als ein Ausdruck eines der angegebenen Typen interpretiert.
Kann der Objekttypeintrag bzw. der Objekttyp-Eigenschaftseintrag nicht gefunden werden, so gibt der Compiler eine Warnung aus.
Das Literal ist in diesem Fall ein Nullwert ("null").
Wird keine Eigenschafts-Id angegeben, so wird der Ausdruck als Array interpretiert. Dies geht jedoch nur falls alle Objekttypen-Eigenschaften
den gleichen nativen, kopierbasierten Typ haben.
Ansonsten kann das Literal nur als eigener Typ interpertiert werden (siehe "Klassen - Zuweisungen").
Die Notation sieht folgendermaßen aus:
<<Objekttyp-Id>[, %]>
<<Objekttyp-Id>,<Eigenschafts-Id>[, %])>

Das Prozentzeichen ist optional und gibt an, dass Fließkommazahlen als Prozentanteile und daher als Ganzzahlen zwischen 0 und 100 interpretiert werden
sollen. Aus 0.30 würde in diesem Fall also die Ganzzahl 30 werden.
Es gilt bei einer reinen Angabe der Objekttyp-Id für alle Eigenschaften.
Es gilt zu beachten, dass Objekttypen-Ids auch als Escape-Sequenzen verwendet werden können.

\section{Farbliterale}
Farbliterale haben stets den nativen Typ "integer". Sie sind daher ein 32-Bit-Wert, welcher eine RGBA-Farbe enthält.
Der RGBA-Wert muss hexadezimal angegeben werden, wobei die ersten beiden Ziffern für den Alphawert, die zweiten beiden für den Grünwert,
die dritten beiden für den Blauwert und die letzten beiden für den Blauwert stehen.
Notation:
|c<Farbe>|r

\section{Escape-Sequenzen}
Eine Escape-Sequenz ist ein Zeichen, welchem das \-Zeichen vorangestellt wird und welches ausschließlich ein einem
Zeichenkettenliteral vorkommen darf.
Es gibt folgende Escape-Sequenzen:
\begin(enumerate}
\item n				Zeilenumbruch
\item t				Horizontaler Tabulator
\item \				Backslash
\item "				Doppeltes Anführungszeichen
\end{enumerate}

Zusätzlich existieren spezielle Escape-Sequenzen für Objekttypen-Ids (siehe "Literale - Objekttypen-Ids") und Farben (siehe "Literale - Farbliterale").
Farbkodierungen und Objekttypen-Ids in Zeichenketten werden zum Kompilierzeitpunkt validiert bzw. kompiliert.
Die speziellen Escapesequenzen werden in den Zeichenketten zum Kompilierzeitpunkt ersetzt, insofern sie gültig sind.
Falls nicht, bleibt der Ausdruck erhalten und der Compiler gibt eine Warnung aus.
Wie bei normalen Objektypen-Literalen kann ein Prozentzeichen angehängt werden.
Bei Farben kann jedoch, anders als bei gewöhnlichen Farbliteralen, Text eingeschlossen werden. Dieser sollte dann in der entsprechenden Farbe
im Spiel bei einer Ausgabe angezeigt werden.

\section{Notation}
|c<Farbe>[<Text>]|r
<<Objekt-Id>,<Feld-Id>[, %]>

	\chapter{Literale}
Ein Literal ist ein sich selbst erklärender Ausdruck, der nicht speziell benannt werden muss also keinen Bezeichner benötigt.
Für folgende native Typen gelten folgende Literalnotationsregeln:

Kopierbasierte Typen:
TODO: Exponentialschreibweise für alle Ganzzahlausdrücke erlauben und auch für Fließkommazahlen?
\begin{enumerate}
\item string : "<Zeichenkettenliteral>" | null
\item integer : 0<Oktalliteral> | 0%<Binärliteral> | 0x<Hexadezimalliteral> | <Dezimalliteral>[e <Exponent>] | '<Id-Literal>' | null
\item real : <Fließkommaliteral> [r | R] | null
\item boolean : true | false | null
\item handle (und Nicht-agent-Kindtypen): null
\end{enumerate}

Referenzbasierte Typen:
\begin{enumerate}
\item code : 0
\item agent (und Kindtypen) : 0
\item Klassenobjekte : 0
\item Funktionszeiger : 0
\end{enumerate}

Dabei können Zeichenkettenausdrücke mehrfach hintereinander geschrieben werden. Diese werden dann automatisch zu einem einzelnen
zusammengefasst. Dies vereinfacht die Aufteilung auf mehrere Zeilen, ohne unnötige Zeilenumbrüche und Einrückungen dazwischen und
dient einer besseren Übersicht:
"<Ausdruckteil 1>"
"<Ausdruckteil n>"

\section{Nullwerte}
Wie oben dargestellt erhalten kopierbasierte Typen einen Nullwert mit dem Literal "null" und referenzbasierte Typen mit dem Literal "0".
Der Zugriff auf Nullwerte kann in manchen Fällen zu einem undefinierten Verhalten führen.
In der Regel ist er im Testmodus jedoch abgesichert und führt zu Fehlermeldungen (siehe "Klassen").

\section{Id-Literale}
Für Id-Literale gelten bei der Validierung selbige Regeln wie in der Sektion "Objekttypen-Literale" definiert. Es wird also
eine Warnung vom Compiler ausgegeben, falls sich die Id statisch überprüfen lässt und in keiner der entsprechenden Dateien
zu finden ist.

\section{Objekttypen-Literale}
Des Weiteren können Eigenschaften sogenannter Objekttypen, die einen bestimmten nativen, kopierbasierten Typ haben abgefragt werden.
Objekttypen erhalten in Warcraft 3 The Frozen Throne eine vierstellige, hexadezimale Id. Daher können Objekttypen im Wertebereich von 16^4 definiert werden.
Jede Id identifiziert genau einen Objekttyp.
Ein Objekttyp kann in JASS++ folgendes sein:
\begin{enumerate}
\item ein Einheitentyp
\item ein Gegenstandstyp
\item ein Doodad-Typ
\item ein Zerstörbares-Typ
\item ein Fähigkeitentyp
\item ein Zauberverstärkertyp
\item ein Forschungstyp
\item ein Wettereffekttyp
\item ein Blitzeffekttyp
\item ein Geländeset
\end{enumerate}

Die Objekttypen stammen aus eine der folgenden Dateien (aufsteigend nach Priorität geordnet):
\begin{enumerate}
\item war3map.u
\item war3map.d
\item UI/UnitData.slk
\item *
\item *
\end{enumerate}
TODO

Dabei werden die angegebenen Dateien in den folgenden Archiven in absteigender Reihenfolge durchsucht:
\begin{enumerate}
\item Karte
\item Kampagne
\item War3Patch.mpq
\item War3Locale.mpq (richtigen Namen finden)
\item War3X.mpq
\item War3.mpq
\end{enumerate}
TODO

Die Eigenschaften eines Objekttyps können einen der folgenden nativen, kopierbasierten Typen haben:
\begin{enumerate}
\item string
\item integer
\item real
\item boolean
\end{enumerate}
TODO

Die Eigenschaften eines Objekttyps haben selbst nochmals eine eigene Id.
Ein Objekttypen-Eigenschafts-Literal wird immer als ein Ausdruck eines der angegebenen Typen interpretiert.
Kann der Objekttypeintrag bzw. der Objekttyp-Eigenschaftseintrag nicht gefunden werden, so gibt der Compiler eine Warnung aus.
Das Literal ist in diesem Fall ein Nullwert ("null").
Wird keine Eigenschafts-Id angegeben, so wird der Ausdruck als Array interpretiert. Dies geht jedoch nur falls alle Objekttypen-Eigenschaften
den gleichen nativen, kopierbasierten Typ haben.
Ansonsten kann das Literal nur als eigener Typ interpertiert werden (siehe "Klassen - Zuweisungen").
Die Notation sieht folgendermaßen aus:
<<Objekttyp-Id>[, %]>
<<Objekttyp-Id>,<Eigenschafts-Id>[, %])>

Das Prozentzeichen ist optional und gibt an, dass Fließkommazahlen als Prozentanteile und daher als Ganzzahlen zwischen 0 und 100 interpretiert werden
sollen. Aus 0.30 würde in diesem Fall also die Ganzzahl 30 werden.
Es gilt bei einer reinen Angabe der Objekttyp-Id für alle Eigenschaften.
Es gilt zu beachten, dass Objekttypen-Ids auch als Escape-Sequenzen verwendet werden können.

\section{Farbliterale}
Farbliterale haben stets den nativen Typ "integer". Sie sind daher ein 32-Bit-Wert, welcher eine RGBA-Farbe enthält.
Der RGBA-Wert muss hexadezimal angegeben werden, wobei die ersten beiden Ziffern für den Alphawert, die zweiten beiden für den Grünwert,
die dritten beiden für den Blauwert und die letzten beiden für den Blauwert stehen.
Notation:
|c<Farbe>|r

\section{Escape-Sequenzen}
Eine Escape-Sequenz ist ein Zeichen, welchem das \-Zeichen vorangestellt wird und welches ausschließlich ein einem
Zeichenkettenliteral vorkommen darf.
Es gibt folgende Escape-Sequenzen:
\begin(enumerate}
\item n				Zeilenumbruch
\item t				Horizontaler Tabulator
\item \				Backslash
\item "				Doppeltes Anführungszeichen
\end{enumerate}

Zusätzlich existieren spezielle Escape-Sequenzen für Objekttypen-Ids (siehe "Literale - Objekttypen-Ids") und Farben (siehe "Literale - Farbliterale").
Farbkodierungen und Objekttypen-Ids in Zeichenketten werden zum Kompilierzeitpunkt validiert bzw. kompiliert.
Die speziellen Escapesequenzen werden in den Zeichenketten zum Kompilierzeitpunkt ersetzt, insofern sie gültig sind.
Falls nicht, bleibt der Ausdruck erhalten und der Compiler gibt eine Warnung aus.
Wie bei normalen Objektypen-Literalen kann ein Prozentzeichen angehängt werden.
Bei Farben kann jedoch, anders als bei gewöhnlichen Farbliteralen, Text eingeschlossen werden. Dieser sollte dann in der entsprechenden Farbe
im Spiel bei einer Ausgabe angezeigt werden.

\section{Notation}
|c<Farbe>[<Text>]|r
<<Objekt-Id>,<Feld-Id>[, %]>

	\chapter{Literale}
Ein Literal ist ein sich selbst erklärender Ausdruck, der nicht speziell benannt werden muss also keinen Bezeichner benötigt.
Für folgende native Typen gelten folgende Literalnotationsregeln:

Kopierbasierte Typen:
TODO: Exponentialschreibweise für alle Ganzzahlausdrücke erlauben und auch für Fließkommazahlen?
\begin{enumerate}
\item string : "<Zeichenkettenliteral>" | null
\item integer : 0<Oktalliteral> | 0%<Binärliteral> | 0x<Hexadezimalliteral> | <Dezimalliteral>[e <Exponent>] | '<Id-Literal>' | null
\item real : <Fließkommaliteral> [r | R] | null
\item boolean : true | false | null
\item handle (und Nicht-agent-Kindtypen): null
\end{enumerate}

Referenzbasierte Typen:
\begin{enumerate}
\item code : 0
\item agent (und Kindtypen) : 0
\item Klassenobjekte : 0
\item Funktionszeiger : 0
\end{enumerate}

Dabei können Zeichenkettenausdrücke mehrfach hintereinander geschrieben werden. Diese werden dann automatisch zu einem einzelnen
zusammengefasst. Dies vereinfacht die Aufteilung auf mehrere Zeilen, ohne unnötige Zeilenumbrüche und Einrückungen dazwischen und
dient einer besseren Übersicht:
"<Ausdruckteil 1>"
"<Ausdruckteil n>"

\section{Nullwerte}
Wie oben dargestellt erhalten kopierbasierte Typen einen Nullwert mit dem Literal "null" und referenzbasierte Typen mit dem Literal "0".
Der Zugriff auf Nullwerte kann in manchen Fällen zu einem undefinierten Verhalten führen.
In der Regel ist er im Testmodus jedoch abgesichert und führt zu Fehlermeldungen (siehe "Klassen").

\section{Id-Literale}
Für Id-Literale gelten bei der Validierung selbige Regeln wie in der Sektion "Objekttypen-Literale" definiert. Es wird also
eine Warnung vom Compiler ausgegeben, falls sich die Id statisch überprüfen lässt und in keiner der entsprechenden Dateien
zu finden ist.

\section{Objekttypen-Literale}
Des Weiteren können Eigenschaften sogenannter Objekttypen, die einen bestimmten nativen, kopierbasierten Typ haben abgefragt werden.
Objekttypen erhalten in Warcraft 3 The Frozen Throne eine vierstellige, hexadezimale Id. Daher können Objekttypen im Wertebereich von 16^4 definiert werden.
Jede Id identifiziert genau einen Objekttyp.
Ein Objekttyp kann in JASS++ folgendes sein:
\begin{enumerate}
\item ein Einheitentyp
\item ein Gegenstandstyp
\item ein Doodad-Typ
\item ein Zerstörbares-Typ
\item ein Fähigkeitentyp
\item ein Zauberverstärkertyp
\item ein Forschungstyp
\item ein Wettereffekttyp
\item ein Blitzeffekttyp
\item ein Geländeset
\end{enumerate}

Die Objekttypen stammen aus eine der folgenden Dateien (aufsteigend nach Priorität geordnet):
\begin{enumerate}
\item war3map.u
\item war3map.d
\item UI/UnitData.slk
\item *
\item *
\end{enumerate}
TODO

Dabei werden die angegebenen Dateien in den folgenden Archiven in absteigender Reihenfolge durchsucht:
\begin{enumerate}
\item Karte
\item Kampagne
\item War3Patch.mpq
\item War3Locale.mpq (richtigen Namen finden)
\item War3X.mpq
\item War3.mpq
\end{enumerate}
TODO

Die Eigenschaften eines Objekttyps können einen der folgenden nativen, kopierbasierten Typen haben:
\begin{enumerate}
\item string
\item integer
\item real
\item boolean
\end{enumerate}
TODO

Die Eigenschaften eines Objekttyps haben selbst nochmals eine eigene Id.
Ein Objekttypen-Eigenschafts-Literal wird immer als ein Ausdruck eines der angegebenen Typen interpretiert.
Kann der Objekttypeintrag bzw. der Objekttyp-Eigenschaftseintrag nicht gefunden werden, so gibt der Compiler eine Warnung aus.
Das Literal ist in diesem Fall ein Nullwert ("null").
Wird keine Eigenschafts-Id angegeben, so wird der Ausdruck als Array interpretiert. Dies geht jedoch nur falls alle Objekttypen-Eigenschaften
den gleichen nativen, kopierbasierten Typ haben.
Ansonsten kann das Literal nur als eigener Typ interpertiert werden (siehe "Klassen - Zuweisungen").
Die Notation sieht folgendermaßen aus:
<<Objekttyp-Id>[, %]>
<<Objekttyp-Id>,<Eigenschafts-Id>[, %])>

Das Prozentzeichen ist optional und gibt an, dass Fließkommazahlen als Prozentanteile und daher als Ganzzahlen zwischen 0 und 100 interpretiert werden
sollen. Aus 0.30 würde in diesem Fall also die Ganzzahl 30 werden.
Es gilt bei einer reinen Angabe der Objekttyp-Id für alle Eigenschaften.
Es gilt zu beachten, dass Objekttypen-Ids auch als Escape-Sequenzen verwendet werden können.

\section{Farbliterale}
Farbliterale haben stets den nativen Typ "integer". Sie sind daher ein 32-Bit-Wert, welcher eine RGBA-Farbe enthält.
Der RGBA-Wert muss hexadezimal angegeben werden, wobei die ersten beiden Ziffern für den Alphawert, die zweiten beiden für den Grünwert,
die dritten beiden für den Blauwert und die letzten beiden für den Blauwert stehen.
Notation:
|c<Farbe>|r

\section{Escape-Sequenzen}
Eine Escape-Sequenz ist ein Zeichen, welchem das \-Zeichen vorangestellt wird und welches ausschließlich ein einem
Zeichenkettenliteral vorkommen darf.
Es gibt folgende Escape-Sequenzen:
\begin(enumerate}
\item n				Zeilenumbruch
\item t				Horizontaler Tabulator
\item \				Backslash
\item "				Doppeltes Anführungszeichen
\end{enumerate}

Zusätzlich existieren spezielle Escape-Sequenzen für Objekttypen-Ids (siehe "Literale - Objekttypen-Ids") und Farben (siehe "Literale - Farbliterale").
Farbkodierungen und Objekttypen-Ids in Zeichenketten werden zum Kompilierzeitpunkt validiert bzw. kompiliert.
Die speziellen Escapesequenzen werden in den Zeichenketten zum Kompilierzeitpunkt ersetzt, insofern sie gültig sind.
Falls nicht, bleibt der Ausdruck erhalten und der Compiler gibt eine Warnung aus.
Wie bei normalen Objektypen-Literalen kann ein Prozentzeichen angehängt werden.
Bei Farben kann jedoch, anders als bei gewöhnlichen Farbliteralen, Text eingeschlossen werden. Dieser sollte dann in der entsprechenden Farbe
im Spiel bei einer Ausgabe angezeigt werden.

\section{Notation}
|c<Farbe>[<Text>]|r
<<Objekt-Id>,<Feld-Id>[, %]>

	\chapter{Literale}
Ein Literal ist ein sich selbst erklärender Ausdruck, der nicht speziell benannt werden muss also keinen Bezeichner benötigt.
Für folgende native Typen gelten folgende Literalnotationsregeln:

Kopierbasierte Typen:
TODO: Exponentialschreibweise für alle Ganzzahlausdrücke erlauben und auch für Fließkommazahlen?
\begin{enumerate}
\item string : "<Zeichenkettenliteral>" | null
\item integer : 0<Oktalliteral> | 0%<Binärliteral> | 0x<Hexadezimalliteral> | <Dezimalliteral>[e <Exponent>] | '<Id-Literal>' | null
\item real : <Fließkommaliteral> [r | R] | null
\item boolean : true | false | null
\item handle (und Nicht-agent-Kindtypen): null
\end{enumerate}

Referenzbasierte Typen:
\begin{enumerate}
\item code : 0
\item agent (und Kindtypen) : 0
\item Klassenobjekte : 0
\item Funktionszeiger : 0
\end{enumerate}

Dabei können Zeichenkettenausdrücke mehrfach hintereinander geschrieben werden. Diese werden dann automatisch zu einem einzelnen
zusammengefasst. Dies vereinfacht die Aufteilung auf mehrere Zeilen, ohne unnötige Zeilenumbrüche und Einrückungen dazwischen und
dient einer besseren Übersicht:
"<Ausdruckteil 1>"
"<Ausdruckteil n>"

\section{Nullwerte}
Wie oben dargestellt erhalten kopierbasierte Typen einen Nullwert mit dem Literal "null" und referenzbasierte Typen mit dem Literal "0".
Der Zugriff auf Nullwerte kann in manchen Fällen zu einem undefinierten Verhalten führen.
In der Regel ist er im Testmodus jedoch abgesichert und führt zu Fehlermeldungen (siehe "Klassen").

\section{Id-Literale}
Für Id-Literale gelten bei der Validierung selbige Regeln wie in der Sektion "Objekttypen-Literale" definiert. Es wird also
eine Warnung vom Compiler ausgegeben, falls sich die Id statisch überprüfen lässt und in keiner der entsprechenden Dateien
zu finden ist.

\section{Objekttypen-Literale}
Des Weiteren können Eigenschaften sogenannter Objekttypen, die einen bestimmten nativen, kopierbasierten Typ haben abgefragt werden.
Objekttypen erhalten in Warcraft 3 The Frozen Throne eine vierstellige, hexadezimale Id. Daher können Objekttypen im Wertebereich von 16^4 definiert werden.
Jede Id identifiziert genau einen Objekttyp.
Ein Objekttyp kann in JASS++ folgendes sein:
\begin{enumerate}
\item ein Einheitentyp
\item ein Gegenstandstyp
\item ein Doodad-Typ
\item ein Zerstörbares-Typ
\item ein Fähigkeitentyp
\item ein Zauberverstärkertyp
\item ein Forschungstyp
\item ein Wettereffekttyp
\item ein Blitzeffekttyp
\item ein Geländeset
\end{enumerate}

Die Objekttypen stammen aus eine der folgenden Dateien (aufsteigend nach Priorität geordnet):
\begin{enumerate}
\item war3map.u
\item war3map.d
\item UI/UnitData.slk
\item *
\item *
\end{enumerate}
TODO

Dabei werden die angegebenen Dateien in den folgenden Archiven in absteigender Reihenfolge durchsucht:
\begin{enumerate}
\item Karte
\item Kampagne
\item War3Patch.mpq
\item War3Locale.mpq (richtigen Namen finden)
\item War3X.mpq
\item War3.mpq
\end{enumerate}
TODO

Die Eigenschaften eines Objekttyps können einen der folgenden nativen, kopierbasierten Typen haben:
\begin{enumerate}
\item string
\item integer
\item real
\item boolean
\end{enumerate}
TODO

Die Eigenschaften eines Objekttyps haben selbst nochmals eine eigene Id.
Ein Objekttypen-Eigenschafts-Literal wird immer als ein Ausdruck eines der angegebenen Typen interpretiert.
Kann der Objekttypeintrag bzw. der Objekttyp-Eigenschaftseintrag nicht gefunden werden, so gibt der Compiler eine Warnung aus.
Das Literal ist in diesem Fall ein Nullwert ("null").
Wird keine Eigenschafts-Id angegeben, so wird der Ausdruck als Array interpretiert. Dies geht jedoch nur falls alle Objekttypen-Eigenschaften
den gleichen nativen, kopierbasierten Typ haben.
Ansonsten kann das Literal nur als eigener Typ interpertiert werden (siehe "Klassen - Zuweisungen").
Die Notation sieht folgendermaßen aus:
<<Objekttyp-Id>[, %]>
<<Objekttyp-Id>,<Eigenschafts-Id>[, %])>

Das Prozentzeichen ist optional und gibt an, dass Fließkommazahlen als Prozentanteile und daher als Ganzzahlen zwischen 0 und 100 interpretiert werden
sollen. Aus 0.30 würde in diesem Fall also die Ganzzahl 30 werden.
Es gilt bei einer reinen Angabe der Objekttyp-Id für alle Eigenschaften.
Es gilt zu beachten, dass Objekttypen-Ids auch als Escape-Sequenzen verwendet werden können.

\section{Farbliterale}
Farbliterale haben stets den nativen Typ "integer". Sie sind daher ein 32-Bit-Wert, welcher eine RGBA-Farbe enthält.
Der RGBA-Wert muss hexadezimal angegeben werden, wobei die ersten beiden Ziffern für den Alphawert, die zweiten beiden für den Grünwert,
die dritten beiden für den Blauwert und die letzten beiden für den Blauwert stehen.
Notation:
|c<Farbe>|r

\section{Escape-Sequenzen}
Eine Escape-Sequenz ist ein Zeichen, welchem das \-Zeichen vorangestellt wird und welches ausschließlich ein einem
Zeichenkettenliteral vorkommen darf.
Es gibt folgende Escape-Sequenzen:
\begin(enumerate}
\item n				Zeilenumbruch
\item t				Horizontaler Tabulator
\item \				Backslash
\item "				Doppeltes Anführungszeichen
\end{enumerate}

Zusätzlich existieren spezielle Escape-Sequenzen für Objekttypen-Ids (siehe "Literale - Objekttypen-Ids") und Farben (siehe "Literale - Farbliterale").
Farbkodierungen und Objekttypen-Ids in Zeichenketten werden zum Kompilierzeitpunkt validiert bzw. kompiliert.
Die speziellen Escapesequenzen werden in den Zeichenketten zum Kompilierzeitpunkt ersetzt, insofern sie gültig sind.
Falls nicht, bleibt der Ausdruck erhalten und der Compiler gibt eine Warnung aus.
Wie bei normalen Objektypen-Literalen kann ein Prozentzeichen angehängt werden.
Bei Farben kann jedoch, anders als bei gewöhnlichen Farbliteralen, Text eingeschlossen werden. Dieser sollte dann in der entsprechenden Farbe
im Spiel bei einer Ausgabe angezeigt werden.

\section{Notation}
|c<Farbe>[<Text>]|r
<<Objekt-Id>,<Feld-Id>[, %]>


	\chapter{Literale}
Ein Literal ist ein sich selbst erklärender Ausdruck, der nicht speziell benannt werden muss also keinen Bezeichner benötigt.
Für folgende native Typen gelten folgende Literalnotationsregeln:

Kopierbasierte Typen:
TODO: Exponentialschreibweise für alle Ganzzahlausdrücke erlauben und auch für Fließkommazahlen?
\begin{enumerate}
\item string : "<Zeichenkettenliteral>" | null
\item integer : 0<Oktalliteral> | 0%<Binärliteral> | 0x<Hexadezimalliteral> | <Dezimalliteral>[e <Exponent>] | '<Id-Literal>' | null
\item real : <Fließkommaliteral> [r | R] | null
\item boolean : true | false | null
\item handle (und Nicht-agent-Kindtypen): null
\end{enumerate}

Referenzbasierte Typen:
\begin{enumerate}
\item code : 0
\item agent (und Kindtypen) : 0
\item Klassenobjekte : 0
\item Funktionszeiger : 0
\end{enumerate}

Dabei können Zeichenkettenausdrücke mehrfach hintereinander geschrieben werden. Diese werden dann automatisch zu einem einzelnen
zusammengefasst. Dies vereinfacht die Aufteilung auf mehrere Zeilen, ohne unnötige Zeilenumbrüche und Einrückungen dazwischen und
dient einer besseren Übersicht:
"<Ausdruckteil 1>"
"<Ausdruckteil n>"

\section{Nullwerte}
Wie oben dargestellt erhalten kopierbasierte Typen einen Nullwert mit dem Literal "null" und referenzbasierte Typen mit dem Literal "0".
Der Zugriff auf Nullwerte kann in manchen Fällen zu einem undefinierten Verhalten führen.
In der Regel ist er im Testmodus jedoch abgesichert und führt zu Fehlermeldungen (siehe "Klassen").

\section{Id-Literale}
Für Id-Literale gelten bei der Validierung selbige Regeln wie in der Sektion "Objekttypen-Literale" definiert. Es wird also
eine Warnung vom Compiler ausgegeben, falls sich die Id statisch überprüfen lässt und in keiner der entsprechenden Dateien
zu finden ist.

\section{Objekttypen-Literale}
Des Weiteren können Eigenschaften sogenannter Objekttypen, die einen bestimmten nativen, kopierbasierten Typ haben abgefragt werden.
Objekttypen erhalten in Warcraft 3 The Frozen Throne eine vierstellige, hexadezimale Id. Daher können Objekttypen im Wertebereich von 16^4 definiert werden.
Jede Id identifiziert genau einen Objekttyp.
Ein Objekttyp kann in JASS++ folgendes sein:
\begin{enumerate}
\item ein Einheitentyp
\item ein Gegenstandstyp
\item ein Doodad-Typ
\item ein Zerstörbares-Typ
\item ein Fähigkeitentyp
\item ein Zauberverstärkertyp
\item ein Forschungstyp
\item ein Wettereffekttyp
\item ein Blitzeffekttyp
\item ein Geländeset
\end{enumerate}

Die Objekttypen stammen aus eine der folgenden Dateien (aufsteigend nach Priorität geordnet):
\begin{enumerate}
\item war3map.u
\item war3map.d
\item UI/UnitData.slk
\item *
\item *
\end{enumerate}
TODO

Dabei werden die angegebenen Dateien in den folgenden Archiven in absteigender Reihenfolge durchsucht:
\begin{enumerate}
\item Karte
\item Kampagne
\item War3Patch.mpq
\item War3Locale.mpq (richtigen Namen finden)
\item War3X.mpq
\item War3.mpq
\end{enumerate}
TODO

Die Eigenschaften eines Objekttyps können einen der folgenden nativen, kopierbasierten Typen haben:
\begin{enumerate}
\item string
\item integer
\item real
\item boolean
\end{enumerate}
TODO

Die Eigenschaften eines Objekttyps haben selbst nochmals eine eigene Id.
Ein Objekttypen-Eigenschafts-Literal wird immer als ein Ausdruck eines der angegebenen Typen interpretiert.
Kann der Objekttypeintrag bzw. der Objekttyp-Eigenschaftseintrag nicht gefunden werden, so gibt der Compiler eine Warnung aus.
Das Literal ist in diesem Fall ein Nullwert ("null").
Wird keine Eigenschafts-Id angegeben, so wird der Ausdruck als Array interpretiert. Dies geht jedoch nur falls alle Objekttypen-Eigenschaften
den gleichen nativen, kopierbasierten Typ haben.
Ansonsten kann das Literal nur als eigener Typ interpertiert werden (siehe "Klassen - Zuweisungen").
Die Notation sieht folgendermaßen aus:
<<Objekttyp-Id>[, %]>
<<Objekttyp-Id>,<Eigenschafts-Id>[, %])>

Das Prozentzeichen ist optional und gibt an, dass Fließkommazahlen als Prozentanteile und daher als Ganzzahlen zwischen 0 und 100 interpretiert werden
sollen. Aus 0.30 würde in diesem Fall also die Ganzzahl 30 werden.
Es gilt bei einer reinen Angabe der Objekttyp-Id für alle Eigenschaften.
Es gilt zu beachten, dass Objekttypen-Ids auch als Escape-Sequenzen verwendet werden können.

\section{Farbliterale}
Farbliterale haben stets den nativen Typ "integer". Sie sind daher ein 32-Bit-Wert, welcher eine RGBA-Farbe enthält.
Der RGBA-Wert muss hexadezimal angegeben werden, wobei die ersten beiden Ziffern für den Alphawert, die zweiten beiden für den Grünwert,
die dritten beiden für den Blauwert und die letzten beiden für den Blauwert stehen.
Notation:
|c<Farbe>|r

\section{Escape-Sequenzen}
Eine Escape-Sequenz ist ein Zeichen, welchem das \-Zeichen vorangestellt wird und welches ausschließlich ein einem
Zeichenkettenliteral vorkommen darf.
Es gibt folgende Escape-Sequenzen:
\begin(enumerate}
\item n				Zeilenumbruch
\item t				Horizontaler Tabulator
\item \				Backslash
\item "				Doppeltes Anführungszeichen
\end{enumerate}

Zusätzlich existieren spezielle Escape-Sequenzen für Objekttypen-Ids (siehe "Literale - Objekttypen-Ids") und Farben (siehe "Literale - Farbliterale").
Farbkodierungen und Objekttypen-Ids in Zeichenketten werden zum Kompilierzeitpunkt validiert bzw. kompiliert.
Die speziellen Escapesequenzen werden in den Zeichenketten zum Kompilierzeitpunkt ersetzt, insofern sie gültig sind.
Falls nicht, bleibt der Ausdruck erhalten und der Compiler gibt eine Warnung aus.
Wie bei normalen Objektypen-Literalen kann ein Prozentzeichen angehängt werden.
Bei Farben kann jedoch, anders als bei gewöhnlichen Farbliteralen, Text eingeschlossen werden. Dieser sollte dann in der entsprechenden Farbe
im Spiel bei einer Ausgabe angezeigt werden.

\section{Notation}
|c<Farbe>[<Text>]|r
<<Objekt-Id>,<Feld-Id>[, %]>


	\chapter{Literale}
Ein Literal ist ein sich selbst erklärender Ausdruck, der nicht speziell benannt werden muss also keinen Bezeichner benötigt.
Für folgende native Typen gelten folgende Literalnotationsregeln:

Kopierbasierte Typen:
TODO: Exponentialschreibweise für alle Ganzzahlausdrücke erlauben und auch für Fließkommazahlen?
\begin{enumerate}
\item string : "<Zeichenkettenliteral>" | null
\item integer : 0<Oktalliteral> | 0%<Binärliteral> | 0x<Hexadezimalliteral> | <Dezimalliteral>[e <Exponent>] | '<Id-Literal>' | null
\item real : <Fließkommaliteral> [r | R] | null
\item boolean : true | false | null
\item handle (und Nicht-agent-Kindtypen): null
\end{enumerate}

Referenzbasierte Typen:
\begin{enumerate}
\item code : 0
\item agent (und Kindtypen) : 0
\item Klassenobjekte : 0
\item Funktionszeiger : 0
\end{enumerate}

Dabei können Zeichenkettenausdrücke mehrfach hintereinander geschrieben werden. Diese werden dann automatisch zu einem einzelnen
zusammengefasst. Dies vereinfacht die Aufteilung auf mehrere Zeilen, ohne unnötige Zeilenumbrüche und Einrückungen dazwischen und
dient einer besseren Übersicht:
"<Ausdruckteil 1>"
"<Ausdruckteil n>"

\section{Nullwerte}
Wie oben dargestellt erhalten kopierbasierte Typen einen Nullwert mit dem Literal "null" und referenzbasierte Typen mit dem Literal "0".
Der Zugriff auf Nullwerte kann in manchen Fällen zu einem undefinierten Verhalten führen.
In der Regel ist er im Testmodus jedoch abgesichert und führt zu Fehlermeldungen (siehe "Klassen").

\section{Id-Literale}
Für Id-Literale gelten bei der Validierung selbige Regeln wie in der Sektion "Objekttypen-Literale" definiert. Es wird also
eine Warnung vom Compiler ausgegeben, falls sich die Id statisch überprüfen lässt und in keiner der entsprechenden Dateien
zu finden ist.

\section{Objekttypen-Literale}
Des Weiteren können Eigenschaften sogenannter Objekttypen, die einen bestimmten nativen, kopierbasierten Typ haben abgefragt werden.
Objekttypen erhalten in Warcraft 3 The Frozen Throne eine vierstellige, hexadezimale Id. Daher können Objekttypen im Wertebereich von 16^4 definiert werden.
Jede Id identifiziert genau einen Objekttyp.
Ein Objekttyp kann in JASS++ folgendes sein:
\begin{enumerate}
\item ein Einheitentyp
\item ein Gegenstandstyp
\item ein Doodad-Typ
\item ein Zerstörbares-Typ
\item ein Fähigkeitentyp
\item ein Zauberverstärkertyp
\item ein Forschungstyp
\item ein Wettereffekttyp
\item ein Blitzeffekttyp
\item ein Geländeset
\end{enumerate}

Die Objekttypen stammen aus eine der folgenden Dateien (aufsteigend nach Priorität geordnet):
\begin{enumerate}
\item war3map.u
\item war3map.d
\item UI/UnitData.slk
\item *
\item *
\end{enumerate}
TODO

Dabei werden die angegebenen Dateien in den folgenden Archiven in absteigender Reihenfolge durchsucht:
\begin{enumerate}
\item Karte
\item Kampagne
\item War3Patch.mpq
\item War3Locale.mpq (richtigen Namen finden)
\item War3X.mpq
\item War3.mpq
\end{enumerate}
TODO

Die Eigenschaften eines Objekttyps können einen der folgenden nativen, kopierbasierten Typen haben:
\begin{enumerate}
\item string
\item integer
\item real
\item boolean
\end{enumerate}
TODO

Die Eigenschaften eines Objekttyps haben selbst nochmals eine eigene Id.
Ein Objekttypen-Eigenschafts-Literal wird immer als ein Ausdruck eines der angegebenen Typen interpretiert.
Kann der Objekttypeintrag bzw. der Objekttyp-Eigenschaftseintrag nicht gefunden werden, so gibt der Compiler eine Warnung aus.
Das Literal ist in diesem Fall ein Nullwert ("null").
Wird keine Eigenschafts-Id angegeben, so wird der Ausdruck als Array interpretiert. Dies geht jedoch nur falls alle Objekttypen-Eigenschaften
den gleichen nativen, kopierbasierten Typ haben.
Ansonsten kann das Literal nur als eigener Typ interpertiert werden (siehe "Klassen - Zuweisungen").
Die Notation sieht folgendermaßen aus:
<<Objekttyp-Id>[, %]>
<<Objekttyp-Id>,<Eigenschafts-Id>[, %])>

Das Prozentzeichen ist optional und gibt an, dass Fließkommazahlen als Prozentanteile und daher als Ganzzahlen zwischen 0 und 100 interpretiert werden
sollen. Aus 0.30 würde in diesem Fall also die Ganzzahl 30 werden.
Es gilt bei einer reinen Angabe der Objekttyp-Id für alle Eigenschaften.
Es gilt zu beachten, dass Objekttypen-Ids auch als Escape-Sequenzen verwendet werden können.

\section{Farbliterale}
Farbliterale haben stets den nativen Typ "integer". Sie sind daher ein 32-Bit-Wert, welcher eine RGBA-Farbe enthält.
Der RGBA-Wert muss hexadezimal angegeben werden, wobei die ersten beiden Ziffern für den Alphawert, die zweiten beiden für den Grünwert,
die dritten beiden für den Blauwert und die letzten beiden für den Blauwert stehen.
Notation:
|c<Farbe>|r

\section{Escape-Sequenzen}
Eine Escape-Sequenz ist ein Zeichen, welchem das \-Zeichen vorangestellt wird und welches ausschließlich ein einem
Zeichenkettenliteral vorkommen darf.
Es gibt folgende Escape-Sequenzen:
\begin(enumerate}
\item n				Zeilenumbruch
\item t				Horizontaler Tabulator
\item \				Backslash
\item "				Doppeltes Anführungszeichen
\end{enumerate}

Zusätzlich existieren spezielle Escape-Sequenzen für Objekttypen-Ids (siehe "Literale - Objekttypen-Ids") und Farben (siehe "Literale - Farbliterale").
Farbkodierungen und Objekttypen-Ids in Zeichenketten werden zum Kompilierzeitpunkt validiert bzw. kompiliert.
Die speziellen Escapesequenzen werden in den Zeichenketten zum Kompilierzeitpunkt ersetzt, insofern sie gültig sind.
Falls nicht, bleibt der Ausdruck erhalten und der Compiler gibt eine Warnung aus.
Wie bei normalen Objektypen-Literalen kann ein Prozentzeichen angehängt werden.
Bei Farben kann jedoch, anders als bei gewöhnlichen Farbliteralen, Text eingeschlossen werden. Dieser sollte dann in der entsprechenden Farbe
im Spiel bei einer Ausgabe angezeigt werden.

\section{Notation}
|c<Farbe>[<Text>]|r
<<Objekt-Id>,<Feld-Id>[, %]>


	\chapter{Literale}
Ein Literal ist ein sich selbst erklärender Ausdruck, der nicht speziell benannt werden muss also keinen Bezeichner benötigt.
Für folgende native Typen gelten folgende Literalnotationsregeln:

Kopierbasierte Typen:
TODO: Exponentialschreibweise für alle Ganzzahlausdrücke erlauben und auch für Fließkommazahlen?
\begin{enumerate}
\item string : "<Zeichenkettenliteral>" | null
\item integer : 0<Oktalliteral> | 0%<Binärliteral> | 0x<Hexadezimalliteral> | <Dezimalliteral>[e <Exponent>] | '<Id-Literal>' | null
\item real : <Fließkommaliteral> [r | R] | null
\item boolean : true | false | null
\item handle (und Nicht-agent-Kindtypen): null
\end{enumerate}

Referenzbasierte Typen:
\begin{enumerate}
\item code : 0
\item agent (und Kindtypen) : 0
\item Klassenobjekte : 0
\item Funktionszeiger : 0
\end{enumerate}

Dabei können Zeichenkettenausdrücke mehrfach hintereinander geschrieben werden. Diese werden dann automatisch zu einem einzelnen
zusammengefasst. Dies vereinfacht die Aufteilung auf mehrere Zeilen, ohne unnötige Zeilenumbrüche und Einrückungen dazwischen und
dient einer besseren Übersicht:
"<Ausdruckteil 1>"
"<Ausdruckteil n>"

\section{Nullwerte}
Wie oben dargestellt erhalten kopierbasierte Typen einen Nullwert mit dem Literal "null" und referenzbasierte Typen mit dem Literal "0".
Der Zugriff auf Nullwerte kann in manchen Fällen zu einem undefinierten Verhalten führen.
In der Regel ist er im Testmodus jedoch abgesichert und führt zu Fehlermeldungen (siehe "Klassen").

\section{Id-Literale}
Für Id-Literale gelten bei der Validierung selbige Regeln wie in der Sektion "Objekttypen-Literale" definiert. Es wird also
eine Warnung vom Compiler ausgegeben, falls sich die Id statisch überprüfen lässt und in keiner der entsprechenden Dateien
zu finden ist.

\section{Objekttypen-Literale}
Des Weiteren können Eigenschaften sogenannter Objekttypen, die einen bestimmten nativen, kopierbasierten Typ haben abgefragt werden.
Objekttypen erhalten in Warcraft 3 The Frozen Throne eine vierstellige, hexadezimale Id. Daher können Objekttypen im Wertebereich von 16^4 definiert werden.
Jede Id identifiziert genau einen Objekttyp.
Ein Objekttyp kann in JASS++ folgendes sein:
\begin{enumerate}
\item ein Einheitentyp
\item ein Gegenstandstyp
\item ein Doodad-Typ
\item ein Zerstörbares-Typ
\item ein Fähigkeitentyp
\item ein Zauberverstärkertyp
\item ein Forschungstyp
\item ein Wettereffekttyp
\item ein Blitzeffekttyp
\item ein Geländeset
\end{enumerate}

Die Objekttypen stammen aus eine der folgenden Dateien (aufsteigend nach Priorität geordnet):
\begin{enumerate}
\item war3map.u
\item war3map.d
\item UI/UnitData.slk
\item *
\item *
\end{enumerate}
TODO

Dabei werden die angegebenen Dateien in den folgenden Archiven in absteigender Reihenfolge durchsucht:
\begin{enumerate}
\item Karte
\item Kampagne
\item War3Patch.mpq
\item War3Locale.mpq (richtigen Namen finden)
\item War3X.mpq
\item War3.mpq
\end{enumerate}
TODO

Die Eigenschaften eines Objekttyps können einen der folgenden nativen, kopierbasierten Typen haben:
\begin{enumerate}
\item string
\item integer
\item real
\item boolean
\end{enumerate}
TODO

Die Eigenschaften eines Objekttyps haben selbst nochmals eine eigene Id.
Ein Objekttypen-Eigenschafts-Literal wird immer als ein Ausdruck eines der angegebenen Typen interpretiert.
Kann der Objekttypeintrag bzw. der Objekttyp-Eigenschaftseintrag nicht gefunden werden, so gibt der Compiler eine Warnung aus.
Das Literal ist in diesem Fall ein Nullwert ("null").
Wird keine Eigenschafts-Id angegeben, so wird der Ausdruck als Array interpretiert. Dies geht jedoch nur falls alle Objekttypen-Eigenschaften
den gleichen nativen, kopierbasierten Typ haben.
Ansonsten kann das Literal nur als eigener Typ interpertiert werden (siehe "Klassen - Zuweisungen").
Die Notation sieht folgendermaßen aus:
<<Objekttyp-Id>[, %]>
<<Objekttyp-Id>,<Eigenschafts-Id>[, %])>

Das Prozentzeichen ist optional und gibt an, dass Fließkommazahlen als Prozentanteile und daher als Ganzzahlen zwischen 0 und 100 interpretiert werden
sollen. Aus 0.30 würde in diesem Fall also die Ganzzahl 30 werden.
Es gilt bei einer reinen Angabe der Objekttyp-Id für alle Eigenschaften.
Es gilt zu beachten, dass Objekttypen-Ids auch als Escape-Sequenzen verwendet werden können.

\section{Farbliterale}
Farbliterale haben stets den nativen Typ "integer". Sie sind daher ein 32-Bit-Wert, welcher eine RGBA-Farbe enthält.
Der RGBA-Wert muss hexadezimal angegeben werden, wobei die ersten beiden Ziffern für den Alphawert, die zweiten beiden für den Grünwert,
die dritten beiden für den Blauwert und die letzten beiden für den Blauwert stehen.
Notation:
|c<Farbe>|r

\section{Escape-Sequenzen}
Eine Escape-Sequenz ist ein Zeichen, welchem das \-Zeichen vorangestellt wird und welches ausschließlich ein einem
Zeichenkettenliteral vorkommen darf.
Es gibt folgende Escape-Sequenzen:
\begin(enumerate}
\item n				Zeilenumbruch
\item t				Horizontaler Tabulator
\item \				Backslash
\item "				Doppeltes Anführungszeichen
\end{enumerate}

Zusätzlich existieren spezielle Escape-Sequenzen für Objekttypen-Ids (siehe "Literale - Objekttypen-Ids") und Farben (siehe "Literale - Farbliterale").
Farbkodierungen und Objekttypen-Ids in Zeichenketten werden zum Kompilierzeitpunkt validiert bzw. kompiliert.
Die speziellen Escapesequenzen werden in den Zeichenketten zum Kompilierzeitpunkt ersetzt, insofern sie gültig sind.
Falls nicht, bleibt der Ausdruck erhalten und der Compiler gibt eine Warnung aus.
Wie bei normalen Objektypen-Literalen kann ein Prozentzeichen angehängt werden.
Bei Farben kann jedoch, anders als bei gewöhnlichen Farbliteralen, Text eingeschlossen werden. Dieser sollte dann in der entsprechenden Farbe
im Spiel bei einer Ausgabe angezeigt werden.

\section{Notation}
|c<Farbe>[<Text>]|r
<<Objekt-Id>,<Feld-Id>[, %]>

	\chapter{Literale}
Ein Literal ist ein sich selbst erklärender Ausdruck, der nicht speziell benannt werden muss also keinen Bezeichner benötigt.
Für folgende native Typen gelten folgende Literalnotationsregeln:

Kopierbasierte Typen:
TODO: Exponentialschreibweise für alle Ganzzahlausdrücke erlauben und auch für Fließkommazahlen?
\begin{enumerate}
\item string : "<Zeichenkettenliteral>" | null
\item integer : 0<Oktalliteral> | 0%<Binärliteral> | 0x<Hexadezimalliteral> | <Dezimalliteral>[e <Exponent>] | '<Id-Literal>' | null
\item real : <Fließkommaliteral> [r | R] | null
\item boolean : true | false | null
\item handle (und Nicht-agent-Kindtypen): null
\end{enumerate}

Referenzbasierte Typen:
\begin{enumerate}
\item code : 0
\item agent (und Kindtypen) : 0
\item Klassenobjekte : 0
\item Funktionszeiger : 0
\end{enumerate}

Dabei können Zeichenkettenausdrücke mehrfach hintereinander geschrieben werden. Diese werden dann automatisch zu einem einzelnen
zusammengefasst. Dies vereinfacht die Aufteilung auf mehrere Zeilen, ohne unnötige Zeilenumbrüche und Einrückungen dazwischen und
dient einer besseren Übersicht:
"<Ausdruckteil 1>"
"<Ausdruckteil n>"

\section{Nullwerte}
Wie oben dargestellt erhalten kopierbasierte Typen einen Nullwert mit dem Literal "null" und referenzbasierte Typen mit dem Literal "0".
Der Zugriff auf Nullwerte kann in manchen Fällen zu einem undefinierten Verhalten führen.
In der Regel ist er im Testmodus jedoch abgesichert und führt zu Fehlermeldungen (siehe "Klassen").

\section{Id-Literale}
Für Id-Literale gelten bei der Validierung selbige Regeln wie in der Sektion "Objekttypen-Literale" definiert. Es wird also
eine Warnung vom Compiler ausgegeben, falls sich die Id statisch überprüfen lässt und in keiner der entsprechenden Dateien
zu finden ist.

\section{Objekttypen-Literale}
Des Weiteren können Eigenschaften sogenannter Objekttypen, die einen bestimmten nativen, kopierbasierten Typ haben abgefragt werden.
Objekttypen erhalten in Warcraft 3 The Frozen Throne eine vierstellige, hexadezimale Id. Daher können Objekttypen im Wertebereich von 16^4 definiert werden.
Jede Id identifiziert genau einen Objekttyp.
Ein Objekttyp kann in JASS++ folgendes sein:
\begin{enumerate}
\item ein Einheitentyp
\item ein Gegenstandstyp
\item ein Doodad-Typ
\item ein Zerstörbares-Typ
\item ein Fähigkeitentyp
\item ein Zauberverstärkertyp
\item ein Forschungstyp
\item ein Wettereffekttyp
\item ein Blitzeffekttyp
\item ein Geländeset
\end{enumerate}

Die Objekttypen stammen aus eine der folgenden Dateien (aufsteigend nach Priorität geordnet):
\begin{enumerate}
\item war3map.u
\item war3map.d
\item UI/UnitData.slk
\item *
\item *
\end{enumerate}
TODO

Dabei werden die angegebenen Dateien in den folgenden Archiven in absteigender Reihenfolge durchsucht:
\begin{enumerate}
\item Karte
\item Kampagne
\item War3Patch.mpq
\item War3Locale.mpq (richtigen Namen finden)
\item War3X.mpq
\item War3.mpq
\end{enumerate}
TODO

Die Eigenschaften eines Objekttyps können einen der folgenden nativen, kopierbasierten Typen haben:
\begin{enumerate}
\item string
\item integer
\item real
\item boolean
\end{enumerate}
TODO

Die Eigenschaften eines Objekttyps haben selbst nochmals eine eigene Id.
Ein Objekttypen-Eigenschafts-Literal wird immer als ein Ausdruck eines der angegebenen Typen interpretiert.
Kann der Objekttypeintrag bzw. der Objekttyp-Eigenschaftseintrag nicht gefunden werden, so gibt der Compiler eine Warnung aus.
Das Literal ist in diesem Fall ein Nullwert ("null").
Wird keine Eigenschafts-Id angegeben, so wird der Ausdruck als Array interpretiert. Dies geht jedoch nur falls alle Objekttypen-Eigenschaften
den gleichen nativen, kopierbasierten Typ haben.
Ansonsten kann das Literal nur als eigener Typ interpertiert werden (siehe "Klassen - Zuweisungen").
Die Notation sieht folgendermaßen aus:
<<Objekttyp-Id>[, %]>
<<Objekttyp-Id>,<Eigenschafts-Id>[, %])>

Das Prozentzeichen ist optional und gibt an, dass Fließkommazahlen als Prozentanteile und daher als Ganzzahlen zwischen 0 und 100 interpretiert werden
sollen. Aus 0.30 würde in diesem Fall also die Ganzzahl 30 werden.
Es gilt bei einer reinen Angabe der Objekttyp-Id für alle Eigenschaften.
Es gilt zu beachten, dass Objekttypen-Ids auch als Escape-Sequenzen verwendet werden können.

\section{Farbliterale}
Farbliterale haben stets den nativen Typ "integer". Sie sind daher ein 32-Bit-Wert, welcher eine RGBA-Farbe enthält.
Der RGBA-Wert muss hexadezimal angegeben werden, wobei die ersten beiden Ziffern für den Alphawert, die zweiten beiden für den Grünwert,
die dritten beiden für den Blauwert und die letzten beiden für den Blauwert stehen.
Notation:
|c<Farbe>|r

\section{Escape-Sequenzen}
Eine Escape-Sequenz ist ein Zeichen, welchem das \-Zeichen vorangestellt wird und welches ausschließlich ein einem
Zeichenkettenliteral vorkommen darf.
Es gibt folgende Escape-Sequenzen:
\begin(enumerate}
\item n				Zeilenumbruch
\item t				Horizontaler Tabulator
\item \				Backslash
\item "				Doppeltes Anführungszeichen
\end{enumerate}

Zusätzlich existieren spezielle Escape-Sequenzen für Objekttypen-Ids (siehe "Literale - Objekttypen-Ids") und Farben (siehe "Literale - Farbliterale").
Farbkodierungen und Objekttypen-Ids in Zeichenketten werden zum Kompilierzeitpunkt validiert bzw. kompiliert.
Die speziellen Escapesequenzen werden in den Zeichenketten zum Kompilierzeitpunkt ersetzt, insofern sie gültig sind.
Falls nicht, bleibt der Ausdruck erhalten und der Compiler gibt eine Warnung aus.
Wie bei normalen Objektypen-Literalen kann ein Prozentzeichen angehängt werden.
Bei Farben kann jedoch, anders als bei gewöhnlichen Farbliteralen, Text eingeschlossen werden. Dieser sollte dann in der entsprechenden Farbe
im Spiel bei einer Ausgabe angezeigt werden.

\section{Notation}
|c<Farbe>[<Text>]|r
<<Objekt-Id>,<Feld-Id>[, %]>

	\chapter{Literale}
Ein Literal ist ein sich selbst erklärender Ausdruck, der nicht speziell benannt werden muss also keinen Bezeichner benötigt.
Für folgende native Typen gelten folgende Literalnotationsregeln:

Kopierbasierte Typen:
TODO: Exponentialschreibweise für alle Ganzzahlausdrücke erlauben und auch für Fließkommazahlen?
\begin{enumerate}
\item string : "<Zeichenkettenliteral>" | null
\item integer : 0<Oktalliteral> | 0%<Binärliteral> | 0x<Hexadezimalliteral> | <Dezimalliteral>[e <Exponent>] | '<Id-Literal>' | null
\item real : <Fließkommaliteral> [r | R] | null
\item boolean : true | false | null
\item handle (und Nicht-agent-Kindtypen): null
\end{enumerate}

Referenzbasierte Typen:
\begin{enumerate}
\item code : 0
\item agent (und Kindtypen) : 0
\item Klassenobjekte : 0
\item Funktionszeiger : 0
\end{enumerate}

Dabei können Zeichenkettenausdrücke mehrfach hintereinander geschrieben werden. Diese werden dann automatisch zu einem einzelnen
zusammengefasst. Dies vereinfacht die Aufteilung auf mehrere Zeilen, ohne unnötige Zeilenumbrüche und Einrückungen dazwischen und
dient einer besseren Übersicht:
"<Ausdruckteil 1>"
"<Ausdruckteil n>"

\section{Nullwerte}
Wie oben dargestellt erhalten kopierbasierte Typen einen Nullwert mit dem Literal "null" und referenzbasierte Typen mit dem Literal "0".
Der Zugriff auf Nullwerte kann in manchen Fällen zu einem undefinierten Verhalten führen.
In der Regel ist er im Testmodus jedoch abgesichert und führt zu Fehlermeldungen (siehe "Klassen").

\section{Id-Literale}
Für Id-Literale gelten bei der Validierung selbige Regeln wie in der Sektion "Objekttypen-Literale" definiert. Es wird also
eine Warnung vom Compiler ausgegeben, falls sich die Id statisch überprüfen lässt und in keiner der entsprechenden Dateien
zu finden ist.

\section{Objekttypen-Literale}
Des Weiteren können Eigenschaften sogenannter Objekttypen, die einen bestimmten nativen, kopierbasierten Typ haben abgefragt werden.
Objekttypen erhalten in Warcraft 3 The Frozen Throne eine vierstellige, hexadezimale Id. Daher können Objekttypen im Wertebereich von 16^4 definiert werden.
Jede Id identifiziert genau einen Objekttyp.
Ein Objekttyp kann in JASS++ folgendes sein:
\begin{enumerate}
\item ein Einheitentyp
\item ein Gegenstandstyp
\item ein Doodad-Typ
\item ein Zerstörbares-Typ
\item ein Fähigkeitentyp
\item ein Zauberverstärkertyp
\item ein Forschungstyp
\item ein Wettereffekttyp
\item ein Blitzeffekttyp
\item ein Geländeset
\end{enumerate}

Die Objekttypen stammen aus eine der folgenden Dateien (aufsteigend nach Priorität geordnet):
\begin{enumerate}
\item war3map.u
\item war3map.d
\item UI/UnitData.slk
\item *
\item *
\end{enumerate}
TODO

Dabei werden die angegebenen Dateien in den folgenden Archiven in absteigender Reihenfolge durchsucht:
\begin{enumerate}
\item Karte
\item Kampagne
\item War3Patch.mpq
\item War3Locale.mpq (richtigen Namen finden)
\item War3X.mpq
\item War3.mpq
\end{enumerate}
TODO

Die Eigenschaften eines Objekttyps können einen der folgenden nativen, kopierbasierten Typen haben:
\begin{enumerate}
\item string
\item integer
\item real
\item boolean
\end{enumerate}
TODO

Die Eigenschaften eines Objekttyps haben selbst nochmals eine eigene Id.
Ein Objekttypen-Eigenschafts-Literal wird immer als ein Ausdruck eines der angegebenen Typen interpretiert.
Kann der Objekttypeintrag bzw. der Objekttyp-Eigenschaftseintrag nicht gefunden werden, so gibt der Compiler eine Warnung aus.
Das Literal ist in diesem Fall ein Nullwert ("null").
Wird keine Eigenschafts-Id angegeben, so wird der Ausdruck als Array interpretiert. Dies geht jedoch nur falls alle Objekttypen-Eigenschaften
den gleichen nativen, kopierbasierten Typ haben.
Ansonsten kann das Literal nur als eigener Typ interpertiert werden (siehe "Klassen - Zuweisungen").
Die Notation sieht folgendermaßen aus:
<<Objekttyp-Id>[, %]>
<<Objekttyp-Id>,<Eigenschafts-Id>[, %])>

Das Prozentzeichen ist optional und gibt an, dass Fließkommazahlen als Prozentanteile und daher als Ganzzahlen zwischen 0 und 100 interpretiert werden
sollen. Aus 0.30 würde in diesem Fall also die Ganzzahl 30 werden.
Es gilt bei einer reinen Angabe der Objekttyp-Id für alle Eigenschaften.
Es gilt zu beachten, dass Objekttypen-Ids auch als Escape-Sequenzen verwendet werden können.

\section{Farbliterale}
Farbliterale haben stets den nativen Typ "integer". Sie sind daher ein 32-Bit-Wert, welcher eine RGBA-Farbe enthält.
Der RGBA-Wert muss hexadezimal angegeben werden, wobei die ersten beiden Ziffern für den Alphawert, die zweiten beiden für den Grünwert,
die dritten beiden für den Blauwert und die letzten beiden für den Blauwert stehen.
Notation:
|c<Farbe>|r

\section{Escape-Sequenzen}
Eine Escape-Sequenz ist ein Zeichen, welchem das \-Zeichen vorangestellt wird und welches ausschließlich ein einem
Zeichenkettenliteral vorkommen darf.
Es gibt folgende Escape-Sequenzen:
\begin(enumerate}
\item n				Zeilenumbruch
\item t				Horizontaler Tabulator
\item \				Backslash
\item "				Doppeltes Anführungszeichen
\end{enumerate}

Zusätzlich existieren spezielle Escape-Sequenzen für Objekttypen-Ids (siehe "Literale - Objekttypen-Ids") und Farben (siehe "Literale - Farbliterale").
Farbkodierungen und Objekttypen-Ids in Zeichenketten werden zum Kompilierzeitpunkt validiert bzw. kompiliert.
Die speziellen Escapesequenzen werden in den Zeichenketten zum Kompilierzeitpunkt ersetzt, insofern sie gültig sind.
Falls nicht, bleibt der Ausdruck erhalten und der Compiler gibt eine Warnung aus.
Wie bei normalen Objektypen-Literalen kann ein Prozentzeichen angehängt werden.
Bei Farben kann jedoch, anders als bei gewöhnlichen Farbliteralen, Text eingeschlossen werden. Dieser sollte dann in der entsprechenden Farbe
im Spiel bei einer Ausgabe angezeigt werden.

\section{Notation}
|c<Farbe>[<Text>]|r
<<Objekt-Id>,<Feld-Id>[, %]>


	\chapter{Literale}
Ein Literal ist ein sich selbst erklärender Ausdruck, der nicht speziell benannt werden muss also keinen Bezeichner benötigt.
Für folgende native Typen gelten folgende Literalnotationsregeln:

Kopierbasierte Typen:
TODO: Exponentialschreibweise für alle Ganzzahlausdrücke erlauben und auch für Fließkommazahlen?
\begin{enumerate}
\item string : "<Zeichenkettenliteral>" | null
\item integer : 0<Oktalliteral> | 0%<Binärliteral> | 0x<Hexadezimalliteral> | <Dezimalliteral>[e <Exponent>] | '<Id-Literal>' | null
\item real : <Fließkommaliteral> [r | R] | null
\item boolean : true | false | null
\item handle (und Nicht-agent-Kindtypen): null
\end{enumerate}

Referenzbasierte Typen:
\begin{enumerate}
\item code : 0
\item agent (und Kindtypen) : 0
\item Klassenobjekte : 0
\item Funktionszeiger : 0
\end{enumerate}

Dabei können Zeichenkettenausdrücke mehrfach hintereinander geschrieben werden. Diese werden dann automatisch zu einem einzelnen
zusammengefasst. Dies vereinfacht die Aufteilung auf mehrere Zeilen, ohne unnötige Zeilenumbrüche und Einrückungen dazwischen und
dient einer besseren Übersicht:
"<Ausdruckteil 1>"
"<Ausdruckteil n>"

\section{Nullwerte}
Wie oben dargestellt erhalten kopierbasierte Typen einen Nullwert mit dem Literal "null" und referenzbasierte Typen mit dem Literal "0".
Der Zugriff auf Nullwerte kann in manchen Fällen zu einem undefinierten Verhalten führen.
In der Regel ist er im Testmodus jedoch abgesichert und führt zu Fehlermeldungen (siehe "Klassen").

\section{Id-Literale}
Für Id-Literale gelten bei der Validierung selbige Regeln wie in der Sektion "Objekttypen-Literale" definiert. Es wird also
eine Warnung vom Compiler ausgegeben, falls sich die Id statisch überprüfen lässt und in keiner der entsprechenden Dateien
zu finden ist.

\section{Objekttypen-Literale}
Des Weiteren können Eigenschaften sogenannter Objekttypen, die einen bestimmten nativen, kopierbasierten Typ haben abgefragt werden.
Objekttypen erhalten in Warcraft 3 The Frozen Throne eine vierstellige, hexadezimale Id. Daher können Objekttypen im Wertebereich von 16^4 definiert werden.
Jede Id identifiziert genau einen Objekttyp.
Ein Objekttyp kann in JASS++ folgendes sein:
\begin{enumerate}
\item ein Einheitentyp
\item ein Gegenstandstyp
\item ein Doodad-Typ
\item ein Zerstörbares-Typ
\item ein Fähigkeitentyp
\item ein Zauberverstärkertyp
\item ein Forschungstyp
\item ein Wettereffekttyp
\item ein Blitzeffekttyp
\item ein Geländeset
\end{enumerate}

Die Objekttypen stammen aus eine der folgenden Dateien (aufsteigend nach Priorität geordnet):
\begin{enumerate}
\item war3map.u
\item war3map.d
\item UI/UnitData.slk
\item *
\item *
\end{enumerate}
TODO

Dabei werden die angegebenen Dateien in den folgenden Archiven in absteigender Reihenfolge durchsucht:
\begin{enumerate}
\item Karte
\item Kampagne
\item War3Patch.mpq
\item War3Locale.mpq (richtigen Namen finden)
\item War3X.mpq
\item War3.mpq
\end{enumerate}
TODO

Die Eigenschaften eines Objekttyps können einen der folgenden nativen, kopierbasierten Typen haben:
\begin{enumerate}
\item string
\item integer
\item real
\item boolean
\end{enumerate}
TODO

Die Eigenschaften eines Objekttyps haben selbst nochmals eine eigene Id.
Ein Objekttypen-Eigenschafts-Literal wird immer als ein Ausdruck eines der angegebenen Typen interpretiert.
Kann der Objekttypeintrag bzw. der Objekttyp-Eigenschaftseintrag nicht gefunden werden, so gibt der Compiler eine Warnung aus.
Das Literal ist in diesem Fall ein Nullwert ("null").
Wird keine Eigenschafts-Id angegeben, so wird der Ausdruck als Array interpretiert. Dies geht jedoch nur falls alle Objekttypen-Eigenschaften
den gleichen nativen, kopierbasierten Typ haben.
Ansonsten kann das Literal nur als eigener Typ interpertiert werden (siehe "Klassen - Zuweisungen").
Die Notation sieht folgendermaßen aus:
<<Objekttyp-Id>[, %]>
<<Objekttyp-Id>,<Eigenschafts-Id>[, %])>

Das Prozentzeichen ist optional und gibt an, dass Fließkommazahlen als Prozentanteile und daher als Ganzzahlen zwischen 0 und 100 interpretiert werden
sollen. Aus 0.30 würde in diesem Fall also die Ganzzahl 30 werden.
Es gilt bei einer reinen Angabe der Objekttyp-Id für alle Eigenschaften.
Es gilt zu beachten, dass Objekttypen-Ids auch als Escape-Sequenzen verwendet werden können.

\section{Farbliterale}
Farbliterale haben stets den nativen Typ "integer". Sie sind daher ein 32-Bit-Wert, welcher eine RGBA-Farbe enthält.
Der RGBA-Wert muss hexadezimal angegeben werden, wobei die ersten beiden Ziffern für den Alphawert, die zweiten beiden für den Grünwert,
die dritten beiden für den Blauwert und die letzten beiden für den Blauwert stehen.
Notation:
|c<Farbe>|r

\section{Escape-Sequenzen}
Eine Escape-Sequenz ist ein Zeichen, welchem das \-Zeichen vorangestellt wird und welches ausschließlich ein einem
Zeichenkettenliteral vorkommen darf.
Es gibt folgende Escape-Sequenzen:
\begin(enumerate}
\item n				Zeilenumbruch
\item t				Horizontaler Tabulator
\item \				Backslash
\item "				Doppeltes Anführungszeichen
\end{enumerate}

Zusätzlich existieren spezielle Escape-Sequenzen für Objekttypen-Ids (siehe "Literale - Objekttypen-Ids") und Farben (siehe "Literale - Farbliterale").
Farbkodierungen und Objekttypen-Ids in Zeichenketten werden zum Kompilierzeitpunkt validiert bzw. kompiliert.
Die speziellen Escapesequenzen werden in den Zeichenketten zum Kompilierzeitpunkt ersetzt, insofern sie gültig sind.
Falls nicht, bleibt der Ausdruck erhalten und der Compiler gibt eine Warnung aus.
Wie bei normalen Objektypen-Literalen kann ein Prozentzeichen angehängt werden.
Bei Farben kann jedoch, anders als bei gewöhnlichen Farbliteralen, Text eingeschlossen werden. Dieser sollte dann in der entsprechenden Farbe
im Spiel bei einer Ausgabe angezeigt werden.

\section{Notation}
|c<Farbe>[<Text>]|r
<<Objekt-Id>,<Feld-Id>[, %]>

	\chapter{Literale}
Ein Literal ist ein sich selbst erklärender Ausdruck, der nicht speziell benannt werden muss also keinen Bezeichner benötigt.
Für folgende native Typen gelten folgende Literalnotationsregeln:

Kopierbasierte Typen:
TODO: Exponentialschreibweise für alle Ganzzahlausdrücke erlauben und auch für Fließkommazahlen?
\begin{enumerate}
\item string : "<Zeichenkettenliteral>" | null
\item integer : 0<Oktalliteral> | 0%<Binärliteral> | 0x<Hexadezimalliteral> | <Dezimalliteral>[e <Exponent>] | '<Id-Literal>' | null
\item real : <Fließkommaliteral> [r | R] | null
\item boolean : true | false | null
\item handle (und Nicht-agent-Kindtypen): null
\end{enumerate}

Referenzbasierte Typen:
\begin{enumerate}
\item code : 0
\item agent (und Kindtypen) : 0
\item Klassenobjekte : 0
\item Funktionszeiger : 0
\end{enumerate}

Dabei können Zeichenkettenausdrücke mehrfach hintereinander geschrieben werden. Diese werden dann automatisch zu einem einzelnen
zusammengefasst. Dies vereinfacht die Aufteilung auf mehrere Zeilen, ohne unnötige Zeilenumbrüche und Einrückungen dazwischen und
dient einer besseren Übersicht:
"<Ausdruckteil 1>"
"<Ausdruckteil n>"

\section{Nullwerte}
Wie oben dargestellt erhalten kopierbasierte Typen einen Nullwert mit dem Literal "null" und referenzbasierte Typen mit dem Literal "0".
Der Zugriff auf Nullwerte kann in manchen Fällen zu einem undefinierten Verhalten führen.
In der Regel ist er im Testmodus jedoch abgesichert und führt zu Fehlermeldungen (siehe "Klassen").

\section{Id-Literale}
Für Id-Literale gelten bei der Validierung selbige Regeln wie in der Sektion "Objekttypen-Literale" definiert. Es wird also
eine Warnung vom Compiler ausgegeben, falls sich die Id statisch überprüfen lässt und in keiner der entsprechenden Dateien
zu finden ist.

\section{Objekttypen-Literale}
Des Weiteren können Eigenschaften sogenannter Objekttypen, die einen bestimmten nativen, kopierbasierten Typ haben abgefragt werden.
Objekttypen erhalten in Warcraft 3 The Frozen Throne eine vierstellige, hexadezimale Id. Daher können Objekttypen im Wertebereich von 16^4 definiert werden.
Jede Id identifiziert genau einen Objekttyp.
Ein Objekttyp kann in JASS++ folgendes sein:
\begin{enumerate}
\item ein Einheitentyp
\item ein Gegenstandstyp
\item ein Doodad-Typ
\item ein Zerstörbares-Typ
\item ein Fähigkeitentyp
\item ein Zauberverstärkertyp
\item ein Forschungstyp
\item ein Wettereffekttyp
\item ein Blitzeffekttyp
\item ein Geländeset
\end{enumerate}

Die Objekttypen stammen aus eine der folgenden Dateien (aufsteigend nach Priorität geordnet):
\begin{enumerate}
\item war3map.u
\item war3map.d
\item UI/UnitData.slk
\item *
\item *
\end{enumerate}
TODO

Dabei werden die angegebenen Dateien in den folgenden Archiven in absteigender Reihenfolge durchsucht:
\begin{enumerate}
\item Karte
\item Kampagne
\item War3Patch.mpq
\item War3Locale.mpq (richtigen Namen finden)
\item War3X.mpq
\item War3.mpq
\end{enumerate}
TODO

Die Eigenschaften eines Objekttyps können einen der folgenden nativen, kopierbasierten Typen haben:
\begin{enumerate}
\item string
\item integer
\item real
\item boolean
\end{enumerate}
TODO

Die Eigenschaften eines Objekttyps haben selbst nochmals eine eigene Id.
Ein Objekttypen-Eigenschafts-Literal wird immer als ein Ausdruck eines der angegebenen Typen interpretiert.
Kann der Objekttypeintrag bzw. der Objekttyp-Eigenschaftseintrag nicht gefunden werden, so gibt der Compiler eine Warnung aus.
Das Literal ist in diesem Fall ein Nullwert ("null").
Wird keine Eigenschafts-Id angegeben, so wird der Ausdruck als Array interpretiert. Dies geht jedoch nur falls alle Objekttypen-Eigenschaften
den gleichen nativen, kopierbasierten Typ haben.
Ansonsten kann das Literal nur als eigener Typ interpertiert werden (siehe "Klassen - Zuweisungen").
Die Notation sieht folgendermaßen aus:
<<Objekttyp-Id>[, %]>
<<Objekttyp-Id>,<Eigenschafts-Id>[, %])>

Das Prozentzeichen ist optional und gibt an, dass Fließkommazahlen als Prozentanteile und daher als Ganzzahlen zwischen 0 und 100 interpretiert werden
sollen. Aus 0.30 würde in diesem Fall also die Ganzzahl 30 werden.
Es gilt bei einer reinen Angabe der Objekttyp-Id für alle Eigenschaften.
Es gilt zu beachten, dass Objekttypen-Ids auch als Escape-Sequenzen verwendet werden können.

\section{Farbliterale}
Farbliterale haben stets den nativen Typ "integer". Sie sind daher ein 32-Bit-Wert, welcher eine RGBA-Farbe enthält.
Der RGBA-Wert muss hexadezimal angegeben werden, wobei die ersten beiden Ziffern für den Alphawert, die zweiten beiden für den Grünwert,
die dritten beiden für den Blauwert und die letzten beiden für den Blauwert stehen.
Notation:
|c<Farbe>|r

\section{Escape-Sequenzen}
Eine Escape-Sequenz ist ein Zeichen, welchem das \-Zeichen vorangestellt wird und welches ausschließlich ein einem
Zeichenkettenliteral vorkommen darf.
Es gibt folgende Escape-Sequenzen:
\begin(enumerate}
\item n				Zeilenumbruch
\item t				Horizontaler Tabulator
\item \				Backslash
\item "				Doppeltes Anführungszeichen
\end{enumerate}

Zusätzlich existieren spezielle Escape-Sequenzen für Objekttypen-Ids (siehe "Literale - Objekttypen-Ids") und Farben (siehe "Literale - Farbliterale").
Farbkodierungen und Objekttypen-Ids in Zeichenketten werden zum Kompilierzeitpunkt validiert bzw. kompiliert.
Die speziellen Escapesequenzen werden in den Zeichenketten zum Kompilierzeitpunkt ersetzt, insofern sie gültig sind.
Falls nicht, bleibt der Ausdruck erhalten und der Compiler gibt eine Warnung aus.
Wie bei normalen Objektypen-Literalen kann ein Prozentzeichen angehängt werden.
Bei Farben kann jedoch, anders als bei gewöhnlichen Farbliteralen, Text eingeschlossen werden. Dieser sollte dann in der entsprechenden Farbe
im Spiel bei einer Ausgabe angezeigt werden.

\section{Notation}
|c<Farbe>[<Text>]|r
<<Objekt-Id>,<Feld-Id>[, %]>


	\chapter{Literale}
Ein Literal ist ein sich selbst erklärender Ausdruck, der nicht speziell benannt werden muss also keinen Bezeichner benötigt.
Für folgende native Typen gelten folgende Literalnotationsregeln:

Kopierbasierte Typen:
TODO: Exponentialschreibweise für alle Ganzzahlausdrücke erlauben und auch für Fließkommazahlen?
\begin{enumerate}
\item string : "<Zeichenkettenliteral>" | null
\item integer : 0<Oktalliteral> | 0%<Binärliteral> | 0x<Hexadezimalliteral> | <Dezimalliteral>[e <Exponent>] | '<Id-Literal>' | null
\item real : <Fließkommaliteral> [r | R] | null
\item boolean : true | false | null
\item handle (und Nicht-agent-Kindtypen): null
\end{enumerate}

Referenzbasierte Typen:
\begin{enumerate}
\item code : 0
\item agent (und Kindtypen) : 0
\item Klassenobjekte : 0
\item Funktionszeiger : 0
\end{enumerate}

Dabei können Zeichenkettenausdrücke mehrfach hintereinander geschrieben werden. Diese werden dann automatisch zu einem einzelnen
zusammengefasst. Dies vereinfacht die Aufteilung auf mehrere Zeilen, ohne unnötige Zeilenumbrüche und Einrückungen dazwischen und
dient einer besseren Übersicht:
"<Ausdruckteil 1>"
"<Ausdruckteil n>"

\section{Nullwerte}
Wie oben dargestellt erhalten kopierbasierte Typen einen Nullwert mit dem Literal "null" und referenzbasierte Typen mit dem Literal "0".
Der Zugriff auf Nullwerte kann in manchen Fällen zu einem undefinierten Verhalten führen.
In der Regel ist er im Testmodus jedoch abgesichert und führt zu Fehlermeldungen (siehe "Klassen").

\section{Id-Literale}
Für Id-Literale gelten bei der Validierung selbige Regeln wie in der Sektion "Objekttypen-Literale" definiert. Es wird also
eine Warnung vom Compiler ausgegeben, falls sich die Id statisch überprüfen lässt und in keiner der entsprechenden Dateien
zu finden ist.

\section{Objekttypen-Literale}
Des Weiteren können Eigenschaften sogenannter Objekttypen, die einen bestimmten nativen, kopierbasierten Typ haben abgefragt werden.
Objekttypen erhalten in Warcraft 3 The Frozen Throne eine vierstellige, hexadezimale Id. Daher können Objekttypen im Wertebereich von 16^4 definiert werden.
Jede Id identifiziert genau einen Objekttyp.
Ein Objekttyp kann in JASS++ folgendes sein:
\begin{enumerate}
\item ein Einheitentyp
\item ein Gegenstandstyp
\item ein Doodad-Typ
\item ein Zerstörbares-Typ
\item ein Fähigkeitentyp
\item ein Zauberverstärkertyp
\item ein Forschungstyp
\item ein Wettereffekttyp
\item ein Blitzeffekttyp
\item ein Geländeset
\end{enumerate}

Die Objekttypen stammen aus eine der folgenden Dateien (aufsteigend nach Priorität geordnet):
\begin{enumerate}
\item war3map.u
\item war3map.d
\item UI/UnitData.slk
\item *
\item *
\end{enumerate}
TODO

Dabei werden die angegebenen Dateien in den folgenden Archiven in absteigender Reihenfolge durchsucht:
\begin{enumerate}
\item Karte
\item Kampagne
\item War3Patch.mpq
\item War3Locale.mpq (richtigen Namen finden)
\item War3X.mpq
\item War3.mpq
\end{enumerate}
TODO

Die Eigenschaften eines Objekttyps können einen der folgenden nativen, kopierbasierten Typen haben:
\begin{enumerate}
\item string
\item integer
\item real
\item boolean
\end{enumerate}
TODO

Die Eigenschaften eines Objekttyps haben selbst nochmals eine eigene Id.
Ein Objekttypen-Eigenschafts-Literal wird immer als ein Ausdruck eines der angegebenen Typen interpretiert.
Kann der Objekttypeintrag bzw. der Objekttyp-Eigenschaftseintrag nicht gefunden werden, so gibt der Compiler eine Warnung aus.
Das Literal ist in diesem Fall ein Nullwert ("null").
Wird keine Eigenschafts-Id angegeben, so wird der Ausdruck als Array interpretiert. Dies geht jedoch nur falls alle Objekttypen-Eigenschaften
den gleichen nativen, kopierbasierten Typ haben.
Ansonsten kann das Literal nur als eigener Typ interpertiert werden (siehe "Klassen - Zuweisungen").
Die Notation sieht folgendermaßen aus:
<<Objekttyp-Id>[, %]>
<<Objekttyp-Id>,<Eigenschafts-Id>[, %])>

Das Prozentzeichen ist optional und gibt an, dass Fließkommazahlen als Prozentanteile und daher als Ganzzahlen zwischen 0 und 100 interpretiert werden
sollen. Aus 0.30 würde in diesem Fall also die Ganzzahl 30 werden.
Es gilt bei einer reinen Angabe der Objekttyp-Id für alle Eigenschaften.
Es gilt zu beachten, dass Objekttypen-Ids auch als Escape-Sequenzen verwendet werden können.

\section{Farbliterale}
Farbliterale haben stets den nativen Typ "integer". Sie sind daher ein 32-Bit-Wert, welcher eine RGBA-Farbe enthält.
Der RGBA-Wert muss hexadezimal angegeben werden, wobei die ersten beiden Ziffern für den Alphawert, die zweiten beiden für den Grünwert,
die dritten beiden für den Blauwert und die letzten beiden für den Blauwert stehen.
Notation:
|c<Farbe>|r

\section{Escape-Sequenzen}
Eine Escape-Sequenz ist ein Zeichen, welchem das \-Zeichen vorangestellt wird und welches ausschließlich ein einem
Zeichenkettenliteral vorkommen darf.
Es gibt folgende Escape-Sequenzen:
\begin(enumerate}
\item n				Zeilenumbruch
\item t				Horizontaler Tabulator
\item \				Backslash
\item "				Doppeltes Anführungszeichen
\end{enumerate}

Zusätzlich existieren spezielle Escape-Sequenzen für Objekttypen-Ids (siehe "Literale - Objekttypen-Ids") und Farben (siehe "Literale - Farbliterale").
Farbkodierungen und Objekttypen-Ids in Zeichenketten werden zum Kompilierzeitpunkt validiert bzw. kompiliert.
Die speziellen Escapesequenzen werden in den Zeichenketten zum Kompilierzeitpunkt ersetzt, insofern sie gültig sind.
Falls nicht, bleibt der Ausdruck erhalten und der Compiler gibt eine Warnung aus.
Wie bei normalen Objektypen-Literalen kann ein Prozentzeichen angehängt werden.
Bei Farben kann jedoch, anders als bei gewöhnlichen Farbliteralen, Text eingeschlossen werden. Dieser sollte dann in der entsprechenden Farbe
im Spiel bei einer Ausgabe angezeigt werden.

\section{Notation}
|c<Farbe>[<Text>]|r
<<Objekt-Id>,<Feld-Id>[, %]>

	\chapter{Literale}
Ein Literal ist ein sich selbst erklärender Ausdruck, der nicht speziell benannt werden muss also keinen Bezeichner benötigt.
Für folgende native Typen gelten folgende Literalnotationsregeln:

Kopierbasierte Typen:
TODO: Exponentialschreibweise für alle Ganzzahlausdrücke erlauben und auch für Fließkommazahlen?
\begin{enumerate}
\item string : "<Zeichenkettenliteral>" | null
\item integer : 0<Oktalliteral> | 0%<Binärliteral> | 0x<Hexadezimalliteral> | <Dezimalliteral>[e <Exponent>] | '<Id-Literal>' | null
\item real : <Fließkommaliteral> [r | R] | null
\item boolean : true | false | null
\item handle (und Nicht-agent-Kindtypen): null
\end{enumerate}

Referenzbasierte Typen:
\begin{enumerate}
\item code : 0
\item agent (und Kindtypen) : 0
\item Klassenobjekte : 0
\item Funktionszeiger : 0
\end{enumerate}

Dabei können Zeichenkettenausdrücke mehrfach hintereinander geschrieben werden. Diese werden dann automatisch zu einem einzelnen
zusammengefasst. Dies vereinfacht die Aufteilung auf mehrere Zeilen, ohne unnötige Zeilenumbrüche und Einrückungen dazwischen und
dient einer besseren Übersicht:
"<Ausdruckteil 1>"
"<Ausdruckteil n>"

\section{Nullwerte}
Wie oben dargestellt erhalten kopierbasierte Typen einen Nullwert mit dem Literal "null" und referenzbasierte Typen mit dem Literal "0".
Der Zugriff auf Nullwerte kann in manchen Fällen zu einem undefinierten Verhalten führen.
In der Regel ist er im Testmodus jedoch abgesichert und führt zu Fehlermeldungen (siehe "Klassen").

\section{Id-Literale}
Für Id-Literale gelten bei der Validierung selbige Regeln wie in der Sektion "Objekttypen-Literale" definiert. Es wird also
eine Warnung vom Compiler ausgegeben, falls sich die Id statisch überprüfen lässt und in keiner der entsprechenden Dateien
zu finden ist.

\section{Objekttypen-Literale}
Des Weiteren können Eigenschaften sogenannter Objekttypen, die einen bestimmten nativen, kopierbasierten Typ haben abgefragt werden.
Objekttypen erhalten in Warcraft 3 The Frozen Throne eine vierstellige, hexadezimale Id. Daher können Objekttypen im Wertebereich von 16^4 definiert werden.
Jede Id identifiziert genau einen Objekttyp.
Ein Objekttyp kann in JASS++ folgendes sein:
\begin{enumerate}
\item ein Einheitentyp
\item ein Gegenstandstyp
\item ein Doodad-Typ
\item ein Zerstörbares-Typ
\item ein Fähigkeitentyp
\item ein Zauberverstärkertyp
\item ein Forschungstyp
\item ein Wettereffekttyp
\item ein Blitzeffekttyp
\item ein Geländeset
\end{enumerate}

Die Objekttypen stammen aus eine der folgenden Dateien (aufsteigend nach Priorität geordnet):
\begin{enumerate}
\item war3map.u
\item war3map.d
\item UI/UnitData.slk
\item *
\item *
\end{enumerate}
TODO

Dabei werden die angegebenen Dateien in den folgenden Archiven in absteigender Reihenfolge durchsucht:
\begin{enumerate}
\item Karte
\item Kampagne
\item War3Patch.mpq
\item War3Locale.mpq (richtigen Namen finden)
\item War3X.mpq
\item War3.mpq
\end{enumerate}
TODO

Die Eigenschaften eines Objekttyps können einen der folgenden nativen, kopierbasierten Typen haben:
\begin{enumerate}
\item string
\item integer
\item real
\item boolean
\end{enumerate}
TODO

Die Eigenschaften eines Objekttyps haben selbst nochmals eine eigene Id.
Ein Objekttypen-Eigenschafts-Literal wird immer als ein Ausdruck eines der angegebenen Typen interpretiert.
Kann der Objekttypeintrag bzw. der Objekttyp-Eigenschaftseintrag nicht gefunden werden, so gibt der Compiler eine Warnung aus.
Das Literal ist in diesem Fall ein Nullwert ("null").
Wird keine Eigenschafts-Id angegeben, so wird der Ausdruck als Array interpretiert. Dies geht jedoch nur falls alle Objekttypen-Eigenschaften
den gleichen nativen, kopierbasierten Typ haben.
Ansonsten kann das Literal nur als eigener Typ interpertiert werden (siehe "Klassen - Zuweisungen").
Die Notation sieht folgendermaßen aus:
<<Objekttyp-Id>[, %]>
<<Objekttyp-Id>,<Eigenschafts-Id>[, %])>

Das Prozentzeichen ist optional und gibt an, dass Fließkommazahlen als Prozentanteile und daher als Ganzzahlen zwischen 0 und 100 interpretiert werden
sollen. Aus 0.30 würde in diesem Fall also die Ganzzahl 30 werden.
Es gilt bei einer reinen Angabe der Objekttyp-Id für alle Eigenschaften.
Es gilt zu beachten, dass Objekttypen-Ids auch als Escape-Sequenzen verwendet werden können.

\section{Farbliterale}
Farbliterale haben stets den nativen Typ "integer". Sie sind daher ein 32-Bit-Wert, welcher eine RGBA-Farbe enthält.
Der RGBA-Wert muss hexadezimal angegeben werden, wobei die ersten beiden Ziffern für den Alphawert, die zweiten beiden für den Grünwert,
die dritten beiden für den Blauwert und die letzten beiden für den Blauwert stehen.
Notation:
|c<Farbe>|r

\section{Escape-Sequenzen}
Eine Escape-Sequenz ist ein Zeichen, welchem das \-Zeichen vorangestellt wird und welches ausschließlich ein einem
Zeichenkettenliteral vorkommen darf.
Es gibt folgende Escape-Sequenzen:
\begin(enumerate}
\item n				Zeilenumbruch
\item t				Horizontaler Tabulator
\item \				Backslash
\item "				Doppeltes Anführungszeichen
\end{enumerate}

Zusätzlich existieren spezielle Escape-Sequenzen für Objekttypen-Ids (siehe "Literale - Objekttypen-Ids") und Farben (siehe "Literale - Farbliterale").
Farbkodierungen und Objekttypen-Ids in Zeichenketten werden zum Kompilierzeitpunkt validiert bzw. kompiliert.
Die speziellen Escapesequenzen werden in den Zeichenketten zum Kompilierzeitpunkt ersetzt, insofern sie gültig sind.
Falls nicht, bleibt der Ausdruck erhalten und der Compiler gibt eine Warnung aus.
Wie bei normalen Objektypen-Literalen kann ein Prozentzeichen angehängt werden.
Bei Farben kann jedoch, anders als bei gewöhnlichen Farbliteralen, Text eingeschlossen werden. Dieser sollte dann in der entsprechenden Farbe
im Spiel bei einer Ausgabe angezeigt werden.

\section{Notation}
|c<Farbe>[<Text>]|r
<<Objekt-Id>,<Feld-Id>[, %]>

	\chapter{Literale}
Ein Literal ist ein sich selbst erklärender Ausdruck, der nicht speziell benannt werden muss also keinen Bezeichner benötigt.
Für folgende native Typen gelten folgende Literalnotationsregeln:

Kopierbasierte Typen:
TODO: Exponentialschreibweise für alle Ganzzahlausdrücke erlauben und auch für Fließkommazahlen?
\begin{enumerate}
\item string : "<Zeichenkettenliteral>" | null
\item integer : 0<Oktalliteral> | 0%<Binärliteral> | 0x<Hexadezimalliteral> | <Dezimalliteral>[e <Exponent>] | '<Id-Literal>' | null
\item real : <Fließkommaliteral> [r | R] | null
\item boolean : true | false | null
\item handle (und Nicht-agent-Kindtypen): null
\end{enumerate}

Referenzbasierte Typen:
\begin{enumerate}
\item code : 0
\item agent (und Kindtypen) : 0
\item Klassenobjekte : 0
\item Funktionszeiger : 0
\end{enumerate}

Dabei können Zeichenkettenausdrücke mehrfach hintereinander geschrieben werden. Diese werden dann automatisch zu einem einzelnen
zusammengefasst. Dies vereinfacht die Aufteilung auf mehrere Zeilen, ohne unnötige Zeilenumbrüche und Einrückungen dazwischen und
dient einer besseren Übersicht:
"<Ausdruckteil 1>"
"<Ausdruckteil n>"

\section{Nullwerte}
Wie oben dargestellt erhalten kopierbasierte Typen einen Nullwert mit dem Literal "null" und referenzbasierte Typen mit dem Literal "0".
Der Zugriff auf Nullwerte kann in manchen Fällen zu einem undefinierten Verhalten führen.
In der Regel ist er im Testmodus jedoch abgesichert und führt zu Fehlermeldungen (siehe "Klassen").

\section{Id-Literale}
Für Id-Literale gelten bei der Validierung selbige Regeln wie in der Sektion "Objekttypen-Literale" definiert. Es wird also
eine Warnung vom Compiler ausgegeben, falls sich die Id statisch überprüfen lässt und in keiner der entsprechenden Dateien
zu finden ist.

\section{Objekttypen-Literale}
Des Weiteren können Eigenschaften sogenannter Objekttypen, die einen bestimmten nativen, kopierbasierten Typ haben abgefragt werden.
Objekttypen erhalten in Warcraft 3 The Frozen Throne eine vierstellige, hexadezimale Id. Daher können Objekttypen im Wertebereich von 16^4 definiert werden.
Jede Id identifiziert genau einen Objekttyp.
Ein Objekttyp kann in JASS++ folgendes sein:
\begin{enumerate}
\item ein Einheitentyp
\item ein Gegenstandstyp
\item ein Doodad-Typ
\item ein Zerstörbares-Typ
\item ein Fähigkeitentyp
\item ein Zauberverstärkertyp
\item ein Forschungstyp
\item ein Wettereffekttyp
\item ein Blitzeffekttyp
\item ein Geländeset
\end{enumerate}

Die Objekttypen stammen aus eine der folgenden Dateien (aufsteigend nach Priorität geordnet):
\begin{enumerate}
\item war3map.u
\item war3map.d
\item UI/UnitData.slk
\item *
\item *
\end{enumerate}
TODO

Dabei werden die angegebenen Dateien in den folgenden Archiven in absteigender Reihenfolge durchsucht:
\begin{enumerate}
\item Karte
\item Kampagne
\item War3Patch.mpq
\item War3Locale.mpq (richtigen Namen finden)
\item War3X.mpq
\item War3.mpq
\end{enumerate}
TODO

Die Eigenschaften eines Objekttyps können einen der folgenden nativen, kopierbasierten Typen haben:
\begin{enumerate}
\item string
\item integer
\item real
\item boolean
\end{enumerate}
TODO

Die Eigenschaften eines Objekttyps haben selbst nochmals eine eigene Id.
Ein Objekttypen-Eigenschafts-Literal wird immer als ein Ausdruck eines der angegebenen Typen interpretiert.
Kann der Objekttypeintrag bzw. der Objekttyp-Eigenschaftseintrag nicht gefunden werden, so gibt der Compiler eine Warnung aus.
Das Literal ist in diesem Fall ein Nullwert ("null").
Wird keine Eigenschafts-Id angegeben, so wird der Ausdruck als Array interpretiert. Dies geht jedoch nur falls alle Objekttypen-Eigenschaften
den gleichen nativen, kopierbasierten Typ haben.
Ansonsten kann das Literal nur als eigener Typ interpertiert werden (siehe "Klassen - Zuweisungen").
Die Notation sieht folgendermaßen aus:
<<Objekttyp-Id>[, %]>
<<Objekttyp-Id>,<Eigenschafts-Id>[, %])>

Das Prozentzeichen ist optional und gibt an, dass Fließkommazahlen als Prozentanteile und daher als Ganzzahlen zwischen 0 und 100 interpretiert werden
sollen. Aus 0.30 würde in diesem Fall also die Ganzzahl 30 werden.
Es gilt bei einer reinen Angabe der Objekttyp-Id für alle Eigenschaften.
Es gilt zu beachten, dass Objekttypen-Ids auch als Escape-Sequenzen verwendet werden können.

\section{Farbliterale}
Farbliterale haben stets den nativen Typ "integer". Sie sind daher ein 32-Bit-Wert, welcher eine RGBA-Farbe enthält.
Der RGBA-Wert muss hexadezimal angegeben werden, wobei die ersten beiden Ziffern für den Alphawert, die zweiten beiden für den Grünwert,
die dritten beiden für den Blauwert und die letzten beiden für den Blauwert stehen.
Notation:
|c<Farbe>|r

\section{Escape-Sequenzen}
Eine Escape-Sequenz ist ein Zeichen, welchem das \-Zeichen vorangestellt wird und welches ausschließlich ein einem
Zeichenkettenliteral vorkommen darf.
Es gibt folgende Escape-Sequenzen:
\begin(enumerate}
\item n				Zeilenumbruch
\item t				Horizontaler Tabulator
\item \				Backslash
\item "				Doppeltes Anführungszeichen
\end{enumerate}

Zusätzlich existieren spezielle Escape-Sequenzen für Objekttypen-Ids (siehe "Literale - Objekttypen-Ids") und Farben (siehe "Literale - Farbliterale").
Farbkodierungen und Objekttypen-Ids in Zeichenketten werden zum Kompilierzeitpunkt validiert bzw. kompiliert.
Die speziellen Escapesequenzen werden in den Zeichenketten zum Kompilierzeitpunkt ersetzt, insofern sie gültig sind.
Falls nicht, bleibt der Ausdruck erhalten und der Compiler gibt eine Warnung aus.
Wie bei normalen Objektypen-Literalen kann ein Prozentzeichen angehängt werden.
Bei Farben kann jedoch, anders als bei gewöhnlichen Farbliteralen, Text eingeschlossen werden. Dieser sollte dann in der entsprechenden Farbe
im Spiel bei einer Ausgabe angezeigt werden.

\section{Notation}
|c<Farbe>[<Text>]|r
<<Objekt-Id>,<Feld-Id>[, %]>

	\chapter{Literale}
Ein Literal ist ein sich selbst erklärender Ausdruck, der nicht speziell benannt werden muss also keinen Bezeichner benötigt.
Für folgende native Typen gelten folgende Literalnotationsregeln:

Kopierbasierte Typen:
TODO: Exponentialschreibweise für alle Ganzzahlausdrücke erlauben und auch für Fließkommazahlen?
\begin{enumerate}
\item string : "<Zeichenkettenliteral>" | null
\item integer : 0<Oktalliteral> | 0%<Binärliteral> | 0x<Hexadezimalliteral> | <Dezimalliteral>[e <Exponent>] | '<Id-Literal>' | null
\item real : <Fließkommaliteral> [r | R] | null
\item boolean : true | false | null
\item handle (und Nicht-agent-Kindtypen): null
\end{enumerate}

Referenzbasierte Typen:
\begin{enumerate}
\item code : 0
\item agent (und Kindtypen) : 0
\item Klassenobjekte : 0
\item Funktionszeiger : 0
\end{enumerate}

Dabei können Zeichenkettenausdrücke mehrfach hintereinander geschrieben werden. Diese werden dann automatisch zu einem einzelnen
zusammengefasst. Dies vereinfacht die Aufteilung auf mehrere Zeilen, ohne unnötige Zeilenumbrüche und Einrückungen dazwischen und
dient einer besseren Übersicht:
"<Ausdruckteil 1>"
"<Ausdruckteil n>"

\section{Nullwerte}
Wie oben dargestellt erhalten kopierbasierte Typen einen Nullwert mit dem Literal "null" und referenzbasierte Typen mit dem Literal "0".
Der Zugriff auf Nullwerte kann in manchen Fällen zu einem undefinierten Verhalten führen.
In der Regel ist er im Testmodus jedoch abgesichert und führt zu Fehlermeldungen (siehe "Klassen").

\section{Id-Literale}
Für Id-Literale gelten bei der Validierung selbige Regeln wie in der Sektion "Objekttypen-Literale" definiert. Es wird also
eine Warnung vom Compiler ausgegeben, falls sich die Id statisch überprüfen lässt und in keiner der entsprechenden Dateien
zu finden ist.

\section{Objekttypen-Literale}
Des Weiteren können Eigenschaften sogenannter Objekttypen, die einen bestimmten nativen, kopierbasierten Typ haben abgefragt werden.
Objekttypen erhalten in Warcraft 3 The Frozen Throne eine vierstellige, hexadezimale Id. Daher können Objekttypen im Wertebereich von 16^4 definiert werden.
Jede Id identifiziert genau einen Objekttyp.
Ein Objekttyp kann in JASS++ folgendes sein:
\begin{enumerate}
\item ein Einheitentyp
\item ein Gegenstandstyp
\item ein Doodad-Typ
\item ein Zerstörbares-Typ
\item ein Fähigkeitentyp
\item ein Zauberverstärkertyp
\item ein Forschungstyp
\item ein Wettereffekttyp
\item ein Blitzeffekttyp
\item ein Geländeset
\end{enumerate}

Die Objekttypen stammen aus eine der folgenden Dateien (aufsteigend nach Priorität geordnet):
\begin{enumerate}
\item war3map.u
\item war3map.d
\item UI/UnitData.slk
\item *
\item *
\end{enumerate}
TODO

Dabei werden die angegebenen Dateien in den folgenden Archiven in absteigender Reihenfolge durchsucht:
\begin{enumerate}
\item Karte
\item Kampagne
\item War3Patch.mpq
\item War3Locale.mpq (richtigen Namen finden)
\item War3X.mpq
\item War3.mpq
\end{enumerate}
TODO

Die Eigenschaften eines Objekttyps können einen der folgenden nativen, kopierbasierten Typen haben:
\begin{enumerate}
\item string
\item integer
\item real
\item boolean
\end{enumerate}
TODO

Die Eigenschaften eines Objekttyps haben selbst nochmals eine eigene Id.
Ein Objekttypen-Eigenschafts-Literal wird immer als ein Ausdruck eines der angegebenen Typen interpretiert.
Kann der Objekttypeintrag bzw. der Objekttyp-Eigenschaftseintrag nicht gefunden werden, so gibt der Compiler eine Warnung aus.
Das Literal ist in diesem Fall ein Nullwert ("null").
Wird keine Eigenschafts-Id angegeben, so wird der Ausdruck als Array interpretiert. Dies geht jedoch nur falls alle Objekttypen-Eigenschaften
den gleichen nativen, kopierbasierten Typ haben.
Ansonsten kann das Literal nur als eigener Typ interpertiert werden (siehe "Klassen - Zuweisungen").
Die Notation sieht folgendermaßen aus:
<<Objekttyp-Id>[, %]>
<<Objekttyp-Id>,<Eigenschafts-Id>[, %])>

Das Prozentzeichen ist optional und gibt an, dass Fließkommazahlen als Prozentanteile und daher als Ganzzahlen zwischen 0 und 100 interpretiert werden
sollen. Aus 0.30 würde in diesem Fall also die Ganzzahl 30 werden.
Es gilt bei einer reinen Angabe der Objekttyp-Id für alle Eigenschaften.
Es gilt zu beachten, dass Objekttypen-Ids auch als Escape-Sequenzen verwendet werden können.

\section{Farbliterale}
Farbliterale haben stets den nativen Typ "integer". Sie sind daher ein 32-Bit-Wert, welcher eine RGBA-Farbe enthält.
Der RGBA-Wert muss hexadezimal angegeben werden, wobei die ersten beiden Ziffern für den Alphawert, die zweiten beiden für den Grünwert,
die dritten beiden für den Blauwert und die letzten beiden für den Blauwert stehen.
Notation:
|c<Farbe>|r

\section{Escape-Sequenzen}
Eine Escape-Sequenz ist ein Zeichen, welchem das \-Zeichen vorangestellt wird und welches ausschließlich ein einem
Zeichenkettenliteral vorkommen darf.
Es gibt folgende Escape-Sequenzen:
\begin(enumerate}
\item n				Zeilenumbruch
\item t				Horizontaler Tabulator
\item \				Backslash
\item "				Doppeltes Anführungszeichen
\end{enumerate}

Zusätzlich existieren spezielle Escape-Sequenzen für Objekttypen-Ids (siehe "Literale - Objekttypen-Ids") und Farben (siehe "Literale - Farbliterale").
Farbkodierungen und Objekttypen-Ids in Zeichenketten werden zum Kompilierzeitpunkt validiert bzw. kompiliert.
Die speziellen Escapesequenzen werden in den Zeichenketten zum Kompilierzeitpunkt ersetzt, insofern sie gültig sind.
Falls nicht, bleibt der Ausdruck erhalten und der Compiler gibt eine Warnung aus.
Wie bei normalen Objektypen-Literalen kann ein Prozentzeichen angehängt werden.
Bei Farben kann jedoch, anders als bei gewöhnlichen Farbliteralen, Text eingeschlossen werden. Dieser sollte dann in der entsprechenden Farbe
im Spiel bei einer Ausgabe angezeigt werden.

\section{Notation}
|c<Farbe>[<Text>]|r
<<Objekt-Id>,<Feld-Id>[, %]>

	\chapter{Literale}
Ein Literal ist ein sich selbst erklärender Ausdruck, der nicht speziell benannt werden muss also keinen Bezeichner benötigt.
Für folgende native Typen gelten folgende Literalnotationsregeln:

Kopierbasierte Typen:
TODO: Exponentialschreibweise für alle Ganzzahlausdrücke erlauben und auch für Fließkommazahlen?
\begin{enumerate}
\item string : "<Zeichenkettenliteral>" | null
\item integer : 0<Oktalliteral> | 0%<Binärliteral> | 0x<Hexadezimalliteral> | <Dezimalliteral>[e <Exponent>] | '<Id-Literal>' | null
\item real : <Fließkommaliteral> [r | R] | null
\item boolean : true | false | null
\item handle (und Nicht-agent-Kindtypen): null
\end{enumerate}

Referenzbasierte Typen:
\begin{enumerate}
\item code : 0
\item agent (und Kindtypen) : 0
\item Klassenobjekte : 0
\item Funktionszeiger : 0
\end{enumerate}

Dabei können Zeichenkettenausdrücke mehrfach hintereinander geschrieben werden. Diese werden dann automatisch zu einem einzelnen
zusammengefasst. Dies vereinfacht die Aufteilung auf mehrere Zeilen, ohne unnötige Zeilenumbrüche und Einrückungen dazwischen und
dient einer besseren Übersicht:
"<Ausdruckteil 1>"
"<Ausdruckteil n>"

\section{Nullwerte}
Wie oben dargestellt erhalten kopierbasierte Typen einen Nullwert mit dem Literal "null" und referenzbasierte Typen mit dem Literal "0".
Der Zugriff auf Nullwerte kann in manchen Fällen zu einem undefinierten Verhalten führen.
In der Regel ist er im Testmodus jedoch abgesichert und führt zu Fehlermeldungen (siehe "Klassen").

\section{Id-Literale}
Für Id-Literale gelten bei der Validierung selbige Regeln wie in der Sektion "Objekttypen-Literale" definiert. Es wird also
eine Warnung vom Compiler ausgegeben, falls sich die Id statisch überprüfen lässt und in keiner der entsprechenden Dateien
zu finden ist.

\section{Objekttypen-Literale}
Des Weiteren können Eigenschaften sogenannter Objekttypen, die einen bestimmten nativen, kopierbasierten Typ haben abgefragt werden.
Objekttypen erhalten in Warcraft 3 The Frozen Throne eine vierstellige, hexadezimale Id. Daher können Objekttypen im Wertebereich von 16^4 definiert werden.
Jede Id identifiziert genau einen Objekttyp.
Ein Objekttyp kann in JASS++ folgendes sein:
\begin{enumerate}
\item ein Einheitentyp
\item ein Gegenstandstyp
\item ein Doodad-Typ
\item ein Zerstörbares-Typ
\item ein Fähigkeitentyp
\item ein Zauberverstärkertyp
\item ein Forschungstyp
\item ein Wettereffekttyp
\item ein Blitzeffekttyp
\item ein Geländeset
\end{enumerate}

Die Objekttypen stammen aus eine der folgenden Dateien (aufsteigend nach Priorität geordnet):
\begin{enumerate}
\item war3map.u
\item war3map.d
\item UI/UnitData.slk
\item *
\item *
\end{enumerate}
TODO

Dabei werden die angegebenen Dateien in den folgenden Archiven in absteigender Reihenfolge durchsucht:
\begin{enumerate}
\item Karte
\item Kampagne
\item War3Patch.mpq
\item War3Locale.mpq (richtigen Namen finden)
\item War3X.mpq
\item War3.mpq
\end{enumerate}
TODO

Die Eigenschaften eines Objekttyps können einen der folgenden nativen, kopierbasierten Typen haben:
\begin{enumerate}
\item string
\item integer
\item real
\item boolean
\end{enumerate}
TODO

Die Eigenschaften eines Objekttyps haben selbst nochmals eine eigene Id.
Ein Objekttypen-Eigenschafts-Literal wird immer als ein Ausdruck eines der angegebenen Typen interpretiert.
Kann der Objekttypeintrag bzw. der Objekttyp-Eigenschaftseintrag nicht gefunden werden, so gibt der Compiler eine Warnung aus.
Das Literal ist in diesem Fall ein Nullwert ("null").
Wird keine Eigenschafts-Id angegeben, so wird der Ausdruck als Array interpretiert. Dies geht jedoch nur falls alle Objekttypen-Eigenschaften
den gleichen nativen, kopierbasierten Typ haben.
Ansonsten kann das Literal nur als eigener Typ interpertiert werden (siehe "Klassen - Zuweisungen").
Die Notation sieht folgendermaßen aus:
<<Objekttyp-Id>[, %]>
<<Objekttyp-Id>,<Eigenschafts-Id>[, %])>

Das Prozentzeichen ist optional und gibt an, dass Fließkommazahlen als Prozentanteile und daher als Ganzzahlen zwischen 0 und 100 interpretiert werden
sollen. Aus 0.30 würde in diesem Fall also die Ganzzahl 30 werden.
Es gilt bei einer reinen Angabe der Objekttyp-Id für alle Eigenschaften.
Es gilt zu beachten, dass Objekttypen-Ids auch als Escape-Sequenzen verwendet werden können.

\section{Farbliterale}
Farbliterale haben stets den nativen Typ "integer". Sie sind daher ein 32-Bit-Wert, welcher eine RGBA-Farbe enthält.
Der RGBA-Wert muss hexadezimal angegeben werden, wobei die ersten beiden Ziffern für den Alphawert, die zweiten beiden für den Grünwert,
die dritten beiden für den Blauwert und die letzten beiden für den Blauwert stehen.
Notation:
|c<Farbe>|r

\section{Escape-Sequenzen}
Eine Escape-Sequenz ist ein Zeichen, welchem das \-Zeichen vorangestellt wird und welches ausschließlich ein einem
Zeichenkettenliteral vorkommen darf.
Es gibt folgende Escape-Sequenzen:
\begin(enumerate}
\item n				Zeilenumbruch
\item t				Horizontaler Tabulator
\item \				Backslash
\item "				Doppeltes Anführungszeichen
\end{enumerate}

Zusätzlich existieren spezielle Escape-Sequenzen für Objekttypen-Ids (siehe "Literale - Objekttypen-Ids") und Farben (siehe "Literale - Farbliterale").
Farbkodierungen und Objekttypen-Ids in Zeichenketten werden zum Kompilierzeitpunkt validiert bzw. kompiliert.
Die speziellen Escapesequenzen werden in den Zeichenketten zum Kompilierzeitpunkt ersetzt, insofern sie gültig sind.
Falls nicht, bleibt der Ausdruck erhalten und der Compiler gibt eine Warnung aus.
Wie bei normalen Objektypen-Literalen kann ein Prozentzeichen angehängt werden.
Bei Farben kann jedoch, anders als bei gewöhnlichen Farbliteralen, Text eingeschlossen werden. Dieser sollte dann in der entsprechenden Farbe
im Spiel bei einer Ausgabe angezeigt werden.

\section{Notation}
|c<Farbe>[<Text>]|r
<<Objekt-Id>,<Feld-Id>[, %]>

	\chapter{Literale}
Ein Literal ist ein sich selbst erklärender Ausdruck, der nicht speziell benannt werden muss also keinen Bezeichner benötigt.
Für folgende native Typen gelten folgende Literalnotationsregeln:

Kopierbasierte Typen:
TODO: Exponentialschreibweise für alle Ganzzahlausdrücke erlauben und auch für Fließkommazahlen?
\begin{enumerate}
\item string : "<Zeichenkettenliteral>" | null
\item integer : 0<Oktalliteral> | 0%<Binärliteral> | 0x<Hexadezimalliteral> | <Dezimalliteral>[e <Exponent>] | '<Id-Literal>' | null
\item real : <Fließkommaliteral> [r | R] | null
\item boolean : true | false | null
\item handle (und Nicht-agent-Kindtypen): null
\end{enumerate}

Referenzbasierte Typen:
\begin{enumerate}
\item code : 0
\item agent (und Kindtypen) : 0
\item Klassenobjekte : 0
\item Funktionszeiger : 0
\end{enumerate}

Dabei können Zeichenkettenausdrücke mehrfach hintereinander geschrieben werden. Diese werden dann automatisch zu einem einzelnen
zusammengefasst. Dies vereinfacht die Aufteilung auf mehrere Zeilen, ohne unnötige Zeilenumbrüche und Einrückungen dazwischen und
dient einer besseren Übersicht:
"<Ausdruckteil 1>"
"<Ausdruckteil n>"

\section{Nullwerte}
Wie oben dargestellt erhalten kopierbasierte Typen einen Nullwert mit dem Literal "null" und referenzbasierte Typen mit dem Literal "0".
Der Zugriff auf Nullwerte kann in manchen Fällen zu einem undefinierten Verhalten führen.
In der Regel ist er im Testmodus jedoch abgesichert und führt zu Fehlermeldungen (siehe "Klassen").

\section{Id-Literale}
Für Id-Literale gelten bei der Validierung selbige Regeln wie in der Sektion "Objekttypen-Literale" definiert. Es wird also
eine Warnung vom Compiler ausgegeben, falls sich die Id statisch überprüfen lässt und in keiner der entsprechenden Dateien
zu finden ist.

\section{Objekttypen-Literale}
Des Weiteren können Eigenschaften sogenannter Objekttypen, die einen bestimmten nativen, kopierbasierten Typ haben abgefragt werden.
Objekttypen erhalten in Warcraft 3 The Frozen Throne eine vierstellige, hexadezimale Id. Daher können Objekttypen im Wertebereich von 16^4 definiert werden.
Jede Id identifiziert genau einen Objekttyp.
Ein Objekttyp kann in JASS++ folgendes sein:
\begin{enumerate}
\item ein Einheitentyp
\item ein Gegenstandstyp
\item ein Doodad-Typ
\item ein Zerstörbares-Typ
\item ein Fähigkeitentyp
\item ein Zauberverstärkertyp
\item ein Forschungstyp
\item ein Wettereffekttyp
\item ein Blitzeffekttyp
\item ein Geländeset
\end{enumerate}

Die Objekttypen stammen aus eine der folgenden Dateien (aufsteigend nach Priorität geordnet):
\begin{enumerate}
\item war3map.u
\item war3map.d
\item UI/UnitData.slk
\item *
\item *
\end{enumerate}
TODO

Dabei werden die angegebenen Dateien in den folgenden Archiven in absteigender Reihenfolge durchsucht:
\begin{enumerate}
\item Karte
\item Kampagne
\item War3Patch.mpq
\item War3Locale.mpq (richtigen Namen finden)
\item War3X.mpq
\item War3.mpq
\end{enumerate}
TODO

Die Eigenschaften eines Objekttyps können einen der folgenden nativen, kopierbasierten Typen haben:
\begin{enumerate}
\item string
\item integer
\item real
\item boolean
\end{enumerate}
TODO

Die Eigenschaften eines Objekttyps haben selbst nochmals eine eigene Id.
Ein Objekttypen-Eigenschafts-Literal wird immer als ein Ausdruck eines der angegebenen Typen interpretiert.
Kann der Objekttypeintrag bzw. der Objekttyp-Eigenschaftseintrag nicht gefunden werden, so gibt der Compiler eine Warnung aus.
Das Literal ist in diesem Fall ein Nullwert ("null").
Wird keine Eigenschafts-Id angegeben, so wird der Ausdruck als Array interpretiert. Dies geht jedoch nur falls alle Objekttypen-Eigenschaften
den gleichen nativen, kopierbasierten Typ haben.
Ansonsten kann das Literal nur als eigener Typ interpertiert werden (siehe "Klassen - Zuweisungen").
Die Notation sieht folgendermaßen aus:
<<Objekttyp-Id>[, %]>
<<Objekttyp-Id>,<Eigenschafts-Id>[, %])>

Das Prozentzeichen ist optional und gibt an, dass Fließkommazahlen als Prozentanteile und daher als Ganzzahlen zwischen 0 und 100 interpretiert werden
sollen. Aus 0.30 würde in diesem Fall also die Ganzzahl 30 werden.
Es gilt bei einer reinen Angabe der Objekttyp-Id für alle Eigenschaften.
Es gilt zu beachten, dass Objekttypen-Ids auch als Escape-Sequenzen verwendet werden können.

\section{Farbliterale}
Farbliterale haben stets den nativen Typ "integer". Sie sind daher ein 32-Bit-Wert, welcher eine RGBA-Farbe enthält.
Der RGBA-Wert muss hexadezimal angegeben werden, wobei die ersten beiden Ziffern für den Alphawert, die zweiten beiden für den Grünwert,
die dritten beiden für den Blauwert und die letzten beiden für den Blauwert stehen.
Notation:
|c<Farbe>|r

\section{Escape-Sequenzen}
Eine Escape-Sequenz ist ein Zeichen, welchem das \-Zeichen vorangestellt wird und welches ausschließlich ein einem
Zeichenkettenliteral vorkommen darf.
Es gibt folgende Escape-Sequenzen:
\begin(enumerate}
\item n				Zeilenumbruch
\item t				Horizontaler Tabulator
\item \				Backslash
\item "				Doppeltes Anführungszeichen
\end{enumerate}

Zusätzlich existieren spezielle Escape-Sequenzen für Objekttypen-Ids (siehe "Literale - Objekttypen-Ids") und Farben (siehe "Literale - Farbliterale").
Farbkodierungen und Objekttypen-Ids in Zeichenketten werden zum Kompilierzeitpunkt validiert bzw. kompiliert.
Die speziellen Escapesequenzen werden in den Zeichenketten zum Kompilierzeitpunkt ersetzt, insofern sie gültig sind.
Falls nicht, bleibt der Ausdruck erhalten und der Compiler gibt eine Warnung aus.
Wie bei normalen Objektypen-Literalen kann ein Prozentzeichen angehängt werden.
Bei Farben kann jedoch, anders als bei gewöhnlichen Farbliteralen, Text eingeschlossen werden. Dieser sollte dann in der entsprechenden Farbe
im Spiel bei einer Ausgabe angezeigt werden.

\section{Notation}
|c<Farbe>[<Text>]|r
<<Objekt-Id>,<Feld-Id>[, %]>

	\chapter{Literale}
Ein Literal ist ein sich selbst erklärender Ausdruck, der nicht speziell benannt werden muss also keinen Bezeichner benötigt.
Für folgende native Typen gelten folgende Literalnotationsregeln:

Kopierbasierte Typen:
TODO: Exponentialschreibweise für alle Ganzzahlausdrücke erlauben und auch für Fließkommazahlen?
\begin{enumerate}
\item string : "<Zeichenkettenliteral>" | null
\item integer : 0<Oktalliteral> | 0%<Binärliteral> | 0x<Hexadezimalliteral> | <Dezimalliteral>[e <Exponent>] | '<Id-Literal>' | null
\item real : <Fließkommaliteral> [r | R] | null
\item boolean : true | false | null
\item handle (und Nicht-agent-Kindtypen): null
\end{enumerate}

Referenzbasierte Typen:
\begin{enumerate}
\item code : 0
\item agent (und Kindtypen) : 0
\item Klassenobjekte : 0
\item Funktionszeiger : 0
\end{enumerate}

Dabei können Zeichenkettenausdrücke mehrfach hintereinander geschrieben werden. Diese werden dann automatisch zu einem einzelnen
zusammengefasst. Dies vereinfacht die Aufteilung auf mehrere Zeilen, ohne unnötige Zeilenumbrüche und Einrückungen dazwischen und
dient einer besseren Übersicht:
"<Ausdruckteil 1>"
"<Ausdruckteil n>"

\section{Nullwerte}
Wie oben dargestellt erhalten kopierbasierte Typen einen Nullwert mit dem Literal "null" und referenzbasierte Typen mit dem Literal "0".
Der Zugriff auf Nullwerte kann in manchen Fällen zu einem undefinierten Verhalten führen.
In der Regel ist er im Testmodus jedoch abgesichert und führt zu Fehlermeldungen (siehe "Klassen").

\section{Id-Literale}
Für Id-Literale gelten bei der Validierung selbige Regeln wie in der Sektion "Objekttypen-Literale" definiert. Es wird also
eine Warnung vom Compiler ausgegeben, falls sich die Id statisch überprüfen lässt und in keiner der entsprechenden Dateien
zu finden ist.

\section{Objekttypen-Literale}
Des Weiteren können Eigenschaften sogenannter Objekttypen, die einen bestimmten nativen, kopierbasierten Typ haben abgefragt werden.
Objekttypen erhalten in Warcraft 3 The Frozen Throne eine vierstellige, hexadezimale Id. Daher können Objekttypen im Wertebereich von 16^4 definiert werden.
Jede Id identifiziert genau einen Objekttyp.
Ein Objekttyp kann in JASS++ folgendes sein:
\begin{enumerate}
\item ein Einheitentyp
\item ein Gegenstandstyp
\item ein Doodad-Typ
\item ein Zerstörbares-Typ
\item ein Fähigkeitentyp
\item ein Zauberverstärkertyp
\item ein Forschungstyp
\item ein Wettereffekttyp
\item ein Blitzeffekttyp
\item ein Geländeset
\end{enumerate}

Die Objekttypen stammen aus eine der folgenden Dateien (aufsteigend nach Priorität geordnet):
\begin{enumerate}
\item war3map.u
\item war3map.d
\item UI/UnitData.slk
\item *
\item *
\end{enumerate}
TODO

Dabei werden die angegebenen Dateien in den folgenden Archiven in absteigender Reihenfolge durchsucht:
\begin{enumerate}
\item Karte
\item Kampagne
\item War3Patch.mpq
\item War3Locale.mpq (richtigen Namen finden)
\item War3X.mpq
\item War3.mpq
\end{enumerate}
TODO

Die Eigenschaften eines Objekttyps können einen der folgenden nativen, kopierbasierten Typen haben:
\begin{enumerate}
\item string
\item integer
\item real
\item boolean
\end{enumerate}
TODO

Die Eigenschaften eines Objekttyps haben selbst nochmals eine eigene Id.
Ein Objekttypen-Eigenschafts-Literal wird immer als ein Ausdruck eines der angegebenen Typen interpretiert.
Kann der Objekttypeintrag bzw. der Objekttyp-Eigenschaftseintrag nicht gefunden werden, so gibt der Compiler eine Warnung aus.
Das Literal ist in diesem Fall ein Nullwert ("null").
Wird keine Eigenschafts-Id angegeben, so wird der Ausdruck als Array interpretiert. Dies geht jedoch nur falls alle Objekttypen-Eigenschaften
den gleichen nativen, kopierbasierten Typ haben.
Ansonsten kann das Literal nur als eigener Typ interpertiert werden (siehe "Klassen - Zuweisungen").
Die Notation sieht folgendermaßen aus:
<<Objekttyp-Id>[, %]>
<<Objekttyp-Id>,<Eigenschafts-Id>[, %])>

Das Prozentzeichen ist optional und gibt an, dass Fließkommazahlen als Prozentanteile und daher als Ganzzahlen zwischen 0 und 100 interpretiert werden
sollen. Aus 0.30 würde in diesem Fall also die Ganzzahl 30 werden.
Es gilt bei einer reinen Angabe der Objekttyp-Id für alle Eigenschaften.
Es gilt zu beachten, dass Objekttypen-Ids auch als Escape-Sequenzen verwendet werden können.

\section{Farbliterale}
Farbliterale haben stets den nativen Typ "integer". Sie sind daher ein 32-Bit-Wert, welcher eine RGBA-Farbe enthält.
Der RGBA-Wert muss hexadezimal angegeben werden, wobei die ersten beiden Ziffern für den Alphawert, die zweiten beiden für den Grünwert,
die dritten beiden für den Blauwert und die letzten beiden für den Blauwert stehen.
Notation:
|c<Farbe>|r

\section{Escape-Sequenzen}
Eine Escape-Sequenz ist ein Zeichen, welchem das \-Zeichen vorangestellt wird und welches ausschließlich ein einem
Zeichenkettenliteral vorkommen darf.
Es gibt folgende Escape-Sequenzen:
\begin(enumerate}
\item n				Zeilenumbruch
\item t				Horizontaler Tabulator
\item \				Backslash
\item "				Doppeltes Anführungszeichen
\end{enumerate}

Zusätzlich existieren spezielle Escape-Sequenzen für Objekttypen-Ids (siehe "Literale - Objekttypen-Ids") und Farben (siehe "Literale - Farbliterale").
Farbkodierungen und Objekttypen-Ids in Zeichenketten werden zum Kompilierzeitpunkt validiert bzw. kompiliert.
Die speziellen Escapesequenzen werden in den Zeichenketten zum Kompilierzeitpunkt ersetzt, insofern sie gültig sind.
Falls nicht, bleibt der Ausdruck erhalten und der Compiler gibt eine Warnung aus.
Wie bei normalen Objektypen-Literalen kann ein Prozentzeichen angehängt werden.
Bei Farben kann jedoch, anders als bei gewöhnlichen Farbliteralen, Text eingeschlossen werden. Dieser sollte dann in der entsprechenden Farbe
im Spiel bei einer Ausgabe angezeigt werden.

\section{Notation}
|c<Farbe>[<Text>]|r
<<Objekt-Id>,<Feld-Id>[, %]>


	\chapter{Literale}
Ein Literal ist ein sich selbst erklärender Ausdruck, der nicht speziell benannt werden muss also keinen Bezeichner benötigt.
Für folgende native Typen gelten folgende Literalnotationsregeln:

Kopierbasierte Typen:
TODO: Exponentialschreibweise für alle Ganzzahlausdrücke erlauben und auch für Fließkommazahlen?
\begin{enumerate}
\item string : "<Zeichenkettenliteral>" | null
\item integer : 0<Oktalliteral> | 0%<Binärliteral> | 0x<Hexadezimalliteral> | <Dezimalliteral>[e <Exponent>] | '<Id-Literal>' | null
\item real : <Fließkommaliteral> [r | R] | null
\item boolean : true | false | null
\item handle (und Nicht-agent-Kindtypen): null
\end{enumerate}

Referenzbasierte Typen:
\begin{enumerate}
\item code : 0
\item agent (und Kindtypen) : 0
\item Klassenobjekte : 0
\item Funktionszeiger : 0
\end{enumerate}

Dabei können Zeichenkettenausdrücke mehrfach hintereinander geschrieben werden. Diese werden dann automatisch zu einem einzelnen
zusammengefasst. Dies vereinfacht die Aufteilung auf mehrere Zeilen, ohne unnötige Zeilenumbrüche und Einrückungen dazwischen und
dient einer besseren Übersicht:
"<Ausdruckteil 1>"
"<Ausdruckteil n>"

\section{Nullwerte}
Wie oben dargestellt erhalten kopierbasierte Typen einen Nullwert mit dem Literal "null" und referenzbasierte Typen mit dem Literal "0".
Der Zugriff auf Nullwerte kann in manchen Fällen zu einem undefinierten Verhalten führen.
In der Regel ist er im Testmodus jedoch abgesichert und führt zu Fehlermeldungen (siehe "Klassen").

\section{Id-Literale}
Für Id-Literale gelten bei der Validierung selbige Regeln wie in der Sektion "Objekttypen-Literale" definiert. Es wird also
eine Warnung vom Compiler ausgegeben, falls sich die Id statisch überprüfen lässt und in keiner der entsprechenden Dateien
zu finden ist.

\section{Objekttypen-Literale}
Des Weiteren können Eigenschaften sogenannter Objekttypen, die einen bestimmten nativen, kopierbasierten Typ haben abgefragt werden.
Objekttypen erhalten in Warcraft 3 The Frozen Throne eine vierstellige, hexadezimale Id. Daher können Objekttypen im Wertebereich von 16^4 definiert werden.
Jede Id identifiziert genau einen Objekttyp.
Ein Objekttyp kann in JASS++ folgendes sein:
\begin{enumerate}
\item ein Einheitentyp
\item ein Gegenstandstyp
\item ein Doodad-Typ
\item ein Zerstörbares-Typ
\item ein Fähigkeitentyp
\item ein Zauberverstärkertyp
\item ein Forschungstyp
\item ein Wettereffekttyp
\item ein Blitzeffekttyp
\item ein Geländeset
\end{enumerate}

Die Objekttypen stammen aus eine der folgenden Dateien (aufsteigend nach Priorität geordnet):
\begin{enumerate}
\item war3map.u
\item war3map.d
\item UI/UnitData.slk
\item *
\item *
\end{enumerate}
TODO

Dabei werden die angegebenen Dateien in den folgenden Archiven in absteigender Reihenfolge durchsucht:
\begin{enumerate}
\item Karte
\item Kampagne
\item War3Patch.mpq
\item War3Locale.mpq (richtigen Namen finden)
\item War3X.mpq
\item War3.mpq
\end{enumerate}
TODO

Die Eigenschaften eines Objekttyps können einen der folgenden nativen, kopierbasierten Typen haben:
\begin{enumerate}
\item string
\item integer
\item real
\item boolean
\end{enumerate}
TODO

Die Eigenschaften eines Objekttyps haben selbst nochmals eine eigene Id.
Ein Objekttypen-Eigenschafts-Literal wird immer als ein Ausdruck eines der angegebenen Typen interpretiert.
Kann der Objekttypeintrag bzw. der Objekttyp-Eigenschaftseintrag nicht gefunden werden, so gibt der Compiler eine Warnung aus.
Das Literal ist in diesem Fall ein Nullwert ("null").
Wird keine Eigenschafts-Id angegeben, so wird der Ausdruck als Array interpretiert. Dies geht jedoch nur falls alle Objekttypen-Eigenschaften
den gleichen nativen, kopierbasierten Typ haben.
Ansonsten kann das Literal nur als eigener Typ interpertiert werden (siehe "Klassen - Zuweisungen").
Die Notation sieht folgendermaßen aus:
<<Objekttyp-Id>[, %]>
<<Objekttyp-Id>,<Eigenschafts-Id>[, %])>

Das Prozentzeichen ist optional und gibt an, dass Fließkommazahlen als Prozentanteile und daher als Ganzzahlen zwischen 0 und 100 interpretiert werden
sollen. Aus 0.30 würde in diesem Fall also die Ganzzahl 30 werden.
Es gilt bei einer reinen Angabe der Objekttyp-Id für alle Eigenschaften.
Es gilt zu beachten, dass Objekttypen-Ids auch als Escape-Sequenzen verwendet werden können.

\section{Farbliterale}
Farbliterale haben stets den nativen Typ "integer". Sie sind daher ein 32-Bit-Wert, welcher eine RGBA-Farbe enthält.
Der RGBA-Wert muss hexadezimal angegeben werden, wobei die ersten beiden Ziffern für den Alphawert, die zweiten beiden für den Grünwert,
die dritten beiden für den Blauwert und die letzten beiden für den Blauwert stehen.
Notation:
|c<Farbe>|r

\section{Escape-Sequenzen}
Eine Escape-Sequenz ist ein Zeichen, welchem das \-Zeichen vorangestellt wird und welches ausschließlich ein einem
Zeichenkettenliteral vorkommen darf.
Es gibt folgende Escape-Sequenzen:
\begin(enumerate}
\item n				Zeilenumbruch
\item t				Horizontaler Tabulator
\item \				Backslash
\item "				Doppeltes Anführungszeichen
\end{enumerate}

Zusätzlich existieren spezielle Escape-Sequenzen für Objekttypen-Ids (siehe "Literale - Objekttypen-Ids") und Farben (siehe "Literale - Farbliterale").
Farbkodierungen und Objekttypen-Ids in Zeichenketten werden zum Kompilierzeitpunkt validiert bzw. kompiliert.
Die speziellen Escapesequenzen werden in den Zeichenketten zum Kompilierzeitpunkt ersetzt, insofern sie gültig sind.
Falls nicht, bleibt der Ausdruck erhalten und der Compiler gibt eine Warnung aus.
Wie bei normalen Objektypen-Literalen kann ein Prozentzeichen angehängt werden.
Bei Farben kann jedoch, anders als bei gewöhnlichen Farbliteralen, Text eingeschlossen werden. Dieser sollte dann in der entsprechenden Farbe
im Spiel bei einer Ausgabe angezeigt werden.

\section{Notation}
|c<Farbe>[<Text>]|r
<<Objekt-Id>,<Feld-Id>[, %]>

	\chapter{Literale}
Ein Literal ist ein sich selbst erklärender Ausdruck, der nicht speziell benannt werden muss also keinen Bezeichner benötigt.
Für folgende native Typen gelten folgende Literalnotationsregeln:

Kopierbasierte Typen:
TODO: Exponentialschreibweise für alle Ganzzahlausdrücke erlauben und auch für Fließkommazahlen?
\begin{enumerate}
\item string : "<Zeichenkettenliteral>" | null
\item integer : 0<Oktalliteral> | 0%<Binärliteral> | 0x<Hexadezimalliteral> | <Dezimalliteral>[e <Exponent>] | '<Id-Literal>' | null
\item real : <Fließkommaliteral> [r | R] | null
\item boolean : true | false | null
\item handle (und Nicht-agent-Kindtypen): null
\end{enumerate}

Referenzbasierte Typen:
\begin{enumerate}
\item code : 0
\item agent (und Kindtypen) : 0
\item Klassenobjekte : 0
\item Funktionszeiger : 0
\end{enumerate}

Dabei können Zeichenkettenausdrücke mehrfach hintereinander geschrieben werden. Diese werden dann automatisch zu einem einzelnen
zusammengefasst. Dies vereinfacht die Aufteilung auf mehrere Zeilen, ohne unnötige Zeilenumbrüche und Einrückungen dazwischen und
dient einer besseren Übersicht:
"<Ausdruckteil 1>"
"<Ausdruckteil n>"

\section{Nullwerte}
Wie oben dargestellt erhalten kopierbasierte Typen einen Nullwert mit dem Literal "null" und referenzbasierte Typen mit dem Literal "0".
Der Zugriff auf Nullwerte kann in manchen Fällen zu einem undefinierten Verhalten führen.
In der Regel ist er im Testmodus jedoch abgesichert und führt zu Fehlermeldungen (siehe "Klassen").

\section{Id-Literale}
Für Id-Literale gelten bei der Validierung selbige Regeln wie in der Sektion "Objekttypen-Literale" definiert. Es wird also
eine Warnung vom Compiler ausgegeben, falls sich die Id statisch überprüfen lässt und in keiner der entsprechenden Dateien
zu finden ist.

\section{Objekttypen-Literale}
Des Weiteren können Eigenschaften sogenannter Objekttypen, die einen bestimmten nativen, kopierbasierten Typ haben abgefragt werden.
Objekttypen erhalten in Warcraft 3 The Frozen Throne eine vierstellige, hexadezimale Id. Daher können Objekttypen im Wertebereich von 16^4 definiert werden.
Jede Id identifiziert genau einen Objekttyp.
Ein Objekttyp kann in JASS++ folgendes sein:
\begin{enumerate}
\item ein Einheitentyp
\item ein Gegenstandstyp
\item ein Doodad-Typ
\item ein Zerstörbares-Typ
\item ein Fähigkeitentyp
\item ein Zauberverstärkertyp
\item ein Forschungstyp
\item ein Wettereffekttyp
\item ein Blitzeffekttyp
\item ein Geländeset
\end{enumerate}

Die Objekttypen stammen aus eine der folgenden Dateien (aufsteigend nach Priorität geordnet):
\begin{enumerate}
\item war3map.u
\item war3map.d
\item UI/UnitData.slk
\item *
\item *
\end{enumerate}
TODO

Dabei werden die angegebenen Dateien in den folgenden Archiven in absteigender Reihenfolge durchsucht:
\begin{enumerate}
\item Karte
\item Kampagne
\item War3Patch.mpq
\item War3Locale.mpq (richtigen Namen finden)
\item War3X.mpq
\item War3.mpq
\end{enumerate}
TODO

Die Eigenschaften eines Objekttyps können einen der folgenden nativen, kopierbasierten Typen haben:
\begin{enumerate}
\item string
\item integer
\item real
\item boolean
\end{enumerate}
TODO

Die Eigenschaften eines Objekttyps haben selbst nochmals eine eigene Id.
Ein Objekttypen-Eigenschafts-Literal wird immer als ein Ausdruck eines der angegebenen Typen interpretiert.
Kann der Objekttypeintrag bzw. der Objekttyp-Eigenschaftseintrag nicht gefunden werden, so gibt der Compiler eine Warnung aus.
Das Literal ist in diesem Fall ein Nullwert ("null").
Wird keine Eigenschafts-Id angegeben, so wird der Ausdruck als Array interpretiert. Dies geht jedoch nur falls alle Objekttypen-Eigenschaften
den gleichen nativen, kopierbasierten Typ haben.
Ansonsten kann das Literal nur als eigener Typ interpertiert werden (siehe "Klassen - Zuweisungen").
Die Notation sieht folgendermaßen aus:
<<Objekttyp-Id>[, %]>
<<Objekttyp-Id>,<Eigenschafts-Id>[, %])>

Das Prozentzeichen ist optional und gibt an, dass Fließkommazahlen als Prozentanteile und daher als Ganzzahlen zwischen 0 und 100 interpretiert werden
sollen. Aus 0.30 würde in diesem Fall also die Ganzzahl 30 werden.
Es gilt bei einer reinen Angabe der Objekttyp-Id für alle Eigenschaften.
Es gilt zu beachten, dass Objekttypen-Ids auch als Escape-Sequenzen verwendet werden können.

\section{Farbliterale}
Farbliterale haben stets den nativen Typ "integer". Sie sind daher ein 32-Bit-Wert, welcher eine RGBA-Farbe enthält.
Der RGBA-Wert muss hexadezimal angegeben werden, wobei die ersten beiden Ziffern für den Alphawert, die zweiten beiden für den Grünwert,
die dritten beiden für den Blauwert und die letzten beiden für den Blauwert stehen.
Notation:
|c<Farbe>|r

\section{Escape-Sequenzen}
Eine Escape-Sequenz ist ein Zeichen, welchem das \-Zeichen vorangestellt wird und welches ausschließlich ein einem
Zeichenkettenliteral vorkommen darf.
Es gibt folgende Escape-Sequenzen:
\begin(enumerate}
\item n				Zeilenumbruch
\item t				Horizontaler Tabulator
\item \				Backslash
\item "				Doppeltes Anführungszeichen
\end{enumerate}

Zusätzlich existieren spezielle Escape-Sequenzen für Objekttypen-Ids (siehe "Literale - Objekttypen-Ids") und Farben (siehe "Literale - Farbliterale").
Farbkodierungen und Objekttypen-Ids in Zeichenketten werden zum Kompilierzeitpunkt validiert bzw. kompiliert.
Die speziellen Escapesequenzen werden in den Zeichenketten zum Kompilierzeitpunkt ersetzt, insofern sie gültig sind.
Falls nicht, bleibt der Ausdruck erhalten und der Compiler gibt eine Warnung aus.
Wie bei normalen Objektypen-Literalen kann ein Prozentzeichen angehängt werden.
Bei Farben kann jedoch, anders als bei gewöhnlichen Farbliteralen, Text eingeschlossen werden. Dieser sollte dann in der entsprechenden Farbe
im Spiel bei einer Ausgabe angezeigt werden.

\section{Notation}
|c<Farbe>[<Text>]|r
<<Objekt-Id>,<Feld-Id>[, %]>

	\chapter{Literale}
Ein Literal ist ein sich selbst erklärender Ausdruck, der nicht speziell benannt werden muss also keinen Bezeichner benötigt.
Für folgende native Typen gelten folgende Literalnotationsregeln:

Kopierbasierte Typen:
TODO: Exponentialschreibweise für alle Ganzzahlausdrücke erlauben und auch für Fließkommazahlen?
\begin{enumerate}
\item string : "<Zeichenkettenliteral>" | null
\item integer : 0<Oktalliteral> | 0%<Binärliteral> | 0x<Hexadezimalliteral> | <Dezimalliteral>[e <Exponent>] | '<Id-Literal>' | null
\item real : <Fließkommaliteral> [r | R] | null
\item boolean : true | false | null
\item handle (und Nicht-agent-Kindtypen): null
\end{enumerate}

Referenzbasierte Typen:
\begin{enumerate}
\item code : 0
\item agent (und Kindtypen) : 0
\item Klassenobjekte : 0
\item Funktionszeiger : 0
\end{enumerate}

Dabei können Zeichenkettenausdrücke mehrfach hintereinander geschrieben werden. Diese werden dann automatisch zu einem einzelnen
zusammengefasst. Dies vereinfacht die Aufteilung auf mehrere Zeilen, ohne unnötige Zeilenumbrüche und Einrückungen dazwischen und
dient einer besseren Übersicht:
"<Ausdruckteil 1>"
"<Ausdruckteil n>"

\section{Nullwerte}
Wie oben dargestellt erhalten kopierbasierte Typen einen Nullwert mit dem Literal "null" und referenzbasierte Typen mit dem Literal "0".
Der Zugriff auf Nullwerte kann in manchen Fällen zu einem undefinierten Verhalten führen.
In der Regel ist er im Testmodus jedoch abgesichert und führt zu Fehlermeldungen (siehe "Klassen").

\section{Id-Literale}
Für Id-Literale gelten bei der Validierung selbige Regeln wie in der Sektion "Objekttypen-Literale" definiert. Es wird also
eine Warnung vom Compiler ausgegeben, falls sich die Id statisch überprüfen lässt und in keiner der entsprechenden Dateien
zu finden ist.

\section{Objekttypen-Literale}
Des Weiteren können Eigenschaften sogenannter Objekttypen, die einen bestimmten nativen, kopierbasierten Typ haben abgefragt werden.
Objekttypen erhalten in Warcraft 3 The Frozen Throne eine vierstellige, hexadezimale Id. Daher können Objekttypen im Wertebereich von 16^4 definiert werden.
Jede Id identifiziert genau einen Objekttyp.
Ein Objekttyp kann in JASS++ folgendes sein:
\begin{enumerate}
\item ein Einheitentyp
\item ein Gegenstandstyp
\item ein Doodad-Typ
\item ein Zerstörbares-Typ
\item ein Fähigkeitentyp
\item ein Zauberverstärkertyp
\item ein Forschungstyp
\item ein Wettereffekttyp
\item ein Blitzeffekttyp
\item ein Geländeset
\end{enumerate}

Die Objekttypen stammen aus eine der folgenden Dateien (aufsteigend nach Priorität geordnet):
\begin{enumerate}
\item war3map.u
\item war3map.d
\item UI/UnitData.slk
\item *
\item *
\end{enumerate}
TODO

Dabei werden die angegebenen Dateien in den folgenden Archiven in absteigender Reihenfolge durchsucht:
\begin{enumerate}
\item Karte
\item Kampagne
\item War3Patch.mpq
\item War3Locale.mpq (richtigen Namen finden)
\item War3X.mpq
\item War3.mpq
\end{enumerate}
TODO

Die Eigenschaften eines Objekttyps können einen der folgenden nativen, kopierbasierten Typen haben:
\begin{enumerate}
\item string
\item integer
\item real
\item boolean
\end{enumerate}
TODO

Die Eigenschaften eines Objekttyps haben selbst nochmals eine eigene Id.
Ein Objekttypen-Eigenschafts-Literal wird immer als ein Ausdruck eines der angegebenen Typen interpretiert.
Kann der Objekttypeintrag bzw. der Objekttyp-Eigenschaftseintrag nicht gefunden werden, so gibt der Compiler eine Warnung aus.
Das Literal ist in diesem Fall ein Nullwert ("null").
Wird keine Eigenschafts-Id angegeben, so wird der Ausdruck als Array interpretiert. Dies geht jedoch nur falls alle Objekttypen-Eigenschaften
den gleichen nativen, kopierbasierten Typ haben.
Ansonsten kann das Literal nur als eigener Typ interpertiert werden (siehe "Klassen - Zuweisungen").
Die Notation sieht folgendermaßen aus:
<<Objekttyp-Id>[, %]>
<<Objekttyp-Id>,<Eigenschafts-Id>[, %])>

Das Prozentzeichen ist optional und gibt an, dass Fließkommazahlen als Prozentanteile und daher als Ganzzahlen zwischen 0 und 100 interpretiert werden
sollen. Aus 0.30 würde in diesem Fall also die Ganzzahl 30 werden.
Es gilt bei einer reinen Angabe der Objekttyp-Id für alle Eigenschaften.
Es gilt zu beachten, dass Objekttypen-Ids auch als Escape-Sequenzen verwendet werden können.

\section{Farbliterale}
Farbliterale haben stets den nativen Typ "integer". Sie sind daher ein 32-Bit-Wert, welcher eine RGBA-Farbe enthält.
Der RGBA-Wert muss hexadezimal angegeben werden, wobei die ersten beiden Ziffern für den Alphawert, die zweiten beiden für den Grünwert,
die dritten beiden für den Blauwert und die letzten beiden für den Blauwert stehen.
Notation:
|c<Farbe>|r

\section{Escape-Sequenzen}
Eine Escape-Sequenz ist ein Zeichen, welchem das \-Zeichen vorangestellt wird und welches ausschließlich ein einem
Zeichenkettenliteral vorkommen darf.
Es gibt folgende Escape-Sequenzen:
\begin(enumerate}
\item n				Zeilenumbruch
\item t				Horizontaler Tabulator
\item \				Backslash
\item "				Doppeltes Anführungszeichen
\end{enumerate}

Zusätzlich existieren spezielle Escape-Sequenzen für Objekttypen-Ids (siehe "Literale - Objekttypen-Ids") und Farben (siehe "Literale - Farbliterale").
Farbkodierungen und Objekttypen-Ids in Zeichenketten werden zum Kompilierzeitpunkt validiert bzw. kompiliert.
Die speziellen Escapesequenzen werden in den Zeichenketten zum Kompilierzeitpunkt ersetzt, insofern sie gültig sind.
Falls nicht, bleibt der Ausdruck erhalten und der Compiler gibt eine Warnung aus.
Wie bei normalen Objektypen-Literalen kann ein Prozentzeichen angehängt werden.
Bei Farben kann jedoch, anders als bei gewöhnlichen Farbliteralen, Text eingeschlossen werden. Dieser sollte dann in der entsprechenden Farbe
im Spiel bei einer Ausgabe angezeigt werden.

\section{Notation}
|c<Farbe>[<Text>]|r
<<Objekt-Id>,<Feld-Id>[, %]>

	\chapter{Literale}
Ein Literal ist ein sich selbst erklärender Ausdruck, der nicht speziell benannt werden muss also keinen Bezeichner benötigt.
Für folgende native Typen gelten folgende Literalnotationsregeln:

Kopierbasierte Typen:
TODO: Exponentialschreibweise für alle Ganzzahlausdrücke erlauben und auch für Fließkommazahlen?
\begin{enumerate}
\item string : "<Zeichenkettenliteral>" | null
\item integer : 0<Oktalliteral> | 0%<Binärliteral> | 0x<Hexadezimalliteral> | <Dezimalliteral>[e <Exponent>] | '<Id-Literal>' | null
\item real : <Fließkommaliteral> [r | R] | null
\item boolean : true | false | null
\item handle (und Nicht-agent-Kindtypen): null
\end{enumerate}

Referenzbasierte Typen:
\begin{enumerate}
\item code : 0
\item agent (und Kindtypen) : 0
\item Klassenobjekte : 0
\item Funktionszeiger : 0
\end{enumerate}

Dabei können Zeichenkettenausdrücke mehrfach hintereinander geschrieben werden. Diese werden dann automatisch zu einem einzelnen
zusammengefasst. Dies vereinfacht die Aufteilung auf mehrere Zeilen, ohne unnötige Zeilenumbrüche und Einrückungen dazwischen und
dient einer besseren Übersicht:
"<Ausdruckteil 1>"
"<Ausdruckteil n>"

\section{Nullwerte}
Wie oben dargestellt erhalten kopierbasierte Typen einen Nullwert mit dem Literal "null" und referenzbasierte Typen mit dem Literal "0".
Der Zugriff auf Nullwerte kann in manchen Fällen zu einem undefinierten Verhalten führen.
In der Regel ist er im Testmodus jedoch abgesichert und führt zu Fehlermeldungen (siehe "Klassen").

\section{Id-Literale}
Für Id-Literale gelten bei der Validierung selbige Regeln wie in der Sektion "Objekttypen-Literale" definiert. Es wird also
eine Warnung vom Compiler ausgegeben, falls sich die Id statisch überprüfen lässt und in keiner der entsprechenden Dateien
zu finden ist.

\section{Objekttypen-Literale}
Des Weiteren können Eigenschaften sogenannter Objekttypen, die einen bestimmten nativen, kopierbasierten Typ haben abgefragt werden.
Objekttypen erhalten in Warcraft 3 The Frozen Throne eine vierstellige, hexadezimale Id. Daher können Objekttypen im Wertebereich von 16^4 definiert werden.
Jede Id identifiziert genau einen Objekttyp.
Ein Objekttyp kann in JASS++ folgendes sein:
\begin{enumerate}
\item ein Einheitentyp
\item ein Gegenstandstyp
\item ein Doodad-Typ
\item ein Zerstörbares-Typ
\item ein Fähigkeitentyp
\item ein Zauberverstärkertyp
\item ein Forschungstyp
\item ein Wettereffekttyp
\item ein Blitzeffekttyp
\item ein Geländeset
\end{enumerate}

Die Objekttypen stammen aus eine der folgenden Dateien (aufsteigend nach Priorität geordnet):
\begin{enumerate}
\item war3map.u
\item war3map.d
\item UI/UnitData.slk
\item *
\item *
\end{enumerate}
TODO

Dabei werden die angegebenen Dateien in den folgenden Archiven in absteigender Reihenfolge durchsucht:
\begin{enumerate}
\item Karte
\item Kampagne
\item War3Patch.mpq
\item War3Locale.mpq (richtigen Namen finden)
\item War3X.mpq
\item War3.mpq
\end{enumerate}
TODO

Die Eigenschaften eines Objekttyps können einen der folgenden nativen, kopierbasierten Typen haben:
\begin{enumerate}
\item string
\item integer
\item real
\item boolean
\end{enumerate}
TODO

Die Eigenschaften eines Objekttyps haben selbst nochmals eine eigene Id.
Ein Objekttypen-Eigenschafts-Literal wird immer als ein Ausdruck eines der angegebenen Typen interpretiert.
Kann der Objekttypeintrag bzw. der Objekttyp-Eigenschaftseintrag nicht gefunden werden, so gibt der Compiler eine Warnung aus.
Das Literal ist in diesem Fall ein Nullwert ("null").
Wird keine Eigenschafts-Id angegeben, so wird der Ausdruck als Array interpretiert. Dies geht jedoch nur falls alle Objekttypen-Eigenschaften
den gleichen nativen, kopierbasierten Typ haben.
Ansonsten kann das Literal nur als eigener Typ interpertiert werden (siehe "Klassen - Zuweisungen").
Die Notation sieht folgendermaßen aus:
<<Objekttyp-Id>[, %]>
<<Objekttyp-Id>,<Eigenschafts-Id>[, %])>

Das Prozentzeichen ist optional und gibt an, dass Fließkommazahlen als Prozentanteile und daher als Ganzzahlen zwischen 0 und 100 interpretiert werden
sollen. Aus 0.30 würde in diesem Fall also die Ganzzahl 30 werden.
Es gilt bei einer reinen Angabe der Objekttyp-Id für alle Eigenschaften.
Es gilt zu beachten, dass Objekttypen-Ids auch als Escape-Sequenzen verwendet werden können.

\section{Farbliterale}
Farbliterale haben stets den nativen Typ "integer". Sie sind daher ein 32-Bit-Wert, welcher eine RGBA-Farbe enthält.
Der RGBA-Wert muss hexadezimal angegeben werden, wobei die ersten beiden Ziffern für den Alphawert, die zweiten beiden für den Grünwert,
die dritten beiden für den Blauwert und die letzten beiden für den Blauwert stehen.
Notation:
|c<Farbe>|r

\section{Escape-Sequenzen}
Eine Escape-Sequenz ist ein Zeichen, welchem das \-Zeichen vorangestellt wird und welches ausschließlich ein einem
Zeichenkettenliteral vorkommen darf.
Es gibt folgende Escape-Sequenzen:
\begin(enumerate}
\item n				Zeilenumbruch
\item t				Horizontaler Tabulator
\item \				Backslash
\item "				Doppeltes Anführungszeichen
\end{enumerate}

Zusätzlich existieren spezielle Escape-Sequenzen für Objekttypen-Ids (siehe "Literale - Objekttypen-Ids") und Farben (siehe "Literale - Farbliterale").
Farbkodierungen und Objekttypen-Ids in Zeichenketten werden zum Kompilierzeitpunkt validiert bzw. kompiliert.
Die speziellen Escapesequenzen werden in den Zeichenketten zum Kompilierzeitpunkt ersetzt, insofern sie gültig sind.
Falls nicht, bleibt der Ausdruck erhalten und der Compiler gibt eine Warnung aus.
Wie bei normalen Objektypen-Literalen kann ein Prozentzeichen angehängt werden.
Bei Farben kann jedoch, anders als bei gewöhnlichen Farbliteralen, Text eingeschlossen werden. Dieser sollte dann in der entsprechenden Farbe
im Spiel bei einer Ausgabe angezeigt werden.

\section{Notation}
|c<Farbe>[<Text>]|r
<<Objekt-Id>,<Feld-Id>[, %]>

	\chapter{Literale}
Ein Literal ist ein sich selbst erklärender Ausdruck, der nicht speziell benannt werden muss also keinen Bezeichner benötigt.
Für folgende native Typen gelten folgende Literalnotationsregeln:

Kopierbasierte Typen:
TODO: Exponentialschreibweise für alle Ganzzahlausdrücke erlauben und auch für Fließkommazahlen?
\begin{enumerate}
\item string : "<Zeichenkettenliteral>" | null
\item integer : 0<Oktalliteral> | 0%<Binärliteral> | 0x<Hexadezimalliteral> | <Dezimalliteral>[e <Exponent>] | '<Id-Literal>' | null
\item real : <Fließkommaliteral> [r | R] | null
\item boolean : true | false | null
\item handle (und Nicht-agent-Kindtypen): null
\end{enumerate}

Referenzbasierte Typen:
\begin{enumerate}
\item code : 0
\item agent (und Kindtypen) : 0
\item Klassenobjekte : 0
\item Funktionszeiger : 0
\end{enumerate}

Dabei können Zeichenkettenausdrücke mehrfach hintereinander geschrieben werden. Diese werden dann automatisch zu einem einzelnen
zusammengefasst. Dies vereinfacht die Aufteilung auf mehrere Zeilen, ohne unnötige Zeilenumbrüche und Einrückungen dazwischen und
dient einer besseren Übersicht:
"<Ausdruckteil 1>"
"<Ausdruckteil n>"

\section{Nullwerte}
Wie oben dargestellt erhalten kopierbasierte Typen einen Nullwert mit dem Literal "null" und referenzbasierte Typen mit dem Literal "0".
Der Zugriff auf Nullwerte kann in manchen Fällen zu einem undefinierten Verhalten führen.
In der Regel ist er im Testmodus jedoch abgesichert und führt zu Fehlermeldungen (siehe "Klassen").

\section{Id-Literale}
Für Id-Literale gelten bei der Validierung selbige Regeln wie in der Sektion "Objekttypen-Literale" definiert. Es wird also
eine Warnung vom Compiler ausgegeben, falls sich die Id statisch überprüfen lässt und in keiner der entsprechenden Dateien
zu finden ist.

\section{Objekttypen-Literale}
Des Weiteren können Eigenschaften sogenannter Objekttypen, die einen bestimmten nativen, kopierbasierten Typ haben abgefragt werden.
Objekttypen erhalten in Warcraft 3 The Frozen Throne eine vierstellige, hexadezimale Id. Daher können Objekttypen im Wertebereich von 16^4 definiert werden.
Jede Id identifiziert genau einen Objekttyp.
Ein Objekttyp kann in JASS++ folgendes sein:
\begin{enumerate}
\item ein Einheitentyp
\item ein Gegenstandstyp
\item ein Doodad-Typ
\item ein Zerstörbares-Typ
\item ein Fähigkeitentyp
\item ein Zauberverstärkertyp
\item ein Forschungstyp
\item ein Wettereffekttyp
\item ein Blitzeffekttyp
\item ein Geländeset
\end{enumerate}

Die Objekttypen stammen aus eine der folgenden Dateien (aufsteigend nach Priorität geordnet):
\begin{enumerate}
\item war3map.u
\item war3map.d
\item UI/UnitData.slk
\item *
\item *
\end{enumerate}
TODO

Dabei werden die angegebenen Dateien in den folgenden Archiven in absteigender Reihenfolge durchsucht:
\begin{enumerate}
\item Karte
\item Kampagne
\item War3Patch.mpq
\item War3Locale.mpq (richtigen Namen finden)
\item War3X.mpq
\item War3.mpq
\end{enumerate}
TODO

Die Eigenschaften eines Objekttyps können einen der folgenden nativen, kopierbasierten Typen haben:
\begin{enumerate}
\item string
\item integer
\item real
\item boolean
\end{enumerate}
TODO

Die Eigenschaften eines Objekttyps haben selbst nochmals eine eigene Id.
Ein Objekttypen-Eigenschafts-Literal wird immer als ein Ausdruck eines der angegebenen Typen interpretiert.
Kann der Objekttypeintrag bzw. der Objekttyp-Eigenschaftseintrag nicht gefunden werden, so gibt der Compiler eine Warnung aus.
Das Literal ist in diesem Fall ein Nullwert ("null").
Wird keine Eigenschafts-Id angegeben, so wird der Ausdruck als Array interpretiert. Dies geht jedoch nur falls alle Objekttypen-Eigenschaften
den gleichen nativen, kopierbasierten Typ haben.
Ansonsten kann das Literal nur als eigener Typ interpertiert werden (siehe "Klassen - Zuweisungen").
Die Notation sieht folgendermaßen aus:
<<Objekttyp-Id>[, %]>
<<Objekttyp-Id>,<Eigenschafts-Id>[, %])>

Das Prozentzeichen ist optional und gibt an, dass Fließkommazahlen als Prozentanteile und daher als Ganzzahlen zwischen 0 und 100 interpretiert werden
sollen. Aus 0.30 würde in diesem Fall also die Ganzzahl 30 werden.
Es gilt bei einer reinen Angabe der Objekttyp-Id für alle Eigenschaften.
Es gilt zu beachten, dass Objekttypen-Ids auch als Escape-Sequenzen verwendet werden können.

\section{Farbliterale}
Farbliterale haben stets den nativen Typ "integer". Sie sind daher ein 32-Bit-Wert, welcher eine RGBA-Farbe enthält.
Der RGBA-Wert muss hexadezimal angegeben werden, wobei die ersten beiden Ziffern für den Alphawert, die zweiten beiden für den Grünwert,
die dritten beiden für den Blauwert und die letzten beiden für den Blauwert stehen.
Notation:
|c<Farbe>|r

\section{Escape-Sequenzen}
Eine Escape-Sequenz ist ein Zeichen, welchem das \-Zeichen vorangestellt wird und welches ausschließlich ein einem
Zeichenkettenliteral vorkommen darf.
Es gibt folgende Escape-Sequenzen:
\begin(enumerate}
\item n				Zeilenumbruch
\item t				Horizontaler Tabulator
\item \				Backslash
\item "				Doppeltes Anführungszeichen
\end{enumerate}

Zusätzlich existieren spezielle Escape-Sequenzen für Objekttypen-Ids (siehe "Literale - Objekttypen-Ids") und Farben (siehe "Literale - Farbliterale").
Farbkodierungen und Objekttypen-Ids in Zeichenketten werden zum Kompilierzeitpunkt validiert bzw. kompiliert.
Die speziellen Escapesequenzen werden in den Zeichenketten zum Kompilierzeitpunkt ersetzt, insofern sie gültig sind.
Falls nicht, bleibt der Ausdruck erhalten und der Compiler gibt eine Warnung aus.
Wie bei normalen Objektypen-Literalen kann ein Prozentzeichen angehängt werden.
Bei Farben kann jedoch, anders als bei gewöhnlichen Farbliteralen, Text eingeschlossen werden. Dieser sollte dann in der entsprechenden Farbe
im Spiel bei einer Ausgabe angezeigt werden.

\section{Notation}
|c<Farbe>[<Text>]|r
<<Objekt-Id>,<Feld-Id>[, %]>


	\chapter{Literale}
Ein Literal ist ein sich selbst erklärender Ausdruck, der nicht speziell benannt werden muss also keinen Bezeichner benötigt.
Für folgende native Typen gelten folgende Literalnotationsregeln:

Kopierbasierte Typen:
TODO: Exponentialschreibweise für alle Ganzzahlausdrücke erlauben und auch für Fließkommazahlen?
\begin{enumerate}
\item string : "<Zeichenkettenliteral>" | null
\item integer : 0<Oktalliteral> | 0%<Binärliteral> | 0x<Hexadezimalliteral> | <Dezimalliteral>[e <Exponent>] | '<Id-Literal>' | null
\item real : <Fließkommaliteral> [r | R] | null
\item boolean : true | false | null
\item handle (und Nicht-agent-Kindtypen): null
\end{enumerate}

Referenzbasierte Typen:
\begin{enumerate}
\item code : 0
\item agent (und Kindtypen) : 0
\item Klassenobjekte : 0
\item Funktionszeiger : 0
\end{enumerate}

Dabei können Zeichenkettenausdrücke mehrfach hintereinander geschrieben werden. Diese werden dann automatisch zu einem einzelnen
zusammengefasst. Dies vereinfacht die Aufteilung auf mehrere Zeilen, ohne unnötige Zeilenumbrüche und Einrückungen dazwischen und
dient einer besseren Übersicht:
"<Ausdruckteil 1>"
"<Ausdruckteil n>"

\section{Nullwerte}
Wie oben dargestellt erhalten kopierbasierte Typen einen Nullwert mit dem Literal "null" und referenzbasierte Typen mit dem Literal "0".
Der Zugriff auf Nullwerte kann in manchen Fällen zu einem undefinierten Verhalten führen.
In der Regel ist er im Testmodus jedoch abgesichert und führt zu Fehlermeldungen (siehe "Klassen").

\section{Id-Literale}
Für Id-Literale gelten bei der Validierung selbige Regeln wie in der Sektion "Objekttypen-Literale" definiert. Es wird also
eine Warnung vom Compiler ausgegeben, falls sich die Id statisch überprüfen lässt und in keiner der entsprechenden Dateien
zu finden ist.

\section{Objekttypen-Literale}
Des Weiteren können Eigenschaften sogenannter Objekttypen, die einen bestimmten nativen, kopierbasierten Typ haben abgefragt werden.
Objekttypen erhalten in Warcraft 3 The Frozen Throne eine vierstellige, hexadezimale Id. Daher können Objekttypen im Wertebereich von 16^4 definiert werden.
Jede Id identifiziert genau einen Objekttyp.
Ein Objekttyp kann in JASS++ folgendes sein:
\begin{enumerate}
\item ein Einheitentyp
\item ein Gegenstandstyp
\item ein Doodad-Typ
\item ein Zerstörbares-Typ
\item ein Fähigkeitentyp
\item ein Zauberverstärkertyp
\item ein Forschungstyp
\item ein Wettereffekttyp
\item ein Blitzeffekttyp
\item ein Geländeset
\end{enumerate}

Die Objekttypen stammen aus eine der folgenden Dateien (aufsteigend nach Priorität geordnet):
\begin{enumerate}
\item war3map.u
\item war3map.d
\item UI/UnitData.slk
\item *
\item *
\end{enumerate}
TODO

Dabei werden die angegebenen Dateien in den folgenden Archiven in absteigender Reihenfolge durchsucht:
\begin{enumerate}
\item Karte
\item Kampagne
\item War3Patch.mpq
\item War3Locale.mpq (richtigen Namen finden)
\item War3X.mpq
\item War3.mpq
\end{enumerate}
TODO

Die Eigenschaften eines Objekttyps können einen der folgenden nativen, kopierbasierten Typen haben:
\begin{enumerate}
\item string
\item integer
\item real
\item boolean
\end{enumerate}
TODO

Die Eigenschaften eines Objekttyps haben selbst nochmals eine eigene Id.
Ein Objekttypen-Eigenschafts-Literal wird immer als ein Ausdruck eines der angegebenen Typen interpretiert.
Kann der Objekttypeintrag bzw. der Objekttyp-Eigenschaftseintrag nicht gefunden werden, so gibt der Compiler eine Warnung aus.
Das Literal ist in diesem Fall ein Nullwert ("null").
Wird keine Eigenschafts-Id angegeben, so wird der Ausdruck als Array interpretiert. Dies geht jedoch nur falls alle Objekttypen-Eigenschaften
den gleichen nativen, kopierbasierten Typ haben.
Ansonsten kann das Literal nur als eigener Typ interpertiert werden (siehe "Klassen - Zuweisungen").
Die Notation sieht folgendermaßen aus:
<<Objekttyp-Id>[, %]>
<<Objekttyp-Id>,<Eigenschafts-Id>[, %])>

Das Prozentzeichen ist optional und gibt an, dass Fließkommazahlen als Prozentanteile und daher als Ganzzahlen zwischen 0 und 100 interpretiert werden
sollen. Aus 0.30 würde in diesem Fall also die Ganzzahl 30 werden.
Es gilt bei einer reinen Angabe der Objekttyp-Id für alle Eigenschaften.
Es gilt zu beachten, dass Objekttypen-Ids auch als Escape-Sequenzen verwendet werden können.

\section{Farbliterale}
Farbliterale haben stets den nativen Typ "integer". Sie sind daher ein 32-Bit-Wert, welcher eine RGBA-Farbe enthält.
Der RGBA-Wert muss hexadezimal angegeben werden, wobei die ersten beiden Ziffern für den Alphawert, die zweiten beiden für den Grünwert,
die dritten beiden für den Blauwert und die letzten beiden für den Blauwert stehen.
Notation:
|c<Farbe>|r

\section{Escape-Sequenzen}
Eine Escape-Sequenz ist ein Zeichen, welchem das \-Zeichen vorangestellt wird und welches ausschließlich ein einem
Zeichenkettenliteral vorkommen darf.
Es gibt folgende Escape-Sequenzen:
\begin(enumerate}
\item n				Zeilenumbruch
\item t				Horizontaler Tabulator
\item \				Backslash
\item "				Doppeltes Anführungszeichen
\end{enumerate}

Zusätzlich existieren spezielle Escape-Sequenzen für Objekttypen-Ids (siehe "Literale - Objekttypen-Ids") und Farben (siehe "Literale - Farbliterale").
Farbkodierungen und Objekttypen-Ids in Zeichenketten werden zum Kompilierzeitpunkt validiert bzw. kompiliert.
Die speziellen Escapesequenzen werden in den Zeichenketten zum Kompilierzeitpunkt ersetzt, insofern sie gültig sind.
Falls nicht, bleibt der Ausdruck erhalten und der Compiler gibt eine Warnung aus.
Wie bei normalen Objektypen-Literalen kann ein Prozentzeichen angehängt werden.
Bei Farben kann jedoch, anders als bei gewöhnlichen Farbliteralen, Text eingeschlossen werden. Dieser sollte dann in der entsprechenden Farbe
im Spiel bei einer Ausgabe angezeigt werden.

\section{Notation}
|c<Farbe>[<Text>]|r
<<Objekt-Id>,<Feld-Id>[, %]>


	\chapter{Literale}
Ein Literal ist ein sich selbst erklärender Ausdruck, der nicht speziell benannt werden muss also keinen Bezeichner benötigt.
Für folgende native Typen gelten folgende Literalnotationsregeln:

Kopierbasierte Typen:
TODO: Exponentialschreibweise für alle Ganzzahlausdrücke erlauben und auch für Fließkommazahlen?
\begin{enumerate}
\item string : "<Zeichenkettenliteral>" | null
\item integer : 0<Oktalliteral> | 0%<Binärliteral> | 0x<Hexadezimalliteral> | <Dezimalliteral>[e <Exponent>] | '<Id-Literal>' | null
\item real : <Fließkommaliteral> [r | R] | null
\item boolean : true | false | null
\item handle (und Nicht-agent-Kindtypen): null
\end{enumerate}

Referenzbasierte Typen:
\begin{enumerate}
\item code : 0
\item agent (und Kindtypen) : 0
\item Klassenobjekte : 0
\item Funktionszeiger : 0
\end{enumerate}

Dabei können Zeichenkettenausdrücke mehrfach hintereinander geschrieben werden. Diese werden dann automatisch zu einem einzelnen
zusammengefasst. Dies vereinfacht die Aufteilung auf mehrere Zeilen, ohne unnötige Zeilenumbrüche und Einrückungen dazwischen und
dient einer besseren Übersicht:
"<Ausdruckteil 1>"
"<Ausdruckteil n>"

\section{Nullwerte}
Wie oben dargestellt erhalten kopierbasierte Typen einen Nullwert mit dem Literal "null" und referenzbasierte Typen mit dem Literal "0".
Der Zugriff auf Nullwerte kann in manchen Fällen zu einem undefinierten Verhalten führen.
In der Regel ist er im Testmodus jedoch abgesichert und führt zu Fehlermeldungen (siehe "Klassen").

\section{Id-Literale}
Für Id-Literale gelten bei der Validierung selbige Regeln wie in der Sektion "Objekttypen-Literale" definiert. Es wird also
eine Warnung vom Compiler ausgegeben, falls sich die Id statisch überprüfen lässt und in keiner der entsprechenden Dateien
zu finden ist.

\section{Objekttypen-Literale}
Des Weiteren können Eigenschaften sogenannter Objekttypen, die einen bestimmten nativen, kopierbasierten Typ haben abgefragt werden.
Objekttypen erhalten in Warcraft 3 The Frozen Throne eine vierstellige, hexadezimale Id. Daher können Objekttypen im Wertebereich von 16^4 definiert werden.
Jede Id identifiziert genau einen Objekttyp.
Ein Objekttyp kann in JASS++ folgendes sein:
\begin{enumerate}
\item ein Einheitentyp
\item ein Gegenstandstyp
\item ein Doodad-Typ
\item ein Zerstörbares-Typ
\item ein Fähigkeitentyp
\item ein Zauberverstärkertyp
\item ein Forschungstyp
\item ein Wettereffekttyp
\item ein Blitzeffekttyp
\item ein Geländeset
\end{enumerate}

Die Objekttypen stammen aus eine der folgenden Dateien (aufsteigend nach Priorität geordnet):
\begin{enumerate}
\item war3map.u
\item war3map.d
\item UI/UnitData.slk
\item *
\item *
\end{enumerate}
TODO

Dabei werden die angegebenen Dateien in den folgenden Archiven in absteigender Reihenfolge durchsucht:
\begin{enumerate}
\item Karte
\item Kampagne
\item War3Patch.mpq
\item War3Locale.mpq (richtigen Namen finden)
\item War3X.mpq
\item War3.mpq
\end{enumerate}
TODO

Die Eigenschaften eines Objekttyps können einen der folgenden nativen, kopierbasierten Typen haben:
\begin{enumerate}
\item string
\item integer
\item real
\item boolean
\end{enumerate}
TODO

Die Eigenschaften eines Objekttyps haben selbst nochmals eine eigene Id.
Ein Objekttypen-Eigenschafts-Literal wird immer als ein Ausdruck eines der angegebenen Typen interpretiert.
Kann der Objekttypeintrag bzw. der Objekttyp-Eigenschaftseintrag nicht gefunden werden, so gibt der Compiler eine Warnung aus.
Das Literal ist in diesem Fall ein Nullwert ("null").
Wird keine Eigenschafts-Id angegeben, so wird der Ausdruck als Array interpretiert. Dies geht jedoch nur falls alle Objekttypen-Eigenschaften
den gleichen nativen, kopierbasierten Typ haben.
Ansonsten kann das Literal nur als eigener Typ interpertiert werden (siehe "Klassen - Zuweisungen").
Die Notation sieht folgendermaßen aus:
<<Objekttyp-Id>[, %]>
<<Objekttyp-Id>,<Eigenschafts-Id>[, %])>

Das Prozentzeichen ist optional und gibt an, dass Fließkommazahlen als Prozentanteile und daher als Ganzzahlen zwischen 0 und 100 interpretiert werden
sollen. Aus 0.30 würde in diesem Fall also die Ganzzahl 30 werden.
Es gilt bei einer reinen Angabe der Objekttyp-Id für alle Eigenschaften.
Es gilt zu beachten, dass Objekttypen-Ids auch als Escape-Sequenzen verwendet werden können.

\section{Farbliterale}
Farbliterale haben stets den nativen Typ "integer". Sie sind daher ein 32-Bit-Wert, welcher eine RGBA-Farbe enthält.
Der RGBA-Wert muss hexadezimal angegeben werden, wobei die ersten beiden Ziffern für den Alphawert, die zweiten beiden für den Grünwert,
die dritten beiden für den Blauwert und die letzten beiden für den Blauwert stehen.
Notation:
|c<Farbe>|r

\section{Escape-Sequenzen}
Eine Escape-Sequenz ist ein Zeichen, welchem das \-Zeichen vorangestellt wird und welches ausschließlich ein einem
Zeichenkettenliteral vorkommen darf.
Es gibt folgende Escape-Sequenzen:
\begin(enumerate}
\item n				Zeilenumbruch
\item t				Horizontaler Tabulator
\item \				Backslash
\item "				Doppeltes Anführungszeichen
\end{enumerate}

Zusätzlich existieren spezielle Escape-Sequenzen für Objekttypen-Ids (siehe "Literale - Objekttypen-Ids") und Farben (siehe "Literale - Farbliterale").
Farbkodierungen und Objekttypen-Ids in Zeichenketten werden zum Kompilierzeitpunkt validiert bzw. kompiliert.
Die speziellen Escapesequenzen werden in den Zeichenketten zum Kompilierzeitpunkt ersetzt, insofern sie gültig sind.
Falls nicht, bleibt der Ausdruck erhalten und der Compiler gibt eine Warnung aus.
Wie bei normalen Objektypen-Literalen kann ein Prozentzeichen angehängt werden.
Bei Farben kann jedoch, anders als bei gewöhnlichen Farbliteralen, Text eingeschlossen werden. Dieser sollte dann in der entsprechenden Farbe
im Spiel bei einer Ausgabe angezeigt werden.

\section{Notation}
|c<Farbe>[<Text>]|r
<<Objekt-Id>,<Feld-Id>[, %]>


	\chapter{Literale}
Ein Literal ist ein sich selbst erklärender Ausdruck, der nicht speziell benannt werden muss also keinen Bezeichner benötigt.
Für folgende native Typen gelten folgende Literalnotationsregeln:

Kopierbasierte Typen:
TODO: Exponentialschreibweise für alle Ganzzahlausdrücke erlauben und auch für Fließkommazahlen?
\begin{enumerate}
\item string : "<Zeichenkettenliteral>" | null
\item integer : 0<Oktalliteral> | 0%<Binärliteral> | 0x<Hexadezimalliteral> | <Dezimalliteral>[e <Exponent>] | '<Id-Literal>' | null
\item real : <Fließkommaliteral> [r | R] | null
\item boolean : true | false | null
\item handle (und Nicht-agent-Kindtypen): null
\end{enumerate}

Referenzbasierte Typen:
\begin{enumerate}
\item code : 0
\item agent (und Kindtypen) : 0
\item Klassenobjekte : 0
\item Funktionszeiger : 0
\end{enumerate}

Dabei können Zeichenkettenausdrücke mehrfach hintereinander geschrieben werden. Diese werden dann automatisch zu einem einzelnen
zusammengefasst. Dies vereinfacht die Aufteilung auf mehrere Zeilen, ohne unnötige Zeilenumbrüche und Einrückungen dazwischen und
dient einer besseren Übersicht:
"<Ausdruckteil 1>"
"<Ausdruckteil n>"

\section{Nullwerte}
Wie oben dargestellt erhalten kopierbasierte Typen einen Nullwert mit dem Literal "null" und referenzbasierte Typen mit dem Literal "0".
Der Zugriff auf Nullwerte kann in manchen Fällen zu einem undefinierten Verhalten führen.
In der Regel ist er im Testmodus jedoch abgesichert und führt zu Fehlermeldungen (siehe "Klassen").

\section{Id-Literale}
Für Id-Literale gelten bei der Validierung selbige Regeln wie in der Sektion "Objekttypen-Literale" definiert. Es wird also
eine Warnung vom Compiler ausgegeben, falls sich die Id statisch überprüfen lässt und in keiner der entsprechenden Dateien
zu finden ist.

\section{Objekttypen-Literale}
Des Weiteren können Eigenschaften sogenannter Objekttypen, die einen bestimmten nativen, kopierbasierten Typ haben abgefragt werden.
Objekttypen erhalten in Warcraft 3 The Frozen Throne eine vierstellige, hexadezimale Id. Daher können Objekttypen im Wertebereich von 16^4 definiert werden.
Jede Id identifiziert genau einen Objekttyp.
Ein Objekttyp kann in JASS++ folgendes sein:
\begin{enumerate}
\item ein Einheitentyp
\item ein Gegenstandstyp
\item ein Doodad-Typ
\item ein Zerstörbares-Typ
\item ein Fähigkeitentyp
\item ein Zauberverstärkertyp
\item ein Forschungstyp
\item ein Wettereffekttyp
\item ein Blitzeffekttyp
\item ein Geländeset
\end{enumerate}

Die Objekttypen stammen aus eine der folgenden Dateien (aufsteigend nach Priorität geordnet):
\begin{enumerate}
\item war3map.u
\item war3map.d
\item UI/UnitData.slk
\item *
\item *
\end{enumerate}
TODO

Dabei werden die angegebenen Dateien in den folgenden Archiven in absteigender Reihenfolge durchsucht:
\begin{enumerate}
\item Karte
\item Kampagne
\item War3Patch.mpq
\item War3Locale.mpq (richtigen Namen finden)
\item War3X.mpq
\item War3.mpq
\end{enumerate}
TODO

Die Eigenschaften eines Objekttyps können einen der folgenden nativen, kopierbasierten Typen haben:
\begin{enumerate}
\item string
\item integer
\item real
\item boolean
\end{enumerate}
TODO

Die Eigenschaften eines Objekttyps haben selbst nochmals eine eigene Id.
Ein Objekttypen-Eigenschafts-Literal wird immer als ein Ausdruck eines der angegebenen Typen interpretiert.
Kann der Objekttypeintrag bzw. der Objekttyp-Eigenschaftseintrag nicht gefunden werden, so gibt der Compiler eine Warnung aus.
Das Literal ist in diesem Fall ein Nullwert ("null").
Wird keine Eigenschafts-Id angegeben, so wird der Ausdruck als Array interpretiert. Dies geht jedoch nur falls alle Objekttypen-Eigenschaften
den gleichen nativen, kopierbasierten Typ haben.
Ansonsten kann das Literal nur als eigener Typ interpertiert werden (siehe "Klassen - Zuweisungen").
Die Notation sieht folgendermaßen aus:
<<Objekttyp-Id>[, %]>
<<Objekttyp-Id>,<Eigenschafts-Id>[, %])>

Das Prozentzeichen ist optional und gibt an, dass Fließkommazahlen als Prozentanteile und daher als Ganzzahlen zwischen 0 und 100 interpretiert werden
sollen. Aus 0.30 würde in diesem Fall also die Ganzzahl 30 werden.
Es gilt bei einer reinen Angabe der Objekttyp-Id für alle Eigenschaften.
Es gilt zu beachten, dass Objekttypen-Ids auch als Escape-Sequenzen verwendet werden können.

\section{Farbliterale}
Farbliterale haben stets den nativen Typ "integer". Sie sind daher ein 32-Bit-Wert, welcher eine RGBA-Farbe enthält.
Der RGBA-Wert muss hexadezimal angegeben werden, wobei die ersten beiden Ziffern für den Alphawert, die zweiten beiden für den Grünwert,
die dritten beiden für den Blauwert und die letzten beiden für den Blauwert stehen.
Notation:
|c<Farbe>|r

\section{Escape-Sequenzen}
Eine Escape-Sequenz ist ein Zeichen, welchem das \-Zeichen vorangestellt wird und welches ausschließlich ein einem
Zeichenkettenliteral vorkommen darf.
Es gibt folgende Escape-Sequenzen:
\begin(enumerate}
\item n				Zeilenumbruch
\item t				Horizontaler Tabulator
\item \				Backslash
\item "				Doppeltes Anführungszeichen
\end{enumerate}

Zusätzlich existieren spezielle Escape-Sequenzen für Objekttypen-Ids (siehe "Literale - Objekttypen-Ids") und Farben (siehe "Literale - Farbliterale").
Farbkodierungen und Objekttypen-Ids in Zeichenketten werden zum Kompilierzeitpunkt validiert bzw. kompiliert.
Die speziellen Escapesequenzen werden in den Zeichenketten zum Kompilierzeitpunkt ersetzt, insofern sie gültig sind.
Falls nicht, bleibt der Ausdruck erhalten und der Compiler gibt eine Warnung aus.
Wie bei normalen Objektypen-Literalen kann ein Prozentzeichen angehängt werden.
Bei Farben kann jedoch, anders als bei gewöhnlichen Farbliteralen, Text eingeschlossen werden. Dieser sollte dann in der entsprechenden Farbe
im Spiel bei einer Ausgabe angezeigt werden.

\section{Notation}
|c<Farbe>[<Text>]|r
<<Objekt-Id>,<Feld-Id>[, %]>


	\chapter{Literale}
Ein Literal ist ein sich selbst erklärender Ausdruck, der nicht speziell benannt werden muss also keinen Bezeichner benötigt.
Für folgende native Typen gelten folgende Literalnotationsregeln:

Kopierbasierte Typen:
TODO: Exponentialschreibweise für alle Ganzzahlausdrücke erlauben und auch für Fließkommazahlen?
\begin{enumerate}
\item string : "<Zeichenkettenliteral>" | null
\item integer : 0<Oktalliteral> | 0%<Binärliteral> | 0x<Hexadezimalliteral> | <Dezimalliteral>[e <Exponent>] | '<Id-Literal>' | null
\item real : <Fließkommaliteral> [r | R] | null
\item boolean : true | false | null
\item handle (und Nicht-agent-Kindtypen): null
\end{enumerate}

Referenzbasierte Typen:
\begin{enumerate}
\item code : 0
\item agent (und Kindtypen) : 0
\item Klassenobjekte : 0
\item Funktionszeiger : 0
\end{enumerate}

Dabei können Zeichenkettenausdrücke mehrfach hintereinander geschrieben werden. Diese werden dann automatisch zu einem einzelnen
zusammengefasst. Dies vereinfacht die Aufteilung auf mehrere Zeilen, ohne unnötige Zeilenumbrüche und Einrückungen dazwischen und
dient einer besseren Übersicht:
"<Ausdruckteil 1>"
"<Ausdruckteil n>"

\section{Nullwerte}
Wie oben dargestellt erhalten kopierbasierte Typen einen Nullwert mit dem Literal "null" und referenzbasierte Typen mit dem Literal "0".
Der Zugriff auf Nullwerte kann in manchen Fällen zu einem undefinierten Verhalten führen.
In der Regel ist er im Testmodus jedoch abgesichert und führt zu Fehlermeldungen (siehe "Klassen").

\section{Id-Literale}
Für Id-Literale gelten bei der Validierung selbige Regeln wie in der Sektion "Objekttypen-Literale" definiert. Es wird also
eine Warnung vom Compiler ausgegeben, falls sich die Id statisch überprüfen lässt und in keiner der entsprechenden Dateien
zu finden ist.

\section{Objekttypen-Literale}
Des Weiteren können Eigenschaften sogenannter Objekttypen, die einen bestimmten nativen, kopierbasierten Typ haben abgefragt werden.
Objekttypen erhalten in Warcraft 3 The Frozen Throne eine vierstellige, hexadezimale Id. Daher können Objekttypen im Wertebereich von 16^4 definiert werden.
Jede Id identifiziert genau einen Objekttyp.
Ein Objekttyp kann in JASS++ folgendes sein:
\begin{enumerate}
\item ein Einheitentyp
\item ein Gegenstandstyp
\item ein Doodad-Typ
\item ein Zerstörbares-Typ
\item ein Fähigkeitentyp
\item ein Zauberverstärkertyp
\item ein Forschungstyp
\item ein Wettereffekttyp
\item ein Blitzeffekttyp
\item ein Geländeset
\end{enumerate}

Die Objekttypen stammen aus eine der folgenden Dateien (aufsteigend nach Priorität geordnet):
\begin{enumerate}
\item war3map.u
\item war3map.d
\item UI/UnitData.slk
\item *
\item *
\end{enumerate}
TODO

Dabei werden die angegebenen Dateien in den folgenden Archiven in absteigender Reihenfolge durchsucht:
\begin{enumerate}
\item Karte
\item Kampagne
\item War3Patch.mpq
\item War3Locale.mpq (richtigen Namen finden)
\item War3X.mpq
\item War3.mpq
\end{enumerate}
TODO

Die Eigenschaften eines Objekttyps können einen der folgenden nativen, kopierbasierten Typen haben:
\begin{enumerate}
\item string
\item integer
\item real
\item boolean
\end{enumerate}
TODO

Die Eigenschaften eines Objekttyps haben selbst nochmals eine eigene Id.
Ein Objekttypen-Eigenschafts-Literal wird immer als ein Ausdruck eines der angegebenen Typen interpretiert.
Kann der Objekttypeintrag bzw. der Objekttyp-Eigenschaftseintrag nicht gefunden werden, so gibt der Compiler eine Warnung aus.
Das Literal ist in diesem Fall ein Nullwert ("null").
Wird keine Eigenschafts-Id angegeben, so wird der Ausdruck als Array interpretiert. Dies geht jedoch nur falls alle Objekttypen-Eigenschaften
den gleichen nativen, kopierbasierten Typ haben.
Ansonsten kann das Literal nur als eigener Typ interpertiert werden (siehe "Klassen - Zuweisungen").
Die Notation sieht folgendermaßen aus:
<<Objekttyp-Id>[, %]>
<<Objekttyp-Id>,<Eigenschafts-Id>[, %])>

Das Prozentzeichen ist optional und gibt an, dass Fließkommazahlen als Prozentanteile und daher als Ganzzahlen zwischen 0 und 100 interpretiert werden
sollen. Aus 0.30 würde in diesem Fall also die Ganzzahl 30 werden.
Es gilt bei einer reinen Angabe der Objekttyp-Id für alle Eigenschaften.
Es gilt zu beachten, dass Objekttypen-Ids auch als Escape-Sequenzen verwendet werden können.

\section{Farbliterale}
Farbliterale haben stets den nativen Typ "integer". Sie sind daher ein 32-Bit-Wert, welcher eine RGBA-Farbe enthält.
Der RGBA-Wert muss hexadezimal angegeben werden, wobei die ersten beiden Ziffern für den Alphawert, die zweiten beiden für den Grünwert,
die dritten beiden für den Blauwert und die letzten beiden für den Blauwert stehen.
Notation:
|c<Farbe>|r

\section{Escape-Sequenzen}
Eine Escape-Sequenz ist ein Zeichen, welchem das \-Zeichen vorangestellt wird und welches ausschließlich ein einem
Zeichenkettenliteral vorkommen darf.
Es gibt folgende Escape-Sequenzen:
\begin(enumerate}
\item n				Zeilenumbruch
\item t				Horizontaler Tabulator
\item \				Backslash
\item "				Doppeltes Anführungszeichen
\end{enumerate}

Zusätzlich existieren spezielle Escape-Sequenzen für Objekttypen-Ids (siehe "Literale - Objekttypen-Ids") und Farben (siehe "Literale - Farbliterale").
Farbkodierungen und Objekttypen-Ids in Zeichenketten werden zum Kompilierzeitpunkt validiert bzw. kompiliert.
Die speziellen Escapesequenzen werden in den Zeichenketten zum Kompilierzeitpunkt ersetzt, insofern sie gültig sind.
Falls nicht, bleibt der Ausdruck erhalten und der Compiler gibt eine Warnung aus.
Wie bei normalen Objektypen-Literalen kann ein Prozentzeichen angehängt werden.
Bei Farben kann jedoch, anders als bei gewöhnlichen Farbliteralen, Text eingeschlossen werden. Dieser sollte dann in der entsprechenden Farbe
im Spiel bei einer Ausgabe angezeigt werden.

\section{Notation}
|c<Farbe>[<Text>]|r
<<Objekt-Id>,<Feld-Id>[, %]>

	\chapter{Literale}
Ein Literal ist ein sich selbst erklärender Ausdruck, der nicht speziell benannt werden muss also keinen Bezeichner benötigt.
Für folgende native Typen gelten folgende Literalnotationsregeln:

Kopierbasierte Typen:
TODO: Exponentialschreibweise für alle Ganzzahlausdrücke erlauben und auch für Fließkommazahlen?
\begin{enumerate}
\item string : "<Zeichenkettenliteral>" | null
\item integer : 0<Oktalliteral> | 0%<Binärliteral> | 0x<Hexadezimalliteral> | <Dezimalliteral>[e <Exponent>] | '<Id-Literal>' | null
\item real : <Fließkommaliteral> [r | R] | null
\item boolean : true | false | null
\item handle (und Nicht-agent-Kindtypen): null
\end{enumerate}

Referenzbasierte Typen:
\begin{enumerate}
\item code : 0
\item agent (und Kindtypen) : 0
\item Klassenobjekte : 0
\item Funktionszeiger : 0
\end{enumerate}

Dabei können Zeichenkettenausdrücke mehrfach hintereinander geschrieben werden. Diese werden dann automatisch zu einem einzelnen
zusammengefasst. Dies vereinfacht die Aufteilung auf mehrere Zeilen, ohne unnötige Zeilenumbrüche und Einrückungen dazwischen und
dient einer besseren Übersicht:
"<Ausdruckteil 1>"
"<Ausdruckteil n>"

\section{Nullwerte}
Wie oben dargestellt erhalten kopierbasierte Typen einen Nullwert mit dem Literal "null" und referenzbasierte Typen mit dem Literal "0".
Der Zugriff auf Nullwerte kann in manchen Fällen zu einem undefinierten Verhalten führen.
In der Regel ist er im Testmodus jedoch abgesichert und führt zu Fehlermeldungen (siehe "Klassen").

\section{Id-Literale}
Für Id-Literale gelten bei der Validierung selbige Regeln wie in der Sektion "Objekttypen-Literale" definiert. Es wird also
eine Warnung vom Compiler ausgegeben, falls sich die Id statisch überprüfen lässt und in keiner der entsprechenden Dateien
zu finden ist.

\section{Objekttypen-Literale}
Des Weiteren können Eigenschaften sogenannter Objekttypen, die einen bestimmten nativen, kopierbasierten Typ haben abgefragt werden.
Objekttypen erhalten in Warcraft 3 The Frozen Throne eine vierstellige, hexadezimale Id. Daher können Objekttypen im Wertebereich von 16^4 definiert werden.
Jede Id identifiziert genau einen Objekttyp.
Ein Objekttyp kann in JASS++ folgendes sein:
\begin{enumerate}
\item ein Einheitentyp
\item ein Gegenstandstyp
\item ein Doodad-Typ
\item ein Zerstörbares-Typ
\item ein Fähigkeitentyp
\item ein Zauberverstärkertyp
\item ein Forschungstyp
\item ein Wettereffekttyp
\item ein Blitzeffekttyp
\item ein Geländeset
\end{enumerate}

Die Objekttypen stammen aus eine der folgenden Dateien (aufsteigend nach Priorität geordnet):
\begin{enumerate}
\item war3map.u
\item war3map.d
\item UI/UnitData.slk
\item *
\item *
\end{enumerate}
TODO

Dabei werden die angegebenen Dateien in den folgenden Archiven in absteigender Reihenfolge durchsucht:
\begin{enumerate}
\item Karte
\item Kampagne
\item War3Patch.mpq
\item War3Locale.mpq (richtigen Namen finden)
\item War3X.mpq
\item War3.mpq
\end{enumerate}
TODO

Die Eigenschaften eines Objekttyps können einen der folgenden nativen, kopierbasierten Typen haben:
\begin{enumerate}
\item string
\item integer
\item real
\item boolean
\end{enumerate}
TODO

Die Eigenschaften eines Objekttyps haben selbst nochmals eine eigene Id.
Ein Objekttypen-Eigenschafts-Literal wird immer als ein Ausdruck eines der angegebenen Typen interpretiert.
Kann der Objekttypeintrag bzw. der Objekttyp-Eigenschaftseintrag nicht gefunden werden, so gibt der Compiler eine Warnung aus.
Das Literal ist in diesem Fall ein Nullwert ("null").
Wird keine Eigenschafts-Id angegeben, so wird der Ausdruck als Array interpretiert. Dies geht jedoch nur falls alle Objekttypen-Eigenschaften
den gleichen nativen, kopierbasierten Typ haben.
Ansonsten kann das Literal nur als eigener Typ interpertiert werden (siehe "Klassen - Zuweisungen").
Die Notation sieht folgendermaßen aus:
<<Objekttyp-Id>[, %]>
<<Objekttyp-Id>,<Eigenschafts-Id>[, %])>

Das Prozentzeichen ist optional und gibt an, dass Fließkommazahlen als Prozentanteile und daher als Ganzzahlen zwischen 0 und 100 interpretiert werden
sollen. Aus 0.30 würde in diesem Fall also die Ganzzahl 30 werden.
Es gilt bei einer reinen Angabe der Objekttyp-Id für alle Eigenschaften.
Es gilt zu beachten, dass Objekttypen-Ids auch als Escape-Sequenzen verwendet werden können.

\section{Farbliterale}
Farbliterale haben stets den nativen Typ "integer". Sie sind daher ein 32-Bit-Wert, welcher eine RGBA-Farbe enthält.
Der RGBA-Wert muss hexadezimal angegeben werden, wobei die ersten beiden Ziffern für den Alphawert, die zweiten beiden für den Grünwert,
die dritten beiden für den Blauwert und die letzten beiden für den Blauwert stehen.
Notation:
|c<Farbe>|r

\section{Escape-Sequenzen}
Eine Escape-Sequenz ist ein Zeichen, welchem das \-Zeichen vorangestellt wird und welches ausschließlich ein einem
Zeichenkettenliteral vorkommen darf.
Es gibt folgende Escape-Sequenzen:
\begin(enumerate}
\item n				Zeilenumbruch
\item t				Horizontaler Tabulator
\item \				Backslash
\item "				Doppeltes Anführungszeichen
\end{enumerate}

Zusätzlich existieren spezielle Escape-Sequenzen für Objekttypen-Ids (siehe "Literale - Objekttypen-Ids") und Farben (siehe "Literale - Farbliterale").
Farbkodierungen und Objekttypen-Ids in Zeichenketten werden zum Kompilierzeitpunkt validiert bzw. kompiliert.
Die speziellen Escapesequenzen werden in den Zeichenketten zum Kompilierzeitpunkt ersetzt, insofern sie gültig sind.
Falls nicht, bleibt der Ausdruck erhalten und der Compiler gibt eine Warnung aus.
Wie bei normalen Objektypen-Literalen kann ein Prozentzeichen angehängt werden.
Bei Farben kann jedoch, anders als bei gewöhnlichen Farbliteralen, Text eingeschlossen werden. Dieser sollte dann in der entsprechenden Farbe
im Spiel bei einer Ausgabe angezeigt werden.

\section{Notation}
|c<Farbe>[<Text>]|r
<<Objekt-Id>,<Feld-Id>[, %]>

	\chapter{Literale}
Ein Literal ist ein sich selbst erklärender Ausdruck, der nicht speziell benannt werden muss also keinen Bezeichner benötigt.
Für folgende native Typen gelten folgende Literalnotationsregeln:

Kopierbasierte Typen:
TODO: Exponentialschreibweise für alle Ganzzahlausdrücke erlauben und auch für Fließkommazahlen?
\begin{enumerate}
\item string : "<Zeichenkettenliteral>" | null
\item integer : 0<Oktalliteral> | 0%<Binärliteral> | 0x<Hexadezimalliteral> | <Dezimalliteral>[e <Exponent>] | '<Id-Literal>' | null
\item real : <Fließkommaliteral> [r | R] | null
\item boolean : true | false | null
\item handle (und Nicht-agent-Kindtypen): null
\end{enumerate}

Referenzbasierte Typen:
\begin{enumerate}
\item code : 0
\item agent (und Kindtypen) : 0
\item Klassenobjekte : 0
\item Funktionszeiger : 0
\end{enumerate}

Dabei können Zeichenkettenausdrücke mehrfach hintereinander geschrieben werden. Diese werden dann automatisch zu einem einzelnen
zusammengefasst. Dies vereinfacht die Aufteilung auf mehrere Zeilen, ohne unnötige Zeilenumbrüche und Einrückungen dazwischen und
dient einer besseren Übersicht:
"<Ausdruckteil 1>"
"<Ausdruckteil n>"

\section{Nullwerte}
Wie oben dargestellt erhalten kopierbasierte Typen einen Nullwert mit dem Literal "null" und referenzbasierte Typen mit dem Literal "0".
Der Zugriff auf Nullwerte kann in manchen Fällen zu einem undefinierten Verhalten führen.
In der Regel ist er im Testmodus jedoch abgesichert und führt zu Fehlermeldungen (siehe "Klassen").

\section{Id-Literale}
Für Id-Literale gelten bei der Validierung selbige Regeln wie in der Sektion "Objekttypen-Literale" definiert. Es wird also
eine Warnung vom Compiler ausgegeben, falls sich die Id statisch überprüfen lässt und in keiner der entsprechenden Dateien
zu finden ist.

\section{Objekttypen-Literale}
Des Weiteren können Eigenschaften sogenannter Objekttypen, die einen bestimmten nativen, kopierbasierten Typ haben abgefragt werden.
Objekttypen erhalten in Warcraft 3 The Frozen Throne eine vierstellige, hexadezimale Id. Daher können Objekttypen im Wertebereich von 16^4 definiert werden.
Jede Id identifiziert genau einen Objekttyp.
Ein Objekttyp kann in JASS++ folgendes sein:
\begin{enumerate}
\item ein Einheitentyp
\item ein Gegenstandstyp
\item ein Doodad-Typ
\item ein Zerstörbares-Typ
\item ein Fähigkeitentyp
\item ein Zauberverstärkertyp
\item ein Forschungstyp
\item ein Wettereffekttyp
\item ein Blitzeffekttyp
\item ein Geländeset
\end{enumerate}

Die Objekttypen stammen aus eine der folgenden Dateien (aufsteigend nach Priorität geordnet):
\begin{enumerate}
\item war3map.u
\item war3map.d
\item UI/UnitData.slk
\item *
\item *
\end{enumerate}
TODO

Dabei werden die angegebenen Dateien in den folgenden Archiven in absteigender Reihenfolge durchsucht:
\begin{enumerate}
\item Karte
\item Kampagne
\item War3Patch.mpq
\item War3Locale.mpq (richtigen Namen finden)
\item War3X.mpq
\item War3.mpq
\end{enumerate}
TODO

Die Eigenschaften eines Objekttyps können einen der folgenden nativen, kopierbasierten Typen haben:
\begin{enumerate}
\item string
\item integer
\item real
\item boolean
\end{enumerate}
TODO

Die Eigenschaften eines Objekttyps haben selbst nochmals eine eigene Id.
Ein Objekttypen-Eigenschafts-Literal wird immer als ein Ausdruck eines der angegebenen Typen interpretiert.
Kann der Objekttypeintrag bzw. der Objekttyp-Eigenschaftseintrag nicht gefunden werden, so gibt der Compiler eine Warnung aus.
Das Literal ist in diesem Fall ein Nullwert ("null").
Wird keine Eigenschafts-Id angegeben, so wird der Ausdruck als Array interpretiert. Dies geht jedoch nur falls alle Objekttypen-Eigenschaften
den gleichen nativen, kopierbasierten Typ haben.
Ansonsten kann das Literal nur als eigener Typ interpertiert werden (siehe "Klassen - Zuweisungen").
Die Notation sieht folgendermaßen aus:
<<Objekttyp-Id>[, %]>
<<Objekttyp-Id>,<Eigenschafts-Id>[, %])>

Das Prozentzeichen ist optional und gibt an, dass Fließkommazahlen als Prozentanteile und daher als Ganzzahlen zwischen 0 und 100 interpretiert werden
sollen. Aus 0.30 würde in diesem Fall also die Ganzzahl 30 werden.
Es gilt bei einer reinen Angabe der Objekttyp-Id für alle Eigenschaften.
Es gilt zu beachten, dass Objekttypen-Ids auch als Escape-Sequenzen verwendet werden können.

\section{Farbliterale}
Farbliterale haben stets den nativen Typ "integer". Sie sind daher ein 32-Bit-Wert, welcher eine RGBA-Farbe enthält.
Der RGBA-Wert muss hexadezimal angegeben werden, wobei die ersten beiden Ziffern für den Alphawert, die zweiten beiden für den Grünwert,
die dritten beiden für den Blauwert und die letzten beiden für den Blauwert stehen.
Notation:
|c<Farbe>|r

\section{Escape-Sequenzen}
Eine Escape-Sequenz ist ein Zeichen, welchem das \-Zeichen vorangestellt wird und welches ausschließlich ein einem
Zeichenkettenliteral vorkommen darf.
Es gibt folgende Escape-Sequenzen:
\begin(enumerate}
\item n				Zeilenumbruch
\item t				Horizontaler Tabulator
\item \				Backslash
\item "				Doppeltes Anführungszeichen
\end{enumerate}

Zusätzlich existieren spezielle Escape-Sequenzen für Objekttypen-Ids (siehe "Literale - Objekttypen-Ids") und Farben (siehe "Literale - Farbliterale").
Farbkodierungen und Objekttypen-Ids in Zeichenketten werden zum Kompilierzeitpunkt validiert bzw. kompiliert.
Die speziellen Escapesequenzen werden in den Zeichenketten zum Kompilierzeitpunkt ersetzt, insofern sie gültig sind.
Falls nicht, bleibt der Ausdruck erhalten und der Compiler gibt eine Warnung aus.
Wie bei normalen Objektypen-Literalen kann ein Prozentzeichen angehängt werden.
Bei Farben kann jedoch, anders als bei gewöhnlichen Farbliteralen, Text eingeschlossen werden. Dieser sollte dann in der entsprechenden Farbe
im Spiel bei einer Ausgabe angezeigt werden.

\section{Notation}
|c<Farbe>[<Text>]|r
<<Objekt-Id>,<Feld-Id>[, %]>


	\chapter{Literale}
Ein Literal ist ein sich selbst erklärender Ausdruck, der nicht speziell benannt werden muss also keinen Bezeichner benötigt.
Für folgende native Typen gelten folgende Literalnotationsregeln:

Kopierbasierte Typen:
TODO: Exponentialschreibweise für alle Ganzzahlausdrücke erlauben und auch für Fließkommazahlen?
\begin{enumerate}
\item string : "<Zeichenkettenliteral>" | null
\item integer : 0<Oktalliteral> | 0%<Binärliteral> | 0x<Hexadezimalliteral> | <Dezimalliteral>[e <Exponent>] | '<Id-Literal>' | null
\item real : <Fließkommaliteral> [r | R] | null
\item boolean : true | false | null
\item handle (und Nicht-agent-Kindtypen): null
\end{enumerate}

Referenzbasierte Typen:
\begin{enumerate}
\item code : 0
\item agent (und Kindtypen) : 0
\item Klassenobjekte : 0
\item Funktionszeiger : 0
\end{enumerate}

Dabei können Zeichenkettenausdrücke mehrfach hintereinander geschrieben werden. Diese werden dann automatisch zu einem einzelnen
zusammengefasst. Dies vereinfacht die Aufteilung auf mehrere Zeilen, ohne unnötige Zeilenumbrüche und Einrückungen dazwischen und
dient einer besseren Übersicht:
"<Ausdruckteil 1>"
"<Ausdruckteil n>"

\section{Nullwerte}
Wie oben dargestellt erhalten kopierbasierte Typen einen Nullwert mit dem Literal "null" und referenzbasierte Typen mit dem Literal "0".
Der Zugriff auf Nullwerte kann in manchen Fällen zu einem undefinierten Verhalten führen.
In der Regel ist er im Testmodus jedoch abgesichert und führt zu Fehlermeldungen (siehe "Klassen").

\section{Id-Literale}
Für Id-Literale gelten bei der Validierung selbige Regeln wie in der Sektion "Objekttypen-Literale" definiert. Es wird also
eine Warnung vom Compiler ausgegeben, falls sich die Id statisch überprüfen lässt und in keiner der entsprechenden Dateien
zu finden ist.

\section{Objekttypen-Literale}
Des Weiteren können Eigenschaften sogenannter Objekttypen, die einen bestimmten nativen, kopierbasierten Typ haben abgefragt werden.
Objekttypen erhalten in Warcraft 3 The Frozen Throne eine vierstellige, hexadezimale Id. Daher können Objekttypen im Wertebereich von 16^4 definiert werden.
Jede Id identifiziert genau einen Objekttyp.
Ein Objekttyp kann in JASS++ folgendes sein:
\begin{enumerate}
\item ein Einheitentyp
\item ein Gegenstandstyp
\item ein Doodad-Typ
\item ein Zerstörbares-Typ
\item ein Fähigkeitentyp
\item ein Zauberverstärkertyp
\item ein Forschungstyp
\item ein Wettereffekttyp
\item ein Blitzeffekttyp
\item ein Geländeset
\end{enumerate}

Die Objekttypen stammen aus eine der folgenden Dateien (aufsteigend nach Priorität geordnet):
\begin{enumerate}
\item war3map.u
\item war3map.d
\item UI/UnitData.slk
\item *
\item *
\end{enumerate}
TODO

Dabei werden die angegebenen Dateien in den folgenden Archiven in absteigender Reihenfolge durchsucht:
\begin{enumerate}
\item Karte
\item Kampagne
\item War3Patch.mpq
\item War3Locale.mpq (richtigen Namen finden)
\item War3X.mpq
\item War3.mpq
\end{enumerate}
TODO

Die Eigenschaften eines Objekttyps können einen der folgenden nativen, kopierbasierten Typen haben:
\begin{enumerate}
\item string
\item integer
\item real
\item boolean
\end{enumerate}
TODO

Die Eigenschaften eines Objekttyps haben selbst nochmals eine eigene Id.
Ein Objekttypen-Eigenschafts-Literal wird immer als ein Ausdruck eines der angegebenen Typen interpretiert.
Kann der Objekttypeintrag bzw. der Objekttyp-Eigenschaftseintrag nicht gefunden werden, so gibt der Compiler eine Warnung aus.
Das Literal ist in diesem Fall ein Nullwert ("null").
Wird keine Eigenschafts-Id angegeben, so wird der Ausdruck als Array interpretiert. Dies geht jedoch nur falls alle Objekttypen-Eigenschaften
den gleichen nativen, kopierbasierten Typ haben.
Ansonsten kann das Literal nur als eigener Typ interpertiert werden (siehe "Klassen - Zuweisungen").
Die Notation sieht folgendermaßen aus:
<<Objekttyp-Id>[, %]>
<<Objekttyp-Id>,<Eigenschafts-Id>[, %])>

Das Prozentzeichen ist optional und gibt an, dass Fließkommazahlen als Prozentanteile und daher als Ganzzahlen zwischen 0 und 100 interpretiert werden
sollen. Aus 0.30 würde in diesem Fall also die Ganzzahl 30 werden.
Es gilt bei einer reinen Angabe der Objekttyp-Id für alle Eigenschaften.
Es gilt zu beachten, dass Objekttypen-Ids auch als Escape-Sequenzen verwendet werden können.

\section{Farbliterale}
Farbliterale haben stets den nativen Typ "integer". Sie sind daher ein 32-Bit-Wert, welcher eine RGBA-Farbe enthält.
Der RGBA-Wert muss hexadezimal angegeben werden, wobei die ersten beiden Ziffern für den Alphawert, die zweiten beiden für den Grünwert,
die dritten beiden für den Blauwert und die letzten beiden für den Blauwert stehen.
Notation:
|c<Farbe>|r

\section{Escape-Sequenzen}
Eine Escape-Sequenz ist ein Zeichen, welchem das \-Zeichen vorangestellt wird und welches ausschließlich ein einem
Zeichenkettenliteral vorkommen darf.
Es gibt folgende Escape-Sequenzen:
\begin(enumerate}
\item n				Zeilenumbruch
\item t				Horizontaler Tabulator
\item \				Backslash
\item "				Doppeltes Anführungszeichen
\end{enumerate}

Zusätzlich existieren spezielle Escape-Sequenzen für Objekttypen-Ids (siehe "Literale - Objekttypen-Ids") und Farben (siehe "Literale - Farbliterale").
Farbkodierungen und Objekttypen-Ids in Zeichenketten werden zum Kompilierzeitpunkt validiert bzw. kompiliert.
Die speziellen Escapesequenzen werden in den Zeichenketten zum Kompilierzeitpunkt ersetzt, insofern sie gültig sind.
Falls nicht, bleibt der Ausdruck erhalten und der Compiler gibt eine Warnung aus.
Wie bei normalen Objektypen-Literalen kann ein Prozentzeichen angehängt werden.
Bei Farben kann jedoch, anders als bei gewöhnlichen Farbliteralen, Text eingeschlossen werden. Dieser sollte dann in der entsprechenden Farbe
im Spiel bei einer Ausgabe angezeigt werden.

\section{Notation}
|c<Farbe>[<Text>]|r
<<Objekt-Id>,<Feld-Id>[, %]>

	\chapter{Literale}
Ein Literal ist ein sich selbst erklärender Ausdruck, der nicht speziell benannt werden muss also keinen Bezeichner benötigt.
Für folgende native Typen gelten folgende Literalnotationsregeln:

Kopierbasierte Typen:
TODO: Exponentialschreibweise für alle Ganzzahlausdrücke erlauben und auch für Fließkommazahlen?
\begin{enumerate}
\item string : "<Zeichenkettenliteral>" | null
\item integer : 0<Oktalliteral> | 0%<Binärliteral> | 0x<Hexadezimalliteral> | <Dezimalliteral>[e <Exponent>] | '<Id-Literal>' | null
\item real : <Fließkommaliteral> [r | R] | null
\item boolean : true | false | null
\item handle (und Nicht-agent-Kindtypen): null
\end{enumerate}

Referenzbasierte Typen:
\begin{enumerate}
\item code : 0
\item agent (und Kindtypen) : 0
\item Klassenobjekte : 0
\item Funktionszeiger : 0
\end{enumerate}

Dabei können Zeichenkettenausdrücke mehrfach hintereinander geschrieben werden. Diese werden dann automatisch zu einem einzelnen
zusammengefasst. Dies vereinfacht die Aufteilung auf mehrere Zeilen, ohne unnötige Zeilenumbrüche und Einrückungen dazwischen und
dient einer besseren Übersicht:
"<Ausdruckteil 1>"
"<Ausdruckteil n>"

\section{Nullwerte}
Wie oben dargestellt erhalten kopierbasierte Typen einen Nullwert mit dem Literal "null" und referenzbasierte Typen mit dem Literal "0".
Der Zugriff auf Nullwerte kann in manchen Fällen zu einem undefinierten Verhalten führen.
In der Regel ist er im Testmodus jedoch abgesichert und führt zu Fehlermeldungen (siehe "Klassen").

\section{Id-Literale}
Für Id-Literale gelten bei der Validierung selbige Regeln wie in der Sektion "Objekttypen-Literale" definiert. Es wird also
eine Warnung vom Compiler ausgegeben, falls sich die Id statisch überprüfen lässt und in keiner der entsprechenden Dateien
zu finden ist.

\section{Objekttypen-Literale}
Des Weiteren können Eigenschaften sogenannter Objekttypen, die einen bestimmten nativen, kopierbasierten Typ haben abgefragt werden.
Objekttypen erhalten in Warcraft 3 The Frozen Throne eine vierstellige, hexadezimale Id. Daher können Objekttypen im Wertebereich von 16^4 definiert werden.
Jede Id identifiziert genau einen Objekttyp.
Ein Objekttyp kann in JASS++ folgendes sein:
\begin{enumerate}
\item ein Einheitentyp
\item ein Gegenstandstyp
\item ein Doodad-Typ
\item ein Zerstörbares-Typ
\item ein Fähigkeitentyp
\item ein Zauberverstärkertyp
\item ein Forschungstyp
\item ein Wettereffekttyp
\item ein Blitzeffekttyp
\item ein Geländeset
\end{enumerate}

Die Objekttypen stammen aus eine der folgenden Dateien (aufsteigend nach Priorität geordnet):
\begin{enumerate}
\item war3map.u
\item war3map.d
\item UI/UnitData.slk
\item *
\item *
\end{enumerate}
TODO

Dabei werden die angegebenen Dateien in den folgenden Archiven in absteigender Reihenfolge durchsucht:
\begin{enumerate}
\item Karte
\item Kampagne
\item War3Patch.mpq
\item War3Locale.mpq (richtigen Namen finden)
\item War3X.mpq
\item War3.mpq
\end{enumerate}
TODO

Die Eigenschaften eines Objekttyps können einen der folgenden nativen, kopierbasierten Typen haben:
\begin{enumerate}
\item string
\item integer
\item real
\item boolean
\end{enumerate}
TODO

Die Eigenschaften eines Objekttyps haben selbst nochmals eine eigene Id.
Ein Objekttypen-Eigenschafts-Literal wird immer als ein Ausdruck eines der angegebenen Typen interpretiert.
Kann der Objekttypeintrag bzw. der Objekttyp-Eigenschaftseintrag nicht gefunden werden, so gibt der Compiler eine Warnung aus.
Das Literal ist in diesem Fall ein Nullwert ("null").
Wird keine Eigenschafts-Id angegeben, so wird der Ausdruck als Array interpretiert. Dies geht jedoch nur falls alle Objekttypen-Eigenschaften
den gleichen nativen, kopierbasierten Typ haben.
Ansonsten kann das Literal nur als eigener Typ interpertiert werden (siehe "Klassen - Zuweisungen").
Die Notation sieht folgendermaßen aus:
<<Objekttyp-Id>[, %]>
<<Objekttyp-Id>,<Eigenschafts-Id>[, %])>

Das Prozentzeichen ist optional und gibt an, dass Fließkommazahlen als Prozentanteile und daher als Ganzzahlen zwischen 0 und 100 interpretiert werden
sollen. Aus 0.30 würde in diesem Fall also die Ganzzahl 30 werden.
Es gilt bei einer reinen Angabe der Objekttyp-Id für alle Eigenschaften.
Es gilt zu beachten, dass Objekttypen-Ids auch als Escape-Sequenzen verwendet werden können.

\section{Farbliterale}
Farbliterale haben stets den nativen Typ "integer". Sie sind daher ein 32-Bit-Wert, welcher eine RGBA-Farbe enthält.
Der RGBA-Wert muss hexadezimal angegeben werden, wobei die ersten beiden Ziffern für den Alphawert, die zweiten beiden für den Grünwert,
die dritten beiden für den Blauwert und die letzten beiden für den Blauwert stehen.
Notation:
|c<Farbe>|r

\section{Escape-Sequenzen}
Eine Escape-Sequenz ist ein Zeichen, welchem das \-Zeichen vorangestellt wird und welches ausschließlich ein einem
Zeichenkettenliteral vorkommen darf.
Es gibt folgende Escape-Sequenzen:
\begin(enumerate}
\item n				Zeilenumbruch
\item t				Horizontaler Tabulator
\item \				Backslash
\item "				Doppeltes Anführungszeichen
\end{enumerate}

Zusätzlich existieren spezielle Escape-Sequenzen für Objekttypen-Ids (siehe "Literale - Objekttypen-Ids") und Farben (siehe "Literale - Farbliterale").
Farbkodierungen und Objekttypen-Ids in Zeichenketten werden zum Kompilierzeitpunkt validiert bzw. kompiliert.
Die speziellen Escapesequenzen werden in den Zeichenketten zum Kompilierzeitpunkt ersetzt, insofern sie gültig sind.
Falls nicht, bleibt der Ausdruck erhalten und der Compiler gibt eine Warnung aus.
Wie bei normalen Objektypen-Literalen kann ein Prozentzeichen angehängt werden.
Bei Farben kann jedoch, anders als bei gewöhnlichen Farbliteralen, Text eingeschlossen werden. Dieser sollte dann in der entsprechenden Farbe
im Spiel bei einer Ausgabe angezeigt werden.

\section{Notation}
|c<Farbe>[<Text>]|r
<<Objekt-Id>,<Feld-Id>[, %]>

\end{document} 
