\chapter { Präprozessoren }

Präprozessoren werden vor der restlichen Syntaxvalidierung ausgewertet.
Sie können zur Vorauswertung verwendet werden, um zu bestimmen, welcher Code kompiliert und welcher vom Compiler ignoriert werden soll.
Präprozessoren werden mit dem #-Zeichen eingeleitet. Darauf muss eine der Anweisungen folgen.
Eine Anweisung mehrzeilig zu gestalten wird momentan nicht unterstützt und ist eigentlich auch nicht
notwendig, da man keine eigenen Makros (wie in C und C++) definieren kann.
Zwischen dem #-Zeichen und der Anweisung können beliebig viele Leerzeichen bzw. Tabulatoren stehen.

#include [ jass | zinc | vjass | jasspp | cjass ] [ once ] "<Dateipfad>" Wird cjass verwendet, so wird die gesamte Datei ignoriert! Wird once angegeben, wird die Datei nur einmal eingebunden.
#inject CustomMapScript | InitGlobals | InitSounds | CreateNeutralHostile | CreateNeutralPassiveBuildings | CreatePlayerBuildings | CreatePlayerUnits | CreateAllUnits | InitCustomTriggers | RunInitializationTriggers | InitCustomPlayerSlots | InitCustomTeams | InitAllyPriorities | main | config // injectet die entsprechende Funktion und ersetzt ihren Inhalt durch den Inhalt des Inject-Blocks. Existiert die Funktion nicht (z. B. da es keine eigenen Teams gibt, so zeigt der Compiler eine Fehlermeldung an)
#endinject
#initjasspp				Initialisiert die globale Hashtable und erzeugt sämtliche Prototypauslöser. Nützlich für eine Injection der main-Funktion
#if (<konstanter Ausdruck>)		Zum Beispiel (WC3_VERSION == "1.21" && constantValue > 10)
#else if (<konstanter Ausdruck>)
#else
#endif
#error <konstanter Ausdruck>		Bricht den Kompiliervorgang ab und gibt den konstanten Ausdruck als Fehlermeldung aus.
#jass
#endjass
#zinc
#endzinc
#vjass
#endvjass
#jasspp
#endjasspp
#cjass					Nur reserviert. Code wird momentan noch ignoriert.
#endcjass				Nur reserviert. Code wird momentan noch ignoriert.
#external <Ausdruck> <Parameter>	Ruft einen externen Befehl mit Parametern auf. Externe Befehle müssen vom Compiler ausgeliefert und ausgewertet werden.

\section { Vordefinierte Konstanten }
Es gilt zu beachten, dass die folgenden Konstanten normale JASS++-Konstanten eines nativen Typs sind und somit auch in Nicht-Präprozessor-Anweisungen verwendet werden können.
Typ Bezeichner					Beschreibung
string LANGUAGE					Enthält eine Zeichenkette mit der aktuellen Skriptsprache ("JASS++", "Zinc", "vJass", "JASS")
string LANGUAGE_VERSION				Enthält eine Zeichenkette mit der Version der aktuellen Skriptsprache
boolean JASS					Enthält einen boolean-Wert, der angibt, ob die aktuelle Skriptsprache JASS ist.
boolean ZINC					Enthält einen boolean-Wert, der angibt, ob die aktuelle Skriptsprache Zinc ist.
boolean VJASS					Enthält einen boolean-Wert, der angibt, ob die aktuelle Skriptsprache vJass ist.
boolean JASSPP					Enthält einen boolean-Wert, der angibt, ob die aktuelle Skriptsprache JASS++ ist.
string COMPILER					Enthält eine Zeichenkette mit dem Namen des verwendeten Compilers.
string COMPILER_VERSION				Enthält eine Zeichenkette mit der Version des verwendeten Compilers.
string WC3_VERSION				Enthält eine Zeichenkette mit der Warcraft-3-Version (ist zur Laufzeit aktuell!).
string WC3_TFT_VERSION				Enthält eine Zeichenkette mit der Warcraft-3-The-Frozen-Throne-Version. Falls The Frozen Throne nicht verwendet wird, ist der Wert gleich 0 gesetzt (ist zur Laufzeit aktuell!).
boolean DEBUG_MODE				Enthält einen boolean-Wert, der angibt, ob der Debug-Modus aktiviert ist.
string FILE_NAME				Enthält eine Zeichenkette des Dateinamens der aktuellen Datei.
string LINE_NUMBER				Enthält eine Zeichenkette der Zeilennummer der aktuellen Zeile.
string SCOPE_NAME				Enthält eine Zeichenkette des Namens des aktuellen Gültigkeitsbereichs.
string SCOPE_NAME_FULL				Enthält eine Zeichenkette des gesamten Namens (auch alle oberen Gültigkeitsbereiche) des aktuellen Gültigkeitsbereichs.
boolean OPTIMIZATION_INLINE_FUNCTIONS		Enthält einen boolean-Wert, der angibt, ob die Optimierungsoption im Compiler aktiviert ist.
boolean OPTIMIZATION_REMOVE_WHITE_SPACES	Enthält einen boolean-Wert, der angibt, ob die Optimierungsoption im Compiler aktiviert ist.
boolean OPTIMIZATION_REMOVE_CONSTANTS		Enthält einen boolean-Wert, der angibt, ob die Optimierungsoption im Compiler aktiviert ist.
boolean OPTIMIZATION_REMOVE_COMMENTS		Enthält einen boolean-Wert, der angibt, ob die Optimierungsoption im Compiler aktiviert ist.
TODO: Hier noch weitere Optimierungsoptionen hinzufügen.

\section { Dateipfade für die "include"-Anweisung }
Dateipfade müssen wie bei UNIX-Systemen geschrieben werden. . steht für das aktuelle und ..
für das darüber liegende Verzeichnis. Verzeichnisse werden durch das /-Zeichen getrennt.
Standardmäßig wird im Verzeichnis der aktuellen Code-Datei nach angegebenen Dateinamen gesucht.