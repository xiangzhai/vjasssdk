\chapter{Lambda-Funktionen}
Lambda-Funktionen sind namenlose Funktionen, sozusagen Funktionen, bei denen nur der Code-Inhalt angegeben werden muss.
In JASS++ ist es möglich Lambda-Funktionen zu deklarieren, welche keinen Rückgabewert haben und keine Parameter entgegennehmen.
Lambda-Funktionen können entweder in Variablen des Typs code oder in Funktionsvariablen gespeichert werden. Bei letzterer Variablenart
müssen sie jedoch als "threaded" deklariert werden, bei ersterer ist dies optional.
Außerdem kann ihr Inhalt an andere Funktionen wie bei gewöhnlichen Funktionsvariablen bzw. Variablen des Typs "code" weitergegeben
werden.
Lambda-Funktionen könne auch direkt an Paramater der beiden Typen übergeben werden.

\section{Notation}
code <Variablenname> = [threaded]
	<Funktions-Code>
abstract void () <Variablenname> = threaded
	<Funktions-Code>

\section{code-Beispiel}
void test()
	code testCode =
		integer i
		for (i = 0; i < 100; i++)
			Print("Iteration " + I2S(i))

\section{Funktionsvariablen-Beispiel}
void test()
	abstract void () testFunctionVariable = threaded
		integer i
		for (i = 0; i < 100; i++)
			Print("Iteration " + I2S(i))
	execute testFunctionVariable
	evaluate testFunctionVariable

\section{Parameter-Beispiel}
void test()
	timer testTimer = CreateTimer()
	TimerStart(testTimer, 2.0, true,
		Print("Timer function runs!")
	)