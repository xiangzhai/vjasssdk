\chapter{Schlüsselwörter}

Im Folgenden wird zwischen Standard, was auf gängige Programmiersprachen verweist, und Warcraft-spezifisch, was auf Warcraft-spezifische
Eigenschaften verweist, unterschieden.

\section{Typ-Schlüsselwörter}
Standard: integer, string, boolean, real
Warcraft-spezifisch: code, handle, agent, event, player, widget, unit, destructable, item, ability, buff, force, group, trigger, triggercondition, triggeraction, timer, location, region, rect, boolexpr, sound, conditionfunc, filterfunc, unitpool, itempool, race, alliancetype, racepreference, gamestate, igamestate, fgamestate, playerstate, playerscore, playergameresult, unitstate, aidifficulty, eventid, gameevent, playerevent, playerunitevent, unitevent, limitop, widgetevent, dialogevent, unittype, gamespeed, gamedifficulty, gametype, mapflag, mapvisibility, mapsetting, mapdensity, mapcontrol, playerslotstate, volumegroup, camerafield, camerasetup, playercolor, placement, startlocprio, raritycontrol, blendmode, texmapflags, effect, effecttype, weathereffect, terraindeformation, fogstate, fogmodifier, dialog, button, quest, questitem, defeatcondition, timerdialog, leaderboard, multiboard, multiboarditem, trackable, gamecache, version, itemtype, texttag, attacktype, damagetype, weapontype, soundtype, lightning, pathingtype, image, ubersplat, hashtable

\section{Prozedual-Schlüsselwörter}
Standard: type, var, function, if, else, switch, case, break, continue, default, while, do, for, foreach, forever, try, catch, finally, throw, enum, size, const, alias, init, static, dynamic, scope
Warcract-spezifisch: native, mapinit, debug, ai, load, local

\section{Operator-Schlüsselwörter}
Warcraft-spezifisch: id, hash

\section{Funktional-Schlüsselwörter}
Standard: operator, return

\section{Generik-Schlüsselwörter}
Standard: template

\section{Modularisierung-Schlüsselwörter}
Standard: scope, private, protected, public

\section{Objektorientierung-Schlüsselwörter}
Standard: class, this, self, parent, new, delete, copy, init, virtual, friend, abstract, mutable

\section{Multi-Threading-Schlüsselwöter}
Warcract-spezifisch: threaded, execute, executewait, evaluate, executions, evaluations, sleeps, sleepon, sleepoff, reset, sleep, wait

\section{Serialisierungs-Schlüsselwörter}
Warcract-spezifisch: save, load, flush, exists

\section{Möglicher Zusatz}
executefast ermöglicht den Aufruf mit ExecuteFunc und Parameterspeicherung, jedoch könnte
der Aufruf nicht mittels calls abgerufen werden und es gäbe eventuell weitere Nachteile
(Geschwindigkeit).

\section{Zusammenfassung der Qualifizierer}
Standard: type, var, function, enum, const, template, private, protected, public, class, init, virtual, static, dynamic, scope, friend, abstract, mutable
Warcraft-spezifisch: native, mapinit, debug, threaded, ai, load, local

\section{Zusammenfassung der Anweisungsblockschlüsselwörter}
Standard: if, else, switch, case, default, while, do, for, foreach, forever, try, catch, finally
Warcraft-spezifisch: threaded

\section{Zusammenfassung der Anweisungsschlüsselwörter}
Standard: break, continue, throw, size, return, new, delete, copy
Warcraft-spezifisch: id, hash, execute, executewait, evaluate, executions, evaluations, sleeps, sleepon, sleepoff, reset, sleep, wait, save, load, flush, exists

\section{Zusammenfassung der Behältertyp-Schlüsselwörter}
Standard: string
Warcraft-spezifisch: force, group

\section{Bezeichnerspezifizierungsschlüsselwörter}
Standard: type, scope, ai, load, var, function, abstract, enum, class

\section{Zusammenfassung der optionalen Schlüsselwörter}
Standard: type, var, function, scope 
