\chapter{Gültigkeitsbereiche}
Gültigkeitsbereiche definieren, innerhalb welches Bereiches eine Deklaration ihre Gültigkeit hat. Außerhalb dieses Bereiches
kann die Deklaration nur über den Bezeichner, insofern ein solcher existiert, des Gültigkeitsbereiches bzw. den .-Operator erreicht werden.
Innerhalb kann der Bezeichner mit dem .-Operator optional verwendet werden.
In bestimmten Fällen geschieht dies implizit (Methodenaufruf, Pseudonyme, Gültigkeitsbereichverwendung usw.).
Außerdem können außerhalb eines Gültigkeitsbereiches nur statische Deklarationen dessen verwendet werden.
Dies sind z. B. statische Variablen, Klassendeklarationen usw. (siehe unten).
Ein Gültigkeitsbereich besitzt in JASS++ nicht zwingend einen Bezeichner und ist durch eine durch einen Tabulator eingerückte Ebene definiert.
Der äußerste Gültigkeitsbereich trägt keinen Bezeichner und befindet sich auf der Ebene ohne jegliche Einrückung.
Er erlaubt es dem Entwickler, eine Funktion direkt zu deklarieren, ohne einen Gültigkeitsbereich für sie zu definieren.
Gültigkeitsbereiche können beliebig verschachtelt werden, unabhängig ihrer Art. Das bedeutet, dass z. B. auch Klassen innerhalb von
Klassendeklarationen deklariert werden können.
Je tiefer der Gültigkeitsbereich, desto höher ist die Priorität des Bezeichners. Die Bezeichner können sich gegenseitig überdecken.
Dies ist explizit gewollt. Über die Bezeichner der äußeren Gültigkeitsbereiche bzw. den .-Operator kann auf Bezeichner höherer
Gültigkeitsbereiche zugegriffen werden. Dies ermöglicht eine Verschachtelung beliebig vieler Gültigkeitsbreiche.
Zu beachten gilt, dass lokale Variablen nur im Funktionskopf deklariert werden können, wodurch eine erneute Deklaration und Überlagerung
der äußeren lokalen Variablen, durch innere in neuen Gültigkeitsbereichen nicht möglich ist.
Ein Gültigkeitsbereich kann mit dem optionalen Schlüsselwort "scope" eingeleitet werden. Dieses wird den weiteren Qualifizierern vorangestellt.

Es existieren folgende Arten von Gültigkeitsbereichen:
\begin{enumerate}
\item benannte Gültigkeitsbereiche
\item unbenannte Gültigkeitsbereiche
\item sprachspezifische Gültigkeitsbereiche
\end{enumerate}

Es existieren folgende sprachenspezifische Arten von Gültigkeitsbereichen:
\begin{enumerate}
\item Anweisungsblöcke
\item Funktionsdeklarationen
\item Aufzählungsdeklarationen
\item Klassendeklarationen
\item Qualifizierer-Gültigkeitsbereiche
\end{enumerate}

Folgende Deklarationen sind nicht statisch sondern dynamisch und können daher von einem äußeren Gültigkeitsbereich bzw. einem inneren, bei welchem es sich um einen
sprachenspezifischen handelt, nicht verwendet werden:
\begin{enumerate}
\item Deklarationen nicht statischer Variabler in Funktionen
\item Deklarationen nicht statischer Elemente in Klassen
\end{enumerate}

\section{Anmerkung zu Funktionen}
Es gilt zu beachten, dass ebenfalls auf Eigenschaften von Funktionen zugegriffen werden kann, jedoch nur, wie auch bei Klassen,
auf die statischen Variablen (siehe "Zugriffsmodifikatoren").

\section{Deklarationsvoraussetzungen}
Benannte Gültigkeitsbereiche können bei ihrer Deklaration eben solche voraussetzen.
Dazu wird der :-Operator verwendet. Dies stellt sicher, dass die Voraussetzungen vor dem jeweiligen
Gültigkeitsbereich deklariert werden. Sollte es dabei zu Kreisabhängigkeiten kommen, indem sich z. B. zwei
Deklarationen gegenseitig voraussetzen, so muss der Compiler eine Fehlermeldung ausgeben.
Ansonsten muss er sicherstellen, dass die Objekte in der richtigen Reihenfolge deklariert werden.
Deklarationsvoraussetzungen sind notwendig, da Funktionen in JASS erst nach ihrer Deklaration im Skript
verwendet werden können.
Eine Umgehung dieser Beschränkung kann nur bei als "threaded" deklarierte Funktionen erreicht werden
(siehe dazu Abschnitt "Funktionen").
Abgeleitete Klassen werden implizit als Deklarationsvoraussetzung angegeben.

\subsection{Notation}
: <Bezeichner 1>, <Bezeichner 2>, ... <Bezeichner n>

\section{Initialisierer}
Die Initialisierungsfunktion wird genau vor dem ersten Zugriff auf ein Gültigkeitsbereichselement aufgerufen.
Dies wird vom Compiler durch eine Laufzeitüberprüfung ermöglicht und kann relativ viel Zusatzcode erzeugen, da bei jedem Zugriff von außerhalb
überprüft werden muss, ob der Gültigkeitsbereich bereits initialisiert wurde.
Die Funktion darf nicht manuell aufgerufen werden und hat deshalb keine Zugriffsmodifikatoren, keinen Rückgabetyp, keinen Bezeichner und keine Parameter.
Werden mehrere Initialisierer im selben Gültigkeitsbereich deklariert, so sollte der Compiler einen Hinweis ausgeben.
ANMERKUNG: Zunächst wurde der Initialisierer mit dem Schlüsselwort "static" deklariert, jedoch gerät diese Art der Deklaration in einen Konflikt mit
den Qualifizierer-Gültigkeitsbereichen.

\subsection{Notation}
static
	<Anweisungen>

\section{Notation von Gültigkeitsbereichen}
[scope] [<weitere Qualifizierer>] [<Bezeichner>] [<Eigenschaften des jeweiligen Sprachkonstrukts>] [<Deklarationsvoraussetzungen>]
	[Initialisierer]
	[<Zugriffsmodifikatoren>] [<Deklarationen>]

\section{Bespiele}
void test()
	int x = 0

	// nächster Gültigkeitsbereich
		.x = .x // Expliziter Verweis auf den äußeren Gültigkeitsbereich mit dem .-Operator
		.x = 10 // x im äußeren Gültigkeitsbereich erhält den Wert 10
		x = 12 // x im äußeren Gültigkeitsbereich erhält den Wert 12, da im aktuellen Gültigkeitsbereich keine Variable namens x deklariert wurde

scope
	int y = 10

	scope test
		int y = 12; // Neue Deklaration, andere Variable!
		y = .y; // Zuweisung des Wertes der Variablen des äußeren Gültigkeitsbereiches.

class Test
	public static const integer i = 10
	public static const integer j = i // Der Bezeichner kann innerhalb des Gültigkeitsbereiches weggelassen werden.
	public static const integer k = Test.j // Er ist optional.