\chapter{Klassen}
JASS++ ist eine objektorientierte Sprache, die ein relativ herkömmliches Klassenkonzept verwendet, welches jedoch um einige
besondere Neuerungen erweitert wurde, um eine gewisse Konsistenz verschiedener Schlüsselwörter aufzubauen und vor allem dem
Entwickler neue Möglichkeiten zu bieten, Code einfacher zu implementieren und das Verhalten seines Programms besser abschätzen
zu können.
JASS++ unterstützt Mehrfachvererbung, abstrakte Klassen, statische, konstante und virtuelle Klassenelemente und -methoden, sowie
Behälterklassen und native Klassen.
Bei den letzteren beiden handelt es sich um spezielle Neuerungen, die für ein konsistentes Konzept eingeführt wurden.
Behälterklassen erlauben die Deklaration eines eigenen Behältertyps.
Native Klassen erlauben die Deklaration eines Typs, welcher rein funktional, sprich ohne Objektorientierung, verwendet werden kann.
Mit dem Schlüsselwort "virtual" überlässt es JASS++ dem Entwickler, auf welche Art er seine Methoden und Elemente implementiert und
welche Typsicherheiten bzw. Ableitungsmöglichkeiten er dabei garantieren möchte

\section{Grundlegendes}
Mit dem Schlüsselwort "self" spricht man die eigene Klasse bzw. deren Gültigkeitsbereich an.
Bei virtuellen statischen Methoden muss darauf geachtet werden, mit welcher Klasse
die erste statische Methode aufgerufen wurde, damit diese die korrekten weiteren statischen Methoden aufruft, insofern sie selbst z. B. zu einer
der Elternklassen und nicht zur Klasse für die der Aufruf galt gehört.
Der Zugriff auf die Klasse kann allerdings auch, wie bei gewöhnlichen Gültigkeitsbereichen, über ihren Namen oder den .-Operator erfolgen.
In diesem Fall wird ebenfalls überprüft, ob es sich um eine virtuelle statische Methode handelt und die der korrekten Klasse aufgerufen.
Mit dem Schlüsselwort "this" spricht man die verwendete Instanz an. "this" ist wie ein referenzbasierter Parameter mit dem Typ
der Klasse, der jeder nicht statischen Methode übergeben wird.
Konstanten Methoden wird der Parameter mit einem konstanten Typ übergeben.

\section{Abstrakte Klassen}
Abstrakte Klassen werden mit dem Schlüsselwort "abstract" deklariert. Von ihnen können keine
Instanzen erzeugt werden. Wird eine abstrakte Klasse abgeleitet, so muss die abgeleitete Klasse sämtliche
abstrakte Methoden und Elemente implementieren, um nicht mehr selbst als abstrakt zu gelten.
Klassen mit mindestens einem abstrakten Element oder einer abstrakten Methode müssen als abstrakt deklariert werden.

\section{Native Klassen}
Native Klassen werden mit dem Schlüsselwort "native" deklariert. Sie können abstrakt sein.
Native Klassen bewirken eine Behandlung der Klasse wie einen nativen, referenzbasierten Typ.
Daher werden für alle Methoden Funktionen erzeugt. Dabei tragen Konstruktoren alle den Bezeichner
"Create<Klassenbezeichner>", der Destruktor den Bezeichner "Destroy<Klassenbezeichner>" und der Kopier-
konstruktor den Bezeichner "Copy<Klassenbezeichner>".
Die Konstruktor- und die Kopierkonstruktorfunktionen erhalten als Rückgabetyp den Klassentyp.
Der Destruktor erhält einen Parameter des Klassentyps.
Alle anderen Funktionen werden ebenfalls aus dem Klassenbezeichner und dem Methodennamen erzeugt:
"<Klassenbezeichner><Methodenbezeichner>" und sie erhalten selbstverständlich das Klassenobjekt als
ersten Parameter.
Es gilt außerdem zu beachten, dass die Methodenbezeichner in den Funktionsbezeichnern und der Klassenbezeichner in den
Funktionsbezeichnern, in denen er an zweiter Stelle kommt automatisch mit einem Großbuchstaben beginnen,
um dem Schema der Nativen aus JASS zu gleichen.
Native Klassen werden hauptsächlich in der Standardbibliothek von JASS++ verwendet, um eine nicht objektorientierte
Zugriffsmöglichkeit zu erlauben.

\section{Behälter-Klassen}
Behälter-Klassen werden definiert, indem die Klasse mindestens die Operatoren "size" und "[]" implementiert (als nicht statische Methoden).
Beide Operatoren können sowohl normal als auch als "const" deklariert werden.
Dabei sollte der "size"-Operator die Größe des Behälters und der "[]" Zugriff auf die einzelnen Elemente zur Verfügung stellen.
Die Instanz entspricht somit einem eindimensionalen Array (siehe "Behältertypen").

\section{Vererbung}
Mit dem Schlüsselwort "base" spricht man die Elternklassen an. Diese können aber auch über ihren Namen, wie bei gewöhnlichen
Gültigkeitsbereichen, angesprochen werden. Da sie normalerweise kein äußerer Gültigkeitsbereich sind, funktioniert die Verwendung des "."-Operators nicht.
Innerhalb der abgeleiteten Klasse können sämtliche Methoden und Elemente aller Elternklassen verwendet werden, sie nicht als "private" deklariert wurden.

\subsection{Mehrfachvererbung}
Haben mehrere geerbte Klassen eine identische Methode (gleicher Bezeichner, gleiche Parameter, konstant/nicht konstant) bzw. ein
identisches Element (gleicher Bezeichner, gleicher Typ, statisch/nicht statisch), so
muss beim Aufruf der Methode bzw. dem Zugriff auf das Element der Gültigkeitsbereich der jeweiligen Klasse vorangestellt werden,
um zu spezifizieren, welche Methode aufgerufen bzw. auf welches Element zugegriffen werden soll.
Unnötig wird dies erst, sobald eine Implementation der Methode bzw. des Elements in der abgeleiteten Klasse selbst existiert.
Dabei gilt es unbedingt zu beachten, dass im Falle einer Implementation der Speicher für alle Elemente belegt wird und
die Elemente der anderen Klassen weiterhin in deren Methoden verwendet werden. Das gleiche gilt auch für Methoden.
Wird also eine Instanz zu einer jeweiligen Elternklasse konvertiert, so wird im Falle einer nicht virutellen, mehrfach
vorhandenen Methode, die Implementation der Elternklasse aufgerufen.
Eine Ausnahme hierbei machen lediglich virtuelle Elemente bzw. Methoden, da sie Objektweise gespeichert werden.
Der Compiler sollte im Fall von Überschneidungen eine Warnung ausgeben, falls diese nicht virtuell "ausgeglichen" werden.

Implementation:
Die Implementation virtueller Methoden und Elemente ist eine nicht besonders einfach zu lösende Aufgabe, besonders ohne eine
globale Hash-Tabelle. Man muss sämtliche Überprüfungen bei Zugriffen (vor allem auch in statischen Methoden) beachten!

\subsection{protected}
Ist eine Deklaration innerhalb des Klassengültigkeitsbereiches "protected", so kann darauf nur von abgleiteten Klassen
der Klasse zugegriffen werden.

\subsection{Notation}
(public | private | protected <Klassenbezeichner>, ...)

\section{Instanzen}
Instanzen werden mit Hilfe des Konstruktors einer Klasse erzeugt und mit Hilfe des Destruktors wieder freigegeben werden.
Es können nur mehr als 8192 Instanzen erzeugt werden, wenn eine Klasse eine Maximalgrößenangabe hat oder eine globale Hashtable exisitiert (Zweiteres ist in der Regel der Fall).
In diesem Fall werden die Instanzdaten in der globalen Hash-Tabelle oder in zusätzlichen globalen Variablen gespeichert.
Falls die angegebene Maximalgröße den Wert 8192 übersteigt und keine globale Hash-Tabelle existiert,
sollte der Compiler eine Warnung bzw. einen Hinweis ausgeben.
Die Maximalgröße einer Klasse kann mit dem "size"-Operator abgefragt werden: <Klassenbezeichner>.size
Im Test-Modus muss zur Laufzeit 0 vom "new"-Operator zurückgegeben und eine Fehlermeldung ausgegeben werden, falls die Instanzgrenze
überschritten wurde.
Außerdem muss im Test-Modus eine Fehlermeldung zurückgegeben bzw. abgebrochen werden, wenn der "delete"-Operator auf eine 0-Instanz angewandt wird.

\section{Elemente}
Klassenelemente entsprechen Variablendeklarationen mit Zugriffsmodifikatoren, die zusätzlich statisch oder mutabel (veränderlich/mutable) sein können.
Statische Elemente werden zur Laufzeit pro Klasse einmal erzeugt und sind wie globale Variablen innerhalb des Gültigkeitsbereiches der Klasse.
Nicht statische Elemente exisitieren zur Laufzeit pro Klasseninstanz einmal.
Sie können per <Klasseninstanzvariable>.<Elementbezeichner> angesprochen werden.
Ihr erster Zugriff findet im Konstruktor statt.
Als "mutable" deklarierte Elemente können auch von konstanten Methoden oder bei einem Direktzugriff bei konstanten Instanzen verändert werden.
Eine Klasse kann ebenfalls Array-Elemente enthalten (siehe "Typen - Arrays").

\subsection{Virtuelle Elemente}
Elemente können mit Hilfe des Schlüsselworts "virutal" als virtuell deklariert werden. Virtuelle nicht statische Elemente werden pro
Klasseninstanz gespeichert. Daher wird nicht das Element einer Basisklasse angesprochen, falls eine Konvertierung der Instanz zu
dieser erfolgt, sondern das virtuelle.
Statische virtuelle Elemente garantieren ebenfalls einen korrekten Zugriff auf die Implementation des eigentlichen Typs der Instanz.
So wird bei Zugriffen auf ein statisches virtuelles Element in einer Methode der Instanz stets überprüft, welches zugehörige für die
Instanz hinterlegt wurde.
Wird in einer virtuellen Methode auf ein nicht virtuelles Element zugegriffen, sollte der Compiler eine Warnung ausgeben.

\subsection{Abstrakte Elemente}
Klassenelemente, die mit dem Schlüsselwort "abstract" deklariert wurden, müssen in der abgeleiteten Klasse implementiert werden,
damit diese nicht als abstrakt gilt, falls sie nicht selbst als abstrakt deklariert wurde.
Sie können erst in der abgeleiteten Klasse bzw. nach ihrer Implementation verwendet werden.
Eine Klasse mit mindestens einem abstrakten Element, muss als abstrakt deklariert werden.

\subsection{Zugriffsmodifikatoren}
Zugriffsmodifikatoren von nicht privaten bzw. vererbten Klassenelementen können in der abgeleiteten Klasse umdeklariert werden,
was ihre Zugreifbarkeit verändert, ohne dass sie neu definiert werden müssen.
Dies geschieht, indem man in der abgeleiteten Klasse lediglich die Zugriffsmodifikatoren und den Bezeichner verwendet:
[ Zugriffsmodfikatoren ] <Bezeichner>;

\subsection{Elemente der Basisklasse}
Mit base.<Elementname> oder <Basisklasse>.<Elementname> kann ein Element der Basisklasse angesprochen werden.
Mehrdeutigkeiten müssen jedoch über den eigentlichen Bezeichner der Basisklasse aufgelöst werden.

\subsection{Mutable Elemente}
Klassenelemente, die als "mutable" deklariert wurden, können auch von konstanten Methoden verändert werden. Dies gestattet die
Verwendung einer Ausnahme, da es in seltenen Fällen vorkommen kann, dass ein Element von bestimmten Stellen aus veränderbar sein
muss.

\subsection{Elementeüberschreibung}
Sämtliche geerbte Elemente der Basisklassen können überschrieben werden.
Wird ein nicht virtuelles Element einer Basisklasse überschrieben und der Typ der Basisklasse irgendwo im Code verwendet, so sollte
der Compiler eine Warnung ausgeben, dass möglicherweise nicht das richtige verwendet wird.
Es gilt zu beachten, dass nicht virtuelle Elemente in der abgeleiteten Klasse mehrfach existieren können (siehe "Mehrfachvererbung").

\subsection{Notation}
[Zugriffsmodifikatoren] [static] [mutable] [virtual] [abstract] <Variablendeklaration>;

\section{Methoden}

\subsection{Virtuelle Methoden}
Als virtuell deklarierte nicht statische Methoden, werden pro Klasseninstanz gespeichert.
So wird immer die Methode der Klasse aufgerufen, welche bei der Erzeugung der Instanz angegeben wurde.
Virtuelle Methoden müssen zwingend "threaded" sein.
Achtung: Virtuelle nicht statische Methoden sollten nicht im Konstruktor aufgerufen werden, da sie durch die Konstruktoren
erst überschrieben und der Klasseninstanz zugewiesen werden. Der Compiler sollte in diesem Fall eine Warnung ausgeben.
Der Konstruktor selbst kann nicht virtuell sein.
Statische Methoden können ebenfalls virtuell sein und werden entsprechend des zugehörigen Objekts innerhalb einer Methode aufgerufen.
Wird eine nicht virtuelle statische Methode aus einer virtuellen Methode heraus aufgerufen, sollte der Compiler eine Warnung ausgeben.
Virtuelle Methoden bewirken, dass diese auch aufgerufen werden, wenn das Objekt zu einem Basistyp konvertiert wurde, da pro
Objekt Verweise auf die entsprechenden Methoden gespeichert werden. Dies gilt für statische und nicht statische virtuelle Methoden.
Bei virtuellen statischen Methoden kommt noch die Besonderheit hinzu, dass bei einem Aufruf der statischen Methode Kindklasse.statischeMethode(),
welche jedoch in der Kindklasse nicht neu definiert wurde, weshalb automatisch die der Elternklasse aufgerufen wird, von der statischen Methode
überprüft werden muss, von welcher Klasse aus sie aufgerufen wurde, damit sie bei virtuellen statischen Methoden die richtigen der Kindklasse und
nicht die der Elternklasse aufruft.
Wird also eine virtuelle Methode aufgerufen, so ist garantiert, dass sämtliche andere virtuellen Methoden, die aus dieser heraus aufgerufen werden,
sich auf der selben bzw. niedrigsten Klassenebene des Objekts befinden, falls sie dort deklariert wurden.

\subsection{Abstrakte Methoden}
Abstrakte Methoden werden mit dem Schlüsselwort "abstract" deklariert.
Sie besitzen keinen eigenen inneren Gültigkeitsbereich, sondern werden mit einem Semikolon abgeschlossen.
Außerdem müssen sie als virtuell deklariert werden.
Abstrakte Methoden müssen allesamt implementiert werden, falls die abgeleitete Klasse nicht ebenfalls als abstrakt deklariert wurde.

\subsection{Zugriffsmodifikatoren}
Zugriffsmodifikatoren von geerbten Methoden können in der abgeleiteten Klasse "umdeklariert" werden, was ihre Zugreifbarkeit verändert,
ohne dass sie neu definiert werden müssen.
Dies geschieht, indem man in der abgeleiteten Klasse lediglich die Zugriffsmodifikatoren und den Bezeichner verwendet:
[ Zugriffsmodfikatoren ] <Bezeichner>;

\subsection{Methoden der Basisklasse}
Mit base.<Methodenname> oder <Basisklasse>.<Methodenname> kann eine Methode der Basisklasse angesprochen werden.
Mehrdeutigkeiten müssen jedoch über den eigentlichen Bezeichner der Basisklasse aufgelöst werden.

\subsection{Konstante Methoden}
Nichtstatische Methoden, die mit dem Suffix "const" deklariert wurden, erhalten das Argument "this" mit konstantem Typ
und gelten somit auch als eine andere Deklaration der gleichen Methode, welche nicht als "const"
deklariert wurde (siehe Funktionen).
Innerhalb ihrer Definition dürfen keine Elemente der Klasse verändert bzw. keine nicht konstanten Methoden der Klasse
aufgerufen werden. Bei nicht statischen Methoden erscheint dies logisch, da der Parameter "this" einen konstanten Typ besitzt
und sowieso nicht übergeben werden könnte.

Statische Methoden, die mit dem Suffix "const" deklariert wurden, dürfen keinerlei nicht als "const" deklarierte, statische
Methoden der Klasse aufrufen, geschweigedenn irgendwelche statischen Elemente der Klasse verändern, sondern nur zurückgeben.

Bei beiden Arten von "const"-Methoden gilt die Ausnahme für "mutable"-Elemente.
Diese dürfen auch von konstanten Methoden verändert werden.

\subsection{Methodenüberschreibung}
Sämtliche geerbte Methoden der Basisklassen können überschrieben werden. Dabei ist es optional, ob die ursprünglichen Methoden innerhalb
der Methode aufgerufen werden.
Wird eine nicht virtuelle Methode einer Basisklasse überschrieben und der Typ der Basisklasse irgendwo im Code verwendet, so sollte
der Compiler eine Warnung ausgeben, dass möglicherweise nicht die richtige aufgerufen wird.

\subsection{Notation}
[static] [virtual] [abstract] Funktionsdeklaration [const]

\subsection{Konstruktor}
Der Konstruktor einer Klasse darf nicht virtuell und somit auch nicht abstrakt sein.
Konstruktoren einer Klassevererbungshierarchie werden von oben nach unten, von links nach rechts, aufgerufen.
Empfangen alle Elternklassenkonstruktoren bestimmte Argumente und wird nicht mindestens einer davon manuell aufgerufen,
so gibt der Compiler einen Fehler zurück, da die Elternklassen nicht richtig konstruiert wurden.
Jede nächstgelegene Elternklasse muss konstruiert werden, ob implizit, falls möglich (keine Parameter), oder explizit.
Wird kein Konstruktor in einer Kindklasse definiert, so können selbstverständlich die der Elternklassen mit dem Bezeichner
der Kindklasse verwendet werden.
Der Konstruktor muss stets den Namen "new" tragen und besitzt keinen Rückgabetyp. Er gibt dennoch das neu erzeugte Objekt zurück,
welches vom Entwickler nicht modifiziert werden kann (die Variable "this" kann im Konstruktor verändert werden, wird aber nicht
zurückgegeben).
Dies dient der Garantie, dass ein neu erzeugtes Objekt bei einer Konstruktion zurückgegeben wird.
Er kann über den Bezeichner der Funktion ("new") aufgerufen werden.
Dabei wird der Typ automatisch anhand des Typs der Variable bestimmt, was jedoch nicht immer (z. B. bei der Verwendung von abstrakten Klassen) erwünscht ist. Daher sollte der Typ der Variable umgewandelt werden, falls eine andere Klasse verwendet werden soll.
Dies ist selbstverständlich auch bei der Übergabe als Funktionsparameter möglich.
Klassen benötigen keinen Standardkonstruktor. Wird kein Konstruktor deklariert, so wird beim Aufruf von "new" eine Funktion in einer höheren
Schicht (Compilerimplementation) aufgerufen und dennoch ein Objekt konstruiert. Jedoch sollte der Compiler eine Warnung ausgeben, falls kein
Konstruktor definiert wurde bzw. Eigenschaftsvariablen der Klasse nicht initialisiert wurden.

\subsubsection{Deklarations-Notation}
[Zugriffsmodifikatoren] new([<Konstruktorparameter>]) { [<Anweisungen>] }

\subsubsection{Aufrufs-Notation}
new ([<Konstruktorparameter>])

\subsection{Kopierkonstruktor}
Weist man einer Objektvariable den Wert einer anderen zu, so zeigen beide Variablen normalerweise auf dasselbe Objekt.
Dies kann umgangen werden, indem man den Kopierkonstruktor verwendet und das Objekt kopiert bzw. ein neues mit den gleichen
Eigenschaften erzeugt.
Dazu dient das Schlüsselwort "copy". Es einer Klassen-Instanz-Variable vorangestellt werden (wie beim Konstruktor).
Wird kein eigener Kopierkonstruktor definiert, so werden sämtliche Elementinhalte ebenfalls kopiert und außerdem
alle Kopierkonstruktoren der Elternklassen in der Reihenfolge, welche auch beim Konstruktor angewandt wird, aufgerufen.
Zu beachten gilt dabei, dass der Inhalt von Array-, Funktions- und Objektelementvariablen ebenfalls kopiert wird, sowie auch deren
Arrays, Funktionen und Objekte usw. (unter Verwendung von deren Kopierkonstruktoren). Dadurch zeigen die Elementvariablen beider Objekte nur auf
dieselben Ziele, falls es sich bei den Elementvariablen um keine Objekte handelt, sondern um referenzbasierte native Typen, wie z. B. "unit".
Der Inhalt von Elementvariablen mit kopierbasierten Typen wird selbstverständlich ebenfalls kopiert (dazu ist natürlich kein Kopierkonstruktor notwendig).
Der Kopierkonstruktor darf virtuell und abstrakt sein.
Er darf keinerlei Parameter enthalten, weshalb nur exakt ein Kopierkonstruktor pro Klasse deklariert werden kann.
Daher benötigt er bei der Deklaration wie der Initialisierer (siehe "Gültigkeitsbereiche - Initialisierer") keine Klammern für Parameter.
Außerdem darf er keinen Rückgabetyp besitzen, da wie beim Konstruktor garantiert wird, dass eine neue Instanz zurückgegeben wird.
Werden in einem selbstdefinierten Kopierkonstruktor nicht die Kopierkonstruktoren der Elternklassen aufgerufen, so sollte der Compiler eine Warnung ausgeben.

\subsubsection{Deklarations-Notation}
[Zugriffsmodifikatoren] copy { [<Anweisungen>] }

\subsubsection{Aufrufs-Notation}
copy <Ausdruck>
<Ausdruck>.copy

\subsubsection{Beispiele}
class X a = new a();
class X b = a.copy;
a = copy b;

\subsection{Destruktor}
Ein Objekt kann mit Hilfe des "delete"-Schlüsselworts gelöscht bzw. freigegeben werden.
Dabei wird der Variableninhalt der betreffenden Variable automatisch auf 0 gesetzt, was Fehlzugriffe einschränkt bzw. verhindert.
Zunächst wird der Destruktor, falls vorhanden, der Klasse aufgerufen und danach bis zur obersten Elternklasse hin alle Destruktoren
der geerbten Klassen. Dabei werden bei einer Mehrfachvererbung zunächst die Vererbungshierachien der ersten Klassen von unten nach oben,
von rechts nach links abgearbeitet (genau andersherum als beim Konstruktor).
Existiert kein selbst definierter Destruktor, so ist das Objekt dennoch löschbar. Hierbei gilt praktisch das Gleiche wie beim Konstruktor:
Es gibt stets einen automatisch erzeugten Destruktor.
Erbt die entsprechende Klasse von mindestens einer anderen Klasse, so muss der Destruktor virtuell sein. Dies soll Nichtlöschen von allozierten
Objekten verhindern.
Der Destruktor einer abstrakten Klasse muss daher schon als virtuell deklariert werden (zwar nicht unbedingt notwendig, da es sonst einen Fehler in
der Kindklasse gäbe, aber warum nicht?).
Der Destruktor besitzt ebenfalls keine Parameter und keinen Rückgabetyp und darf pro Klasse nur einmal definiert werden.
Daher benötigt er bei der Deklaration wie der Initialisierer (siehe "Gültigkeitsbereiche - Initialisierer") keine Klammern für Parameter.

\subsubsection{Deklarations-Notation}
[Zugriffsmodifikatoren] delete { }

\subsubsection{Aufrufs-Notation}
delete <Ausdruck>

\section{Notation}
[native][abstract] class <Klassenbezeichner> [ [<Maximalgröße>] ] [ Vererbungsnotation ]
{
	[ Deklarationen ]
} [globale Instanzen durch Kommata getrennt];
