\chapter{Deklaration nativer Funktionen}
Native Funktionen werden wie native Typen lediglich deklariert und nicht definiert. Die Definition erfolgt in der verwendeten Engine und kann
undefiniert bleiben, solange der Compiler weiß, wie die Funktion aufzurufen ist.
Native Funktionen aus JASS sollten mittels der JASS++-Syntax erneut bzw. ein einziges Mal deklariert werden, da man sie bei solchen Deklarationen
um konstante Parameter usw. erweitern kann.
Dies muss jedoch nicht geschehen (siehe "Kompatibilät zu JASS").
Das "function"-Schlüsselwort ist wie bei selbstdefinierten Funktionen optional und dient besserer Lesbarkeit des Codes.

\section{Notation}
native [const] [function] <Rückgabetyp> <Funktionsname>([<Parametertyp> <Parametername>, ...])