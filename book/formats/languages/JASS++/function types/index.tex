\chapter{Funktionstypen}
Funktionstypen sind Typen, die es Variablen erlauben, Verweise auf als "threaded" deklarierte Funktionen zu speichern.
Dabei müssen der Rückgabetyp und sämtliche Parametertypen übereinstimmen.
Wird eine Variable mit einem Funktionstyp bei ihrer Deklaration als "const deklariert, hat dies zur Folge, dass die Funktionen,
auf welche die Variable zeigt nicht mit den Schlüsselwörtern "execute", "executewait", "evaluate", "sleepon", "sleepoff" und "reset" aufgerufen
bzw. diese auf sie angewandt werden dürfen.
Es gilt zu beachten, dass jede Funktionsdeklaration implizit einen Funktionstypen deklariert. Funktionsbezeichner können daher als Typen
verwendet werden.
Das Schlüsselwort "function" ist optional und dient einer besseren Lesbarkeit des Codes.

\section{Notation}
abstract [function] [<Funktionstypname>] <Rückgabetyp> (Parametertyp1, Parametertyp2, ... ParametertypN) [Instanzendeklarationen]

\section{Beispiel}
abstract Predicate boolean () a = function b