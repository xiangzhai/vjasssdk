\chapter{Die Standardbibliothek}
Die Standardbibliothek ist innerhalb eines jeden JASS++-Skripts verfügbar.
Die Bezeichner von Gültigkeitesbereichen, Funktionen, Klassen, Aufzählungen und Funktionstypen beginnen groß. Die Bezeichner von Parametern, Methoden und Elementen
beginnen klein.
Eine Ausnahme macht hierbei der Gültigkeitsbereich der Standardbibliothek selbst, welcher den Bezeichner "jasspp" trägt.
Die Bibliothek ist umfangreicher als bei manch anderen Sprachen und soll nicht nur grundlegende Funktionen der Datenverarbeitung anbieten, sondern eben auch
erweiterte.
Daher existieren Unterbereiche wie "RpgApi" und "VideoApi".
Die Unterbereiche tragen alle samt das Suffix "Api" im Bezeichner, um Namenskonflikte zu vermeiden.
Die Testfunktionen des Unterbereichs "DebugApi" sind nur im Testmodus verfügbar!
Die hier zusammengestellte Gesamt-API basiert auf meinen persönlichen Erfahrungen mit JASS und vJass und sollte alles nötige enthalten und in einigen Bereichen
Standards setzen. Virtuelle Methoden, Vorlagen und Parameter mit Funktionstypen bieten dennoch eine gewisse generische Schnittstelle.
Auf das Wichtigste und möglichst Abstrakte beschränken!

\section{Definition}
\include{sl.jpp}