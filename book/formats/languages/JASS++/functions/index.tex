\chapter{Funktionen}

\section{Funktionsaufrufe}
Mehrere Funktionsdefinitionen können den gleichen Bezeichner haben, insofern sich mindestens die Parametertypen oder die Parameteranzahl unterscheiden.
Unterscheidet sich lediglich der Rückgabetyp, so gelten beide Definition nicht als unterschiedlich, da im Falle eines Aufrufes ohne eine Speicherung
des Rückgabetyps nicht sichergestellt werden könnte, welche Version der Funktion aufgerufen werden sollte.
Parametertypen unterscheiden sich auch, sobald nur einer der beiden als "const" deklariert wurde. In diesem Fall wird je nach konstanter oder nicht
konstanter Parameterübergabe die jeweilge Version der Funktionsdeklaration ausgewählt.
Wird eine Funktion aufgerufen, so können Werte für einzelne Parameter, in beliebiger Reihenfolge übergeben werden, insofern sämtliche
Parameter, welche kein Standardargument besitzen gesetzt werden:
MyFunction(y = 10, x = 100, 100, 3, 5)

\section{Standardargumente}
Parameter mit Standardargumenten erhalten automatisch das Standardargument, insofern sie nicht explizit beim Aufruf gesetzt werden.
Standardargumente müssen keine konstanten Ausdrücke sein, jedoch sollte der Compiler in diesem Fall eine Warnung ausgeben!

\section{Rückgabewerte}
Eine Funktion, die einen Rückgabetyp ungleich "void" besitzt und keine "return"-Anweisung in ihrem äußersten Gültigkeitsbereich, liefert an dieser Stelle
automatisch einen Nullwert ihres Rückgabetyps zurück. Der Compiler sollte jedoch in diesem Fall eine Warnung ausgeben.
Dies dient z. B. Situationen, in denen z. B. durch "return"-Anweisungen im Block einer "default"-Anweisung in einer "switch"-Verzweigung
garantiert ist, dass die Funktion immer einen Wert zurückgibt und dennoch keine "return"-Anweisung im äußersten Gültigkeitsbereich steht.

\section{Variablendeklarationen}
Lokale nicht statische Variable der Funktion dürfen nur am Anfang des Funktionskörpers stattfinden. Innerhalb einer Funktion deklarierte
Variable können ebenfalls Zugriffsmodifikatoren besitzen, da der Funktionskörper als Gültigkeitsbereich zählt.

\section{Ausnahmebehandlung}
Es können nur Ausnahmen der Typen direkt geworfen werden, die nach dem Schlüsselwort "throw" angegeben wurden.
Mit direkt ist gemeint, dass sie nicht von einer verschachtelten Funktion geworfen werden (siehe "Ausnahmebehandlung").

\section{Notation}
[threaded] [mapinit] [const] [function] [Rückgabetyp] Funktionsname([Parametertyp Parametername [= Standardargument]] ...) [<Deklarationsvoraussetzungen>] [throw (Typen getrennt durch Kommata)]
	<Deklarationen lokaler nicht statischer Variablen>
	<Anweisungen>

Das Schlüsselwort "function" ist optional und dient besserer Lesbarkeit des Codes.

\section{"wait"-Anweisungen}
Das Schlüsselwort "wait" sorgt für ein Warten in Spielzeit und wartet somit auch synchron (das heißt auch bei Pausierungen des Spiels wird angehalten).
Dem "wait"-Schlüsselwort folgt eine Fließkommazahlenausdruck. Bei der Kompilierung wird vor jeder "wait"-Anweisung der Zustand der Funktion mit dem Schlüssel eines neu erzeugten Zeitgebers gespeichert.
Mit Zustand sind sämtliche lokale Variablen gemeint.
Alle Teile zwischen bzw. nach "wait"-Anweisungen werden auf separate Funktionen aufgeteilt, die die entsprechenden Daten aus ihrem ablaufenden Zeitgeber laden.
Dabei muss garantiert sein, dass auch Ereignisdaten korrekt übergeben werden. Wird also eine "wait"-Anweisung ausgeführt, bricht die Funktion ab, startet einen Zeitgeber mit dem angegebenen Intervall und ruft beim Ablauf den Funktionsteil auf, in welchem die Daten geladen werden.
Daher muss die Funktion auch nicht "threaded" sein, insofern der Aufruf eines Zeitgebers in Auslöserbedingungen möglich ist (nachprüfen!).
Bei mehreren "wait"-Anweisungen entsteht so eine Verschachtelung von Zeitgeberaufrufen.
Hinweis: "executewait" hat große Ähnlichkeit mit dieser Funktionalität. "executewait" wird jedoch benötigt, damit die Funktion nach dem Aufruf weiterlaufen kann.
"wait" kann z. B. bei Videosequenzen verwendet werden, damit die Sequenz synchron abläuft.

\subsection{Notation}
wait <Fließkommazahlausdruck>

\section{"threaded"-Funktionen}
Ist eine Funktion "threaded", kann sie von überall aus mit einem der Schlüsselwörter "execute", "executewait", oder "evaluate" aufgerufen werden.
Rückgabewerte können nur bei "evaluate" zurückgegeben werden.
"evaluate" erzeugt keinen neuen Thread, es wird daher gewartet, bis der Funktionsaufruf abgearbeitet wurde. Funktionen, die mit
"evaluate" aufgerufen werden, dürfen keine "sleep"-Anweisungen bzw. "TriggerSleepAction"-Aufrufe enthalten, ob direkt oder indirekt spielt dabei keine Rolle.
Der Compiler sollte in diesem Fall eine Fehlermeldung ausgeben.
Bei "execute" und "executewait" dagegen, läuft der Code unmittelbar nach dem Aufruf weiter. Der Funktionsaufruf findet in einem eigenen
Thread statt.
"execute" und "executewait" bewirken eigentlich das Gleiche, insofern "TriggerExecute" und "TriggerExecuteWait" das Gleiche bewirken.
Jedoch gibt es eine weitere Verwendungsmöglichkeit von "executewait", die sich von "execute" deutlich unterscheidet.
Gibt man bei "executewait" vor dem Funktionsbezeichner eine "real"-Wert an, so wird die Funktion erst nach Ablauf dieser Zeit
aufgerufen, jedoch wie bei "execute" nicht auf den Aufruf gewartet.
Intern wird dies durch einen speziell dafür erzeugten Zeitgeber "timer" realisiert. Theoretisch wäre daher die Anweisung "executewait" mit einem
"real"-Ausdruck auch in Nicht-"threaded"-Funktionen möglich (erlauben?).
Der Rückgabewert von Funktionen, die mit "execute" bzw. "executewait" aufgerufen wurden, dürfen nicht verwendet werden (zeitliche Differenz!).

\subsection{Notation}
evaluate <Funktionsbezeichner>
execute <Funktionsbezeichner>
executewait [<real-Ausdruck>] <Funktionsbezeichner>

Mittels der Schlüsselwörter "executions" und "evaluations" kann herausgefunden werden, wie oft eine "threaded" Funktion per "execute" und "executewait"
bzw. "evaluate" aufgerufen wurde.
Zurückgegeben wird die Anzahl als integer.

\subsection{Notation}
executions <Funktionsbezeichner>
evaluations <Funktionsbezeichner>
<Funktionsbezeichner>.executions
<Funktionsbezeichner>.evaluations

Mittels des Schlüsselwortes "sleep", kann in einer "threaded"-Funktion gewartet werden.
Sie entspricht einem Aufruf der Funktion "TriggerSleepAction".

\subsection{Notation}
sleep <real-Ausdruck>

Mittels der Schlüsselwörter "sleeps", "sleepon" und "sleepoff" kann eine "threaded" Funktion darauf überprüft werden, ob ihr Thread gerade wartet bzw.
das Warten aktiviert oder deaktiviert werden.
Die Aktivierung und Deaktivierung des Wartens bezieht sich NUR auf alle laufenden "execute"- und "executewait"-Aufrufe der Funktion.
Sämtliche Aufrufe nach der Aktivierung bzw. Deaktivierung sind davon ausgeschlossen.
Dies liegt am Verhalten der nativen JASS-Funktion "TriggerWaitOnSleeps".
Test-Anmerkung:
"IsTriggerWaitOnSleeps" gibt nur "true" zurück, falls explizit TriggerWaitOnSleeps(<Auslöser>, true) aufgerufen wurde. Am Verhalten des Auslösers
scheint sich aber nichts zu ändern.

\subsection{Notation}
sleeps <Funktionsbezeichner> // Liefert einen Boolean-Wert.
<Funktionsbezeichner>.sleeps // Liefert einen Boolean-Wert.
sleepon <Funktionsbezeichner> // Funktion wartet bei "sleep"-Anweisungen.
sleepoff <Funktionsbezeichner> // Funktion wartet nicht bei "sleep"-Anweisungen.

Mittels des Schlüsselwortes "reset" kann eine "threaded" Funktion zurückgesetzt werden. Dies bewirkt, dass sie bei der nächsten "sleep"-Anweisung
abbricht. Dies gilt für sämtliche "execute"- und "executewait"-Aufrufe der Funktion.

\subsection{Notation}
reset <Funktionsname> // Setzt den Thread der Funktion zurück.

\section{Karteninitialisierer}
Ist eine Funktion als "mapinit" deklariert, so wird diese während der Karteninitialisierung exakt vor der Funktion RunInitializationTriggers
aufgerufen. Eine "mapinit"-Funktion darf keine Parameter entgegennehmen. Außerdem sollte der Compiler warnen, falls sie einen Rückgabetyp besitzt,
manuell aufgerufen wird oder "threaded" ist (einstellbar, "threaded" ist ebenfalls eher unnötig).
Exisitieren mehrere Funktionen, die als "mapinit" deklariert wurden, so lässt sich die exakte Aufrufreihenfolge nur durch Deklarationsvoraussetzungen
bestimmen.
Setzt eine als "mapinit" deklarierte Funktion eine andere voraus (ob direkt oder indirekt), so ist garantiert, dass sie erst nach dieser vorausgesetzten
aufgerufen wird.

\section{Konstante Funktionen}
Ist eine Funktion als "const" deklariert, so muss sie einen global konstanten Wert zurückgeben. Beim Kompilieren, wird
sie normalerweise als "constant" JASS-Funktion erzeugt. Global konstant bedeutet in diesem Zusammenhang, dass der Wert
zur Laufzeit immer je nach übergebenen Parametern derselbe ist.