\chapter{Funktionsaufrufe}
Funktionen können über ihren Bezeichner und der Angabe der notwendigen Parameterwerte aufgerufen werden. Ist ihr Rückgabetyp nicht "void", so kann
der Rückgabewert als Ausdruck verwendet werden (z. B. bei einer Zuweisung).
Des Weiteren ist es erlaubt als "threaded" deklarierte Funktionen mit einem der Schlüsselwörter
\begin{enumerate}
\item execute
\item executewait
\item evaluate
\end{enumerate}
aufzurufen.
Die Parameter können bei einem Aufruf (wie z. B. in der Skriptsprache Python) explizit gesetzt werden. Parameter mit einem vordefinierten Wert (siehe
"Funktionen") müssen nicht explizit gesetzt werden.
Diese beiden Regeln entfallen bei Funktionsvariablen, da hierbei kein fester Bezeichner und somit auch keine bestimmte Funktionsdeklaration zugeordnet werden kann.

\section{Notation}
[<execute | executewait | evaluate>] <Funktionsbezeichner | Funktionsvariable>(<Parameterwert> ...);
[<execute | executewait | evaluate>] <Funktionsbezeichner | Funktionsvariable>(<Parametername> = <Parameterwert>);