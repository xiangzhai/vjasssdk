\chapter{Bezeichner (identifiers)}

Ein Bezeichner ist ein vom Entwickler definierter Name, welcher ein bestimmtes Code-Element bezeichnet und darauf verweist.
Für Bezeichner gelten bestimmte Regeln bezüglich ihrer Definition. Bezeichner dürfen in JASS++ anders als in JASS nicht mit
dem Präfix "jasspp" beginnen, da dieses für den Compiler reserviert ist (theoretisch könnte dieser auch sämtliche Bezeichner ersetzen,
um so dagegen vorzugehen, jedoch wird es vorläufig trotzdem reserviert). Ansonsten gelten dieselben Regeln wie in JASS (z. B. keine Unterstriche am Anfang,
keine Schlüsselwörter - auch nicht die von JASS++).
Außerdem kann dem Bezeichner in bestimmten Fällen ein Schlüsselwortpräfix vorangestellt werden, welches die Art des Objekts genauer spezifiziert (siehe dazu Abschnitt "Schlüsselwörter").